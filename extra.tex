\documentclass[12pt]{article}
\usepackage[left=2cm,right=2cm, top=2cm,bottom=2cm,bindingoffset=0cm]{geometry}
\usepackage{fontspec}
\usepackage{polyglossia}
\setdefaultlanguage{russian}
\setmainfont[Mapping=tex-text]{CMU Serif}

\begin{document}

\tableofcontents

\newpage
\section{Античная философия эпохи эллинизма (эпикурейцы, стоики, скептики)}
Эллинистическая ф-я - греко-римская ф-я в период от начала походов Александра Великого до овладения
римлян Египтом. до Августина и в более поздн. эпоху - до конца Др мира (сер.6 в). 
Эконом и полит упадок Греции, закат роли полиса отражаются в греческой ф-ии. Усилия, направленные на
познание объективного мира (ф-я Аристотеля), актив участие в полит жизни, кот-е проявилось у греч ф-ов,
постепенно  замещаются  индивидуализмом,  и  морализированием  либо  скептицизмом  и  агностицизмом.  Со
временем интерес к ф-му мышлению вообще резко падает. Приходит период мистики, религиозно-фил-ого
синкретизма, христианской философии. 
Скептицизм. (ко 4 в. до н)Осн-ль - Пиррон. Как и Сократ излагал свои идеи устно. Скепсис имел место в греч ф-
ии и раньше. В эллен. эпоху складываются его принципы, ибо скепсис опред-ся не методич установками в
невозможности  дальнейшего  познания,  а  отказом  от  возможности  дойти  до  истины.  Доводы  против
правильности как чувственных восприятий, так и познаний мысли скептики объединили в десять тезисов,
тропов: 
1 подвергает сомнению положения о действтель-ности разв-я физиолог структуры животных, 
2 - подчеркивает индивидуал различия людей с точки зрения физиолог и психики, 
3 - о различии чувственных органов, в которых вещи вызывают разные ощущения и т.д. Вообщем они делали
акцент на субъективизм познания ч. Исходя из принципа "ничего не утверждать", подкрепленного тропами,
скептики отвергали любые попытки познания мира в котором всё взаимос-но и подчин-ся единым законам. 
Кинизм – (циники) (Антисфен, Диоген Синопский идр) стремились не сколько к построению законченной
теории бытия и познания, сколько к отработке и эксперемент. проверке на себе определ. образа жизни. Киники с
вызовом именовали себя “гражданами мира” и обязывались жить в любом общ-ве не по его з-нам, а по своим
собственным, с готовностью приемля статус нищих, юродивых. Положение не только крайне бедственное, но и
унизительное избирается ими как наилучшее. Киники хотели быть нагими и одинокими, соц связи и культурные
навыки – мнимость. Все виды духовной и физич. бедности предпочтительнее богатства. 
Эпикуризм. Эпикуреизм - учение и образ жизни, исходящие из идей Эпикура и его послед-ей, отдающих не
задумываясь  предпочтение  матер  радостям  жизни.  Видимо,  наиболее  выдающимся  мыслителем
эллинистического периода был Эпикур. Гл.произ: "Правило" (канон), "О природе" и.т.д. Учение Демокрита
Эпикур не принимает пассивно, но поправляет его, дополняет и развивает. Если Демокрит х-зует атомы по
величине, форме и положению в простр-ве, то Эпикур им приписывает еще одно свойство - тяжесть. Вместе с
Демокритом он признает, что атомы, движутся в пустоте. Эпикур допускает и признает закономерным и опред
отклонение  от  прямолинейного,  движ-я.  Эпикурово  понимание  случайности  не  исключает,  причинного
объяснения.  У  ч.  есть  свобода  выбора,  а  не  все  предопределено.  В  у-ии  о  душе  Эпикур  отстаивает
материалистич взгляды. Согласно Эпикуру, душа - это не нечто бестелесное, а структура атомов, тончайшая
материя, рассеянная по всему организму. Отсюда вытекает и отрицание бессмертия души. В области теории
познания Эпикур - сенсуалист. В основе всякого познания лежат ощущения, которые возникают при отделении
отражений  от  объективно  существующих  предметов  и  проникают  в  наши  органы  чувств.  Таким  образом,
основной  предпосылкой  всякого  познания  является  сущ-е  объективной  реальности  и  ее  познаваемость  с
помощью чувств. 
Стоицизм В конце IV в. до н. э. в Греции формируется стоицизм, который в эллинистическом, а также в более
позднем римском периоде становится одним из самых распространенных ф-х течений. Его основ-м был Зенон.
Трактат  "О  человеческой  природе".  Стоики  часто  сравнивали  ф-ю  с  чел  организмом.  Логику  они  считали
скелетом, этику - мышцами, а физику - душой. Более определенную форму стоическому мышлению придает
Хрисипп. Он  превращает стоическую  ф-ю  в  обширную  систему.  Стоики  хар-вали ф-ю  как "упражнение  в
мудрости". Орудием философии, ее основной частью они считали логику. Она учит обращаться понятиями,
образовывать  суждения  и  умозаключения.  Без  нее  нельзя  понять  ни  физику,  ни  этику,  которая  является
центральной частью стоической ф-ии. В онтологии стоики признают два осн принципа: материальный принцип
(материал), который считается основой, и духовный принцип - логос (бог), который проникает ч/з всю материю
и образует конкрет единичные  вещи.  Стоики,  в отличие от Аристотеля  сущностью  считали материальный
принцип (хотя, так же как и он, признавали материю пассивным, а логос (бог) - активным принципом). Понятие
бога в стоической ф-ии можно охарак-вать как пантеистическое. Логос, согласно их взглядам, пропитывает всю
природу, проявляется везде в мире. Он явл законом необходимости, провидением. Понятие бога сообщает всей
их концепции бытия детерминистский, вплоть до фатализма, х-р, который пронизывает и их этику. В области
теории  познания  стоики  представляют  по  преимуществу  античную  форму  сенсуализма.  Стоики  упрощают
аристотелевскую  систему  категорий  -  четыре  основными  категориями:  субстанция  (сущность),  количество,
определенное  качество  и  отношение,  согласно  определенному  качеству.  С  помощью  данных  категорий
постигается  дейст-сть.  Центром  и  носителем  познания  явл  душа.  Она  понимается  как  нечто  телесное,
материальное. Иногда ее обозначают как пневма (соединение воздуха и огня). Ее централь часть, в которой
локализируется способность к мышлению и вообще все то, что можно определить в нынешних терминах как
психическую деятельность, стоики называют разумом (гегемоником). Разум связывает человека со всем миром.
Индивидуальный разум явл частью мирового разума. Хотя стоики считают основой всякого познания чувства,
большое внимание они уделяют и проблемам мышления. 


\newpage
\section{Принципы марксистской философии}
Сложность и особый интерес к философии К. Маркса определены тем обстоятельством, что существовал и
существует марксизм - массовая идеология, сыгравшая огромную роль в Х1Х-ХХ вв. Как и всякая другая, эта
идеология вобрала в себя значительные идейные фрагменты философско-мировоззренческого порядка, причем
их авторство с известной долей справедливости обычно приписывали Марксу. Найти в этом переплетении чисто
философское содержание, различить Маркса-философа и Маркса-идеолога было и остается непростой задачей.
Философское  творчество  Маркса,  независимо  ни  от  каких  обстоятельств,  обладает  историко-философской
ценностью и тем самым требует изучения в русле истории немецкой философской мысли XIX в.
В последние годы в нашей стране от официально неумеренного восхищения творчеством Маркса перешли к
резко негативным оценкам, что объясняется, конечно, общей политико-идеологической ситуацией. Историку
философии, однако, не пристало разделять и пафос былых восторгов, и энтузиазм продолжающихся проклятий.
Авторитет  Маркса-философа  основан  на  несомненности  оригинального  вклада  в  гегелевское  движение.  В
области философии Маркс сам всегда считал себя учеником и последователем Гегеля, претендуя лишь на
относительную самостоятельность, и это тот именно случай, когда нужно прислушатьс к авторской самооценке.
Принадлежность Маркса-философа к гегелевской школе не вызывает сомнений. С университетских лет Маркс
близко контактировал с наиболее видными гегельянцами - Б. Бауэром, Ф. Кеппеном, затем с А. Руге, М. Гессом,
Ф. Энгельсом, состоял в переписке с Л. Фейербахом.
Маркс участвовал в гегельянских периодических изданиях, был в 1842-1843 гг. редактором "Новой Рейнской
газеты", преимущественно младогегельянского органа. Вместе с другими членами школы Маркс постепенно
перешел от увлечения "философией самосознания" Б. Бауэра к гуманистической антропологии Л. Фейербаха,
когда в школе шла "смена лидера".
Некоторые тексты молодого Маркса написаны в сотрудничестве с Б. Бауэром, А. Руге, Ф. Энгельсом, М. Гессом.
В то же время Маркс-гегельянец проявил высокую степень самостоятельности, что в итоге только обогатило
достижения школы. Вступив в движение позже других, Маркс смог более критически подойти к гегелевской
традиции.
Он по-своему оригинально реализовал некоторые потенции антропологии Л. Фейербаха. Так, Маркс принял
фейербаховское  толкование  принципа  тождества  бытия  и  мышления,  но  конкретизировал  абстрактный
философский принцип первичности бытия, обратив внимание на социальное бытие, на историю человечества.
Своеобразным развитием фейербаховской концепции религии как иллюзорного самосознания стало у Маркса
истолкование идеологии как ложного, превращенного сознания, выражающего реальность в "перевернутом"
виде.
Маркс,  сочетая  гегельянский  критицизм  с  его  гегелевским  прообразом,  сформировал  оригинальный  метод
интерпретации социальной истории. Называя этот метод диалектическим, Маркс отдавал пальму первенства
Гегелю, оставляяза собой приоритет лишь в приспособлении метода для своих, по преимуществу социально-философских  и  социально-экономических  исследовательских  задач.  Наиболее  оригинальны,  влиятельны  и
интересны идеи Маркса, развитые в сфере философии истории и философской антропологии. 
Черты  характера  Маркса:  эгоизм,  самовлюбленность,  крайняя  властность,  нетерпимость  к  другим  точкам
зрения. Раздражение от религии; земная власть аристократов поддерживается религией => религия – главный
враг.
Отчужденный труд М рассм в 4 аспектах.
1.Рабоч использует матер, взятые у природы и получает в итоге нужные для жизни предметы, продукты труда.
Ни исх материал, ни прод ему не принадлежат - они ему чужие. Чем больше р. работает, тем больше мир предм,
не принадл ему. Природа делается для раб только средством труда, а предметы, кот создаются в производстве -средством жизни, физ существования. Раб полностью от них зависит.
2. Процесс труда для р принудителен. Но такой труд - это не удовлетвор потребности в труде, а только средство
для удовлетвор др потребностей. Только вне труда р. распоряжается собой - т.е. свободен. Т.о он свободен
только осущ. жизненные функции, общие у чел с животными. А труд - форма деят, специф для чел, для раб
представляется унижением в себе человека.
3. Труд подневольный отнимает у чел его "родовую" жизнь. Род челов. живет в природе. Жизнь чел неразрывно
связана с прир. Эта связь - деятельный контакт с прир, в кот главное - труд, производство: "...производственная
жизнь и есть родовая жизнь". Но для раб труд - лишь средство для поддержания собственной жизни, а не рода.
Р относится к прир и производству не как своб человек, а как рабочий, т.е отчужденно. Это и значит,что у раб
отобраны и родовая жизнь и чел сущность.
4.Подневольный труд порождает отчежд между людьми. Раб чужды друг другу, поскольку они конкурируют за
возм трудиться.
Не только р. но и все люди являются отчужденными. Отнош между людьми тоже отчужденные и различия
только в видах и уровнях отч. М указывает на сущ первичных и вторичных уровней отч. Почему же чел
становится отчужденным?
Отч труд равнозначен сущ частной собственности. Ч собств - основа экон жизни. На частнособственической
экономике держится вся история. Это значит, что эконом история - ключ к пониманию челов жизни как таковой.
"Религия, семья, гос-во, право, мораль, наука, искусство... суть лишь особые виды производства и подчиняются
его всеобщему закону". Жизнь людей в усл отчуждения калечит их, делает "частичными индивидами" или
неразвитыми, недочелов существами. "Чатсная собств сделала нас настолько глупыми и односторонними, что
какой-нибудь предет явл нашим лишь когда мы обладаем им... когда мы им непосредственно владеем, едим его,
пьем - употребляем... Поэтому на место всех физ и духовных чувств стало простое отчуждение всех этих чувств
- чувство обладания".
Устранение  отчуждения.  Универсальный  человек.  Процесс,  обратный  отчеждению,  -  присвоение  чел
собственной подлинной сущности. М связывает его с общ преобразованиями, с освобождением кот в основе
имеет уничтожение отчужд труда. Что будет , если чел начнет производить как человек, т.е. не подневольно. В
этом случае труд станет средством саморазвития человека, в реализацию человеком своих лучших сторон.
Характеристика  присвоения  челов  собственной  сущности,  или  превращ  труда  из  принуд  в  человеческий
рассматр М по тем же параметрам, что и процесс отчуждения: 1. по присвоению предм труда и его результата 2.
по  освобождению  самой  деятельности  3.присв  человеком  труда  общей  родовой  сущност  4.гармонизации
отношений между людбми.
Здесь  М  создает  грандиозную  по  своему  пафосу  картину  челов,  живущего  в  единстве  с  природой,
преобразующего прир в соответствии с ее законами. Гармония с внешней прир осущ в деятельности, в кот
человек реализует свои цели не по законам утилитарной пользы, а по зак красоты. Внутренняя прир чел такжк
преобразуется.  Вместо  отчежденного  недочеловека  появляется  чеовек,  само  природное  развитие  есть
гармоничный  р-т  всей  истории  чел  общества.  В  человеке  начнут  реализовываться  способности,  пока  еще
реализующиеся не у всех (музык ухо, художественно развитый глаз). - творческие способности.
Универсально развитый, жив в единстве и гармонии с внешней и внутренней природой чел - таков идеальный
фил образ, рисующийся М в качестве ядра коммунист идеала. Уничтожение ч собственности необходимо, но
недостаточно для присвоения людьми челов сущности.


\newpage
\section{Религиозная метафизика и философия всеединства Соловьева}
В философии Соловьева, как и в учении Гегеля, онтология и гносеология, бытие и познание неразделимы и
опираются на единую основу. 
Идея всеединства является центральной в философии В. Соловьева, поэтому всю его систему часто называют
философией всеединства.
В философии всеединства речь шла о единении Бога и человека; идеальных и материальных начал; единого и
множественного; рационального, эмпирического и религиозно-мистического знания; нравственности, науки,
религии, эстетики.
Создавая новую синтетическую философию, Соловьев обратился к анализу предшествующей философской.
Философия, по мнению Соловьева, возникает в период напряженного кризиса, когда религиозная социальная
роль не разрывает человеческое общество, сознание. Идея всеединства есть та цементирующая основа, которая
предает целостность всей философии, несмотря на ее бесконечную вариативность и разнообразие.
Его  философия  начинается  с  понятия  не  бытия,  а  сущего.  Абсолютном  сущем,  по  мысли  Соловьева,
содержаться два центра – абсолютное начало, как таковое, и первоматерия. Для первоматерии, выражающей
начало многообразия, вводится понятие София (мудрость). В Философии Соловьева человек “совечен” Богу, он
говорит о человеке как идее бытия, которая заложена в самой основе мира в целом. Софийный идеальный
человек принадлежит вечности, а она ему, поэтому он едино с Богом. 
Принципы  онтологии,  которые  лежат  в  основе  философской  концепции  Владимира  Соловьева  неразрывно
вязаны  с  его  гносеологическим  учением.  В  своей  основе  единство  онтологии  и  гносеологии  у  Соловьева
базируется  на  платоновской  идее  единства  истины,  добра  и  красоты.  На  основе  этой  идеи  Соловье
разрабатывает концепцию целостного знания, которое предполагает постепенный синтез религии, философии,
науки.
Познание  у  Соловьева  связано  с  этикой,  с  эстетическими  чувствами,  но,  главным  образом,  с  реальным
“собирательным творчеством”. В реальном творчестве преобразуются общество, земная природа, универсум.
Средством для решения этих колоссальных проблем Соловьев предлагает единение свободно-нравственного
человечества,  развивающегося  благодаря  нравственному  совершенствованию  каждой  личности  и  всего
общества.


\newpage
\section{Тема свободы в философии Бердяева}
Философия Бердяева впитала в себя множество разнообразных источников. Ранний Бердяев пытался сочетать
гуманизм Маркса с антропологическим социализмом Михайловского и метафизикой неокантианства. Зрелые
философские воззрения Бердяева представляют собой одну из первых в Европе разновидностей христианского
экзистенциализма.  Согласно  экзистенциализму,  задача  философии  -  заниматься  не  проблемами  науки,  а
вопросами сугубо человеческого бытия (существования). Человек помимо своей воли заброшен в этот мир, в
свою судьбу, и он живет в чуждом ему мире: его бытие со всех сторон окружено таинственными знаками,
символами. Страх важнейшее понятие философии экзистенциализма. 
Большое место в экзистенциализме занимает проблема свободы, определяемая как выбор человеком самого
себя: человек таков, каким он себя свободно выбирает. “Чувство вины за все совершающееся вокруг него -чувство свободного человека” (Бердяев).
Экзистенциализм различают религиозный и атеистический. Именно к религиозному - и относится Бердяев.
Философия Бердяева антропоцентрична - проблема духовности, свободы и творчества, судьбы, смысла жизни и
смерти всегда были в центре его философских размышлений. По Бердяеву “личность вообще первичнее бытия”,
бытие  -  воплощение  причинности,  необходимости,  пассивности,  духовное  начало  свободное,  активное,
творческое. Понятие объективного мира Бердяев заменяет термином “объективированный мир”, интерпретируя
его  как  “объективацию  реальности”,  порожденную  субъективным  духом.  Частично  признавая  социальную
обособленность  бытия  личности,  он  вместе  с  тем  считает  главным  в  человеке  то,  что  определяется  его
внутренним миром, а не внешним окружением. Личность, по Бердяеву прежде всего категория религиозного
сознания,  и  поэтому  проявление  человеческой  сущности,  ее  уникальности  и  неповторимости  может  быть
понято лишь в ее отношении к богу.
Бердяев  рассматривая  три  типа  времени  (космическое,  историческое  и  экзистенциальное  или  мета
историческое),  он  главным  образом  озабочен  предсказанием  того,  как  “мета  история  входит  в  историю”,
обоснованием приближения конца истории. Эти мотивы особенно сильно проявились в его последних работах.
Бердяев считал, что философия хочет не только познания мира но и улучшения его. Мораль и нравственность
Бердяев строит на христианских заповедях.
«Свобода первична, по отношению к бытию». 
/*Творческое развитие христианства Бердяева стало в противоречие с догмами христианства*/
Проблема Теодицея [тео=бог, дицей=оправдывать] – проблема оправдания бога: как это абсолютно доброе,
всеблагое начало создало такой мир.
Решение  Бердяева:  Если  бог  абсолют,  то  непонятно  как  решать  проблему.  «Свобода  есть  изначальная
координата бытия, или свобода предшествует бытию и предшествует ему в абсолютном смысле». Бог, прежде
всего, творец, но тогда, по определению, он изначально нуждается в свободе (нет свободы – нет творчества).
/*Версия свободы навеяна идеями Шеллинга.*/ Бог, таким образом, не является абсолютом и всемогущим. Бог
тоже некогда рождается или возникает. Бог и свобода, которая всегда рядом с ним, появляются из неизвестной
нам  бездны  (под  бездной  понимается  не  пространственно-временное  нечто,  а  поле,  сфера,  куда  не  может
проникнуть наше познание).
Свобода существует наряду с богом!
Возникновение мира – процесс, связанный с рождением и становлением самого (мира) бога. Бог развивается
через  становление  мира,  а  становление  мира  идет  медленно  и  проблематично.  В  несовершенстве  мира
отражается несовершенство бога. Это один процесс. Насколько мир и человек заинтересованы в боге, настолько
бог заинтересован в человеке, мире. Два начала заинтересованы друг в друге.
Встреча человека и Иисуса Христа – точка первой встречи бога и человека. Кардинальный момент в развитии
мира. Человек понял выразителем какой силы он является, кем он ведомый, на кого он должен положиться. И
тогда, бог не несет ответственности за зло в мире, ибо он действует у условиях свободы. 
Свобода – это не хорошо и не плохо, потому что дорогой свободы можно прийти куда угодно. Свобода – это не
гарант удач, успехов, она может привести куда угодно. Свобода – это просто наличие выбора, который мы
должны делать всегда сами. Выбор всегда таков, каковы мы. Пути, которые мы выбираем, зависят от нашей
духовности. 
Быть свободным гораздо труднее и опаснее, чем рабом. Западный мир идет по пути «несвободы», так как там,
главным образом, заботятся о комфорте и благополучии. Человек должен развивать свое духовное начало,
только это ведет по пути свободы.


\newpage
\section{Феноменология Гуссерля}
Гуссерель  рассматривает  психический  акт  как  самостоятельну  особую  реальность.  Особенности  учения
Гуссереля:
1.Гуссерель пытается преодолеть психологизм.
2.Показывает, что общие понятия, которые Брентано выдаются как языковые фикции, существуют и обладают
логическим реальным бытием.
3.Пытается рассмотреть философию исходя из нового метода: созерцание сущностей.
Феномен – явление.
Гуссерель пытается вявить чистую логику, но как он констатирует: познавательная значимость логических
суждений должна быть в свою очередь фундирована (предварительно) в другого рода значимость, не связанную
с познавательной логикой (допредекативную значимость предмета). Отход от общей значимости суждения к
допредикативной значимости восприятия – это первый и решающий методологический шаг феноменологии
духа. Допредикативная значимость – предзаданное восприятие предметов, явлений.
Предрассудки, присущие психологизму:
1.Значимость мышления должна быть обоснована психологически. С т.з. Гуссерля логические законы имеют
идеальное общезначимое содержание.
2.Психологизм: суждение – это психологический феномен. Гуссерель считает, что в логике мы имеем дело с
идеальным содержанием, освобожденным от его эмпирических составляющих.
3.В логике нужно опереться на очевидность, но с позиции феноменологии (ф) очевидность — это не просто
одно чувство наряду с другими. В очевидности (по Г) феноменологической, истина представляет собой скрыто
присутствующсю  изначальность.  Г.  преодолевает  психологическо-эмпирическую  позицию  Брентано  в
«Логических  исследованиях»,  где  он  исследует  идеальное  бытие  понятий,  называемых  эйдосами.  Эйдосы
станут предметом описания феноменов.
Гуссерель ставит задачу разделить акты восприятия и предмета изнутри сознания средствами самого сознания.
Методологический фундамент феноменологии – это метод очевидности или метод редукии. – метод сведения к
очевидности.
Редукция  меняет  установку  сознания:  от  естественной  направленности  на  вещи  и  явления  как  на
существующее, от «захваченности ими», точнее догматической уверенности в их реальном существовании к
фактам и предметам самого сознания., которые называются феноменами сознания (то есть к фактам самого
сознания, а не природного).
2 этапа редукции:
1.Эйдетическая редукция – освобождение от чувственных явлений и переход к чистым феноменам. Природное
существование вещи отбрасывается и образуется феноменологическая предметность в которой можно прежить.
2.Феноменологическая или трансцедентальная редукция – это освобождение от уверенности в укорененности
«я» в мире вещей. Следовательно я (субъект) должно превратиться в чистый поток сознания.
Чистый поток сознания не должен содержать внетренний опыт, наших ощущений.
Это приводит к понятию «интенциональность» (И.)
Интенциональность – это направленность сознания на другое, чем оно само является, то есть на предмет,
понятый как данность, которая хотя и является данностью сознания. И. Не является выходом сознания за свои
собственные пределы, т.к. нет ничего изначально присущего сознанию. В И. Совпадает чистая предметнсть и
чистая субъективность, причем чистую субъективность Гуссерель определяет как ноэзис.
Ноэзис – интенциональный акт, акт восприятия, направленность на предмет.
Чистая предметность – ноэма – то, на что направлен акт сознания.
Интенциональный акт – это единство субъективности и эйдоса (читсая идея).
Чистая логика – это чистая априорность, которая позволит потом развиться интенциональности.


\newpage
\section{Принцип детерминизма и проблема причинности}
Принцип  детерминизма - все реальные явления, процессы детерминированы, т.е. возникают,  развиваются и
уничтожаются в результате действия определённых причин.Причина - явление,  повлёкшее за собой другое
явление.  Следствие  -  явление,  возникшее  в  результате   действия  причины.  Типы  причинных  связей:I.
однонаправленные  причинно-следственные   связи.II.  взаимодействиеВзаимодействие  -  это,  когда  причина
испытывает обратное влияние со стороны следствия.Причинные основания - совокупность всех обстоятельств,
при которых наступает следствие:- причина- условие- мотивы и т.д.Детерминизм - концепция мира, которая
основывается  на  принципах  причинности  и  закономерности.Формы  детерминизма:-   механистический
(лапласовский)  -  причинная  связь  понимается  как  однозначная,  т.е.   определённое  состояние  системы
определяет  последующее  состояние  системы.Причинность   -  необходимость.-  статистический  -  признание
многозначного  соотношения  между  причиной   и  следствием.В  причине  заключается  ряд  возможностей  и
вариантов развития.Индетерминизм - отрицание закономерностей и причинной обусловленности явлений.
Детерминизм (от лат. determine — определяю) — учение о первоначальной определяемости всех происходящих
в мире процессов, включая все процессы человеческой жизни, со стороны Бога (теологический детерминизм,
или учение о предопределении), или только явлений природы (космологический детерминизм), или специально
человеческой  воли  (антропологическо-этический  детерминизм),  для  свободы  которой,  как  и  для
ответственности, не оставалось бы тогда места. Под определяемостью, здесь подразумевается философское
утверждение, что каждое произошедшее событие, включая, и человеческие поступки и поведение однозначно 
определяется  множеством  причин,  непосредственно  предшествующим  данному  событию.  В  таком  свете
детерминизм может быть также определен, как тезис утверждающий, что имеется только одно, точно заданное,
возможное будущее.
Все  определено  в  этом  мире  и  ничто  не  в  состоянии  этого  изменить.  Однако,  всякое  действие  вызывает
следствие  подобно  всему  тому,  что  происходит  в  этой  жизни.  На  принципе  детерминизма  построена  вся
классическая физика, за исключением термодинамики и молекулярной физики. Детерминизм подразумевает
выполнение обратимости времени, т. е. частица прийдет в исходное состояние, если обратить время. Каждая
траектория единственным образом определяется начальными условиями. Всё это находится в замечательном
согласии с экспериментальными данными макромира.
Между детерминизмом и индетерминизмом имеются также переходы, например в учениях Лютера, Цвингли и
Канта: так, если учение детерминизма распространяется на эмпирическую (естественную) природу человека, то
его моральная сторона становится объектом разновидности индетерминизма.


\newpage
\section{Формы чувственного и рационального знания}
Итак, ощущение есть приблизительно верное отражение действительности. Ощущение предполагает наличие
ощущаемого, образ вещи не может быть без самой вещи, отражение без отражаемого. Ощущение есть первая
форма чувственной ступени познания. Но чувственное познание развивается диалектически от простого к более
сложному , не заканчивается на стадии ощущений.
На  основе  ощущений  возникают  восприятия,  являющиеся  более  высокой  и  более  сложной  формой
чувственного познания. Если ощущение отражает отдельные свойства вещей, то восприятие отражает предмет
как совокупность его объективных свойств - форму, цвет, запах, вкус и т. п. В восприятии различные ощущения
находятся  в  единстве,  в  обобщенном  виде.  Восприятие  является  результатом  совокупности  отражательной
деятельности нескольких органов чувств, давая нам поэтому более полное знание о вещах, чем ощущение.
От восприятия чувственных познаний идем к представлениям. Здесь уже нет непосредственной связи с вещами,
так  как  представления  возникают  на  основе  прошлых  ощущений  и  восприятий.  Представление-это
воспроизведенные памятью образы предметов или процессов, основанные на прошлом опыте. Так, например,
мы  не  можем  в  данный  момент  непосредственно  ни  ощущать,  ни  воспринимать  новый  электровоз,  но
представить себе его можем, поскольку видели его раньше. Представления создаются также на основе изучения
снимков, чертежей и т. д. Представление, оставаясь формой чувственного познания, является в то же время
первым  шагом  по  пути  к  абстрактному  мышлению.  Таким  образом,  в  ощущениях,  восприятиях  и
представлениях человек приблизительно верно отражает предметы и явления окружающего мира и тем самым
получает достоверные знания.
Познание есть поступательный процесс. Оно развивается по восходящей линия от простого к сложному, от
низшего к высшему. В процессе познания человек идет от явления к сущности, от познания поверхностных
явлений к пониманию глубинных процессов и внутренне присущих им законов. Однако познание сущности и
законов возможно только на ступени, абстрактного, логического мышления.
Процесс познания есть бесконечный процесс восхождения от живого созерцания к абстрактному мышлению.
Чувственное  и  рациональное  познание  находятся  в  неразрывном  единстве.  Оба  эти  моментах  познании
являются  отражением  внешнего  мира  в  сознании  человека.  Содержанием  чувственного  и  рационального
познания является материальный мир. Живое созерцание и абстрактное мышление есть лишь различные формы
познания одной и той же объективной реальности.
Без  живого  созерцания  немыслимо  абстрактное  мышление.  В  свою  очередь  без  абстрактного  мышления
познание  материального  мира-невозможно.  Органы  чувств  дают  нам  достоверные  знания  о  материальных
вещах  только  в  неразрывной  связи  с  мышлением  и  практикой.  Наоборот,  абстрактное  мышление  без
чувственного познания и практики не может дать истинных знаний.
Признавая неразрывное единство чувственного и рационального познания, диалектический материализм вместе
с  тем  считает,  что  оба  эти  момента  познания  качественно  различны.  Это  различие  состоит  в  том,  что
чувственное  познание  дает  нам  поверхностные  сведения  о  вещах,  а  рациональное  познание  раскрывает
сущность и закономерности их развития. Переход от живого созерцания .к абстрактному мышлению является
качественным скачком. Этот скачок знаменует собой переход от познания явления к познанию сущности.
Процесс познания и получаемые в нем знания в ходе исторического развития практики и самого познания все
более  дифференцируется  и  воплощается  в  различных  своих  формах.  Последние  хотя  и  связаны,  но  не
тождественны одна другой, каждая из них имеет свою специфику. Выделим основные формы познания в их
историческом поступательном процессе.
На  ранних  этапах  истории  существовало  обыденно-практическое  познание,  поставлявшее  элементарные
сведения о природе, а также о самих людях, условиях их жизни, общении и т. д. Основой данной формы
познания  был  опыт  повседневной  жизни,  практики  людей.  Полученные  на  эти  базе  знания  носят  хотя  и
прочный, но хаотический, разрозненный характер, представляя собой простой набор сведений, правил и т. п.
Сфера  обыденного  познания  многообразна.  Она  включает  в  себя  здравый  смысл,  верования,  приметы,
первичные  обобщения  наличного  опыта,  закрепляемые  в  традициях,  преданиях,  назиданиях  и  т.  п.,
интуитивные убеждения, предчувствия и пр.
Одна из исторически первых форм - игровое познание как важный элемент деятельности. В ходе игры индивид
осуществляет активную познавательную деятельность, приобретает большой объем новых знаний, впитывает в
себя богатство культуры -деловые игры, спортивные игры, игра актеров и т. п.
В настоящее время понятие игры широко используется в математике, экономике, кибернетике и других науках.
Здесь все чаще применяются специальные игровые модели и игровые сценарии, где проигрываются различные
варианты течения сложных процессов и решения научных и практических проблем.
Важную  роль,  особенно  на  начальном  этапе  истории  человечества,  играло  мифологическое  познание.  Его
специфика в том, что оно представляет собой фантастическое отражен»» реальности, является бессознательно-художественной переработкой природы и общества народной фантазией. В рамках мифологии вырабатывались
определенные  знания  о природе,  космосе,  о самих  людях, их условиях бытия,  формах  общения и т. д.  В
последнее время было выяснено (особенно в философии структурализма), что мифологическое мышление - это
не  просто  безудержная  игра  фантазии,  а  своеобразное  моделирование  мира,  позволяющее  фиксировать  и
передавать опыт поколений. Так, Леви-Строс указывал на конкретность и метафоричность мифологического
мышления, его способность к обобщению, классификациям и логическому анализу.
Некоторые  современные  исследователи  полагают,  что  в  наше  время  значение  мифологического  познания
отнюдь нс уменьшается. Так, П. Фейерабенд убежден, что достижения мифа несравненно более значительны
чем научные: изобретатели мифа, по его мнению, положили начало культуре, в то время как рационалисты
только изменяли ее, причем не всегда в лучшую сторону.
Уже  в  рамках  мифологии  зарождается  художественно-образная  форма  познания,  которая  в  дальнейшем
получила наиболее развитое выражение в искусстве. Хотя оно специально и не решает познавательные задачи,
но содержит в себе достаточно мощный гносеологический потенциал. Более того, например, в герменевтике
искусство считается важнейшим способом раскрытия истины. Хотя, конечно, художественная деятельность
несводима целиком к познанию, но познавательная функция искусства посредством системы художественных
образов -одна из важнейших для него. Художественно осваивая действительность в различных своих видах
(живопись, музыка, театр и т. д.), удовлетворяя эстетические потребности людей, искусство одновременно
познает мир, а человек творит его - в том числе и по законам красоты. В структуру любого произведения
искусства всегда включаются в той или другой форме определенные знания о разных людях и их характерах, о
тех или иных странах и народах, их обычаях, нравах, быте, об их чувствах, мыслях и т. д.
Одними  из  древних  форм  познания,  генетически  связанными  с  мифологией,  являются  философское  и
религиозное познание. Особенности последнего определяются тем, что оно обусловлено непосредственной
эмоциональной  формой  отношения  людей  к  господствующим  над  ними  земными  силами  (природными  и
социальными). Будучи фантастическим отражением последних, религиозные представления содержат в себе
определенные  знания  о  действительности,  хотя  нередко  и  превратные.  Достаточно  мудрой  и  глубокой
сокровищницей  религиозных  и  других  знаний,  накопленных  людьми  веками  и  тысячелетиями,  являются,
например, Библия и Коран. Однако религия (как и мифология) не воспроизводили знание в систематической и,
тем  более,  теоретической  форме.  Она  никогда  не  выполняла  и  не  выполняет  функции  производства
объективного знания, носящего всеобщий, целостный, самоценностный и доказательный характер.
Говоря о формах знания, нельзя обойти вниманием достаточно известную (особенно в современной западной
гносеологии)  концепцию  личностного  знания.  Знание  по  этой  концепции  -  это  активное  постижение
познаваемых вещей, действие, требующее особого искусства и особых инструментов. Поскольку науку делают
люди,  то  получаемые  в  процессе  научной  деятельности  знания  (как  и  сам  этот  процесс)  не  могут  быть
деперсонифицированными. Личностное знание необходимо предполагает интеллектуальную самоотдачу. В нем
запечатлена не только познаваемая действительность, но сама познающая личность, ее заинтересованное (а не
безразличное) отношение к знанию, личный подход к его трактовке и использованию, собственное осмысление
его  в  контексте  специфических,  сугубо  индивидуальных,  изменчивых  и,  как  правило,  неконтролируемых
ассоциаций.
Личностное знание - это не просто совокупность каких-то утверждений, но и переживание индивида. Личность
живет в нем «как в одеянии из собственной кожи», а не просто констатирует его существование. Тем самым в
каждом акте познания присутствует .страстный вклад познающей личности, и что эта добавка не свидетельство
несовершенства,  но  насущно  необходимый  элемент  знания.  Но  такая  добавка  не  делает  последнее  чисто
субъективным.
В  настоящее  время усиливается  интерес к  проблеме  иррационального, т. е.  того, что лежит  за  пределами
досягаемости разума и недоступно постижению с помощью известных рациональных средств, и вместе с тем
все более укрепляется убеждение в том, что наличие иррациональных пластов в человеческом духе порождает
ту  глубину,  из  которой  появляются  все  новые  смыслы,  идеи,  творения.  Взаимопереход  рационального  и
иррационального - одно из фундаментальных оснований процесса познания. Однако значение внерациональных
факторов не следует преувеличивать, как это делают сторонники иррационализма.
Типологизация форм самого знания может быть проведена по самым различным основаниям (критериям). В
этой связи выделяют,  например,  знания рациональные и  эмоциональные, феноменолистские  (качественные
концепции)  и  эссенцианалистские  (сооруженные  в  основном  количественными  средствами  анализа),
эмпирические и теоретические, фундаментальные и прикладные, философские и чистнонаучные, естественно-научные и гуманитарные и т. д.
Специфическими формами или методами познания являются наблюдение, описание и т. д. На этой ступени
познавательного процесса происходит сбор фактов, фиксирующих внешние проявления, свойства предметов.
Теоретическая форма познания- это углубление человеческой мысли в сущность явлений действительности.
При этом научное познание пользуется такими методами, как моделирование, создание гипотез, теорий и т. д.
Действительность отражается человеком с помощью различных форм познавательной деятельности.
Таким образом человек ничего не может знать о предметах и явлениях внешнего мира без того материала,
который  он  получает  от  органов  чувств  в  формах  чувственного  созерцания  (ощущение,  восприятие,
представление).  Они  дают  чувственно  наглядное  воспроизведение  предметов,  их  отдельных  свойств.
Существенные  связи  между  предметами,  закономерности  их  развития  отражаются  в  формах  абстрактного
мышления - понятиях, суждениях, умозаключениях. В процессе познания человек использует также различные
логические приемы (анализ и синтез, дедукцию, индукцию и т. д.), позволяющие ему теоретически воссоздавать
изучаемый предмет.

\newpage
\section{Основные факторы и этапы антропосоциогенеза}
Антропосоциогенез - процесс происхожнения человека, становление его как биологического вида в процессе
формирования  общества,  которое,  в  свою  очередь  обозначается  социальным  развитием  или  социогенезом.
Методологической основой антропогенеза является естественно-материалистическая теория развития, генетика
и  дарвинизм,  диалектически  объясняющие  взаимоотношения  биологического  и  социального  факторов
эволюции человека как замещение более высокой формой материи - социальной более низких - биологических,
которые не отменяются, а лишь подчиняются и преобразуются первой.
Важную роль в процессе антропогенеза играла осознанная целенаправленная трудовая деятельность, повлекшая
совершенствование головного мозга, развитие конечностей, формирование сознания. Роль труда как основного
фактора  антропогенеза  была  неодинаковой  на  разных  этапах  его  развития,  поскольку  в  ранней  стадии 
первобытного  общества  (стаде)  прогресс  в  социальной  организации  в  значительной  мере  зависел  от
биологических изменений человека; в целом процесс антропогенеза сопровождался постепенным сужением
сферы действия естественного отбора в сторону возникновения общественных закономерностей и создания
социальной и культурной Среды . 
Согласно  археологическим  открытиям  последних  десятилетий  20  в.  на  африканском  континенте,  удалось
доказать, что прямохождение у древних предков современного человека функционально развилось около 6-5
млн. лет тому назад . Этот исторический период и стал периодом возникновения древнейших представителей
семейства гоминид (человекообразных обезьян) африканских двуногих приматов австралопитеков афарского
Australopithecus afarensis и южно-африканского австралопитека Australopithecus africanus, найденных в Южной
Африке, Кении (район озера Рудольф) и Танзании (ущелье Олдовай). Строение их скелетов свидетельствует о
прямохождении, зубная система близка к человеческой, а общая масса мозга составляет 380-500 г при общей
массе  тела  от  25  до  65  кг.  В  местах  обнаружения  остатков  австралопитеков  имеется  множество  костей,
расколотых  тяжелыми  предметами.  Большое  число  черепов  животных  расколоты  с  левой  стороны,  что
свидетельствует о австралопитеки были в основном правшами. Некоторые австралопитеки, видимо, начали
осваивать огонь .
Австралопитеки имеют сходство с человеком не по объему и строению головного мозга, а в основном по
способу передвижения. Изучение австралопитековых показало, что именно двуногость, а не большой объем
мозга, явилась ключевой адаптацией ранних гоминид. В 1964 г. по находкам, сделанным в Танзании, был
выделен  новый  вид  Homo  habilis  -человек  умелый,  имеющий  абсолютный  возраст  2-1.7  млн.  лет,
отличительными особенностями которого от австралопитеков являются двуногость, в целом прогрессивное
строение  кисти,  зубной  системы,  объем  мозговой  коробки  от  540-700  см3,  что  примерно  в  полтора  раза
превышает  объем  мозга  австралопитеков.  На  внутренней  поверхности  черепа  обнаруживаются  признаки
прогрессивных  нейроморфологических  изменений,  определяющиеся  по  отпечаткам  головного  мозга:
выраженная  ассиметрия  полушарий  и  развитие  двух  речевых  центров  как  условие  для  возникновения
членораздельной  речи.  Большой  палец  стопы  не  отведен  в  сторону,  что  свидетельствует  о  том,  что
морфологические  перестройки,  связанные  с  прямохождением,  у  него  полностью  завершились.  Вместе  с
остатками Homo habilis найдены орудия труда со следами целенаправленной обработки, свидетельствующие о
наличии ранних форм трудовой деятельности .
Перечисленные признаки, ведущим из которых является прогрессивное развитие мозга характеризует уже иной
уровень  морфофункциональной  организации,  характерной  для  рода  человека   Homo.  Сопоставление
морфологии африканского и афарского австралопитеком с современным человеком Homo sapiens позволяет
предположить,  что  общим  предком  человека  разумного  является  австралопитек  афарский  Australopithecus
afarensis. Африканский австралопитек Australopithecus africanus является в этой схеме представителем боковой
ветви  эволюции,  приведший  к  узкой  специализации  и  образованию  форм  типа  Australopithecus  robustus,
вымершего около 1 млн. лет назад. Таким образом, на протяжении 1-1.5 млн лет представители двух близких
родов и, возможно, нескольких видов семейства гоминид сосуществовали, причем не только во времени, но и на
перекрывающихся природных территориях. В основе дивергенции (разделения признаков) различных линий
ранних  гоминид  и  австралопитеков  могли  лежать  разного  рода  механизмы  изоляции,  в  первую  очередь
генетические  мутации  в  виде  хромосомных  перестроек.  Это  означает,  что  эволюция  австралопитековых,
ведущим фактором которой являлся естественный отбор, шла постепенно, приводя благодаря дивергенции к
морфологическому и экологическому разнообразию.
Ведущими  факторами  эволюции  на  прегоминидной  стадии  антропогенеза  являлись,  несомненно,  факторы
биологической  эволюции,  главным  из  которых  является  естественный  отбор.  Об  этом  свидетельствует,  в
частности, большое видовое разнообразие австралопитековых, обитавших в различных условиях практически
на всей территории Южной, Центральной и Северо-Восточной Африки. В то же время в происхождении рода
Homo имело место скачкообразное изменение наследственного материала.
В  разных  органах  и  системах  прогоминид  обнаруживалась  ассинхронность  исторического  развития  -филогенеза.  Есть  предположение  о  том,  что  эволюция  коры  больших  полушарий  мозга  состоит  из  двух
компонентов,  разобщенных  по  времени:  соматического,  обеспечивающего  сенсорно-моторные  функции,  и
несоматического, связанного с высшими психическими функциями. Если локомоторный комплекс подвергался
длительным  постоянным  изменениям,  то  головной  мозг  эволюционировал  скачкообразно.  Элементы
скачкообразности в
эволюции  некоторых  структур  ранних  гоминид  могли  быть  обусловлены  генными  мутациями  (реверсии,
транзиции, транслокации и т. п.), что могло повлечь за собой развитие других морфофизиологических свойств в
результате накопления мутаций под контролем естественного отбора. Но именно в период становления Homo
habilis возникла, вероятно, часть хромосомных перестроек в геноме человека, о которых говорилось выше. 
Следущим этапом эволюции после появления Homo habilis считается возникновение около 1.5 млн. лет назад
архантропов,  представителем  которого  является  вид  Homo  erectus.  Трудовая  деятельность,  материальная
культура и ярко выраженная социальность позволили ему быстро расселиться на территории Африки и Азии и
освоить  обширный  ареал,  разнообразный  в  климатическом  отношении.  Орудия  труда  Homo  erectus  более
прогрессивны, чем у Homo habilis, а масса мозга (800-1000 г) превышает минимальную массу (750 г), при
которой возможно существование речи. Наличие при этом речевых центров, возникших впервые у Homo habilis,
предполагает и развитие второй сигнальной системы.
Выделяют  три  группы  Homo  erectus,  обитающие  в  Европе,  Азии  и  Африке.  Долгое  время  древнейшими
архатропами считались азиатские представители их Индонезии и Восточного Китая - питекантроп и синантроп.
Однако находки последних лет на территории Израиля (1982) и Кении (1984), датирующиеся соответственно 2.0
и  1.6  млн. лет,  сопровождающиеся  элементами  материальной культуры и  признаками использования огня,
показали,  что  эволюция  гоминид  происходила  на  Африканском  контитенте  и  на  Ближнем  Востоке.  Это
доказательство позволило связать Homo erectus с восточноафриканскими формами Homo habilis.
Наличие большого количества находок архантропов древностью 1.5-0.1 млн. лет в отдаленных от Африки
регионах - в Юго-Восточной и Восточной Азии, в Центральной Европе и даже на Британских островах -свидетельствует об активных адаптациях их к разнообразным условиям существования. В связи с тем что
небольшое  различие  ископаемых  остатков  Homo  erectus  не  соответствует  разнообразию  природно-климтических условий указанных территорий, можно заключить, что в этих адаптациях значительную роль
играли наряду с факторами биологической эволюции также и социальные факторы: совместное изготовление
орудий труда и использование огня.
Роль  Homo  erectus  в  качестве  этапа  антропогенеза  никогда  не  подвергалась  сомнению.  Что  же  касается
палеонтропа, или неандертальского человека, то его роль в процессе эволюции в настоящее время оспаривается.
Это  связано  в  первую  очередь  с  обнаружением  большого  количества  ископаемых  остатков  человека  с
промежуточными  чертами  между  Homo  erectus  и  человеком  современного  типа.  Кроме  того,
палеонтологические находки последних лет позволяют судить о недооценке интеллектуальных способностей
неандертальцев. На всех стоянках неандертальца обнаружены следы кострищ и обгоревшие кости животных,
что свидетельствует об использовании огня при приготовлении пищи. Орудия труда неандертальца гораздо
совершеннее, чем у предковых форм. Масса головного мозга составляет приблизительно 1500 г, причем сильное
развитие получили отделы, ответственные за логическое мышление. Костные остатки неандертальца из Сен-Сезер (Франция) были найдены вместе с орудиями труда, свойственными верхнепалеолитическому человеку,
что  свидетельствует  об  отсутствии  резкой  интеллектуальной  грани  между  неандертальцем  и  современным
человеком.  Также  имеются  данные  о  ритуальных  захоронениях  неандертальцев  на  территории  Ближнего
Востока.
Эти и ряд других находок позволили в конце 60-х годов выделить палеонтропов в отдельный подвид Homo
sapiens neanderthalensis в отличие от палеонтропа Homo sapiens sapiens, который, таким образом, также получил
видовую классификацию. Наиболее древние ископаемые остатки его возрастом 100 тыс. лет обнаружены также
на  территории  Северо-Восточной  Африки.  Многочисленные  находки  палеонтропов  и  неоантропов  на
территории Европы, датирующиеся 37-25 тыс. лет, свидетельствуют о существовании обоих подвидов человека
в течение нескольких тысячелетий.
В тот же период неоантропы обитали уже не только в Европе и Африке, но и в отдаленных районах Азии (о.
Тайвань, о. Окинавы) и даже в Америке. Эти данные указывают на необычайно быстрый процесс разделения
современного  человека,  что  может  быть  доказательством  скачкообразного  характера  антропогенеза  в  этот
период как в биологическом, так и в социальном смысле. Homo sapiens neanderthalensis в виде ископаемых
остатков не обнаруживается позже рубежа в 25 тыс. лет. Быстрое исчезновение палеонтропов может быть
объяснено  вытеснением  их  людьми  с  более  совершенной  техникой  изготовлений  труда  и  возможной
скрешиваемости (метисации) с ними.
Данные антропологии и археологии доказывают роль целенаправленной трудовой деятельности как основного
фактора антропосоциогенеза в процессе социального развития человека. Однако она была неодинаковой на
разных этапах его развития, поскольку в ранней стадии первобытного общества (стаде) прогресс в социальной
организации в значительной мере зависел от биологических изменений человека; однако процесс антропогенеза
сопровождался  постепенным  сужением  сферы  действия  естественного  отбора  в  сторону  возникновения
общественных закономерностей и создания социальной и культурной среды.

\newpage
\section{Понятия «индивид», «индивидуальность», «личность»}
Индивид (от лат. Individuum - неделимое), первоначально - лат. Перевод греческого понятия 'атом' (впервые у
Цицерона), в дальнейшем - обозначение единичного в отличие от совокупности, массы; отд. Живое существо,
особь, отд. Человек - в отличие от коллектива, социальной группы, общества в целом.
Индивидуальность - неповторимое своеобразие какого-либо явления, отделяющее существа, человека. В самом
общем плане И. В качестве особенного, характеризующего данную единичность в ее качественных отличиях,
противопоставляется типичному как общему, присущему всем элементам данного класса или значительной
части их.
Индивидуальность  не  только  обладает  различными  способностями,  но  еще  и  представляет  некую  их
целостность.  Если  понятие  индивидуальности  подводит  деятельность  человека  под  меру  своеобразия  и
неповторимости,  многосторонности  и  гармоничности,  естественности  и  непринужденности,  то  понятие
личности поддерживает в ней сознательно-волевое начало. Человек как индивидуальность выражает себя в
продуктивных действиях, и поступки его интересуют нас лишь в той мере, в какой они получают органичное
предметное воплощение. О личности можно сказать обратное, в ней интересны именно поступки.
Личность - общежитейский и научный термин, обозначающий:
 1. человечность индивида как субъекта отношений и сознательной деятельности (лицо, в широком смысле
слова) или
  2. устойчивую систему социально-значимых черт, характеризующих индивида как члена того или иного
общества или общности. 
Жизнеспособность  человека  покоится  на  воле  к  жизни  и  предполагает  постоянное  личностное  усилие.
Простейшей,  исходной  формой  этого  усилия  является  подчинение  общественным  нравственным  запретам,
зрелой и развитой - работа по определению смысла жизни.
Человек - совокупность всех общественных отношений.
1. Идеалистическое и религиозно-мистическое понимание человека;
2. натуралистическое (биологическое) понимание человека;
3. сущностное понимание человека;
4. целостное понимание человека. 
Человека  философия  понимает  как  целостность.  Сущность  человека  связана  с  обществ.  Условиями  его
функционирования и развития, с деятельность, в ходе которой он оказывается и предпосылкой и продуктом
истории. 


\newpage
\section{Основные концепции философии истории}
Философия истории анализирует проблемы смысла и цели существования общества, перспективы его развития.
С  возникновением  христианства  оказалось  возможным  иное  понимание  истории.  Оно  предполагало
поступательное прогрессивное развитие в истории человечества: от акта творения Богом до ее "финала" -второго пришествия Христа и Страшного суда. В новое время земная история перестала восприниматься как
священная история. В XVIII веке в работах просветителей постоянно звучала тема мощи человеческого разума,
который следовало лишь освободить от пут религии и предрассудков. История оценивалась теперь как история
разума. Наиболее системно представление о неизбежности прогресса в движении человечества было дано в
философии Гегеля. В марксистской концепции общества прогресс рассматривался как результат неуклонного
развития производительных сил. С конца XX века понятие прогресса общества и истории все более связывается
с развитием телесных и духовных характеристик самого человека. С особой силой философия истории XX века
поставила проблему коммуникации как основания человеческого существования. История возможна лишь в той
мере, в какой люди открыты миру и друг другу. История реализуется через общение.
Философы и социологи используют различные подходы к изучению общественного прогресса. В марксистской
философии,  например,  был  разработан  формационный  подход.  С  точки  зрения  формационного  подхода,
исторический  прогресс  понимается  как  смена  общественно-экономических  формаций.  Общественно-экономическая формация - это исторически конкретное общество на данном этапе его развития. Она включает в  
себя все явления, которые имеются в обществе: материальные, духовные, политические, социальные, семейно-бытовые. Основу общественно-экономической формации составляет способ производства материальной жизни
в единстве производительных сил и производственных отношений. В развитии, исторического процесса Маркс
выделял  пять  формаций:  первобытнообщинную,  рабовладельческую,  феодальную,  капиталистическую  и
коммунистическую. Формационный подход к развитию общества существует наряду с культурологическим и
цивилизационным подходами.
Культурологический подход к истории использовал Освальд Шпенглер (1880- 1936). Он исходил из того, что
каждая  культура  существует  изолированно  и  замкнуто.  Появляясь  на  определенном  этапе  исторического
процесса,  она  переживает  возрасты  отдельного  человека  (детство,  юность,  зрелость  и  старость)  и  затем
погибает. Смерть культуры, по Шпенглеру, начинается с возникновения цивилизации.
Понятие  "культура"  относится  к  числу  фундаментальных  в  современном  обществознании.  Культуру  часто
определяют как "вторую природу", т.е. культура есть природа, обработанная человеком в целях удовлетворения
тех  или  иных  потребностей.  Однако  культуру  нельзя  свести  только  к  вещам,  произведенным  человеком.
Понятие культуры охватывает собой и продукты духовного производства, распространяется на общественные
отношения. Суть культуры заключается в том, что она несет в себе систему и природных, и социальных качеств.
Культура обнаруживает себя в истории. Одним из главных критериев жизнестойкости и прогресса культуры
является способность одной культуры вбирать и осваивать достижения других культур. Культура обусловлена
потребностью общества в закреплении и передаче совокупного духовного опыта.
Категория  цивилизации  охватывает  природу  и  уровень  развития  материальной  и  духовной  культуры.
Цивилизация воплощает в себе технологический аспект культуры. Главное в цивилизации - это непрерывная
смена  технологий  для  удовлетворения  столь  же  непрерывно  растущих  потребностей  и  возможностей
человечества.

\newpage
\section{Ценности человеческого бытия (труд, творчество, любовь, игра)}
Разломленность человеческого бытия на фрагментарные формы жизни, мужскую и женскую,
есть нечто большее, нежели случайные биологические состояния, нежели чисто внешняя
обусловленность  психофизической  организации:  двойственность  полов  относится  в
бытийному строю нашего конечного существования и является фундаментальным моментом
нашей конечности как таковой.
Каждый из нас выступает одновременно личностью и носителем пола, индивидом лишь в пространстве рода,
каждый  из  нас  лишен  другой  половины  человеческого  бытия,  лишен  в  такой  степени,  что  именно  эта
лишенность и порождает величайшую и могучую страсть, глубочайшее чувство, смутную волю к восполнению
и томление по непреходящему бытию -- загадочное стремление обреченных на смерть людей к некоей вечной
жизни.  О  том,  как  Эрос  в  своей  последней  смысловой  глубине  отнесен  к  бессмертию  смертных,  Платон
высказывает в "Пире" устами пророчицы Диотимы: тайна всякой человеческой любви -- воля к вечности во
времени, влечение к устоянию, к длительности именно конечного во времени человека, гонимого раздирающим
потоком времени, знающего о своей бренности [1]. К тому, что без труда дается бессмертным богам в их
самодостаточности, стремятся смертные люди, которые не в состоянии уберечь свое бытие от разрушительной
силы времени, -- и они почти обретают вечность в объятии. Возможно, доставляемое Эросом переживание
вечности содействовало выработке человеческого представления о вечности и бессмертии богов, содействовало
возникновению понятия бытия, разделившего смертное и бессмертное: бытие во времени и бытие по ту сторону
всякого времени. Возможно, в человеческой любви коренится та поэтическая сила, что создала миф, и тогда
Эрос  на  самом  деле  оказался  бы  старейшим  из  богов.  Все  рассмотренные  до  сих  пор  основные
экзистенциальные феномены суть не только существенные моменты человеческого бытия, но также и источник
человеческого  понимания  бытия, не  только онтологические  структуры человека,  но  и  смысловой  горизонт
человеческой онтологии. Тот род и способ, каким мы понимаем бытие, как мы рассматриваем многообразное
сущее, как мыслим себе очертания вещи, делаем различие между безжизненным и одушевленным бытием, 
между видами и родами разнооформленных вещей, как мы толкуем сущность и существование, различаем
действительность  и  возможность,  необходимость  и  случайность  и  тому  подобное  --  все  это  определено  и
обусловлено своеобразием нашего разума, структурой познавательной способности. Но ведь наш разум есть
разум  открытого  смерти  и  смерти  предуготовленного  существа,  разум  действующего,  трудящегося  и
борющегося создания, разум преимущественно практический, наконец -- разум творения, раздвоенного на две
полярные формы жизни и томящегося по единению, исцелению и восполнению. Наш разум не безразличен по
отношению  к  основным  феноменам  нашего  существования,  неизбежно  он  является  разумом  конечного
человека,  определенного  и  обусловленного  в  своем  бытии  смертью,  трудом,  гocподством  и  любовью.
Конечность  человеческого  разума  постигается  недостаточно,  когда  ее  истолковывают  в  качестве
ограниченности,  суженности,  стесненности,  то  есть  пытаются  определить  через  дистанцию,  отделяющую
человеческий  разум  от  некоего  гипотетического  разума  божества  или  мирового  духа.  Измеренный
божественной меркой, человеческий разум оказывается несущественным, убогим, жалким, тусклым огоньком,
изгнанным в дальние дали от сияния, озаряющего вселенную. Разум бога не знает ни смерти, ни труда, ни
господства над равным, ни любви как стремления по утраченной другой половине своего бытия. Считается, что
божественный разум безграничен, закончен, завершен и блаженно покоится в себе. Для нас непостижимо,
каким образом бог понимает бытие, исходя из своего всемогущества, всеприсутствия и всезнания. Но поэтому
он и не может быть меркой для конечного человеческого разума. Всякая попытка уподобить себя богу есть
высокомерие.  Неоднократно  в  истории  западной  метафизики  создавалась  трагическая  ситуация,  в  которой
истолкование бытия человеком связывалось с желанием поставить себя на место божественного разума или хотя
бы по аналогии снять "дистанцию", перебросить мостик между конечным и бесконечным бытием с помощью
analogia entis [2]. С этой традицией следует порвать, если мы готовы вступить в истину нашего конечного
существования и адекватно воспринять нашу антропологическую реальность.
Какие  же  имеются  человеческие  основания  для  того,  чтобы  человек  постоянно  перескакивал  через  свое
"condition humaine" [3], казался способным отринуть свою конечность, мог овладевать сверх-человеческими
возможностями,  грезить  об  абсолютном  разуме  или  абсолютной  власти,  мог  измыслить  действительное  и
примыслить недействительное, был в состоянии освободиться от тягот нашей жизни -- бремени труда, остроты
борьбы,  тени  смерти  и  мук  любовного  томления?  Пожалуй,  не  следует  торопиться  с  психологическим
объяснением и указывать на особую душевную способность -- способность фантазии. Невозможно оспаривать
существование этой способности. Всякий знает ее и бесчисленные формы ее выражения. Несомненно, сила
воображения относится к основным способностям человеческой души; она проявляется в ночном сновидении, в
полуосознанной дневной грезе, в представляемых влечениях нашей инстинктивной жизни, в изобретательности
беседы, в многочисленных ожиданиях, которые сопровождают и обгоняют, прокладывая ему путь, процесс
нашего восприятия. Фантазия действует почти повсеместно: она гнездится в нашем самосознании, определяя
тот образ, который складывается у нас о себе, или же тот, в котором нам хотелось бы видеться ближним, она
ловко  сопротивляется  беспощадному  самопознанию,  приукрашивает  или  искажает  для  нас  образ  другого,
определяет отношение человека к смерти, наполняет нас страхом или надеждой, она -- в качестве творческого
озарения -- направляет и окрыляет труд, она открывает возможность политического действия и просветляет
друг  для  друга  любящих.  Тысячью  способов  фантазия  проницает  человеческую  жизнь,  таится  во  всяком
проекте будущего, во всяком идеале и всяком идоле, выводит человеческие потребности из их естественного
состояния к роскоши; она присутствует при всяком открытии, разжигает войну и кружит у пояса Афродиты.
Фантазия открывает нам возможность освободиться от фактичности, от непреклонного долженствования так-бытия, освободиться хотя бы не в действительности, а "понарошку", забыть на время невзгоды и бежать в более
счастливый  мир  грез.  Она  может  обратиться  в  опиум  для  души.  С  другой  стороны,  фантазия  открывает
великолепный доступ к возможному как таковому, к общению с быть-могущим, она обладает силой раскрытия,
необычайной по значению. Фантазия -- одновременно опасное и благодатное достояние человека, без нее наше
бытие оказалось бы безотрадным и лишенным творчества. Проницая все сферы человеческой жизни, фантазия
все же обладает особым местом, которое можно счесть ее домом: это игра.
Так называем пятый из основных феноменов человеческого существования. Если он назван последним, то не
потому, что является "последним" в иерархическом смысле -- менее значительным и весомым, нежели смерть,
труд, господство и любовь. Игра столь же изначальна, как и эти феномены. Она охватывает всю человеческую
жизнь до самого основания, овладевает ею и существенным образом определяет бытийный склад человека, а
также  способ  понимания  бытия  человеком.  Она  пронизывает  другие  основные  феномены  человеческого
существования,  будучи  неразрывно  переплетенной  и  скрепленной  с  ними.  Игра  есть  исключительная
возможность человеческого бытия. Играть может только человек. Ни животное, ни бог играть не могут. Лишь
сущее, конечным образом отнесенное к всеобъемлющему универсуму и при этом пребывающее в промежутке
между действительностью и возможностью, существует в игре. Эти "тезисы" нуждаются в пояснении, так как
на первый взгляд противоречат привычному жизненному опыту. Каждый знает игру, это совершенно знакомое
явление. Но, по Гегелю, знакомое еще не есть познанное [4]. Как раз то, что кажется нам привычным и само
собой разумеющимся, порой наиболее упрямо ускользает от какого бы то ни было понятийного постижения.
Каждый  знает  игру  по  своей  собственной  жизни,  имеет  представление  об  игре,  знает  игровое  поведение
ближних,  бесчисленные  формы  игры,  знает  общественные  игры,  цирцеевские  массовые  представления,
развлекательные игры и несколько более напряженные, менее легкие и привлекательные, нежели детские игры,
игры взрослых; каждый знает об игровых элементах в сферах труда и политики, в общении полов друг с 
другом, игровые элементы почти во всех областях культуры. Home ludens неотделим от homo faber и homo
politicus. Игра есть такое измерение существования, которое многочисленными нитями сплетено с другими
измерениями.  Всякий человек  играл и  может  высказаться  об  игре,  опираясь на собственный опыт. Чтобы
сделать  игру  предметом  размышления,  ее  не  нужно  привносить  откуда-либо  извне:  сообразно  с
обстоятельствами мы обнаруживаем, что вовлечены в игру, мы накоротке с этой ключевой возможностью даже
тогда, когда на самом деле не играем или полагаем, что давно оставили позади игровую стадию своей жизни.
Каждому  известно  несчетное  число  игровых  ситуаций  в  частной,  семейной  и  общественной  сферах.  Они
изобилуют игровыми действиями, которые суть повседневные события и происшествия в человеческом мире.
Никому игра не чужда, всякий знает ее по свидетельству собственной жизни. Будничная привычность игры,
однако, зачастую препятствует более глубокой постановке вопроса о сущности, бытийном смысле и статусе
игры. Такая привычность совершенно заслоняет вопрос о том, действительно ли и в какой мере игровое начало
человека определяет и оформляет его понимание бытия в целом. Будничная привычность игры чаще всего
остается без вопросов благодаря будничному толкованию игры. В качестве основного феномена, игра обладает
структурой истолкованности. И это толкование не сводится к примеси частного или общественного сознания,
которая могла бы и отсутствовать. Основные Экзистенциальные феномены -- не просто бытийные способы
человеческого существования: они также и способы понимания, с помощью которых человек понимает себя как
смертного, как трудящегося, как борца, любящего и игрока и стремится через такие смысловые горизонты
объяснить одновременно бытие всех вещей.
Что же характеризует будничное толкование человеческой игры? Не что иное, как попытку вытеснить игру из
сущностного центра человеческого бытия, лишить ее сути, понять ее как "пограничный феномен" нашей жизни,
забрать у нее весомость и подлинное значение. Хотя очевидны частота игровых действий, интенсивность, с
какой предаются игре, ее растущая оценка в связи с возрастанием свободного времени в технизированном
обществе,  по-прежнему  в  игре  принято  усматривать  прежде  всего  "отдых",  "расслабление",
времяпрепровождение  и  радостную  праздность,  благотворную  "паузу",  прерывающую  рабочий  день  или
присущую дню праздничному. Там, где толкование игры исходит из ее противопоставления труду или вообще
серьезности жизни, там мы имеем дело с наиболее поверхностным, но преобладающим в повседневности
пониманием  игры.  Игра  при  этом  считается  неким  дополнительным  феноменом,  чем-то  несерьезным,
необязательным,  произвольно-самовольным.  Даже  признавая,  что  игра  имеет  власть  над  людьми  и  своим
очарованием прельщает их, игру все же не рассматривают с точки зрения ее позитивного значения и неверно
толкуют  как  некую  интермедию  между  серьезными  жизненными  занятиями,  как  паузу,  как  наполнение
свободного времени. Сказанное о будничном толковании игры, которое ее умаляет, относится прежде всего к
жизни взрослых. Играют -- да ведь только между делом, шутки ради, для разлечения, времяпрепровождения,
ради  того,  чтобы  на  время  выпрячься  из  кабалы  труда,  а  может  даже  и  с  терапевтическими  целями:
расслабиться, восстановиться, отстраниться от серьезности жизни -- игрой пользуются как сном. Считается, что
реальность взрослой жизни -- решения, решения моральные и политические, тягость труда, острота борьбы,
ответственность  за  себя  и  за  близких.  Будто  бы  только  ребенку  пристало  жить  игрой,  проводить  часы  в
радостной  беззаботности,  попусту  расточать  время.  Счастье  детства,  блаженство  игры  --  мимолетны,  как
мимолетен этот период времени нашей жизни, когда мы еще имеем время, потому что еще не знаем о нем, еще
не видим в "теперь" "уже", "никогда больше" и "еще не", когда наша жизнь мчит в глубоком и неосознанном
настоящем, когда жизненный поток увлекает нас, не ведающих о течении, стремящемся к нашему концу. Чистое
настоящее детства и считается обычно временем игры. Играет ли по-настоящему и в подлинном смысле слова
только  дитя,  а  во  взрослой  жизни  присутствуют  лишь  какие-то  реминисценции  детства,  неосуществимые
попытки  "повторения",--  или  же  игра  остается  основным  феноменом  и  для  других  возрастов?  Понятие
"основной феномен" не подразумевает требования, чтобы явленный образ человеческой жизни непременно и
непрестанно  выказывал  какой-то  определенный  признак.  Вопрос  о  том,  является  ли  игра  основным
экзистенциальным феноменом, не зависит от того, играем ли мы постоянно или же только иногда. Основным
феноменам  вовсе  не  обязательно  проявляться  всегда  и  во  всех  случаях  в  виде  какой-то  постоянной
документации. Да это и не необходимо -- чтобы они "могли" проявляться непрестанно. То, что определяет
человека как существо временное в самом его основании, вовсе не должно происходить в каждый момент
"теперь" его жизни. Смерть все же расположена в конце времени жизни, любовь -- на вершине жизни, игра (как
детская  игра)  --  в  ее  начале.  Подобная  фиксация  и  датировка  во  времени  упускает  то,  что  основные
экзистенциальные феномены захватывают человека всецело. Смерть -- не просто "событие", но и бытийное
постижение  смертности  человеком.  Так  и  игра:  не  просто  калейдоскоп  игровых  актов,  но  прежде  всего
основной способ человеческого общения с возможным и недействительным. Мы начинаем с краткого анализа
игрового поведения, то есть занятия игрой. Из-за своей краткости и сжатости этот анализ может показаться
абстрактно-формальным, но выводимые структуры каждый может, учитывая определенные единичные случаи,
проверить на самом себе. При различении "структуры" и "единичного случая" последний принято обозначать
как  пример  (Bei-Spiel)  структуры.  Многоразличное,  в  котором  утверждается  структура,  понимается  как
случайное, привнесенное игрой случая. Отношение постоянного к изменчивому, необходимого к случайному,
достаточно примечательно характеризуется метафорой игры, причем поначалу должен оставаться открытым
вопрос  о  том,  является  ли  применение  идеи  игры  к  онтологическим  отношениям  неосмотрительным
"антропоморфизмом" или же оно выводимо из самого предмета размышления. Каковы же существенные черты
человеческой игры? Мы начинаем с формы исполнения. Игра -- это импульсивное, спонтанное протекающее  
вершение,  окрыленное  действование,  подобное  движению  человеческого  бытия  в  себе  самом.  Но  игровая
подвижность  не  совпадает  с  обычной  формой  движения  человеческой  жизни.  Рассматривая  обычное
действование,  во  всем  сделанном  мы  обнаруживаем  указание  на  конечную  цель  человека,  на  счастье,
эвдаймонию. Жизнь принимается в качестве урока, обязательного задания, проекта; у нас нет места для отдыха,
мы воспринимаем себя "в пути" и обречены вечно быть изгнанными из всякого настоящего, увлекаемыми
вперед силой внутреннего жизненного проекта, нацеленного на эвдаймонию. Мы все неустанно стремимся к
счастью, но не едины во мнении, в чем же оно заключается. В напряжении нас держит не только беспокойный
порыв к счастью, но и неопределенность в толковании "истинного счастья". Мы пытаемся заработать, завоевать,
за-любить себе счастье и полноту жизни, но нас постоянно влечет за пределы достигнутого, всякое доброе
настоящее мы жертвуем неведомому "лучшему" будущему. Хотя игра как играние есть импульсивно подвижное
бытие,  она  находится  в  стороне  от  всякого  беспокойного  стремления,  проистекающего  из  характера
человеческого бытия как "задачи": у нее нет никакой цели, ее цель и смысл -- в ней самой. Игра -- не ради
будущего  блаженства,  она  уже  сама  по  себе  есть  "счастье",  лишена  всеобщего  "футуризма",  это  дарящее
блаженство настоящее, непредумышленное свершение. Никоим образом это не исключает того, чтобы игра
содержала в себе моменты значительного напряжения, как, например, игра-состязание. Но игра, со своими
волнениями, со всей шкалой внутреннего напряжения и проектом игрового действия, никогда не выходит за
свои пределы и остается в себе самой. Глубокий парадокс нашего существования состоит в том, что в своей
продолжающейся всю жизнь охоте за счастьем мы никогда не настигаем его, никого нельзя перед смертью
назвать счастливым в полном смысле слова, и что мы тем не менее, оставив на мгновение свое преследование,
нежданно  оказываемся  в  "оазисе"  счастья.  Чем  меньше  мы  сплетаем,  игру  с  прочими  .жизненными
устремлениями,  чем  бесцельней  игра,  тем  раньше  мы  находим  в  ней  малое,  но  полное  в  себе  счастье.
Дионисийский  дифирамб  Ницше  "Среди  дочерей  пустыни"  [5],  зачастую  недооцениваемый  и  неправильно
толкуемый, воспевает как раз чары и оазисное счастье игры в пустыне и бессмысленности современного бытия,
порождаемых обесцениванием некогда высших ценностей. Игра не имеет "цели", она ничему не служит. Она
бесполезна, и никчемна: она не соотнесена с какой-то конечной целью -- конечной целью человеческой жизни, в
которую верят или которую провозглашают. Подлинный игрок играет ради того, чтобы играть. Игра -- для себя
и в себе, она более, нежели в одном смысле, есть "исключение". Часто утверждают, что игра целедостаточна в
самой себе, что она несет в себе цели, которые, однако, не выходят за пределы игровой структуры. Но ведь и
всякое  законченное  трудовое  действие  несет  цели  в  себе,  единичные  приемы  согласованы  друг  с  другом,
происходят по единому плану, направляются единым замыслом. Однако трудовое действие в целом служит
выходящим за его пределы целям, вплетено в более широкий смысловой контекст. Игровому действию присущи
лишь имманентные ему цели. Если мы играем ради того, чтобы за счет игры достичь какой-то, иной цели, если
мы играем ради закалки тела, ради здоровья, приобретения военных навыков, играем, чтобы избавиться от
скуки и провести пустое, бессмысленное время,-- тогда мы упускаем из виду собственное значение игры.
Считается,  что  игре  воздается  сполна,  если  ей  приписывается  биологическое  значение  какой-то  еще  пока
безопасной, лишенной риска тренировки и отработки будущих серьезных дел нашей жизни. Игра в этом случае
служит  для  подготовки  --  сначала  посредством  ни  к  чему  не  обязывающих  проб-поступков  и  способов
поведения, которые позднее станут обязательными и неотменимыми. Именно в педагогике обнаруживается
значительное число теорем, низводящих игру до предварительной пробы будущего серьезного действия, до
маневренного  поля  для  опытов  над  бытием.  При  таком  понимании  игры  ее  польза  и  целительная  сила
усматриваются в том, чтобы в направляемой и контролируемой детской игре предвосхитить будущую взрослую
жизнь и плавно, через игровой маскарад, подвести питомца ко времени, когда лишнего времени у него не
останется: все поглотят обязанности, дом, заботы и звания. Оставляем открытым вопрос, исчерпывается ли
подобным  пониманием  игры  ее  педагогическая  значимость  и  вообще  --  ухватывается  ли  хотя  бы
приблизительно. Мы скептически относимся к широко распространенному мнению, будто бы игра принадлежит
исключительно детскому возрасту. Конечно, дети играют более открыто, притворяясь и маскируясь меньше,
нежели это делают взрослые, но игра есть возможность не только ребенка, но человека вообще. Человек как
человек  есть  игрок.  Игровому  свершению  присуща  особая  настроенность,  настроение  окрыленного
удовольствия,  которое  больше  простой  радости  от  свершения,  сопровождающего  спонтанные  поступки,
радость, в которой мы наслаждаемся своей свободой, своим деятельным бытием. Игровое удовольствие -- не
только удовольствие в игре, но и удовольствие от игры, удовольствие от особенного смешения реальности и
нереальности. Игровое удовольствие объемлет также и печаль, ужас, страх: игровое удовольствие античной
трагедии  охватывает  и  страдания  Эдипа.  И  игра-страсть,  переживаемая  как  удовольствие,  влечет  за  собой
катарсис души, который есть нечто большее,
 нежели разрядка застоявшихся аффектов. Далее, игра связана с правилами. То, что ограничивает произвол в
действиях играющего человека, -- не природа, не ее сопротивление человеческому вторжению, не враждебность
ближних, как в сфере господства, -- игра сама полагает себе пределы и границы, она покоряется правилу,
которое сама же и ставит. Играющие связаны игровым правилом, будь то соревнование, карточная игра или игра
детей. Можно отменить "правила", договориться о новых. Но пока человек играет и осмысленно понимает
процесс игры, он остается связанным правилами. Первым делом играющие договариваются о правилах -- пусть
это даже будет условленная импровизация. Конечно, не все время изобретаются "новые" игры -- готовые игры с
твердыми,  известными  правилами  существуют  в  любой  социальной  ситуации.  Но  есть  и  творческое
изобретение  новых  игр,  возникающих  из  спонтанной  деятельности  фантазии  и  затем  "фиксируемых"  во 
взаимной договоренности. Однако мы играем не потому, что в окружающем социуме имеются игры: игры
наличны и возможны лишь потому, что мы играем в сущностной основе своей.
Чем мы играем? На этот вопрос нельзя ответить сразу и недвусмысленно. Всякий игрок играет прежде всего
самим собой, принимая на себя определенную смысловую функцию в смысловом целом общественной игры: он
играет средствами игры (игралищами), вещами, признанными подходящими для игры или специально для нее
изготовленными.  К  таким  средствам  относятся:  игровое  поле,  обозначения  границ,  отметки,  необходимые
инструменты, вспомогательные средства вещественного характера. Не все игралища есть игрушки в строгом
смысле слова. Там, где игра является в чистой двигательной форме (спорт, соревнования и т.д.), она нуждается в
разнообразном игровом инвентаре. Но чем больше игра приобретает черты игры-представления, тем больше в
игровом инвентаре от настоящей игрушки. Кажется, что об игрушке может рассказать любой ребенок, и, однако,
природа игрушки -- темная, запутанная проблема. Само название двусмысленно: мы зовем какую-либо вещь
игрушкой, когда считаем, что можно приспособить ее для игры. Мы говорим сейчас как бы со стороны; с точки
зрения  неиграющего,  не  вовлеченного  в  игру.  Какие-то  чисто  природные  вещи  могут  показаться  нам
пригодными для чужой игры, например ракушки на берегу для детской игры. С другой стороны, нам известно
об искусственном производстве и изготовлении игрушек для определенной игровой потребности. Значит, люди
не  производят  игрушки  в  игре:  они  производят  их  в  труде,  серьезном  трудовом  действии,  снабжающем
игрушками  рынок?  Человеческий  труд,  таким  образом,  производит  не  только  средства  пропитания  и
инструменты для обработки природного материала, он производит жизненно необходимые вещи и для других
измерений бытия, производит оружие воина, украшения женщин, культовый инвентарь для богослужения и --игрушку, насколько игрушкой могут быть искусственные вещи. С этой точки зрения, игрушка есть один из
предметов в общем контексте единой мировой реальности, бытующий иначе, но все же не менее реально, чем,
например,  играющий  ребенок.  Кукла  --  чучело  из  пластмассы,  приобретаемое  за  определенную  цену.  Для
девочки, играющей в куклу, кукла -- "ребенок", а сама она -- его "мама". Конечно же, девочка не становится
жертвой заблуждения, она не путает безжизненную куклу с живым ребенком.
Играющая девочка живет одновременно в двух царствах: в обычной действительности и в сфере нереального,
воображаемого.  В  своей  игре  она  называет  куклу  ребенком:  игрушка  обладает  магическими  чертами,  она
возникает, в строгом смысле, не благодаря промышленному производству, она возникает не в процессе труда, но
в игре и из игры, насколько последняя является проектом особого смыслового измерения, не включающаяся в
действительность, но, скорее, парящего над нею в качестве некоей неуловимой видимости. Здесь раскрывается
сфера возможного, не связанная с течением реальных событий, область, которая, хоть и нуждается в месте и
использует его, занимает пространство и время, но сама по себе не является частью реального пространства и
времени:  нереальное  место  в  нереальном  пространстве  и  времени.  Игрушка  возникает  тогда,  когда  мы
перестаем рассматривать ее в качестве фабричного изделия, извне, и начинаем смотреть на нее глазами игрока,
в рамках единого смыслового контекста игрового мира. Творческое порождение игрового мира -- особенная
продуктивность игры-представления -- чаще всего имеет место в рамках коллективного действия, сыгранности
игрового сообщества. Созидая игровой мир, играющие не остаются в стороне от своего создания, они не
остаются  вовне,  но  сами  вступают  в  игровой  мир  и  играют  там  определенные  роли.  Внутри  созданного
фантазией творческого проекта игрового мира играющие маскируют себя как "творцов", некоторым образом
теряются в своих созданиях, погружаются в свою роль и встречаются с партнером по игре, которые также
играют определенные роли. Конечно, вещи игрового мира никоим образом не перекрывают реальные вещи
реального мира: они лишь преобразуют их в атмосфере продуцированного смысла, но не меняют их реально в
их бытии. Сила игровой фантазии в реальности, разумеется, есть бессилие. Если говорить об изменении бытия,
то здесь игра, очевидно, не может сравниться с человеческим трудом или борьбой за власть. Что же, значит
ничтожное свидетельство нашей творческой силы, которая едва набрасывает очертания воздушных замков в
податливом  материале  фантазии,  несущественно?  Или  оно  свидетельствует  об  исключительном  умении
вступать в контакт с возможностями посреди прочно установленной реальности, к которой мы привязаны
самыми различными способами? Не есть ли это освобождающее, вызволяющее общение с возможностями
также и общение с первоистоком, откуда вообще только и произошло прочное, устойчивое и неизменное бытие?
Является  ли  такая  изначальность  игры  человеческим,  слишком  человеческим  заблуждением,  чрезмерной
оценкой совершенно бессильного что-либо изменить способа поведения или же в человеческой игре нам явлено
указание  на  то,  что  более  всего  остального  может  быть  названным  первоистоком?  Бытийный  строй
человеческой игры совсем не легко прояснить, еще труднее указать присущий игре особый род понимания
бытия. Человек втянут в игру, в трагедию и комедию своего конечного бытия, из которого он никак не может
ускользнуть в чистое, нерушимое самостояние божества. "Вокруг героев", -- говорил Ницше, -- все обращается
в трагедию, вокруг полубогов -- в сатиру, а вокруг богов -- как? -- вероятно, в мир" [6].
Игра как фундаментальная особенность нашего бытия
Игра, которую знает всякий, знает по собственному опыту задолго до того, как вообще научится надежно
управлять своим разумом, игра, в которой всякий свободен задолго до того, как сможет различать понятия
свободы и несвободы, -- эта игра не есть пограничный феномен нашей жизни или преимущество одного только
детства. Человек как человек играет -- и лишь он один, один среди всех существ. Игра есть фундаментальная
особенность нашего существования, которую не может обойти вниманием никакая антропология. Уже чисто
эмпирическое изучение человека выявляет многочисленные феномены явной и замаскированной игры в самых 
различных сферах жизни, обнаруживает в высшей степени интересные образцы игрового поведения в простых
и  сложных  формах,  на  всех  ступенях  культуры  --  от  первобытных  пигмеев  до  позднеиндустриальных
урбанизированных  народов.  Все  возрасты  жизни  причастны  игре,  все  опутаны  игрой  и  одновременно
"освобождены",  окрылены,  осчастливлены  в  ней  --  ребенок  в  песочнице  точно  так  же,  как  и  взрослые  в
"общественной  игре"  своих  конвенциональных  ролей  или  старец,  в  одиночестве  раскладывающий  свой
"пасьянс". Подлинно эмпирическому исследованию следовало бы когда-нибудь собрать и сравнить игровые
обычаи всех времен и народов, зарегистрировать и классифицировать огромное наследие объективированной
фантазии, запечатленное в человеческих играх. Это была бы история "изобретений" -- другого рода, конечно,
чем изобретения орудий труда, машин и оружия, изобретений, которые могут показаться менее полезными, но
которые в основе своей были чрезвычайно необходимыми. Нет ничего необходимее избытка, ни в чем человек
не  нуждается  столь  остро,  как  в  "цели"  для  своей  бесцельной  деятельности.  Естественные  потребности
понуждают нас к действию, нужда учит трудиться и бороться. Затруднение ясно дает нам понять, что нам
следует делать в том или ином случае. А как обстоит дело тогда, когда потребности на время утихают, когда их
неумолимый бич не подгоняет нас, когда у нас есть время, которое буйно для нас разрастается, растягивается и
угрожает вовсе опустеть. Без игры человеческое бытие погрузилось бы в растительное существование. Игра к
тому  же  вливает  многие  смысловые  мотивы  в  жизненные  сферы  труда  и  господства:  как  говорится,  игра
оборачивается серьезностью. Иной раз сделанные в игре изобретения внезапно получают реальное значение.
Человеческое общество  многообразно экспериментирует  на  игровом поле  прежде,  чем  испробованные там
возможности станут твердыми нормами и. обычаями, обязательными правилами и предписаниями. Игра как
испытание  возможностей  занимает  в  системе  экономии  социальной  практики  громадное  место,  хотя  ее
экзистенциальный смысл никогда не исчерпывается этой функцией. Философская антропология обязана выйти
за пределы эмпирического понимания игры и прежде всего разработать концепцию принципиальной структуры,
бытийного строя и имманентного бытийного понимания игры.
Человеческую игру сложно разграничить с тем, что в
биолого-зоологическом  исследовании  поведения  зовется  игрой  животных.  Разве  не  бесспорно  наличие  в
животном  царстве  многочисленных  и  многообразных  способов  поведения,  которые  мы  совершенно  не
задумываясь должны назвать "играми"? Мы не можем найти для этого никакого другого выражения. Поведение
детей и поведение детенышей животных кажется особенно близкими одно другому. Взаимное преследование и
бегство, игра в преследование добычи, проба растущих сил в драках и притворной борьбе, беспокойное, живое
проявление энергии и радости жизни -- все это мы замечаем как у животного, так и у человека. Внешне --прямо-таки  поразительное  сходство.  Но  сходство  между  животными  и  человеком  сказывается  не  только  в
поведении человеческих детей и детенышей животных. Человек -- живое существо, "animal": бесчисленные
черты сближают и роднят его с животными, и близость эта столь велика, что тысячелетиями человек ищет все
новые формулы, чтобы отличить себя от животного. Вероятно, один из сильнейших стимулов антропологии --стремление к подобному различению. Животное избегает человека. По крайней мере дикое животное со своим
ненарушенным инстинктом старается обойти нас стороной, оно чуждается нарушителя спокойствия в природе,
но не "различает" себя от нас. Человек есть природное создание, которое неустанно проводит границы, отделяет
самого себя от природы, от природы вокруг и внутри себя -- обездоленное животное, не управляемое уже
надежными  инстинктами,  обреченное  отстранять  себя,  --  оно  уже  не  существует  просто  так,  но,  скорее,
отброшено назад на свое бытие, отражено к нему, оно относится к самому себе и к бытию всего сущего,
неустанно ищет потерянные тропы и нуждается в определениях самого себя, чувствует себя "венцом творения",
"подобием  бога",  местом,  где  все,  что  есть,  обращается  в  слово,  или  же  вместилищем  мирового  духа.
Человеческий  дух  уже  разработал  многочисленные  формулы  для  того,  чтобы  утвердиться  в  своей
исключительности  и  необыкновенной  весомости,  чтобы  дистанциироваться  от  всех  прочих  природных
созданий. Возможно, трудным делом окажется отобрать среди подобных различений те, которые идут от нашей
гордости и высокомерия, и те, которые на самом деле истинны. Пусть некоторые из этих формул ложны --несомненно то, что мы различаем и существуем в подобных различениях. Акт постижения человеком самого
себя  имеет  предпосылкой  противопоставление  себя  всему  остальному  сущему.  Животное  не  знает  игры
фантазии как общения с возможностями, оно не играет, относя себя к воображаемой видимости. С точки зрения
науки  о  поведении,  специфически  человеческое  в  игре  выявлено  быть  не  может.  Неотложной  задачей
философского осмысления остается утверждение понятия игры, означающего основной феномен нашего бытия,
вопреки широкому и неясному использованию слова "игра" в рамках зоологического исследования поведения.
Задача эта тем неотложней, чем обширнее материалы о психологии животных. То обстоятельство, что человек
нуждается в "антропологии", в понятийном самопонимании, что он живет с им самим созданным образом
самого себя, с видением своей задачи и определением своего места, постоянно пеленгуя свое положение в
космосе, что он может понимать себя, лишь отделив себя от всех остальных областей сущего и в то же время
относя  себя  к  совокупному  целому,  ко  вселенной,  уже  само  это  есть  антропологический  факт  огромного
значения.
У животного нет никакой "зоологии", и она ему не нужна, тем более -- как бы с противоположной стороны -- у
него нет "антропологии". Конечно, домашнее животное знает человека, собака -- своего хозяина, дикий зверь --своего врага. Но подобное знание инакового сущего не составляет момента самопознания. Антропология -- не
какая-то случайная наука в длинном ряду прочих человеческих наук. Никогда мы не становимся для себя
"темой", предметом обсуждения, как природное вещество, безжизненная материя, растительное и животное  
царства. Человек действительно бесконечно интересуется собой и именно ради себя исследует предметный мир.
Всякое познание вещей в конечном счете -- ради самопознания. Все обращенные вовне науки укоренены в
антропологическом  интересе  человека  к  самому  себе.  Субъект  всех  наук  ищет  в  антропологии  истинное
понимание самого себя, понимание себя как существа, которое понимает. Особое положение антропологии -- не
только  в  системе  наук,  которым  предается  человек,  но  и  в  совокупности  всех  человеческих  интересов  и
устремлений основывается на изначальной самоозабоченности человеческого существования. Труд есть явное
выражение подобной самозаботы; только потому, что в "теперь" человек предвидит "позже", в "сегодня" --"завтра", он может позаботиться, спланировать, потрудиться, принять на себя теперешние тяготы ради будущего
удовольствия. В сфере же господства, борьбы за власть людей над людьми возможно обеспечение будущего,
стабилизация  отношений  насилия  институционально  закрепленными  правовыми  отношениями.  Труд  и
господство свидетельствуют об отнесенной к будущему самозаботе человеческого бытия.
А как обстоит дело с игрой? Не является ли ее именно глубокая беззаботность, ее радостное, пребывающее в
себе настоящее, ее бесцельность и бесполезность, ее блаженное парение и удаленность от всех насущных
жизненных нужд тем, что придает ей волшебную силу, пленительное очарование и способность осчастливить?
Разве  игра  не  противоречит  тому,  что  мы  только  что  назвали  центральной  антропологической  структурой
человеческого  интереса  --  "заботой?"  Разве  теперь  не  могут  нам  возразить,  что  беззаботность  игры  есть
указание  на  то,  что  игра  изначально  есть  нечто  нечеловеческое,  что  она,  скорее,  принадлежит  к  еще  не
потревоженной,  не  нарушенной  никакой  рефлексией  животной  жизни  природного  создания,  что  человек
обладает естественной способностью к игре преимущественно лишь в детском возрасте, в состоянии, ближе
всего  находящемся  к  растительной  и  животной  природной  жизни,  что  он  все  больше  утрачивает
непринужденность игры, когда начинается серьезность жизни? Подобное возражение упустило бы из виду,
сколь  велико  отличие  человеческой  беззаботности  от  всякого  лишь  по  видимости  сходного  поведения
животного.  Животное  не  "заботится"  и  не  бывает  "беззаботным"  в  нашем  смысле  слова.  Лишь  сущее,  в
существе своем определенное "заботой", может также и быть "беззаботным". В строгом смысле животное -- ни
"свободно", ни "несвободно", ни "разумно", ни "неразумно". Лишь у человека есть возможность прожить жизнь
по-рабски и неразумно. Беззаботность игры по существу своему не имеет негативного характера, подобно
неразумию  или  рабскому  сознанию.  Здесь  как  раз  все  наоборот:  именно  бесполезная  игра  аутентична  и
подлинна, а не такая,  которая служит  каким-то  внеигровым  целям, как  то:  тренировка тела, установление
рекорда,  времяпрепровождение  как  средство  развлечься.  В  новейших  теориях  игры  сделана  попытка
представить игру как феномен, который присущ не только живому, но известным образом встречается повсюду.
Как утверждают сторонники этих теорий, отражения лунного света на волнующейся водной поверхности есть
игра света; череда облаков в небесах отбрасывает игру теней на леса и луга. Определенная замкнутость места
действия, движение, производимое на фоне ландшафтной декорации лунным светом, тенью от облаков и тому
подобным,  будто  бы  позволяют  предположить,  что  где-то  посреди  реального,  опытно  постигаемого  мира
является  некий  игровой  феномен,  "парящий"  над  реальными  вещами  в  качестве  прекрасной  эстетической
видимости. Игра есть прежде всего якобы свободно парящий эпифеномен, прекрасное сияние, скольжение
теней. Подобные игры можно обнаружить во всем просторе открывающейся нам природы. Это нечто вроде
эстетического  творчества  природы,  и  тогда  с  полным  правом  можно  говорить,  например,  об  игре  волн;
оказывается, что это вовсе не человеческая метафора, не перенос человеческих отношений на явления природы.
Напротив, природа играет в самом изначальном смысле, а игры природных созданий, животных и людей,
производны. На первый взгляд в этом утверждении содержится нечто подкупающее. Можно увязать его с
красочной, образной повседневной речью, которая постоянно подхватывает игровую модель, чтобы выразить в
языке по-человечески переживаемую, трогающую нас своей красотой и очарованием природу. Игра выводится
из теснины
только-человеческого  явления  в  качестве  оптического  события  огромного  диапазона.  Очевидно,  подобные
"игры", которым не нужен никакой игрок-человек, возможны повсюду: человек, в крайнем случае, может быть
вовлечен  в  такую  игру.  Итак,  человеческие  игры  представляются  частными  случаями  всеобщей,
распространенной на всю природу "игры".
Нам  кажется,  что  такое  понимание  игры  неправильно.  Здесь  основанием  анализа  делается  определенное
эстетическое или даже эстетизирующее отношение к природе, но это основание остается в тени и явно не
признается. Световые эффекты и скользящие тени столь же реальны, как и вещи, которые они освещают или
затемняют. Природные  вещи  окружающего нас мира  всегда выступают при  определенных обстоятельствах
своего об-стояния: на рассвете, под бросающим тень облачным небом, в сумеречной ночной тьме, полной
лунного сияния. И каждая вещь на берегу водоема бросает свое зеркальное отражение на поверхность воды. Так
что так называемые игры света и тени -- не более чем лирическое описание тех способов, какими даны нам
вещи окружающего мира. Естественно, мы не случайно используем подобные "метафоры", говорим об игре
волн  или  об  игре  световых  бликов  на  водной  зыби.  Однако  не  сама  природа  играет,  поскольку  она  есть
непосредственный феномен, а мы сами, по существу своему игроки, усматриваем в природе игровые черты, мы
используем  понятие  игры  в  переносном  смысле,  чтобы  приветствовать  вихрь  прекрасного  и  кажущегося
произвольным  танца  света  на  волнующейся  водной  поверхности.  На  деле  танец  света  на  тысячегранных
гребешках  волн  никогда  не  "произволен",  никогда  не  свободен,  никогда  он  не  бывает  исходящим  из  себя
творческим движением. Нерушимо и недвусмысленно здесь властвуют оптические законы. Световые эффекты
--  "игра"  в  столь  же  малой  степени,  в  какой  гребешки  волн,  с  их  ломающимися  пенными  гребнями,  --  
белогривые кони Посейдона. Поэтические метафоры здесь с наивным правом может использовать грезящая,
погруженная в прекрасную видимость душа -- но не человек, который мыслит, постигает и занимается наукой
или который занят выработкой философского понятия игры. Мы не хотели этим сказать, что не может и не
должно  быть  осмысленного  переноса  идеи  игры  на  внечеловеческое  сущее.  Там,  где  метафорическое  или
символическое  понимание  игры  оказывается  шире  человеческой  сферы,  необходимо  просто  критически
выверить и разъяснить оправданность, смысл и пределы подобного перехода границ. Но ни в коем случае не
следует  отдаваться  полупоэтической  манере  эстетизирующего  созерцания  природы.  Проблема
"антропоморфизма" столь же стара, как и стремление европейской метафизики выработать онтологические и
космологические понятия. То понимание бытия и мира, которого мы можем достичь, всегда и неизбежно будет
человеческим, то есть пониманием бытия и мира конечным созданием, которое рождается, любит, зачинает и
рожает, которое трудится и борется, играет и умирает. Элеат Парменид сделал попытку помыслить бытие в
чистом  виде,  исходя  из  него  самого,  а  с  другой  стороны,  представить  понимание  бытия  человеком  как
ничтожное и иллюзорное: он попытался мысленно взглянуть глазами бога. Но его мышление осталось вместе с
тем связанным с неким путем, hodos dizesios (фрагм. 2), путем исследования. То же можно сказать и о Гегеле,
который  переосмыслил  путь  человеческого  мышления  в  путь  бытия,  самопознающего  себя  в  человеке  и
благодаря человеку. Антропоморфизм не преодолевается просто отказом от наивного языка образов и заменой
его строгими понятиями. Наша голова, мыслящий мозг не менее человечны, чем наши органы чувств. Для
проблемы  игры  из  этого  следует,  что  игра  есть  онтологическая  структура  человека  и  путь  человеческой
онтологии. Содержащиеся здесь соотношения между игрой и пониманием бытия могут быть замечены лишь
тогда, когда феномен человеческой игры будет достаточным образом разъяснен в своей структуре. Наш анализ с
самого начала отказался от того поэтизирующего способа рассмотрения, который надеется обнаружить феномен
игры  повсюду,  где  вольная,  наивная  в  своем  антропоморфизме  речь  метафорически  говорит  об  "игре"  --например, о серебристых лунных лучах на волнуемой ветром морской поверхности. По видимости "более
широкое" понятие игры, включающее в себя "игры" луны, воды и света наравне с играми колышащейся нивы,
детеныша животного или человеческого ребенка, а то и вовсе -- ангела и бога, в действительности не дает
ничего, кроме эстетического впечатления, впечатления витающей необязательности, прекрасного, произвола и
сценической замкнутости. Мы настаиваем на том, что игра в основе своей определяется печатью человеческого
смысла. Выше мы упомянули в нашем изложении моменты протекания игры: настроение удовольствия, которое
может охватывать и свою противоположность -- печаль, страдание, отчаяние, пребывающее в себе "чистое
настоящее", не перебиваемое футуризмом нашей повседневной жизни; затем мы перешли к разъяснению правил
игры как самополагания и самоограничения игроков и коснулись коммуникативного характера человеческой
игры, играния-друг-с-другом, игрового сообщества, чтобы наконец наметить тонкое различие между "средством
игры"  и "игрушкой". Особо  важным нам представляется различение  внешней перспективы чужой игры,  в
которой  зритель  не  принимает  участия,  и  внутренней  перспективы,  в  которой  игра  является  игроку.
Деятельность игрока есть необычное производство -- производство "видимости", воображаемое созидание и все
же не ничто, а, скорее, порождение какой-то нереальности, которая обладает чарующей и пленяющей силой и не
противостоит  игроку,  но  втягивает  его  в  себя.  Понятие  "игрок"  столь  же  двусмысленно,  как  и  понятие
"игрушка". Подобно тому как игрушка является реальной вещью в реальном мире и одновременно вещью в
воображаемом мире видимости с действующими только в нем правилами, так и игрок есть человек, который
играет, и одновременно человек согласно его игровой "роли". Играющие словно погружаются в свои роли,
"исчезают" в них и скрывают за разыгранным поведением свое играющее поведение.
"Игровой мир" -- ключевое понятие для истолкования всякой
игры-представления. Этот игровой мир не заключен внутри самих людей и не является полностью независимым
от их душевной жизни, подобно реальному миру плотно примыкающих друг к другу в пространстве вещей.
Игровой  мир  --  не  снаружи  и  не  внутри,  он  столь  же  вовне,  в  качестве  ограниченного  воображаемого
пространства, границы которого знают и соблюдают объединившиеся игроки, сколь и внутри: в представлениях,
помыслах  и  фантазиях  самих  играющих.  Крайне  сложно  определить  местоположение  "игрового  мира".
Феномен, с которым легко обращается даже ребенок, оказывается почти невозможно зафиксировать в понятии.
Маленькая девочка, играющая со своей куклой, уверенно и со знанием дела
движется по переходам из одного "мира" в другой, она без труда снует из воображаемого мира в реальный и
обратно  и  даже  может  одновременно  находиться  в  обоих  мирах.  Она  не  становится  жертвой  обмана  или
самообмана, она знает о кукле как игрушке и одновременно об игровых ролях куклы и себя самой. Игровой мир
не существует нигде и никогда, однако он занимает в реальном пространстве особое игровое пространство, а в
реальном времени -- особое игровое время. Эти двойные пространство и время не обязательно перекрываются
одно другим: один час "игры" может охватывать всю жизнь. Игровой мир обладает собственным имманентным
настоящим. Играющее Я и Я игрового мира должны различаться, хотя и составляют одно и то же лицо. Это
тождество есть предпосылка для различения реальной личности и ее "роли". Определенная аналогия между
игрой и картиной поможет нам несколько это прояснить. Когда мы рассматриваем предметное изображение,
представляющее какую-то вещь, мы совершенно свободно можем различить: висящая на стене картина состоит
из  холста,  красок и рамки,  а также изображенного  на ней пейзажа.  Мы одновременно  видим реальные  и
представленные на картине  вещи. Краска холста  не  заслоняет от  нас цвет неба  в  изображенном пейзаже,
напротив: сквозь цвет холста мы просматриваем краски вещей, изображенных на картине. Мы может также
различить реальные краски и представленный ими цвет, место в пространстве и размеры единого предмета 
"картина" и пространственность внутри картины, изображенную на ней величину вещей. Стоя у изображающей
пейзаж картины, мы словно смотрим на простор за окном -- сходно с этим, но все же не точно так же. Картина
позволяет нам заглянуть в "образный мир", мы вглядываемся через узко ограниченный кусок пространства,
охваченный рамой, в некий "пейзаж", но при этом знаем, что он не раскинулся за стеной комнаты, что действие
картины сходно с действием окна, но на деле не является таковым. Окно позволяет выглянуть из замкнутого
пространства на простор, картина -- вглядеться в "образный мир", который мы видим фрагментарно. Свободное
пространство за окном переходит в пространство комнаты не прерываясь. Напротив, пространство комнаты не
переходит  непрерывно  в  пейзажное  пространство  картины,  оно  определяет  только  то,  что  есть  в  картине
"реального":  изрисованное  полотно.  Пространство  образного  мира  --  не  часть  реального  пространства,  в
котором занимает какое-то определенное место и картина как вещь. Находясь в каком-то определенном месте
реального пространства, мы вглядываемся в "нереальное" пространство пейзажа, принадлежащего к образному
миру. Изображение нереального пространства использует пространство реальное, но они не совпадают. Важно
не  то,  на  чем  основывается  иллюзорное,  лучше  --  воображаемое,  явление  пейзажа  образного  мира,
действительно  ли  и  каким  образом  осознанные  и  освоенные  иллюзионистические  эффекты  стали
использоваться как художественные средства искусства для создания идеальной видимости. В нашем контексте
важно отметить ту свободу и легкость, с какой мы принимаем различение картины и изображенного на ней
"образного мира" (не употребляя никаких понятийных различений). Мы не смешиваем две области: область
реальных вещей и область вещей внутри картины. Если же случится подобное смешение, то мы вовсе не
заметим никакой "картины". Восприятие картины (не касаясь здесь художественных проблем) относится к
объективно наличной "видимости", представляющей собой медиум, тот, в котором мы видим пейзаж образного
мира. В самом образном мире -- опять же реальность, но не та, в которой мы живем, страдаем и действуем, не
подлинная реальность, а как бы "реальность". Мы можем представить себя внутри пейзажа образного мира и
людей, для которых образный мир означал бы их "реальное окружение"; они оказались бы субъектами внутри
мирообраза данного образного мира, мы же -- субъектами восприятия изображения. Мы находимся в иной
ситуации, чем изображенные на картине люди. Мы одновременно видим картину и видим внутри картины,
находимся в реальном сосуществовании с другими наблюдателями картины и в как-бы-сосуществовании с
лицами внутри образного мира. Положение дел, однако, еще более запутано следующими обстоятельствами.
Поскольку всякое изображение, отвлекаясь от присущего ему "образного мира", нуждается также и в реальных
носителях изображения (полотно, краски, зеркальные эффекты и т.д.) и оказывается в этом отношении частью
простой  реальности,  картина  снова  может  быть  изображена  на  другой  картине,  и  так  мы  сталкиваемся  с
повторениями (итерацией) образов. Например, картина изображает "интерьер", культивированное внутреннее
пространство с зеркалами и картинами на стенах. Тогда декорация образного мира относится к имеющимся
внутри него картинам как наша реальность -- ко всей картине как таковой. Модификация "как бы" образной
"видимости" может быть воспроизведена -- нам легко представить себе картины внутри картин образного мира.
Но разгадать итерационные отношения не так-то легко. Лишь в воображаемом медиуме образной видимости
кажется возможным сколь угодно частое воспроизведение отношения реальности к образному миру; в строгом
смысле, образность высшего порядка ничего не прибавляет к воображаемому характеру картины. Изображение
внутри  изображения  не  является  более  воображаемым,  чем  исходное  изображение.  Усиление,  которое  мы,
наверное, интенционально понимаем, само есть только "видимость". Указание на эти сложные соотношения в
картине, которые, правда, всегда известны нам, но едва ли могут быть изложены с понятийной строгостью,
послужит  путеводной  нитью  для  структуро-аналитического  понимания  игры.  В  игре  мы  производим
воображаемый игровой мир. Реальными поступками, которые, однако, пронизаны магическим действием и
смысловой  мощью  фантазии,  мы  создаем  в  игровом  сообществе  с  другими  (иногда  в  воображаемом
сосуществовании с воображаемыми партнерами) ограниченный игровыми правилами и смыслом представления
мир игры. И мы не остаемся перед ним как созерцатели картины, но сами входим в него и берем внутри этого
игрового мира определенную роль. Роль может переживаться с различной интенсивностью. Есть такие игры, в
которых  человек  до  известной  степени  теряет  себя,  идентифицирует  себя  со  своей  ролью  почти  до
неразличимости, погружается в свою роль и ускользает от самого себя. Но подобные погружения нестабильны.
Всякой игре приходит конец, и мы просыпаемся от пленившего нас сна. А есть игры, в которых играющий
обращается со своей ролью суверенно легко, наслаждается своей свободой в сознании, что в любой момент он
может отказаться от роли. Игру можно играть с глубокой, почти неосознаваемой творческой активностью, а
можно -- с порхающей легкостью и грациозной элегантностью. Игровое представление не охватывает одних
только играющих, закуклившихся в свои роли: оно соотнесено и со зрителями, игровым сообществом, для
которого поднят занавес. Об этом ясно свидетельствует зрелищная игра. Зрители здесь не случайные свидетели
чужой игры, они небезучастны, к ним с самого начала обращена игра, она дает им что-то понять, завлекает в
сети своих чар. Даже не действуя, зрители оказываются околдованными. Представление в его традиционной
форме, с окружающей его декорацией, подобно картине. Зрители видят раскрывающийся перед ними игровой
мир. Пространство, в котором они себя ощущают, не переходит в сценическое пространство -- или же переходит
только  в  пространство  сцены,  поскольку  оно  есть  все  же  лишь  игровой  реквизит,  а  не  дорога  в  Колон.
Пространство игрового мира использует реальное место, действие игрового мира -- реальное время, и все же
его невозможно определить и датировать в системе координат реальности. Раскрытая сцена -- словно окно в
воображаемый  мир.  И  этот  необычный  мир,  открывающийся  в  игре,  не  только  противостоит  привычной
реальности,  но  обладает  возможностью  воспроизвести  внутри  себя  это  противостояние  и  свой  контраст  с 
реальностью. Подобно картинам в картинах существуют и игры в играх. И здесь итерация многоступенчата по
интенции, но удерживается в одном и том же медиуме видимости игрового мира. По своему воображаемому
содержанию игра третьей ступени не более воображаема, чем игра второй или первой ступени. И все же такая
итерация не лишена значения. Когда долго колебавшийся принц Датский велит поставить внутри игрового мира
еще одну игру, изображающую цареубийство, и этим разоблачающим представлением ставит в безвыходное
положение причастную к убийству мать и ее любовника, то при этом игровое сообщество взирает в игре на
другое игровое сообщество, становится свидетелем ужасной завороженности -- и само подпадает под власть
колдовских чар.
Двоякое самопонимание человеческой игры: непосредственность жизни и рефлексия
Игра принадлежит к элементарным экзистенциальным актам человека, которые знакомы и самому неразвитому
самосознанию и, стало быть, всегда находятся в поле сознания. Игра неизменно ведет с собой самотолкование.
Играющий человек понимает себя и участвующих в игре других только внутри общего игрового действа; ему
известна разрешающая, облегчающая и освобождающая сила игры, но также и ее колдовская, зачаровывающая
сила. Игра похищает нас из-под власти привычной и будничной "серьезности жизни", проявляющейся прежде
всего в суровости и тягости труда, в борьбе за власть: это похищение порой возвращает нас к еще более
глубокой серьезности, к бездонно-радостной, трагикомической серьезности, в которой мы созерцаем бытие
словно  в  зеркале.  Хотя  человеческой  игре  неизменно  присуще  двоякое  самопонимание,  при  котором
"серьезность" и "игра" кажутся
противоположностями  и  в  качестве  таковых  вновь  снимают  себя,  играющий  человек  не  интересуется
мыслительным самопониманием,  понятийным расчленением своего окрыленного,  упоительно настроенного
действования. Игра любит маску, закутывание, маскарад, "непрямое сообщение", двусмысленно-таинственное:
она бросает завесу между собой и точным понятием, не выказывает себя в недвусмысленных структурах, каком-то  одном простом  облике.  Привольная переполненность  жизнью,  радость от воссоединения противоречий,
наслаждение  печалью,  сознательное  наслаждение  бессознательным,  чувство  произвольности,  самоотдача
поднимающимся из темной сердцевины жизни импульсам, творческая деятельность, которая есть блаженное
настоящее, не приносящее себя в жертву далекому будущему, -- все это черты человеческой игры, упорно
сопротивляющиеся с самого начала всякому мыслительному подходу. Выразить игру в понятии -- разве это не
противоречие  само  по  себе,  невозможная  затея,  которая  как  раз  усложняет  постановку  интересующей  нас
проблемы?  Разрушает  ли  здесь  рефлексия  феномен,  являющийся  чистой  непосредственностью  жизни?
Осмыслить игру  -- разве это  не все равно что грубыми пальцами схватить крыло бабочки?  Возможно.  В
напряженном  соотношении  игры  и  мышления  парадигматически  выражается  общее  противоречие  между
непосредственностью жизни и рефлексией, между в-себе-бытием и понятием, между экзистенцией и сознанием,
между мышлением и. бытием, и именно у того самого существа, которое существует в качестве понимающего
бытие  существа.  Игра  есть  такой  основной  экзистенциальный  феномен,  который,  вероятно,  более  всех
остальных отталкивает от себя понятие.
Но разве не относится это в еще большей степени к смерти? Человеческая смерть ускользает от понятия совсем
на иной  манер: она непостижима для нас,  как  конец  сущего,  которое  было  уверено  в  своем  бытии; уход
умирающего,  его  отход  из  здешнего,  из  пространства  и  времени  немыслимы.  Зафиксировать  пустоту  и
неопределенность царства мертвых, помыслить это "ничто" оказывается для человеческого духа предельной
негативностью,  поглощающей  всякую  определимость.  Но  смерть  --  это  именно  темная,  устрашающая,
пугающая сила, неумолимо выводящая человека перед самим собой, устраивающая ему очную ставку со всей
его судьбой, пробуждающая размышление и смущающие души вопросы. Из смерти, затрагивающей каждого,
рождается  философия  (хотя  не  только  и  не  исключительно  из  этого).  Страх  перед  смертью,  перед  этой
абсолютной  властительницей есть, по  существу, начало  мудрости.  Человек  будучи смертным,  нуждается в
философии. Это melete thanatou -- "забота о смерти", -- как звучит одно из величайших античных определений
философствования. И Аристотель, понявший источник мышления как удивление, изумление, thaumazein, -- и он
говорит о, том, что философия исходит из "меланхолии" -- не из болезненной тоски, но, наверное, из тоски
естества.  Человеческий  труд  изначально  открыт  для  понятия,  он  не  отгорожен  от  него  подобно  игре.  Он
направлен на самопрояснение, на рациональную ясность, его эффективность возрастает, когда он постигает
себя, методически рефлектирует над собой: познание и труд взаимно повышают свой уровень. Пусть античная
theoria  и  основывается,  по-видимому,  на  каком-то  ином  опыте,  и  развивается  в  первую  очередь  в  рамках
внетрудового досуга: наука нового времени своей прагматической структурой указывает все же на тесную связь
труда и познания. Не случайно метафоры, характеризующие познавательный процесс, взяты из области труда и
борьбы, не случйно мы говорим о "работе понятия", о борьбе человеческого духа с потаенностью сущего.
Познание  и  постижение  того,  что  есть,  часто  понимается  как  духовная  обработка  вещей,  как  ломающий
сопротивление натиск, и философию зовут гигантомахией, борьбой гигантов. "Прежде потаенная и замкнутая
сущность вселенной, -- говорит Гегель в гейдельбергском "Введении в историю философии", -- не имеет сил,
чтобы  оказать  сопротивление  дерзновению  познания;  она  должна  раскрыться  перед  ним,  показать  свои
богатства и глубины, предоставив ему все это для его наслаждения" [7]. Наверное, больше нельзя разделять этот
триумфальный пафос, осмысленный лишь на почве абсолютной философии тождества. Но едва ли есть повод
отрицать  близость  познания  и борьбы  или  же  познания  и труда.  Напротив,  когда  понятийные объяснения
характеризуются  метафорой  "игры",  когда  говорят  об  игре  понятия,  это  воспринимается  почти  как 
девальвирующее  возражение.  Ненависть  к  "произвольному",  "необязательному"  и  "несерьезному"  крепко
связана  с  игривым  или  наигранным  мышлением:  это  некая  бессмысленная  радость  от  дистинкций,
псевдопроблем,  пустых  упражнений  в  остроумии.  Мышление  считается  слишком  серьезной  вещью,  чтобы
можно было  допустить его  сравнение  с  игрой. Игра якобы изначально не  расположена  к  мышлению,  она
избегает понятия, она теряет свою непринужденность, свою нерушимую импульсивность, свою радостную
невинность, когда педант и буквоед хотят набросить на нее сеть понятий.
Пока человек играет, он не мыслит, а пока он мыслит, он не играет. Таково расхожее мнение о соотношении
игры и мышления. Конечно, в нем содержится нечто истинное. Оно высказывает частичную истину. Игра чужда
понятию, поскольку сама по себе не настаивает на каком-то структурном самопонимании. Но она никоим
образом не чужда пониманию вообще. Напротив. Она означает и позволяет означать: она представляет. Игра как
представление  есть  преимущественно  оповещение.  Этот  структурный  момент  оповещения  трудно
зафиксировать  и  точно  определить.  Всякое  представление  содержит  некий  "смысл",  который  должен  быть
"возвещен". Соотношение между игрой и смыслом отличается от соотношения словесного звучания и значения.
Игровой смысл не есть нечто отличное от игры: игра -- не средство, не орудие, не повод для выражения смысла.
Она сама есть собственный смысл. Игра осмысленна в себе самой и через себя самое. Играющие движутся в
смысловой атмосфере своей игры. Но, может быть, смысл имеет место лишь для них?
Понятие зрителей игры двояко, и его следует различать. Иногда подразумевается безучастный, равнодушный
зритель,  который  воспринимает  игровое  поведение  Других,  понимает,  что  эти  другие  "играют",  то  есть
разряжаются,  отдыхают,  рассеиваются,  развлекаются  прекрасным  времяпрепровождением.  Мы  видим
играющих на улице детей, картежников в трактире, спортсменов на спортивной арене. Мы наблюдаем это
издалека, нас разделяет дистанция равнодушия, мы не принимаем в этом никакого участия. Игра -- известная
нам по своему типу форма поведения, одна среди прочих. Как безучастные зрители мы наблюдаем также за
трудом других людей, их политическими действиями, за прогулкой влюбленных. Мы можем неожиданно стать
свидетелями какого-нибудь несчастного случая, чужой смерти. Все подобные свидетельства мы совершаем
"проходя мимо". Прохождение-мимо вообще есть преимущественный способ человеческого сосуществования.
Мы высказываем это без примеси сожаления или обвинения. Совершенно невозможно вести диалог со всеми
людьми окружающего нас мира, поддерживать с ними подлинные отношения интенсивного сосуществования.
Люди, с которыми мы можем действительно сосуществовать в позитивном сообществе, -- это всегда лишь узкий
круг близких, приверженцев, друзей. Эмфатическое "обнимитесь, миллионы, в поцелуе слейся, свет" можно
представить  себе  лишь  в  предельно  абстрактной  всеобщности,  в  качестве  универсальной  филантропии.
Изначальные,  подлинные  переживания  общности  всегда  избирательны.  Люди,  которые  нас  затрагивают,  к
которым привязано наше сердце, выделены из огромной безликой массы безразличных нам людей. Неверно,
когда социальный мир представляют в виде концентрических колец, полагая внутри наиболее узкого кольца
жизненное соединение мужчины и женщины и располагая вокруг семью, родственников, друзей, друзей друзей
и т.д.  вплоть  до чужих и безразличных нам  людей.  Чужие,  мимо которых "проходят", образуют  столь же
элементарную  структуру  совместного  бытия,  как  и  сфера  интимности.  Последняя  сама  по  себе  вовсе  не
первичнее  окружающей  все  близкое  сферы  "чужого".  Человек,  становящийся  важным  для  другого  --  как
возлюбленный, как дитя, как друг, -- всегда находится "на людях". "Люди", то есть открытое, неопределенное
множество сосуществующих других, которыми мы не интересуемся, наблюдая их лишь со стороны, образуют
непременный  фон  всей  нашей  избирательной  коммуникации.  Безучастно  и  равнодушно  созерцаем  мы  не
трогающую нас игру чужих людей: именно "проходя мимо".
Совсем другое дело, когда зрители принадлежат к игровому сообществу. Игровое сообщество охватывает как
играющих, так и заинтересованных, соучаствующих и затронутых их игрой свидетелей. Играющие -- "внутри"
игрового мира, зрители -- "перед" ним. Мы уже говорили о том, что структурные особенности представления
особенно  хорошо  проясняются  на  примере  зрелища  (Schau-Spiel).  Участвующие  здесь  зрители  созерцают
чужую игру не просто так, "проходя мимо", но соучаствуя в игровом сообществе и ориентируясь на то, что
возвещает им игра. К игре они относятся иначе, нежели играющие-актеры, у них нет роли в игре, воображаемые
маски не утаивают и не скрывают их. Они созерцают игру-маскарад ролей, но при этом сами они не являются
персонажами игрового мира, они смотрят в него как бы извне и видят видимость игрового мира "перед собой".
Вместе с тем они не впадают в обманчивое заблуждение, не путают события в игровом мире с происшествиями
в реальной действительности. Само собой разумеется, игровые действия всегда остаются также и реальными
происшествиями,  однако  происшествия  играния  не  идентичны  с  событиями,  сыгранными  или,  лучше,
разыгранными на показных, воображаемых подмостках игрового мира. Зрители знают, что на подмостках --реальный актер, но видят его именно "в этой роли". Они знают: падет занавес, и он смоет грим, отложит в
сторону реквизит, снимет маску и из героя обратится в простого гражданина. Иллюзорный характер игрового
явления известен. Это знание, однако, не есть самое существенное. Не бутафория субстанция игры. И актер
может различать себя как актера от того человека, каким он является в показном медиуме игрового мира, он не
заблуждается относительно самого себя и не станет обманывать зрителей -- не станет, как порой неверно
утверждают, пробуждать в них обманчивую иллюзию и стараться, чтобы они приняли представленный им
"театр" за чистую действительность. Величие актера не в том, чтобы зрители были ослеплены и предположили
себя  свидетелями  исключительных  событий  в  будничной  действительности.  Имагинативный,  подчеркнуто
показной характер игрового мира вовсе не должен исчезнуть для игрового сообщества зрителей, они не должны
быть застигнуты врасплох мощным гипнотизирующим, ослепляющим воздействием и утратить сознание того, 
что они присутствуют при игре. Возвещающая сила представления достигает кульминации не тогда, когда
игровое  событие  смешивается  с  повседневной  действительностью,  когда  страх  и  сострадание  так  же
захватывают зрителей, как это может случиться, окажись мы свидетелями какого-нибудь ужасного дорожного
происшествия. При виде подобной катастрофы, затрагивающей чужих нам людей, мы ощущаем в себе порывы,
общечеловеческого чувства солидарности, мы жалеем несчастных, страшимся опасности, столь впечатляюще
продемонстрированной катастрофой, -- опасности, которая подстерегает всех и когда-нибудь может настигнуть
и нас. Мы не идентифицируем себя с потерпевшими, настаиваем на чуждости нам жертв катастрофы. Элемент
"показного", ирреального и воображаемого, характеризующий игровой мир, еще более, очевидно, отстраняет
деяния и страдания выступающих в игре фигур: ведь они только "выдуманы", вымышлены, сочинены, просто
сыграны. Это "как бы" деяния и "как бы" страдания, совершенно лишенные весомости и принадлежащие к
воздушному  царству  парящих  в  эфире  образов  фантазии.  Когда  же  речь  заходит  о  воспроизведении  этой
модификации "как бы", об играх внутри игр, о грезах внутри грез, тогда мы, как зрители уж вовсе не уязвимы
для свирепствующей в игровом мире судьбы: с надежного расстояния мы наслаждаемся зрелищем счастья и
гибели Эдипа. Надежно ли наше построение? Что же тогда столь сильно трогает, потрясает и поражает зрителей
в самое сердце? На чем основывается чарующая, волшебная сила представления, почему мы смотрим затаив
дыхание, как представление цареубийства -- игра в игре -- накладывает в шекспировском "Гамлете" чары на
игровое сообщество, которое само принадлежит к игровому миру, почему и сами мы оказываемся под властью
этих чар? Ведь считается, что игра есть нечто "нереальное", но при этом не имеется в виду, конечно, что игра
как играние не существует вовсе. Скорее, именно потому, что играние как таковое существует, разыгранного им
игрового мира нет. Нет в той простой реальности, где развертываются игровые действия, но игровой мир --вовсе не "ничто", не иллюзорный образ, он обладает приданным ему содержанием, это сценарий со множеством
ролей. Нереальность игрового мира есть предпосылка для того, чтобы в нем мог сказаться некий "смысл",
затягивающий нечто такое, что "реальнее" так называемых фактов. Чтобы сохранить выглядывающий из-за
фактов  смысл,  игровой  мир  должен  казаться  "ничтожнее"  фактов.  В  нереальности  игры  выявляется
сверхреальность сущности. Игра-представление нацелена на возвещение сущности. Мы слишком привыкли
полагать отнесенность к сущности преимущественно в мышлении, в качестве мыслительного отношения к
"идее", ко "всеобщему" как инвариантной структуре, виду, роду и т.д. Единичное лишь несовершенным образом
участвует во всеобщности сущности, единичное связано с идеей через участие, methexis, participatio, отделено
от непреходящей и постоянной идеи своей бренностью. Фактическая вещь относится к сущности как экземпляр
-- к виду или роду. Все совершенно иначе в сфере игры, особенно представления. Здесь налицо следующая
проблема: каким образом в определенных ролях выявляются сущностные возможности человечности. Всегда
перед  нами  здесь  "этот"  человек, который действует и страдает, утверждает себя и погибает, сущностный
человек в своем бессилии перед судьбой, опутанный виной и страданием, со своими надеждами и со своей
погибелью.  Нереальность  сплетенного  из  видимости  игрового  мира  есть  дереализация,  отрицание
определенных единичных случаев, и в то же время -- котурн, на котором получает репрезентацию фигура
игрового мира. В своем возвещении игра символична. Структура понятийного высказывания не пронизывает ее,
она не оперирует с различением единичного и всеобщего, но возвещает в символе, в совпадении единичного и
всеобщего, возвещает парадигматической фигурой, которая "не-реальна", потому что не подразумевает какого-то определенного реального индивида, и "сверх-реальна", потому что имеет в виду сущностное и возможное в
каждом. Смысл представления, одновременно воображаемый и сущностный, нереален и сверхреален. Зритель
игрового сообщества становится свидетелем события, которое не произошло в повседневной действительности,
которое кажется удаленным в некую утопию и все же открыто для зрящего -- то, что он видит в игровом
медиуме, не есть какой-то произвольный вымысел, который затрагивает чужих ему людей и, по существу, не
может коснуться его самого. Свидетель представления, который действительно вовлечен в игровое сообщество,
а  не  просто  "проходит  мимо"  него,  не  может  больше  делать  расхожего  различия  между  собой  и  своими
близкими,  с  одной  стороны,  и  безразличными  ему  другими  --  с  другой.  Нет  больше  противопоставления
человека и людей. Зрящий познает, узревает сущностно бесчеловечное -- его потрясает понимание того, что он
сам в своей сущностной глубине идентичен с чужими персонажами, что разбитый горем сын Лайя, несущий
бремя проклятия Орест, безумный Аякс -- все в нем, в качестве жутких, страшных и страшащих возможностей.
Страх и сострадание лишены здесь какой бы то ни было рефлексивной структуры, указывающей на примере
чужого страдания сходную угрозу собственному бытию и, таким образом, подводящей отдельного человека под
всеобщее понятие. Сострадание и страх обычно считаются
разнонаправленными порывами души: можно сказать, что один нацелен на других людей, второй есть забота о
собственном устоянии. Потрясение, вызываемое трагической игрой, в известной мере снимает различие между
мною  как  отдельным  существом  и  другими  людьми  --  столь  же  обособленными  экзистенциями.  Страх
оказывается  не  заботой  о  моем  эмпирическом,  реальном,  находящемся  под  угрозой  Я,  но  заботой  о
человеческом существе, угрозу которому мы видим в зеркале зрелища. И страдание обращено не вовне, к
другим, но уводит внутрь, туда, где всякий индивид соприкасается с до-индивидуальной основой. Если мы
хотим  получить  удовольствие  от  парадоксального  способа  выражения,  можно  сказать  --  в  противовес
аристотелевскому учению о том, что действие трагедии основано на эффектах "страха" и "сострадания" и их
катарсиса, -- что это верно, лишь когда учитывают обращение обоих аффектов, их структурное изменение: оба
меняют, так сказать, свое интенциональное направление, страх принимает структуру, прежде принадлежавшую
состраданию, и наоборот. Метафизика искусства европейской традиции отправлялась от платоновской борьбы 
против поэтов и его истолкования игры, а также от аристотелевской поэтики. Вновь обратить эту традицию в
открытую проблему, поставить под вопрос верность проторенных путей, критически перепроверить чрезмерное
сближение  понимания  игры  с  пониманием,  присущим  понятийному  мышлению,  поддержать  различение
символа и эйдетического понятия -- все это неотложные вопросы, стоящие перед философской антропологией
при истолковании ею одной из сфер человеческой жизни -- игры. Естественно, подобная задача не может быть
решена достаточно полно в узких рамках данной работы. Она лишь обозначена здесь как указание на более
широкий проблемный горизонт.
Указав  на  "страх"  и  "сострадание",  мы  попытались  пояснить  ситуацию  зрителя,  включенного  в  игровое
сообщество, не созерцающего игру с безучастным равнодушием и не "проходящего мимо" нее, захваченного и
затронутого миром игры. Зрелищу нужны отнюдь не только актеры и их роли, но и игровое сообщество.
Конечно, попытка охарактеризовать зрителя игры-представления лишь через взволнованную затронутость была
бы односторонней. Ведь есть совершенно иное поведение зрителей -- например, зрителей комедии и сатиры,
вызывающих  смех.  И  в  этом  случае  верно,  что  смеемся  мы  над  самими  собою,  не  над  эмпирическими
недостатками,  слабостями  и  дурачествами  тех  или  иных  индивидов,  но  над  слабостью  и  глупостью
человеческого существа. Комедия не менее символична, нежели трагедия. Она освобождает нас, устанавливая
ироническую дистанцию между человеком и человеческим. Шутка, юмор, ирония -- эти основные элементы
игровой веселеет" прокладывают путь временному освобождению человека в смеховом возвышении над самим
собой. Комедия снимает с нас бремя гнетущей кабалы труда, подчиненности чьему-то господству, любовной
страсти и мрачной тени смерти. Смех как смех-над-самим-собой свойственен лишь существу, существующему
как  конечная  свобода.  Ни  одно  животное  смеяться  не  может.  Бергсон  написал  знаменитое  сочинение  под
заглавием "Le Rire", где он представил смех сущностным отличием человека. Правда, греки верили, что их боги,
поскольку им не нужно трудиться для поддержания своей бессмертной жизни, либо -- по аналогии с человеком
-- проводят дни в веселой игре, как говорил Гомер, либо заняты непрестанным мышлением и управлением
миром, как говорили философы. Олимпийские боги изображались игроками, люди -- их игрушками, которыми
они распоряжались по своему усмотрению. И то, что для смертного было сокрушающей сердце трагедией,
бессмертному вполне могло показаться комедией. Наверное, их смех имел недобрый оттенок злорадства -- и
тогда,  когда  раскат  гомерического  хохота  потряс  Олимп,  когда  Гефест,  искусный  и  "рогатый"  бог  поймал
Афродиту в объятиях Ареса в нервущуюся сеть. Лишь исполненный достоинства неизменный бог метафизики
далек от смеха. Образу христианского бога также чужды смех, юмор, ирония, обращенная на себя самого,
совершенное существо не знает никакого смеха, никакого радостного игрового самоосвобождения.
Отсюда  вытекает  со  строгой  логической  последовательностью,  что  первый  истинно  безбожный  человек,
ницшевский Заратустра, упоенно славит смех: "Этот венец смеющегося, этот венок из роз: я сам возложил на
себя этот венец, я сам освятил свой хохот. Никого другого я не нашел сегодня достаточно сильным для этого" .
Игровое сообщество, принадлежащее к игровому представлению, еще недостаточно полно охарактеризовано в
своем отношении к игровому миру тем, что выделяется символическое понимание, использующее нереальность
сцены  в  качестве  условия  для  явления  сверхреальной  сущности.  Обстоятельства  человеческого  существа,
однако, совсем не так просты. Человек, как мы выяснили в ходе рассмотрения основных экзистенциальных
феноменов,  не  обладает  твердо  определенной  сущностью,  которая  затем  сопровождалась  бы  множеством
случайных обстоятельств: человек есть смертный и он есть трудящийся, борец, любящий и игрок. Эти сферы
жизни  никогда  не  изолированы  друг  от  друга,  ни  по  бытию,  ни  по  пониманию.  Труд  и  господство  в
бесчисленных формах переплетаются в истории человеческого рода, любовь и смерть смыкаются друг с другом,
как  мы  попытались  показать.  Игра  стоит  в  оппозиции  к  тем  феноменам  жизни,  которые  принято  считать
тягостной серьезностью жизни. Игра -- иная, она есть колеблющееся в элементе "нереального" активное и
импульсивное  общение  с  воображаемым,  туманным  царством  возможностей.  Игрой,  вполне  реальным
действием, мы создаем "нереальный" игровой мир и глубоко рады этому созданию, мы в восторге от его
фантастичности, которой, впрочем, меньше, чем пены, выбрасываемой на берег волнами. Хотя в ходе анализа
игры мы и объяснили, что для игрока, в игровом мире вещи "реальны", все же следует уточнить, что это
реальность в кавычках: ее не путают с подлинной реальностью. Как игрок в своей роли, так и зритель внутри
игрового сообщества -- оба знают о фиктивности реальности в игровом мире. Они сохраняют это знание, когда
речь идет уже об игре внутри игры, -- как сохраняется различие реальной вещи и "картины", когда в картинах в
свою  очередь  встречаются  картины  и  т.д.  Итерационные  возможности  игры  родственны  итерационным
структурам образности. Игра может воспроизводиться внутри игры. Попросту говоря, дети, играющие в самую
древнюю подражательную игру, выступающие в своем игровом мире "отцами" и "матерями", у которых есть
"дети",  вполне  могут  послать  этих  детей  из  игрового  мира  "поиграть"  на  улице,  пока  дома  не  будут
приготовлены куличики из песка. Уже на этом примере можно узнать нечто значительное. Игре известна не
простая  однородная  итерация,  повторение  игры  в  игре,  в  своей  воображаемой  зоне  представления  игра
объемлет и внеигровое поведение людей. Как один из пяти основных феноменов, игра охватывает не только
себя, но и четыре других феномена. Содержание нашего существования вновь обнаруживается в игре: играют в
смерть, похороны, поминовение мертвых, играют в любовь, борьбу, труд. Здесь мы имеем дело вовсе не с
какими-то искаженными, неподлинными формами данных феноменов человеческого бытия, их розыгрыш --вовсе не обманчивое действие, с помощью которого человек вводит других в заблуждение, притворяется, будто
на самом деле трудится, борется, любит. Эту неподлинную модификацию, лицемерную симуляцию подлинных
экзистенциальных актов, часто, но неправомерно, зовут "игрой". В столь же малой мере это игра, в какой ложь  
является  поэзией.  Ведь  произвольным  все  это  оказывается  только  для  обманывающих,  но  никак  не  для
обманутых.  В  игре  же  не  бывает  лживой  подтасовки  с  намерением  обмануть.  Игрок  и  зритель  игрового
представления знают о фиктивности игрового мира. Об игре в строгом смысле слова можно говорить лишь там,
где воображаемое осознано и открыто признано как таковое. Это не противоречит тому, что игроки иногда
попадают под чары собственной игры, перестают видеть реальность, в которой они играют и имагинативно
строят свой игровой мир. От погруженности в игру можно очнуться. Сыгранная борьба, сыгранный труд и т.д.--опять-таки весьма двусмысленные понятия. Иной раз имеют в виду притворную борьбу, притворный труд, в
другой  раз  --  подлинную  борьбу  и  подлинный  труд,  но  в  качестве  событий  внутри  игрового  мира.  У
человеческой игры нет каких-то иных возможностей для своего выражения, кроме жизненных сфер нашего
существования. Отношение игры к другим основным феноменам -- не просто соседство и соотнесенность, как в
случае с трудом и борьбой  или любовью  и  смертью.  .Игра  охватывает  и  объемлет все другие феномены,
представляет их в непривычном элементе воображаемого и тем самым дает человеческому бытию возможность
самопредставления и самосозерцания в зеркале чистой видимости. Следует еще поразмыслить над тем, что это
означает.
Всеобъемлющая структура человеческой игры.
Игра  объемлет  все.  Она  вершится  человеческим  действованием,  окрыленным  фантазией,  в  чудесном
промежуточном  пространстве  между  действительностью  и  возможностью,  реальностью  и  воображаемой
видимостью и представляет на учиненной ею идеальной сцене -- в себе самой -- все другие феномены бытия, да
вдобавок самое себя. Подобная всеобъемлющая структура необыкновенно сложна в своем интенциональном
строе  и  предполагает  не  только  сообразную  переживанию  классификацию  пережитых  игровых  миров
различных ступеней, но и взаимопроникновение "возможного и действительного", становящееся прямо-таки
проблемой  калькуляции.  К  сказанному  следует  еще  добавить,  что  игровые  элементы  присутствуют  во
множестве форм неигровой жизни, зачастую в виде маленьких увеселений в лощинах серьезного жизненного
ландшафта,  так  что  посреди  сурового,  мрачного  и  отягченного  страданиями  человеческого  существования
всплывают "острова" игрового блаженства. Что было бы с влюбленными с их поистине бесконечной задачей без
разыгранной шутки, без радостных сердечных арабесок? Чем была бы война без авантюры, без игровых правил
рыцарственности, чем был бы труд без игрового гения, чем была бы политическая сцена без добровольного или
недобровольного фарса властителей? Иногда выказываемая во всех этих сферах серьезность есть не более чем
хорошо сидящая маска скрытой игры. Именно потому, что игра способна менять облачения, ее присутствие не
всегда  легко  установить.  Порой  люди  застают  друг  друга  за  игровыми  действиями,  которые  совсем  и  не
выглядят  таковыми.  Феноменология  шутки  как  конституирующего  социальность  фактора  все  еще  не
разработана. От всякой игры, открытой и скрытой, как бы замаскированной, следует строго отличать лицемерие
с  целью  обмана,  подложную  "как  бы"  модификацию  чувств,  умонастроений  и  действий,  в  которой  люди
"представляются" друг перед другом, обманывают не только словами, но и образом поведения, поступками,
когда, например, "играют в любовь", не ощущая ее, когда, как говорится, устраивают "спектакль". Ложь, которая
может быть не только словесной, но и ложью жестов, мимики, даже "молчания", есть жуткая, зловещая тень,
ложащаяся  на  межчеловеческие  отношения  и  угрожающая  им.  Человеческая  ложь  --  это  не  мимикрия
животных,  но  хитрость,  притворство,  коварство  зверей  в  борьбе  за  добычу,  человеческая  ложь  лишена
невинности хищника. По сравнению с животным человек оснастил поведение, имеющееся уже в животном
мире,  средствами  своего  интеллекта,  когда  из  доисторического  собирателя  превратился  в  охотника,
преследующего дичь с помощью всевозможных уловок -- ям, приманок и тому подобного. Рафинированные
обманные  средства  охоты  затем  были  перенесены  в  сферу  охоты  на  людей  --  в  военную  борьбу  и  ее
продолжение в политической риторике, вторглись во взаимное приманивание и вечную войну полов. Охота как
область хитрости и обмана есть поведение, характеризующее животный и человеческий мир, это область своего
рода естественной лжи. Но человек отличается от животного среди прочего и тем, что понимает истину как
таковую, что он открыт смыслу и способен разделять понятый смысл с другими людьми. Межчеловеческое
сообщение -- будь то в мимике, жесте или слове -- есть нечто большее, чем сигнал, и существенно отличается от
предупреждающего свиста серны или же призывного рева оленя. Так как человек знает об истине, об истине
смыслового  и  о  смысле  истины,  он  знает  и  о  мучительной  потаенности  сущего,  может  спрашивать  и
высказываться,  знает  о  неистинности  как  проходящей  замкнутости  вещей,  о  неистинности  как  следствии
притворства, способность к которому он постигает как власть. Именно потому, что человек определяется через
свое отношение к истине, он обладает возможностью лжи. Это опасная, злая возможность -- злая не потому, что
непрестанно учиняет вред, отравляя ядом недоверия отношения между людьми, но потому, что делает само
отношение человека к истине двусмысленным, неверным и ненадежным. Притворное знание хуже незнания,
ложь хуже заблуждения. Легче примириться с тем, что -- как существо конечное -- мы можем познать лишь
весьма  ограниченный  круг  вещей,  чем  вынести  ложь  и  обман  со  стороны  других  людей.  Лицемерная
неподлинность в нагруженных смыслом словах и поступках часто зовется "игрой", а игра противопоставляется
подлинному и правдивому, истинному. Конечно, это неправильное толкование зла и злоупотребление понятиям
игры.  И  все  же  между  обманом  и  игрой  есть  связь.  Сама  игра  --  это  не  обман,  но  она  пользуется
иллюзионистскими эффектами, которыми обычно оперирует и обман, игра воспринимает элементы показного --не для того, чтобы выдать показное за реальность, а чтобы использовать его в качестве средства выражения.
Маски  в  игре  не  должны  вводить  в  заблуждение,  они  должны  зачаровывать,  это  --  реквизит  практики 
волшебства.  Игра  развертывается  внутри  условной  "видимости",  она  ее  не  отрицает,  но  и  не  выдает  за
неподдельно реальное. Всякая игра связана с иллюзорной, воображаемой "видимостью", но не затем, чтобы
обмануть,  а  с  целью  завоевать  измерение  магического.  Когда  в  игровом  мире  представления  "являются"
внеигровые феномены бытия, когда в игре борются, трудятся, любят, а то и вовсе умирают, это не значит, что
игра, с целью обмана, устроила неподлинный спектакль. Это как раз подлинный театр, подмостки зрелища,
выводящего человеческую жизнь перед ней самой. Игра -- исключительный способ для-себя-бытия. Это не для-себя-бытие, происходящее от рассудочной рефлексии, не сознательное обращение представляющей жизни на
себя самое. Ведь игра есть действие, практика общения с воображаемым. В человеческой игре наше бытие
действенно отражается в себе самом, мы показываем себе, чем и как мы являемся. Игровое для-себя-бытие
человека прагматично. Оно существенно отличается от чисто интеллектуального для-себя-бытия, идущего от
рефлексии. Игра принадлежит к элементарному, дорефлективному бытию. Однако она не "непосредственна".
Она обладает структурой "опосредствования", она проста, пока остается игрой, двойственна -- когда выступает
как действие в реальном мире и одновременно -- в мире игровом. Деятельный характер игрового действия,
выделяющий игру в сравнении с рефлексией сознания, очевидно, не дан в игровом сообществе. Не есть ли это
характерная черта "созерцания" в широком смысле слова? Как мы сказали, игровое сообщество включает в себя
игроков  и  свидетелей  игры,  ограниченную  сцену  игрового  мира  и  людей  перед  подмостками.  Последние
включены  в  игру  постольку,  поскольку  они  очарованы  игрой.  При  этом  они  не  "действуют"  сами,  они
погружены в созерцание, которое их захватывает или забавляет. Но "со-зерцатели" игрового представления
"идентифицируют"  себя  с  игроками  имагинативным  образом.  На  этой  идентификации,  наверное,  и
основывается в значительной степени зачаровывающая силы игры. В обесцвеченной и ослабленной форме этот
момент  "идентификации  зрителей  и  игроков"  сказывается  и  на  всяких  цирцеевских  представлениях,  на
развлекательных играх, устраиваемых для масс большого города. Наверное, неправильно насмехаться над тем,
что в современных футбольных состязаниях 22 человека бегают по полю, а сотни тысяч наблюдают за ними.
Оставив в стороне значение подобных грандиозных представлений, определяемое социологически (например,
как "содержание сознания" или "тема разговора" масс), укажем на то, что быть зрителем -- это само по себе есть
род  сильной  эмоциональной  причастности,  способ  идентифицирующего  соучастия  в  игре,  поднимающий
множество интересных  проблем.  Игровое  сообщество зрелища  объединено и собрано  в  общеколлективной
иллюзии, которую могут сознавать и признавать "нереальной" и в то же время понимать как место явления
"сверхреальной"  сущности.  Сцена  представляет,  подмостки  ее  --  весь  мир.  Прежде  всего  это,  наверное,
"человеческий мир", совокупность человеческого бытия, а сверх того, вероятно, и совокупность всего бытия, к
которому  человек  постоянно  себя  относит.  Зрелище,  по  существу,  есть  пример  (Bei-Spiel),  образец,
парадигматическое представление того, что мы есть и каковы мы. Этот имеющийся в человеческой игре образец
заключается в осмысленном представлении бытия во всех его жизненных измерениях для него самого. Владея
способностью играть, человек может созерцать себя, обретать образ собственной жизни во всей его высоте и
глубине, задолго до того, как он начинает размышлять и понятийно постигать истину своего существования.
Игровая рефлексия образна, игра выводит сущность в явление до всякого явного размышления. Человек как
человек по своему бытию есть отношение. Он не сходен с вещью, которая сначала есть в самой себе и только
потом  относится  к  чему-либо.  Категориальная  модель  "субстанции"  не  подходит  к  человеку.  Человек
существует как отношение, отношение к себе, вещам и миру, он существует в пространстве и относит себя к
родине и чужбине, существует во времени и относит себя к собственному прошлому, обусловлен родом и полом
и относит себя к собственному полу в стыде и институциональных отношениях (брак, семья). Культ умерших,
труд, господство, любовь -- все это ключевые способы самоотнесения человека. Игра же есть отношение к
относительному бытию человека. Все основные феномены бытия сплетены друг с другом, игра отражает в себе
их все, в том числе и саму себя. Это и придает игре исключительный статус. С представляющей способностью
игры  связано  и  то,  что  она  может  иначе  наполнять  время,  нежели  остальные  феномены  человеческого
существования.  Предстояние  смерти  бросает  тень  на  всякий  человеческий  поступок.  Мера  времени  нам
установлена, пусть даже мы и не знаем о ней. В свете последнего мгновения все часы, дни и годы как бы даны
нам взаймы. В труде и борьбе человеческое время заполнено нужными, необходимыми делами: время всегда у
нас отнимают. То же можно сказать и о любви, которая, наверное, есть самое трудное, менее всего осваиваемое
занятие нашего бытия, постоянно терпящее крушение.
Лишь у игры есть "свободное время". Слишком часто проблему игры и свободного времени рассматривали
поверхностно,  брали  преимущественно  как  проблему  заполнения  имеющегося  в  распоряжении  свободного
времени всевозможными играми, чтобы  оно  оказалось  осмысленным  и  дарило бы  счастье.  Играем ли мы
потому, что у нас есть свободное время, или же у нас есть свободное время как раз потому, что мы играем?
Такая формулировка проблемы -- не простое переворачивание. В том и другом случае слова "свободное время"
имеют разный смысл. В первом случае свободное время идентично с жизненным временем, не заполненным и
не блокированным неотложными задачами. Время нужно нам, чтобы спать, есть, производить средства к жизни,
поддерживать  общественные  установления  исполнением  самых  различных  служебных  обязанностей.
Свободное время -- тот промежуток, который "остается" после того, как мы сделали все насущные для жизни
дела. Треть 24-часового дня используется на сон, треть -- на труд, треть на жизнь, ради которой трудятся, то
есть на семью, исполнение гражданских обязанностей, развлечения и наслаждение жизнью, занятия любовью,
отдых  и  т.д.  Свободным  временем  человек  обладает  в  той  степени,  в  какой  он  освобожден  от  нужды
естественных  потребностей  и  давления  общественных  необходимостей.  Можно,  например,  по  своему 
усмотрению и произвольно распоряжаться свободным временем, можно, говоря вместе с ранним Марксом,
"охотиться,  рыбачить",  быть  "критическим  критиком",  человек  извлекается  из-под  гнета  необходимости  и
беспрепятственно  передвигается  в  "царстве  свободы".  Свободное  время  --  комплементарная  оппозиция
"рабочего времени" (или служебного). Когда рабочее время сокращается, свободное время расширяется. В
прежние времена большинство людей должно было трудиться большую часть времени своего бодрствования,
отдавая все силы для производства средств жизни для себя и
привилегированного  слоя  господ,  которые  не  "трудились",  а,  скорее,  "правили",  владели  землей,  оружием,
военной мощью. Вместе с
индустриализацией человеческая рабочая сила и прежде всего рабочее время высвобождаются за счет машин в
никогда  не  предполагавшихся  прежде  масштабах:  свободное  время  во  всех  индустриальных  странах
значительно  возрастает.  Возникли  совершенно  новые,  насущные  и  требующие  немедленного  решения
проблемы. Урбанизация сконцентрировала в разных местах земного шара огромные массы людей, чья занятость
в свободное время не формируется "естественно" врожденными склонностями и интересами, но, скорее, ее
необходимо организовывать, направлять, ею нужно руководить. В век техники и человеческое свободное время
приобретает технократический оттенок. Было бы интеллигентской спесью романтически-анахронистического
толка разделять людей на две категории: тех, которые сами могут распорядиться своим свободным временем,
потому что ими движут собственные и самостоятельные интересы, и тех, которые растерянно и беспомощно
стоят перед данным им свободным временем, не зная, что им предпринять, которые остаются
"несовершеннолетними" в формировании собственной жизни, которые нуждаются в руководстве, в поводе и
месте для развлечения. Значительный размах индустрии развлечений подверг наличную здесь потребность
инфляции нормированию и нивелировке. Гигантский аппарат современного технического оснащения жизни не
мог  бы  развиваться  и  функционировать,  если  бы  речь  шла  только  об  удовлетворении  элементарных
естественных потребностей в пище, одежде, жилье. Человеку нужно не столько "необходимое", сколько как раз
"избыточное". Индустриальная экспансия неизбежно должна была распространиться и на поведение человека в
свободное время; и, как я полагаю, она все больше будет формировать содержание сознания людей.
"Идеологическая  война",  крупномасштабная  обработка  сознания  делается  в  преуспевающей  экономике
условием  "полной  занятости".  Другими  словами,  технизация  не  может  ограничиться  воздействием  на
"экономическое" и "военное" поведение людей, как его понимали до сих пор, она все больше будет вторгаться в
резервацию  индивидуального  произвола,  производя  "промышленно  изготовленные  patterns  of  life".  Этим
процессом вряд ли будут затронуты только не-, полу- и едва образованные люди: так называемая элита также не
избежит вовлечения в этот "trend". В прежние времена люди, эпизодически сбрасывавшие с себя тяготы труда,
не нуждались для своего услаждения ни в чем другом, кроме природы, ее мира и красоты. Однако эта мирная и
прекрасная  природа  была  все  же  "Аркадией  человеческой  души",  как  культ  природы,  исполненный
пантеистического  чувства  и  во  многом  обязанный  антикизирующей  учености,  она  возникла  в  эпоху
Возрождения. В отношении к природе современного человека меньше этого возвышающего душу почитания,
дивящегося вселенской гармонии, математической и органической красоте, толкующего природные законы как
истечение  сверхъестественной  мудрости.  Человек  нашего  времени  относится  к  природе  практически-технически,  подходит  к  ней  как  завоеватель  или  по  крайней  мере  как  разведчик.  Туризм,  который  стал
возможным благодаря современным транспортным средствам и был ими вызван, во много раз превосходит по
своим  масштабам  великое  переселение  народов.  То,  что  преподносится  человеческому  любопытству  из
увиденного и услышанного благодаря кино, радио и телевидению, -- вовсе не суррогат естественного опыта, не
предложение  консервированной  духовной  пищи,  но  совершенно  новые  и  оригинальные  источники
переживания,  которые  нацелены  на  планетарную  тотальность  информации,  подобно  тому  как  экономика
развивается с расчетом на мировой рынок. Следует оставить излюбленный "критиками культуры" плач по
"утрате почвы". Техническая регламентация свободного времени -- вовсе не зло априори, даже если учитывать и
признавать, что часто она оборачивается уродством. Пока свободное время рассматривается как время, не
занятое ни трудом, ни политическими делами и обязанностями, оно продолжает контрастировать с трудом и
политической деятельностью. Чем в большей степени труд берут на себя машины, тем больше человек получает
времени для себя самого, однако оно оказывается "опустошенным", незаполненным временем, которое может
быть употреблено для любых целей и занято всем, чем угодно. Пустое время легко обращается в пустыню
скуки, которую приходится разгонять. Таким образом, когда у нас появляется свободное время просто потому,
что мы должны трудиться, после удовлетворения потребности в отдыхе оказывается, что этот промежуток
времени совершенно стерилен и может быть произвольно заполнен каким угодно содержанием. Иначе обстоит
дело, когда мы говорим, что у нас есть свободное время, поскольку и пока мы играем. Свобода времени теперь
означает не "пустоту", а творческое исполнение жизни, а именно -- осуществление воображаемого творчества,
смысловое представление бытия, в известной мере освобождающее нас от свершившихся ситуаций нашей
жизни. Такое освобождение, конечно, не реально и не истинно, мы не избегаем последствий своих поступков.
Человеческая свобода не в силах перескочить свои последствия. Но у нас есть выбор, в сделанном выборе
соустановлена цепочка следований. В игре у нас нет реальной возможности действительно возвратиться к
состоянию перед выбором, но в воображаемом игровом мире мы можем все еще или снова быть тем, кем мы
давно  и  безвозвратно  перестали  быть  в  реальном  мире.  Всякий  акт  свободного  самоосуществления
осуществляет горизонт заранее готовых возможностей. Играя, человек может отстранить от себя ("как бы") все
свое прошлое и вновь начать с точки отсчета. Прошлое, которым мы не располагаем, вновь оказывается в 
нашем  распоряжении.  Возможна  аналогичная  позиция  по  отношению  к  будущему:  реальные  шансы  не
взвешиваются,  не  питается  никаких  ограниченных  надежд,  в  игре  мы  способны  к  самому  свободному
предвосхищению, для нас нет никаких препятствий, мы можем из-мыс-лить их прочь, убрать все оказывающее
сопротивление, можем создать себе на арене игрового мира желанную декорацию. Задержанный временем
человек теряет связь с течением времени, в которое он обычно неизменно вступает или в которое он вовлечен.
Нельзя против этого возразить, что речь все же идет только об иллюзорном, утопическом "освобождении": эта
игровая  свобода  есть  свобода  для  "не-реального"  и  в  нереальном.  В  игровой  "видимости"  упраздняется
историчность человека, игра уводит его из состояний, закрепленных необратимыми решениями, в простор
вообще никогда не  фиксированного бытия, где  все возможно. В игре жизнь представляется нам "легкой",
лишенной тягостей: с нас словно сваливается бремя обязанностей, знаний и забот, игра приобретает черты
грезы, становится общением с "возможностями", которые скорее были из-обретены, нежели обретены. Если
свести  воедино  все  указанные  характеристики  игры  --  магическое  созидание  видимости  игрового  мира,
завороженность  игрового сообщества,  идентификацию  зрителей  с игроками,  самосозерцание человеческого
бытия  в  игре  как  "зерцале  жизни",  до-рациональную  осмысленность  игры,  ее  символическую  силу,
парадигматическую  функцию  и  освобождение  времени  ввиду  обратимости  всех  решений  в  игре,  игровое
облегчение  бытия  и  способность  игры  охватывать  в  себе  все  другие  основные  феномены  человеческого
существования, включая самое себя, то есть способность играть не только в труд, борьбу, любовь и смерть, но и
в игру, -- если мы сведем это воедино, то раскроется праздничный характер игры как общий ее строй. Человек
играет тогда, когда он празднует бытие. "Праздник" прерывает череду отягченных заботами дней, он отграничен
от серого однообразия будней, отделен и возвышен как нечто необычное, особенное, редкое. Но совершенно
недостаточно определять праздник только через противопоставление его будням, ибо праздник имеет значение
и  для  будней,  которым  необходимы  возвеселение,  радость  и  про-яснение.  Праздник  извлечен  из  потока
будничных событий, чтобы служить им маяком, чтобы озарять их. Он обладает репрезентативной, замещающей
функцией. В архаическом обществе яснее видна сущность праздника, нежели в нивелированной временной
последовательности нашей действительности. Там, где верх берут часы, хронометры, точные механизмы для
измерения времени, там человечеству остается все меньше времени для настоящего праздника. Там, где дни и
годы все еще измеряются по ходу солнца и звезд, там празднуют солнцевороты, времена года, различные
космические события, от которых зависит земная жизнь, там празднуют также урожай, который принесло
обработанное  поле,  победу  над  врагом  отчизны,  брачные  торжества  и  роды,  даже  смерть  празднично
справляется как поминовение предков. В праздничном хороводе переплетаются музыка и танец, в хороводе,
который есть нечто большее, нежели непосредственное выражение радости. С музыкой и танцем смыкается
мимический жест -- все это на праздничном игрище, где сообщество празднующих преображается в сообщество
созерцающих,  которые  осмысленно  созерцают  отраженный  образ  бытия  и  приходят  к  предчувственному
прозрению того, что есть. Как коллективное действие игра, наверно, изначально существует в виде праздника.
На заре истории праздник украшался боевыми играми воинов, благодарением за урожай земледельцев, жертвой,
приносимой мертвым, танцевальной игрой юношей и девушек и маскарадом, который ставил все бытие в
зримое присутствие сценического представления. Украшение праздника, которое могло далеко превосходить
будничную потребность в украшении, стало существенным импульсом для возникновения искусства. Конечно,
есть много веских оснований выводить искусство и из мастерства ремесленного умения. Но праздник был
могучим  прорывом  творческих  игровых  сил  человеческого  существа.  Праздничная  игра  стала  корнем  и
основанием человеческого искусства. Игра и искусство внутренне связаны. Конечно, не все игры -- искусство,
но  искусство  есть  наиболее  оригинальная  форма  игры,  она  есть  высочайшая  возможность  посредством
"видимости" явить сущность. Могут возразить: разве искусство не оканчивается на произведении искусства, на
реальном образе, который лучится собственной непотаенной красотой? Нельзя этого отрицать, но способ бытия
произведения  искусства  как  такового  все  же  остается  проблемой.  Что  есть  произведение  искусства:
неподдельно реальная вещь или же вокруг этой вещи всегда есть некая аура -- как бы незримая сцена? Является
ли микельанджеловский "Давид" мастерски высеченной мраморной глыбой на одной из площадей Флоренции
или же он стоит в собственном "воображаемом" мире -- спокоен, в сознании своей силы, праща закинута через
плечо, холодный испытующий взгляд устремлен на превосходящего мощью врага? "Давид" -- и то, и другое;
искусно высеченная мраморная глыба, но и юноша, приготовившийся к борьбе не на жизнь, а на смерть.
Произведения  искусства  озаряются собственным сиянием,  они  стоят словно в "просвете",  к которому мы,
созерцающие и рассматривающие, направляем свой взор и который "раскрывается" нам. Здесь мы не пытаемся
дать  философскую  теорию  произведению  искусства.  Речь  идет  исключительно  о  сжатом  указании  на
соотношение  игры  и  искусства.  Игра  есть  корень  всякого  человеческого  искусства.  Ребенку  и  художнику
наиболее  ясно  открывается  контур  игры  как  творчески-созидательного  общения  с  раскрывающимися
возможностями. Праздник как собирание и представление всех бытийных отношений имеет также и еще одно
важное значение. В архаическом обществе праздник понимался как волшебное заклятие сверх-человеческих
сил, как призыв добрых демонов, изгнание злобных кобольдов, как исключительно благоприятная возможность
для эпифании богов. Праздничное пиршество превращается в культовое жертвенное пиршество, на котором
смертные смешиваются с бессмертными, вкушают в хлебе земную плоть, в вине -- земную кровь. Зрелище
объединяет  культовое  сообщество  в  мироозначающей  и  мироистолковывающей  игре,  в  замаскированных
персонажах сцены игрового мира указывает воочию богов и полубогов, обычно ускользающих от человеческого
глаза. Репрезентирующая функция игры исполняется тут двояко: фигура игрового мира замещает нечто, что 
обладает сверхреальностью сущности, а декорация замещает всю вселенную. По отношению к богам человек
здесь выступает не так, как по отношению к себе или себе подобным: он относит, себя, веруя, к существу,
которому  принадлежит  управление  миром.  В  человеческой  игре,  очевидно,  легче  символизировать
человеческое, нежели то умозрительное существо, которое не трудится и не борется, не любит и не умирает, как
мы. Бог всегда сведущ во всем, без усилий проникает он своим ясновидящим взором от одного конца света до
другого и силой своего помысла, по Анаксагору, сотрясает все. Есть ли подобное существо, мы не можем знать
с уверенностью и подлинной надежностью. То, что в человеческой игре, с тех пор, вероятно, как человек играет,
имеются роли и образы богов, еще не доказывает, что они есть на самом деле. В явление игрового мира игра
может выводить не только то, что имеется вне игры в жизненных сферах труда, господства и т.д. Игра -- не
всегда и не исключительно осмысленное зерцало действительности. Не все, что может быть сыграно, обязано
поэтому  и  существовать.  Играя,  игра  может  воспроизводить  собственную  силу  вымысла  и  выводить  в
воображаемое присутствие творения грезы. Фантазия поэтов создала вымышленные существа различного рода,
сирен  и  лемуров;  химера  существует  в  воображении.  Человеческая  игра,  --  нечто  большее,  нежели
"самоизмышление"  различных  химер,  больше,  нежели  только  представляемое  поведение.  В  своем
прагматическом и опредмечивающем видении сцен игрового мира игра открывает возможности, которые мы
созерцаем именно в качестве являющей себя видимости. Боги приходят в человеческую игру и "пребывают" в
ней, захватывая и завораживая нас. Культ, миф, религия, поскольку они человеческого происхождения, равно
как  и  искусство,  уходят  своими  корнями  в  бытийный  феномен  игры.  Но  кто  сможет  недвусмысленно
утверждать, что религия и искусство лишь отражаются в игре или что они как раз произошли из игровой
способности  человеческого  рода?  Как  бы  то  ни  было,  человеческую  игру,  это  глубоко  двусмысленное
экзистенциальное состояние, кажется, озаряет милость небожителей и, уж конечно, -- улыбки муз.


\newpage
\section{Социальные общности: поколение, пол, семья, профессия, этнос, народ, нация}
Социальная  общность  —  широкое  понятие,  объединяющее  различные  совокупности  людей,  для  которых
характерны некоторые одинаковые черты жизнедеятельности и сознания.
Общности  различного  типа  —  это  формы  совместной  жизнедеятельности  людей,  формы  человеческого
общежития. Они складываются на различной основе и крайне многообразны. Это общности, формирующиеся в
сфере общественного производства (классы, профессиональные группы и т. п.), вырастающие на этнической
основе (народности, нации), на основе демографических различий (половозрастные общности) и др.
Исторически первой формой социальной общности была семья и такие, основанные на кровнородственных
отношениях, социальные общности, как род и племя. В дальнейшем социальные общности формируются также
и на других основаниях и несут на себе отпечатки конкретного социально-экономического строя.
Для социальных общностей характерно не только наличие общих объективных характеристик, но и осознание
единства своих интересов по сравнению с др.общностями, более или менее развитое чувство «мы». Именно на
этой основе происходит превращение простой (статистической) совокупности людей, обладающих общими
объективными характеристиками, в реальную социальную общность (в частности, «класса в себе» в «класс для
себя»).
Люди  одновременно  являются  членами  различных  общностей,  с  разной  степенью  внутреннего  единства.
Поэтому часто единство в одном (напр., в национальной принадлежности) может уступить место различию в
другом (например, в классовой принадлежности).
Функционирование социальных отношений , институтов контроля и организаций порождает сложную систему
социальных связей , управляющую потребностями , интересами и целями людей . Эта система сплачивает 
индивидов и их группы в единое целое - социальную общность и через нее - в социальную систему . Характер
социальных связей определяет как внешнюю структуру социальных общностей , так и ее функции . Внешняя
структура  общности  может  быть  определена  ,  например  ,  ее  объективными  данными  :  сведениями  о
демографической структуре общности , профессиональной структуре , об образовательной характеристике ее
членов и т. п.
Функционально социальные общности направляют действия своих членов на достижения групповых целей .
Социальная  общность  обеспечивает  координацию  этих  действий  ,  что  ведет  к  повышению  ее  внутренней
сплоченности  .  Последняя  возможна  благодаря  образцам  поведения  ,  нормам  ,  определяющим  отношения
внутри  этой  общности  ,  а  также  социально-психологическим  механизмам  ,  направляющим  поведение  ее
членов .
Среди многих видов социальных общностей особое значение с точки зрения влияния на поведение имеют
такие, как семья , трудовой коллектив , группы совместного проведения досуга , а также различные социально-территориальные общности ( поселок, небольшой город , крупные города , регион и т. д. ) . Скажем , семья
осуществляет социализацию молодежи в ходе освоения ею нормативов общественной жизни, формирует у нее
чувство  безопасности  ,  удовлетворяет  эмоциональную  потребность  в  совместных  переживаниях  ,
предотвращает психологическую неуравновешенность , помогает преодолеть состояние изолированности и т. д.
Территориальная общность , ее состояние также влияют на характер поведения ее членов , в особенности в
сфере  неформальных  контактов  .  Профессиональные  группы  кроме  возможности  решения  чисто
профессиональных  вопросов  формируют  у  членов  чувство  трудовой  солидарности  ,  обеспечивают
профессиональный  престиж  и  авторитет  ,  контролируют  поведение  людей  с  позиций  профессиональной
морали. 

\newpage
\section{Происхождение, сущность и функции государства}
Пристального внимания заслуживает вопрос о происхождении, сущности и функциях государства,т.к. именно
государство является ядром политической системы, самым древним и развитым политическим институтом. Вот
почему во всех крупных философских системах, начиная с античности, присутствует учение о государстве. Это
теория идеального государства Платона, "Политика" Аристотеля, идеи Т.Гоббса и Дж.Локка об общественном
договоре, диалектика гражданского общества и государства Гегеля, классовая концепция государства К.Маркса
и В.И.Ленина.
Предметом особых дискуссий является вопрос о происхождении государства. Рассмотрим некоторые точки
зрения на эту проблему. Античная философия в лице Платона и Аристотеля рассматривала возникновение
государства  как  проявление  естественных  потребностей,  присущих  человеку.  Формирование  сословий
философов,  воинов  и  работников  и  было  проявлением  этих  потребностей.  У  Аристотеля  государство
отождествлялось  с  обществом,  поэтому  политическая  сфера  охватывала  все:  семью,  религию,  культуру,
искусство. Быть членом общества означало быть членом государства, следовательно действовать в соответствии
с  его  законами.  Философия  нового  времени  выдвигает  договорную  теорию  происхождения  государства,
представителями  которой  были  Т.Гоббс,  Ж.Руссо,  А.Радищев.  Они  полагали,  что  государство  возникло  в
результате общественного договора, сознательно заключенного людьми. Это - сила, служащая всему обществу:
и богатыми и бедными, оно формируется для обеспечения мира и безопасности граждан. Положительным
моментом  этой  теории  было  то,  что  впервые  исследователи  подчеркнули  земное,  человеческое,  а  не
божественное происхождение государства.
Теория насилия, представленная Е.Дюрингом и Л.Гумпловичем, объясняла появление государства в результате
войн и политического насилия, которые углубляют социальное неравенство, приводят к образованию классов и
эксплуатации.
Существует так же психологическая теория, которая рассматривает возникновение государства как результат
того, что у одних людей существует потребность подчинять, а у других подчиняться. Представителем этой
точки зрения был, например, Шопенгауэр.
С середины XIX века возникает новая, классовая теория происхождения государства, которая получает свое
логическое  завершение  в  марксистской  теории  государства.  Согласно  К.Марксу  и  Ф.Энгельсу  государство
возникает  в  результате  раскола  общества  на  антагонистические  классы.  Развивая  именно  эту,  классовую
особенность государственного господства, В.И.Ленин пришел к выводу о том, что государство - это машина для
подавления  одних  классов  другими,  аппарат  насилия,  который  возникает  в  силу  того,  что  классовые
противоречия объективно не могут быть примирены. Оно появляется, возникает не в силу внешних причин
(войны  и  политическое  насилие),  хотя  и  они  тоже  играют  свою  роль,  но  под  воздействием  внутренних
экономических причин.
Решение  проблемы  происхождения  государства  определяет  и  ответ  на  вопрос  о  его  сущности.  Следует
подчеркнуть,  что  в  настоящее  время  постепенно  преодолевается  свойственный  марксизму  узкоклассовый
подход к сути государства, как только органа одного, господствующего класса, противостоящего другим, как
аппарата  насилия,  давления  на  общество.  Государство  остается  институтом,  при  помощи  которого
экономически господствующий класс становится господствующим и политически. Но при этом подчеркивается
и  всеобщность  ряда  функций  государства  как  органа,  управляющего  делами  всего  общества,  а  также
выражающего  в  той  или  иной  степени  интересы  и  защищающего  всех  и  каждого  конкретного  человека. 
Государство,  таким  образом,  представляет  систему  организованной  политической  власти,  направленной  на
регулирование  как  классовых,  так  и  других  социально-политических  отношений.  Оно  является  формой
общественного  сосуществования  всех  членов  общества,  выражая  и  охраняя  права,  обязанности  и  свободу
каждого.  Если  партии  и  другие  общественные  организации  представляют  интересы  отдельных  классов  и
социальных групп в политической системе, то государство выражает всеобщий интерес.
Все это дает право дать следующее его определение: государство есть социальный институт, осуществляющий
всю полноту политической власти и управления на определенной территории.
Изначально любое государство выполняло триединую задачу: 
-управлять хозяйством и обществом; 
-защищать власть класса эксплуататоров и подавлять сопротивление эксплуатируемых; 
-оборонять собственную территорию и (если имеется возможность) грабить чужую. 
По  мере  развития  общественных  отношений  появилась  возможность  более  цивилизованного  поведения
государства.
Природа государства и его положение в политической системе предполагают наличие ряда специфических
функций, отличающих его от других политических институтов. Функциями государства называются основные
направления его деятельности, связанные с суверенитетом государственной власти. От функций отличаются
цели и задачи государства, отражающие основные направления избираемой тем или иным правительством или
режимом политической стратегии, средства её реализации.
Функции государства классифицируются:
* по сфере общественной жизни: на внутренние и внешние,
* по продолжительности действия: на постоянные (осуществляемые на всех этапах развития государства) и
временные (отражающие определённый этап развития государства),
* по значению: на основные и неосновные,
* по влиянию на общество: на охранительные и регулятивные.
Основной классификацией является деление функций государства на внутренние и внешние. К внутренним
функциям государства относятся:
* Правовая функция — обеспечение правопорядка, установление правовых норм, регулирующих общественные
отношения и поведение граждан, охрана прав и свобод человека и гражданина.
* Политическая функция — обеспечение политической стабильности, выработка программно-стратегических
целей и задач развития общества.
*  Организаторская  функция  —  упорядочивание  всей  властной  деятельности,  осуществление  контроля  за
исполнением законов, координация деятельности всех субъектов политической системы.
* Экономическая функция — организация, координация и регулирование экономических процессов с помощью
налоговой и кредитной политики, планирования, создания стимулов экономической активности, осуществления
санкций.
* Социальная функция — обеспечение солидарных отношений в обществе, сотрудничества различных слоев
общества, реализации принципа социальной справедливости, защита интересов тех категорий граждан, которые
в  силу  объективных  причин  не  могут  самостоятельно  обеспечить  достойный  уровень  жизни  (инвалиды,
пенсионеры, матери, дети), поддержка жилищного строительства, здравоохранения, системы общественного
транспорта.
*  Экологическая  функция  —  гарантирование  человеку  здоровой  среды  обитания,  установление  режима
природопользования.
* Культурная функция — создание условий для удовлетворения культурных запросов людей, формирования
высокой духовности, гражданственности, гарантирование открытого информационного пространства.
* Образовательная функция — деятельность по обеспечению демократизации образования, его непрерывности
и качественности, предоставлению людям равных возможностей получения образования.
К внешним функциям государства относятся:
* Функция обеспечения национальной безопасности — поддержание достаточного уровня обороноспособности
общества, защита территориальной целостности, суверенитета государства.
*  Функция  поддержания  мирового  порядка  —  участие  в  развитии  системы  международных  отношений,
деятельность по предотвращению войн, сокращению вооружений, участие в решении глобальных проблем
человечества.
* Функция взаимовыгодного сотрудничества в экономической, политической, культурной и других сферах с
другими государствами.
Также проводится разделение между:
* деятельностью по выработке политических решений и
* деятельностью по выполнению этих решений — государственному управлению.

\newpage
\section{Общественное сознание и его основные формы}
Общественное  сознание  представляет  собой  многогранный  динамический  процесс,  поддерживаемый
активностью  индивидуальных  сознаний.  В  общественном  сознании  содержатся  устойчивые  представления,
связанные с некоторой системой норм и принципов, теории, пытающиеся обобщить особенности различных
сторон общественной жизни. Когда говорят об общественном сознании в собственном смысле слова, имеют в
виду,  прежде  всего  то,  чем  сознание  людей,  объединенных  в  некоторые  группы,  отличается  от  сугубо
индивидуального сознания человека, направленного, скажем, на решение его личных проблем, на организацию
индивидуальной  жизни.  В  этом  смысле  общественное  сознание  —  это  сознание,  всегда  направленное  на
решение общих проблем устройства общественной жизни в целом и на изучение таких свойств окружающего
мира, которые имеют общее значение.
Благодаря наличию у людей общего сознания и закреплению в сознании устойчивых образов лишь таких идей,
которые  оказываются  перспективными  в  практическом  смысле,  общество  функционирует  как  целостный
организм, то есть оно представляет не просто стихийно сложившиеся в процессе производства отношения, но
содержит сознательно упорядочиваемые людьми связи.
Идеи, получающие закрепление в общественном сознании — это не просто отражение действительности, это
еще и реорганизация действительности, практическое приспособление человека к миру. Такое приспособление
осуществляется  за  счет  того,  что  вырабатываются  новые  формы  социальной  связи,  утверждаются  новые
социальные нормы и те идеи, которые оказываются необходимыми для их воспроизводства.
Формы общественного сознания
Общественное  сознание  представлено  в  различных  формах,  в  которых  выражена  специфическая
направленность отражения действительности. Она зависит от объекта отражения и его целей. Среди форм
общественного сознания можно выделить:
* Экономическое сознание;
* Политическое сознание;
* Правовое сознание;
* Нравственное сознание;
* Эстетическое сознание;
* Религиозное сознание;
* Научное сознание;
* Философское сознание;
Во всех формах общественного сознания происходит объединение каждого отдельного человека с некоторой
общностью людей  или со всем  обществом в  целом, при  чем строится такое объединение  на базе  общего
решения  специфических  вопросов  организации  жизни,  устройства  социальных  институтов,  организации
процесса познания и т. д. Формы общественного сознания, поэтому, всегда тесно связаны с определенного типа
общественными  отношениями:  экономическими,  политическими,  нравственными,  эстетическими,
отношениями между членами научного сообщества и др.

\newpage
\section{Этика и нравственное сознание}
ЭТИКА (греч. ethika: от ethos — нрав, обычай, характер, образ мысли) — 1) на уровне самоопределения —
теория морали, видящая свою цель в обосновании модели достойной жизни; 2) практически — на протяжении
всей истории Э. — обоснование той или иной конкретной моральной системы, фундированное конкретной
интерпретацией  универсалий  культуры,  относящихся  к  субъектному  ряду  (см.  Универсалии,  Категории
культуры): добро и зло, долг, честь, совесть, справедливость, смысл жизни и т.д. В силу этого в традиционной
культуре Э. как теоретическая модель морали и Э. как моральное поучение дифференцировались далеко не
всегда  (от  восточных  кодексов  духовной  и  телесной  гигиены  до  Плутарха);  для  классической  культуры
характерна  ситуация,  когда  этики-теоретики  выступали  одновременно  и  моралистами  —  создателями
определенных этических систем (см. Сократ, Эпикур, Спиноза, Гельвеций, Гольбах, Дидро, Руссо, Кант, Гегель);
неклассическая  культура  конституирует  постулат  о  том,  что  Э.  одновременно  выступает  и  теорией
нравственного  сознания,  и  самим  нравственным  сознанием  в  теоретической  форме  (см.  Марксизм).
Фундаментальная презумпция практической морали (так называемое "золотое правило поведения": поступай по
отношению к другому так, как ты хотел бы. чтобы он поступал по отношению к тебе) в то же время выступает и
предметом обоснования для самых различных этических систем — в диапазоне от конфуцианства и вплоть до
категорического императива Канта, Э. ненасилия Л.Н.Толстого (см. Толстой), этической программы Мартина
Лютера Кинга и др. Согласно ретроспективе Шопенгауэра, "основное положение, относительно содержания
которого согласны ... все моралисты, таково: neminem laede, immo omnes, quantum potes, juva /лат. "никому не
вреди и даже, сколь можешь, помогай" — M.M./,— это, собственно, и есть... собственный фундамент этики,
который  в  течение  целых  тысячелетий  разыскивают,  как  философский  камень".  Термин  "этос"  исходно
употреблялся (начиная с древнегреческой натурфилософии) для фиксации комплекса атрибутивных качеств: от
"этоса праэлементов" у Эмпедокла до расширительного употребления термина "Э." в философской традиции: 
"Э." как название общефилософских произведений у Абеляра, Спинозы, Н.Гартмана. Вместе с тем (также
начиная с античной философии) сфера предметной аппликации данного термина фокусируется на феномене
человеческих  качеств,  в  силу  чего  по  своему  содержанию  Э.  фактически  совпадает  с  философской
антропологией  (дифференциация  философии  на  логику,  физику  и  Э.  у  стоиков,  впоследствии
воспроизводящаяся  в  философской  традиции  вплоть  до  трансцендентализма,—  см.  Стоицизм,  Юм,
Возрождение, Философия Возрождения, Просвещение). На основе дифференциации добродетелей человека на
"этические" как добродетели нрава и  "дианоэтические" как добродетели разума  Аристотель конституирует
понятие "Э." как фиксирующее теоретическое осмысление проблемного поля, центрированного вопросом о том,
какой  "этос"  выступает  в  качестве  совершенного.  Нормативный  характер  Э.  эксплицитно  постулируется
кантовской рефлексией над теорией морали, — Э. конституируется в качестве учения о должном, обретая
характер "практической философии" (см. Кант). Содержательная сторона эволюции Э. во многом определяется
конкретными историческими конфигурациями, которые имели место применительно к оппозиции интернализма
и экстернализма в видении морали (которым соответствуют зафиксированные Кантом трактовки Э. в качестве
"автономной" и "гетерономной"). Если в контексте историцизма мораль рассматривалась как сфера автономии
человеческого  духа,  то  в  рамках  традиций  социального  реализма  и  теизма  она  выступала  как
детерминированная  извне  (в  качестве  внешних  детерминант  морали  рассматривались  —  в  зависимости  от
конкретного содержания этических систем — Абсолют как таковой (см. Предопределение, Провиденциализм);
традиции  национальной  культуры  (этноэтика);  сложившиеся  социальные  отношения  (от  Э.Дюркгейма  до
неомарксизма); корпоративный (классовый) интерес (классический марксизм); уровень интеллектуального и
духовного  развития,  характерный  для  субъекта  морали  и  социального  организмов  целом  (практически  вся
философия  Просвещения,  исключая  Ж.Ж.Руссо,  и  отчасти  философия  Возрождения);  специфика
доминирующих воспитательных стратегий (от Д.Дидро до М.Мид) и пр. Однако в любом случае — в системе
отсчета субъекта — Э. конституируется в концептуальном пространстве совмещения презумпций интернализма
и экстернализма: с одной стороны, фиксируя наличие нормативной системы должного, с другой же — оставляя
за индивидуальным субъектом прерогативу морального выбора. — В этом отношении свобода как таковая
выступает  в  концептуальном  пространстве  Э.  в  качестве  необходимого  условия  возможности  моральной
ответственности (в системах теистической Э. именно в предоставлении права выбора проявляется любовь
Господа к человеку, ибо дает ему возможность свободы и ответной любви — см., например, у В.Н.Лосского).
Содержание конкретных систем Э. во многом вторично по отношению к фундирующим его общефилософским
презумпциям (которые при этом могут выступать для автора этической концепции в качестве имплицитных):
эмотивная  Э.  как  выражение  презумпций  позитивизма,  Э.  диалога  как  предметная  экземплификация
диалогической  философии  в  целом,  аналогичны  эволюционная  Э.,  аналитическая  Э.,  Э.  прагматизма,
феноменологическая Э., Э. экзистенциализма и т.п. К центральным проблемам традиционной Э. относятся
проблема соотношения блага и должного (решения которой варьируются в диапазоне от трактовки долга как
служения  благу  —  до  понимания  блага  как  соответствия  должному);  проблема  соотношения  мотивации
нравственного  поступка  и  его  последствий  (если  консеквенциальная  Э.  полагает  анализ  мотивов
исчерпывающим для оценки нравственного поступка, то альтернативная позиция сосредоточивает внимание на
оценке  его  объективных  последствий,  возлагая  ответственность  за  них  на  субъекта  поступка);  проблема
целесообразности морали (решения которой варьируются от артикуляции нравственного поступка в качестве
целерационального  до  признания  его  сугубо  ценностно-рациональным)  и  т.п.  В  рамках  неклассической
традиции статус Э. как универсальной теории морали подвергается существенному сомнению даже в качестве
возможности: согласно позиции Ницше, Э. как "науке о нравственности ...до настоящего времени недоставало,
как  это  ни  покажется  ...странным,  проблемы  самой  морали:  отсутствовало  даже  всякое  подозрение
относительно  того,  что  тут  может  быть  нечто  проблематичное.  То,  что  философы  называли  "основанием
морали...,  было  ...  только  ученой  формой  доброй  веры  в  господствующую  мораль,  новым  средством  ее
выражения".  В  этом  контексте  любое  суждение  морального  характера  оказывается  сделанным  изнутри
определенной  моральной  системы,  что  обусловливает  фактическую  невозможность  мета-уровня  анализа
феноменов нравственного порядка. В силу этого "само слово "наука о морали" в применении к тому, что им
обозначается,  оказывается  слишком  претенциозным  и  не  согласным  с  хорошим  вкусом,  который  всегда
обыкновенно  предпочитает  слова  более  скромные"  (Ницше).  В  качестве  альтернативы  традиционным
претензиям  на  построение  Э.  как  аксиологически  нейтральной  теоретической  модели  морали  Ницше
постулирует создание "генеалогии морали": "право гражданства" в этом контексте имеет лишь реконструкция
процессуальности  моральной  истории,  пытающаяся  "охватить  понятиями  ...  и  привести  к  известным
комбинациям огромную область тех нежных чувств ценности вещей и тех различий в этих ценностях, которые
живут, растут, оставляют потомство и гибнут", — реконструкция, являющая собой "подготовительную ступень
к учению о типах морали", но не претендующая при этом на статус универсальной теории, обладающей правом
и  самой  возможностью  якобы  нейтральных  аксиологически  суждений.  Таким  образом,  радикальный  отказ
неклассической философии от Э. в ее традиционном понимании фундирует собой идею "генеалогии морали",
т.е.  реконструкцию  ее  исторических  трансформаций,  вне  возможности  конституирования  универсальной
системы Э. на все времена (см. Ницше). (Позднее предложенный Ницше в этом контексте генеалогический
метод выступит основой конституирования генеалогии как общей постмодернистской методологии анализа
развивающихся систем — см. Генеалогия, Фуко.) Что же касается культуры постнеклассического типа, то она не
только  углубляет  критику  в  адрес  попыток  построения  универсально-нейтральной  Э.:  в  семантико-
аксиологическом  пространстве  постмодернизма  Э.  в  традиционном  ее  понимании  вообще  не  может  быть
конституирована как таковая. Тому имеется несколько причин: 1). В контексте радикального отказа постмодерна
от  ригористических  по  своей  природе  "метанарраций"  (см.  Закат  метанарраций)  культурное  пространство
конституирует себя как программно плюралистичное (см. Постмодернистская чувствительность) и ацентричное
(см. Ацентризм), вне какой бы то ни было возможности определения аксиологических или иных приоритетов.
Э. же не просто аксиологична по самой своей сути, но и доктринально-нормативна, в силу чего не может быть
конституирована в условиях мозаичной организации культурного целого (см. Номадология), предполагающего
принципиально  внеоценочную  рядоположенность  и  практическую  реализацию  сосуществования  различных
(вплоть до альтернативных и взаимоисключающих) поведенческих стратегий. 2). Современная культура может
быть  охарактеризована  как  фундированная  презумпцией  идиографизма,  предполагающей  отказ  от
концептуальных  систем,  организованных  по  принципам  жесткого  дедуктивизма  и  номотетики:  явление  и
(соответственно)  факт  обретают  статус  события  (см.  Событие,  Событийность),  адекватная  интерпретация
которого предполагает его рассмотрение в качестве единично-уникального, что означает финальный отказ от
любых универсальных презумпций и аксиологических шкал (см. Идиографизм). — В подобной системе отсчета
Э., неизменно предполагающая подведение частного поступка под общее правило и его оценку, исходя из
общезначимой нормы, не может конституировать
 свое содержание. 3). Необходимым основанием Э. как таковой является феномен субъекта (более того, этот
субъект, как отмечает К.Венн, является носителем "двойной субъективности", ибо интегрирует в себе субъекта
этического рассуждения и морального субъекта как предмета этой теории), — между тем визитной карточкой
для современной культуры может служить фундаментальная презумпция "смерти субъекта", предполагающая
отказ от феномена Я в любых его артикуляциях (см. "Смерть субъекта", "Смерть Автора", "Смерть Бога", Я). 4).
Э. по своей природе атрибутивно метафизична (см. Метафизика): роковым вопросом для Э. стал вопрос о
соотношении конкретно-исторического и общечеловеческого содержания морали, и несмотря на его очевидно
проблемный  статус  (см.  "Скандал  в  философии")  история  Э.  на  всем  своем  протяжении  демонстрирует
настойчивые  попытки  конституирования  системы  общечеловеческих  нравственных  ценностей.  Между  тем
современная культура эксплицитно осмысливает себя как фундированную парадигмой "постметафизического
мышления",  в  пространстве  которого  осуществляется  последовательный  и  радикальный  отказ  от  таких
презумпций  классической  метафизики,  как  презумпция  логоцентризма  (см.  Логоцентризм,  Логотомия,
Логомахия), презумпция имманентности смысла (см. Метафизика отсутствия) и т.п. (см. Постметафизическое
мышление). 5). Все уровни системной организации Э. как теоретической дисциплины фундированы принципом
бинаризма: парные категории (добро/зло, должное/сущее, добродетель/порок и т.д.), альтернативные моральные
принципы (аскетизм/гедонизм, эгоизм/коллективизм, альтруизм/утилитаризм и т.д.), противоположные оценки и
т.п. — вплоть до необходимой для конституирования Э. презумпции возможности бинарной оппозиции добра и
зла,  между  тем  культурная  ситуация  постмодерна  характеризуется  программным  отказом  от  самой  идеи
бинарных  оппозиций  (см.  Бинаризм),  в  силу  чего  в  ментальном  пространстве  постмодерна  в  принципе
"немыслимы  дуализм  или  дихотомия,  даже  в  примитивной  форме  добра  и  зла"  (Делез  и  Гваттари).  6).
Современная культура осуществляет рефлексивно осмысленный поворот к нелинейному видению реальности
(см. Нелинейных динамик теория, Неодетерминизм). В этом контексте Фуко, например, решительно негативно
оценивает историков морали, выстраивавших "линейные генезисы". Так, в концепции исторического времени
Делеза (см. Делез, Событийность, Эон) вводится понятие "не-совозможных" миров, каждый из которых, вместе
с тем, в равной мере может быть возведен к определенному состоянию, являющемуся — в системе отсчета как
того,  так  и  другого  мира  —  его  генетическим  истоком.  "Не-совозможные  миры,  несмотря  на  их  не-совозможность,  все  же  имеют  нечто  общее  —  нечто  объективно  общее,  —  что  представляет  собой
двусмысленный знак генетического элемента, в отношении которого несколько миров являются решениями
одной и той же проблемы" (Делез). Поворот вектора эволюции в сторону оформления того или иного "мира"
объективно  случаен,  и  в  этом  отношении  предшествовавшие  настоящему  моменту  (и  определившие  его
событийную специфику) бифуркации снимают с индивида ответственность за совершенные в этот момент
поступки (по Делезу, "нет больше Адама-грешника, а есть мир, где Адам согрешил"), но налагают на него
ответственность  за  определяемое  его  поступками  здесь  и  сейчас  будущее.  Эти  выводы  постмодернизма
практически  изоморфны  формулируемым  синергетикой  (см.  Синергетика)  выводами  о  "новых  отношениях
между человеком и природой и между человеком и человеком" (Пригожин, И.Стенгерс), когда человек вновь
оказывается в центре мироздания и наделяется новой мерой ответственности за последнее. В целом, таким
образом, Э. в современных условиях может быть конституирована лишь при условии отказаот традиционно
базовых своих характеристик: так, если Й.Флетчер в качестве атрибутивного параметра этического мышления
фиксирует его актуализацию в повелительном наклонении (в отличие, например, от науки, чей стиль мышления
актуализирует себя в наклонении изъявительном), то, согласно позиции Д.Мак-Кенса, в сложившейся ситуации,
напротив, "ей не следует быть внеконтекстуальной, предписывающей ... этикой, распространяющей вполне
готовую всеобщую Истину". Если Э. интерпретирует регуляцию человеческого поведения как должную быть
организованной по сугубо дедуктивному принципу, то современная философия ориентируется на радикально
альтернативные стратегии: постмодернизм предлагает модель самоорганизации человеческой субъективности
как автохтонного процесса — вне навязываемых ей извне регламентации и ограничений со стороны тех или
иных  моральных кодексов, —  "речь идет об образовании себя через разного  рода  техники жизни,  а  не о
подавлении  при  помощи  запрета  и  закона"  (Фуко).  По  оценке  Кристевой,  в  настоящее  время  "в  этике 
73
неожиданно возникает вопрос, какие коды (нравы, социальные соглашения) должны быть разрушены, чтобы,
пусть  на  время  и  с  ясным  осознанием  того,  что  сюда  привлекается,  дать  простор  свободной  игре
отрицательности".  С  точки  зрения  Фуко,  дедуктивно  выстроенный  канон,  чья  реализация  осуществляется
посредством механизма запрета, вообще не является и не может являться формообразующим по отношению к
морали. Оценивая тезис о том, что "мораль целиком заключается в запретах", в качестве ошибочного, Фуко
ставит "проблему этики как формы, которую следует придать своему поведению и своей жизни" (см. Хюбрис).
Соответственно постмодернизм артикулирует моральное поведение не в качестве соответствующего заданной
извне норме, но в качестве продукта особой, имманентной личности и строго индивидуальной "стилизации
поведения"; более того, сам "принцип стилизации поведения" не является универсально необходимым, жестко
ригористичным и требуемым от всех, но имеет смысл и актуальность лишь для тех, "кто хочет придать своему
существованию возможно более прекрасную и завершенную форму" (Фуко). Аналогично Э.Джердайн делает
акцент не на выполнении общего предписания, а на сугубо ситуативном "человеческом управлении собою"
посредством абсолютно неуниверсальных механизмов. В плоскости идиографизма решается вопрос о взаимной
адаптации соучастников коммуникации в трансцендентально-герменевтической концепции языка Апеля. В том
же  ключе  артикулируют  проблему  отношения  к  Другому  поздние  версии  постмодернизма  (см.  Afterpostmodernism).  Конкретные  практики  поведения  мыслятся  в  постмодернизме  как  продукт  особого
("герменевтического") индивидуального опыта, направленного на осознание и организацию себя в качестве
субъекта  —  своего  рода  "практики  существования",  "эстетики  существования"  или  "техники  себя",  не
подчиненные  ни  ригористическому  канону,  ни  какому  бы  то  ни  было  общему  правилу,  но  каждый  раз
выстраиваемые  субъектом  заново  —  своего  рода  "практикование  себя,  целью  которого  является
конституирование  себя  в  качестве  творца  своей  собственной  жизни"  (Фуко).  Подобные  "самотехники"
принципиально идиографичны, ибо не имеют, по оценке Фуко, ничего общего с дедуктивным подчинением
наличному  ценностно-нормативному  канону  как  эксплицитной  системе  предписаний  и,  в  первую  очередь,
запретов:  "владение  собой  ...  принимает  ...  различные  формы,  и  нет  ...  одной  какой-то  области,  которая
объединила бы их". Д.Мак-Кенс постулирует в этом контексте возможность Э. лишь в смысле "открытой" или
"множественной",  если  понимать  под  "множественностью",  в  соответствии  со  сформулированной  Р.Бартом
презумпцией,  не  простой  количественный  плюрализм,  но  принципиальный  отказ  от  возможности
конституирования канона, т.е. "множественность", которая, согласно Кристевой, реализуется как "взрыв".

\newpage
\section{Эстетика и художественное сознание}
ЭСТЕТИКА — термин, разработанный и специфицированный А.Э.Баумгартеном в трактате "Aesthetica" (1750
—1758).  Предложенное  Баумгартеном  новолатинское  лингвистическое  образование,  восходит  к  греч.
прилагательному  "эстетикос"  —  чувствующий,  ощущающий,  чувственный,  от  "эстесис"  —  ощущение,
чувствование,  чувство,  а  также,  перен.  "замечание,  понимание,  познание".  Латинский  неологизм  позволил
Баумгартену  обозначить  "эстетическое"  как  первую,  низшую  форму  познания  (gnoseologia  inpherior),
отличающую ее от чувственности (sensus) — всеобщего условия преданности Универсума (Декарт, Ньютон,
Лейбниц, Вольтер) и его непосредственной обращенности к мыслящей субстанции (sensorium Dei). Тем самым
Э. и эстетическое изначально в понятийном плане определялись в гносеологическом статусе. Э., по понятию,
была заявлена как теоретическая дисциплина, изучающая область смыслообразующих выразительных форм
действительности,  обращенных  к  познавательным  процедурам  на  основе  чувства  прекрасного,  а  также  их
художественных экспликаций. В последнем случае Э. получает статус "философии искусства" (Гегель). Идея
прекрасного определяет всю содержательную суть и направленность классической Э., одновременно становясь
предметом эстетического познания и принципом его организации. Прекрасное, таким образом, предположено
как субстанциальная основа эстетического опыта вообще, через который мир воспринимается в своей свободе и
высшей представленной гармонии, где понятие свободы должно сделать действительной в чувственном мире
заданную его законами цель, и, следовательно, природу должно быть возможно мыслить таким образом, чтобы
закономерность ее формы соответствовала по крайней мере возможности целей, заданных ей законами свободы
(Кант).  Прекрасное  будучи  необходимым  условием,  оформляющим  содержание  эстетического  опыта,
осуществляется  в  последнем  виде  идеала,  обладающего  законодательной  и  нормативной  значимостью
(ценностью).  Идеал,  нормативно  определяющий  (специфицирующий)  содержание  эстетического  опыта,
предполагает  собственный  принцип  обнаружения  его  законов  (законосообразности)  —  рефлектирующую
способность  суждения  или  вкус.  Вкус  в  эстетическом  опыте  обладает  законодательной  способностью
спецификации природы, ее эмпирических законов по принципу целесообразности для наших познавательных
способностей (Кант) и таким образом устанавливает постижимую иерархию родов и видов, переход от одного к
другому  и  высшему  роду  (прекрасному  идеалу),  а,  следовательно,  и  закон  и  порядок  суждения  вкуса  в
специфичной  логике  "эстетических  категорий".  Специфичность  категорий  эстетического  суждения  вкуса
заключается  в  том,  что  они  не  являются  логическими  (рассудочными)  понятиями  и  не  могут  быть
дефиницированы  как  понятия,  а  лишь  выражают  (изображают)  чувство  удовольствия  или  неудовольствия
(характер  благорасположения),  которое  предполагается  наличествующим  у  каждого,  однако  не  является
всеобщей определенностью. Выразительно-изобразительный характер способности эстетического суждения в
его  устанавливающей  (законодательной)  форме,  а  равно  и  в  форме  рассуждения  вкуса  об  установленной
эстетической предметности, эксплицирует Искусство как сферу, не принадлежащую уже абсолютно никакой 
эмпирической действительности (природе вообще), полностью иррелевантную ей. Искусство, постулирующее в
форме  воплощенного  идеала,  понимаемого  как  высшее  выражение  мировой  социальной  гармонии,  акта
свободного  творчества  свободного  народа,  является,  одновременно,  манифестацией  духа  "оживляющего
принципы в душе". Принцип, в свою очередь "есть не что иное, как способность изображения эстетических
идей" (Кант). Именно Искусство, взятое в его эстетическом измерении теоретизирует, т.е. представляет для
рассуждающего разума (духа)  чистую эстетическую  предметность,  извлеченную  художественным  гением  и
обращенную  к  чистому  (лишенному  интереса)  чувству  удовольствия  и  неудовольствия.  Э.  эмансипирует
Искусство в феноменологической форме свободного духовного творчества на основе воображения и по законам
прекрасного идеала, тем самым отделяя его как от праксеологической, так и от гносеологической (научно-теоретической)  сфер  деятельности.  В  концептуальном  плане,  Э.  генерирует  феномен  "классической
художественной  культуры",  "классики",  нормативно  специфицирующей  европейскую  культуру  вообще,  где
искусство представлено идеально, в форме теории ("классицизм", "просвещение") и нормативного образца,
установленного законодательным суждением вкуса и законосозидающим творчеством художественного гения.
Нормативным идеалом в европейской классике становится "Античность", однако не исторический античный
(греко-римский) мир, а его художественная реконструкция, произведенная искусством, коллекционированием,
философией, начиная от Ренессанса вплоть до середины 17 в. и завершенная И.И.Винкельманом ("История
искусства  древности", 1764). Винкельман  предложил  формирующейся  Э. то, в  чем она  крайне  нуждалась:
теоретическим  рассуждениям  о  вкусе,  о  прекрасном.  Была  предложена  аппелирующая  к  истории
художественная  модель,  представляющая  прекрасный  идеал  в  наглядно-чувственной,  воплощенной  форме.
Модель крайне условная, поскольку Винкельман создавал ее не столько опираясь на методику исследования,
атрибуции и систематизации реальных художественных произведений (в число "антиков" попадали и прямые
подделки), сколько на уже разработанную и разрабатываемую новоевропейскую метафизику Античности, ее
теоретический, дискурсивный образ (Буало, Берк, Батте, Вольтер). Такая нормативность классики содержит
известный парадокс историзма мира искусства: в своих прекрасных произведениях (классических шедеврах) он
экстраисторичен (обращен к вечности), но в своем генезисе — историчен. Исторический вектор формирования
эстетической культуры наиболее полно и всесторонне разработан Гегелем в "Феноменологии Духа" и в "Э.". В
"Феноменологии Духа" Гегель рассматривает искусство как художественную религию, которая определяется
исключительно феноменально, как "самосознание" духа, взятое в своей сугубой субъектности (самоположения
для  себя  же):  "дух...  который  свою  сущность,  вознесенную  над  действительностью,  порождает  теперь  из
чистоты самости" ("Феноменология духа"). В таком виде дух является "абсолютным произведением искусства и
одновременно  столь  же  абсолютным  "художником".  Однако  такое  абсолютное  произведение  искусства,  в
котором произведение тождественно художнику, не дано в наличности, как то постулировалось классической
культурой  и,  соответственно,  Э.,  а  есть  результат  становления,  переживающего  различные  этапы  (формы)
своего  осуществления.  И  здесь  Гегель  подвергнет  радикальной  деконструкции  абсолют  воплощенного
прекрасного  идеала,  рассматривая  таковой  только  в  темпорально  связанных  формах  самосозерцания
художественных  индивидуаций  "всеобщей  человечности",  участвующих  в  воле  и  действиях  целого.  Так
возникают  виды  самосозерцающего  свое  становление  духа,  проявляющиеся  в  соответствующих  видах
художественного  произведения:  эпос,  трагедия,  комедия.  Они-то  и  есть  сущностный,  субстанциальный
"эстесис",  который в феноменологическом плане и становится предметом  Э. как  философии искусства.  Э.
Гегеля  завершает  формирование  европейской  классической  художественной  культуры,  возвещая  уже
случившийся  "конец  искусства",  что  художественно  воплощается  в  творческой  индивидуальности  Гёте,
особенно в поздний, "классический" период. Установка Э. относительно полного типа культуры взятого как
завершающий этап Всемирной истории, способствовал и формированию истории Э., когда предшествующие
исторические периоды (эпохи) в своих творческих художественных феноменах стали рассматриваться именно с
позиции эстетической телеологии и заданности. Так появляются "Э. античного мира", "Э. средних веков", "Э.
Возрождения",  художественный  опыт  которых  осмысливается  исключительно  с  позиций  нормативно-эстетической телеологии и заданности. Собственно, такая установка была сформулирована в "Э." (Философии
искусства Гегеля), его последователя Ф.Т.Фишера ("Astentik oder Wissansaft der Schonen" тт. 1—3, 1846—1858) и
заложила  основы  традиции  истории  эстетических  учений.  В  этой  истории  характерна  провиденциально-телеологическая установка на приближение к прекрасному, художественно-выраженному идеалу (эстетической,
т.е.  обращенной  к  восприятию  на  уровне  разума  чувственного  созерцания,  манифестации  художественно
выраженного  духа).  Но  эта  же  история  изначально  оказалась  ограничена  как  в  применимости  своих
методологических  установок,  так  и  в  полноте  охвата  продуктов  художественного  опыта  (праксиса).  За  ее
пределами  оказались  многие  артефакты,  кстати  существенные  для  понимания  исторической  специфики
художественного  творчества  даже  самой  европейской  культуры,  не  говоря  уже  о  иных  культурно-художественных  мирах:  Дальний  и  Ближний  Восток,  Африка,  доколумбова  Америка  и  т.п.,  которые
рассматривались  скорее  этнографически,  чем  эстетически.  Что  же  касается  Европы,  то  целые  периоды
Античности (архаика, этрусское искусство, искусство народов Северной Европы и т.п.) оказались за пределами
не только истории Э., но и истории искусства. В равной степени средневековая культура была представлена
исторически  и  теоретически  лишь  отдельными  нормативными  формами:  философско-теологические
интерпретации  искусства  и  т.п.  проторенессансные  феномены,  либо  некоторые  ретроспекции  античной
классики.  Не  входили  в  таким  образом  сконструированную  модель  исторического  развития  эстетической
культуры  и  вопросы,  связанные  с  региональными,  национально-этническими  спецификациями  творчески-
художественной феноменологии. В связи с тем, что классическая Э. и классическая художественная культура
предполагают творчество как духовное самостановление для самосознания в форме созерцания, то и человек
субъектно  определяется  как  "живое  произведение  искусства"  (Гегель),  формирующееся  лишь  в  процессе
эстетического  воспитания,  педагогика  которого  становится  существеннейшим  элементом  Э.  Теоретико-эстетическая концепция эстетического воспитания была наиболее полно разработана
 Шиллером ("Uber die asthetische Erzichung des Menschen", 1795), где постулируется необходимость творения
"эстетической  реальности",  реальности  высшего  порядка,  в  которой  только  возможно  художественное
сотворение  человеческой  личности,  взятой  в  целостности  (гармонии)  свободного  и  универсального
развертывания своих способностей (сущностных сил). Основной идеей эстетического воспитания становится
игра  как  единственная  форма  свободного  (от  утилитарных  влечений)  проявления  бытия  человека.  В
художественно-классической  форме  идея  эстетического  воспитания  была  раскрыта  Гёте  ("Театральное
призвание  Вильгельма  Мейстера",  "Годы  учений"  и  "Годы  странствий  Вильгельма  Мейстера").  Идея
эстетического воспитания наметила тему европейской художественной культуры, продолжающуюся и в 20 в.,
что же касается педагогики, то эстетическое воспитание оказалось мало востребованным, хотя и вдохновило
ряд педагогических концепций, в частности К.Д.Ушинского ("Человек как предмет воспитания", тт. 1—2, 1868
—1869) в России, Морриса в Англии, а также Бергсона и Дьюи. Однако все эти педагогические устремления не
выходили за рамки эксперимента, подчас, крайне драматического. Парадокс эстетического воспитания был
художественно осмыслен Манном ("Волшебная гора") и особенно остро — Гессе ("Игра в бисер"). Однако
тематизация эстетического воспитания способствовала становлению "Э." как учебной (школьной) дисциплины
и формированию европейской классической модели воспитания и образования. Педагогический модус Э. и
эстетического, тем не менее выводил эстетическую проблематику в более общую, антрополого-гуманитарную
перспективу,  где  эстетическое  стало  все  в  большей  степени  рассматриваться  как  феномен  развертывания
сущностных  аспектов  человеческого  бытия,  а  эстетическая  теория,  соответственно,  как  проработка
возможностей развертывания гуманистической перспективы. Гегелевская концепция человека в виде "живого
произведения искусства" явилась импульсом, весьма существенно преобразующим смысл Э. (в теоретическом и
праксеологическом планах). Суть его в том, что Э. (эстетическая теория) становится одной из парадигмальных
(даже  процедурных)  установок  формирующейся  философской  антропологии  во  всех  ее  многообразных
модификациях. Эстетический и равно художественный опыт начинает интенсивно проникать в сущностные
основы бытия, экзистенциальность человека, которая в значительной степени мыслится аутопойетически. Так
формируется  неклассическая  Э.,  связанная  с  разнообразными  формами  жизнетворчества,  и  имеет  своим
следствием  формирование  инновационных  художественных  течений,  принципиально  антиномичных  и
парадоксально-маргинальных  в  отношении  к  классической,  эстетически-концептуализированной  культуре.
Неклассическая  Э.  в  своем  жизнетворческом  модусе  заявляет  о  себе  как  негативная  по  отношению  к
классической  форме,  прежде  всего  в  отношении  искусства,  которое  понимается  как  художественно-жизнетворческий акт. Художник вступает здесь как автор и исполнитель, непосредственно включенный в бытие
произведения и постоянно осуществляющий его пойесис. Аутопойетический пафос направлен, прежде всего, на
создание  авторизованного  мифопоэтического  мира,  относительно  самодостаточного  и  самодостоверного,
включенного  в  него  творца-художника.  Такой  мир  обладает  собственной  Э.,  чаще  всего  достаточно  полно
разработанной  в  концептуальном  плане.  Цельность  культуры  претерпевает  радикальную  дифференциацию;
художественная  культура  теперь  представляет  настающий  и  калейдоскопический  универсум  множества
автономных  миров.  Возможность  построения  аутопойетического  художественного  мира  была  заявлена
творчеством  Гельдерлина  ("Гиперион"),  но  в  радикальной  форме  проработана  Ницше  ("Происхождение
трагедии из духа музыки", "Казус Вагнера" и "Так говорил Заратустра"). Ницше тем более радикализирует
установку  на  возможность (невозможность)  целостности  культурно-художественного  мира, предлагая тезис
"Смерти  Бога".  "Смерть  Бога"  лишает  мир,  прежде  всего  в  его  творческом  осуществлении  всеобщей
телеологической  задачи,  следовательно  —  всеобщего  нормативного  идеала,  следовательно  —  возможности
континуальной  "всеобщей  человечности".  Мир  теперь  —  дионисийский  экстаз,  трагедия  и  индивидуации
самоудовлетворяющего "Эго", где аполлоническая завершенность представления для созерцающей рефлексии
существует лишь как перманентная целесообразность абсолютной самоутверждающей и законосозидающей
воли  (воли  к  власти).  Художественная  культура,  взятая  в  подобном  ракурсе,  становится  принципиально
культурой модерна, современной и актуальной и одновременно-исключительной и оригинальной. Формируется
возможность  многообразия  "Э.",  где  само  понятие  Э.  теперь  не  ограничивается  ни  статусом  философско-теоретической  рефлексии  по  поводу  прекрасного,  ни  статусом  "рефлесирующей  способности  суждения"
(вкуса), устанавливающей всеобщую сообщаемость систематизированных ценностей, ни даже спецификацией
"гениальности",  с  ее  претензией  на  всеобщее  эстетическое  законодательство  (нормативный  стиль).  Э.
становится  понятием,  характеризующим  целостность  конкретного  "субкультурного"  художественного  мира,
обладающего  своим  авторским  стилем,  рисунком.  Такого  рода  модернизация  неизбежно  приводит  к
разнообразию, эстетически себя определяющих, направлений, течений, вполне суверенных и генетически никак
не связанных. Происходит разрушение Всемирной истории Э. и вместе с этим и Всемирной истории искусств,
как детерминированного процесса сменяющих, однако, зависимых друг от друга исторических стилей, эпох и
т.п.  История  эстетической  культуры  теперь  мыслится  как  трансформация  и  трансмутация  множества
коммуницирующих художественных систем в хронотопе социо-культурного события. Э. получает еще одно
измерение-социологическое (И.Тэн, Я.Буркхардт, Маркс, Лукач, Ортега-и-Гассет). Художественно-эстетические 
феномены  представляются  здесь  как  выражение  социальных  статусов  (интересов,  целей,  представлений)
социальных  групп  в  системах  экономических  и  политических  отношений  (обменов,  коммуникаций,
конфликтов).  Социологизирующая  Э.  развивается  в  двух  конкурирующих  системах  самоописания  и
саморепрезентаций  общества:  политической  и  экономической  и  непосредственно  смыкается  с  социальной
антропологией. Таким образом, по определению Хайдеггера, "искусство вдвигается в горизонт Э. и становится
непосредственным  выражением  жизни  человека",  однако  жизни,  погруженной  в  глобальную  социальную
коммуникацию,  что,  во-первых  требует  описания  эстетического  феномена  в  контексте  этой  коммуникации
(структурализм,  постструктурализм,  лингвистические  и  семиологические  процедуры  описания)  и
контекстуального истолкования эстетических событий (герменевтика). Проблематичной остается классическая
тема  эстетической  культуры  в  модернистской  и  постмодернистской  ситуации.  Здесь  необходимо  отметить
статус  Музея,  удерживающего  в  определенной  степени  классическую  парадигму  как  ориентирующую
координату, в отношении которой модернистские (неомодернистские) проекты опознают и определяют себя как
явления  художественной  культуры.  Присутствие  классической  парадигмы  в  коммуницирующей  системе
эстетических проектов удерживает феноменологичекую суть европейской эстетической культуры, концентрируя
ее на выявление общегуманитарной телеологической установки, которая определяется Хайдеггером термином
"алетейя"  в  смысловом  аспекте  "процветания"  и  "несокрытости"  истины,  открытости  для  обоснования  и
утверждения верховных ценностей человечества.

\newpage
\section{Глобализация как тенденция развития современной цивилизации и ее проблемы}
ГЛОБАЛИЗАЦИЯ - термин для обозначения ситуации изменения всех сторон жизни общества под влиянием
общемировой  тенденции  к  взаимозависимости  и  открытости.  Двигателем  Г.  выступает  современный
капитализм (см.), понимаемый как фаза истории человечества и как несущий определенную (неолиберальную)
геополитическую  и  политическую  программу.  По  мысли  Р.  Робертсона  ("Глобализация".  Лондон,  1992),
"понятие глобализации относится как к компрессии мира, так и к интенсификации осознания мира как целого...
как к конкретной глобальной взаимозависимости, так и осознанию глобального целого в 20 в.". М. Уотерс
("Глобализация". Лондон; Нью-Йорк, 1995) отмечает: Г. - это процесс, "в ходе которого и благодаря которому
определяющее воздействие географии на социальное и культурное структурирование упраздняется и в котором
люди это упразднение все в большей мере осознают".
Впервые  Г.  как  феномен  слияния  рынков  отдельных  видов  продукции  и  услуг,  производимых  крупными
транснациональными корпорациями, была осмысленно зафиксирована, по-видимому, американским ученым Т.
Левиттом в статье, опубликованной в журнале "Гарвард бизнес ревью" в 1983. Более широкое значение новому
термину придали в Гарвардской школе бизнеса, а главным его популяризатором стал японский консультант этой
школы К. Омэ, опубликовавший  в  1990 книгу "Мир без границ".  Полагая,  что  мировая экономика теперь
определяется  взаимозависимостью  трех  центров  -  ЕС,  США,  Япония  -  он  утверждал,  что  экономический
национализм отдельных государств стал бессмысленным, в роли же главных "актеров" на экономической сцене
выступают "глобальные фирмы".
Г. - это признание растущей взаимозависимости современного мира, главным следствием которой является
значительное  ослабление  (некоторые  исследователи  настаивают  даже  на  разрушении)  национального
государственного  суверенитета  под  напором  действий  иных  субъектов  современного  мирового  процесса  -прежде  всего  транснациональных  корпораций  и  иных  транснациональных  образований,  например,
международных компаний, финансовых институтов, этнических диаспор, религиозных движений, мафиозных
групп и т.д. Не связанные, в отличие от национальных государств, условиями международных договоров и
конвенций, эти транснациональные образования оказываются в более выгодном положении, перераспределяя в
свою  пользу  властные  полномочия.  Основной  сферой  Г.  является  международная  экономическая  система
(мировая экономика), т.е. глобальные производство, обмен и потребление, осуществляемые предприятиями в
национальных экономиках и на всемирном рынке. Хотя основная часть глобального продукта потребляется в
странах-производителях,  национальное  развитие  все  более  увязывается  с  глобальными  структурами  и
становится более многосторонним и разноплановым, чем это было в прошлом. Успех глобальных корпораций в
значительной мере опирается на признание их продукции потребителями разных стран. Даже в странах с
высоким  уровнем  развития товарного национализма продукция  глобальных корпораций  находит  успешный
сбыт  по  причине  ее  относительной  дешевизны  и  качества.  Кроме  того,  огромные  прибыли  глобальных
корпораций позволяют вкладывать гигантские средства в рекламу, внушающую потребителям, что это именно
те товары и услуги, которые им нужны. Глобальные производство и сбыт формируют глобального потребителя -"гражданина мира", который во всех странах ищет и находит одно и то же, наслаждается одним и тем же. Г.
представляет  собой  комплексную  тенденцию  в  развитии  современного  мира,  затрагивающую  его
экономические, политические, культурные, но в первую очередь информационно-коммуникационные аспекты.
Так, мощным двигателем процесса Г. стали электронные СМИ. Спутниковое телевидение, получающее все
большее распространение, ломает национальные границы, превращает постоянно растущую часть населения
Земли в одну гигантскую телеаудиторию, которая смотрит одни и те же фильмы, любит и знает одних и тех же
звезд, стремится к одним и тем же символам материального успеха (см. Всемирная деревня). Другой важной
составляющей  процесса  Г.  выступили  современные  информационные  технологии  (Интернет,  компьютерная
связь,  мультимедиа  и  т.п.),  позволившие  транснациональным  организациям  размещать  различные 
составляющие  производственного  процесса  в  разных  странах,  сохраняя  при  этом  тесные  контакты.  Так,
мировой финансовый рынок являет собой в настоящее время интегрированную глобальную систему, четко
скоординированную через мгновенные телекоммуникации. Уменьшение необходимости физических контактов
между  производителями  и  потребителями  позволило  некоторым  услугам,  которые  ранее  невозможно  было
продать на международных рынках, стать объектом специфической торговли. Так, любую деятельность, которая
осуществима на экране или по телефону (от подготовки программного обеспечения до продажи авиабилетов),
стало возможным произвести в любой точке мира, связавшись через спутник или компьютер. Формирующаяся
система глобальной информации формирует потребности и интересы, общие для жителей всех стран. В свою
очередь глобальные потребности ведут к появлению глобальных продуктов. Это проявляется в стандартизации
товаров и унификации торговых марок. К числу товаров, которые приобретают вид стандартных (глобальных),
уже  можно  отнести  мотоциклы,  аудиокассеты,  диски  СD,  стереоаппаратуру,  компьютеры  и  ряд  других.
Следующий уровень Г. - унификация менеджерской стратегии и тактики, разрабатываемой для глобального
рынка без попыток адаптации управления к  местной специфике. Таким  образом, Г.  предполагает также и
тенденцию  к  унификации  мира,  к  жизни  по  единым  принципам,  приверженности  единым  ценностям,
следованию единым обычаям и нормам поведения. (Индивидуальные феноменологии во все большей степени
релятивизируются  в  глобальной  системе  координат.)  Г.  фундирована  социальным  состоянием,  когда
символические  практики  приоритетны  по  отношению  к  материальным,  начинает  доминировать  Г.
символических  обменов,  освобождающих  (уже  в  силу  своей  природы)  социальные  отношения  от
пространственных  референций.  Г.  ведет  к  взаимному  уподоблению  социальных  практик  на  уровне
повседневной  жизни.  Именно  этот  аспект  Г.  является  традиционным  объектом  критики  защитников
национальных культурных традиций от эскалации западной, и прежде всего американской, информационной
продукции.  Они  полагают,  что  Г.  информационно-коммуникационных  процессов  ведет  к  нивелированию
самобытности культур в различных странах и подчинению СМИ не гуманитарным, а коммерческим интересам.
Как  отмечает  французский  ученый  Д.  Коэн,  "прошло  немного  лет,  а  к  понятию  глобализации  уже  стали
относиться с подозрением: одни считают, что оно предполагает фаталистическое отношение к происходящим в
мире  изменениям,  в  то  время  как  другие  призывают  к  защите  нынешнего  дорогой  ценой  завоеванного
общественного порядка от тенденций столь сомнительного качества". По мысли же директора Центра Фернанда
Броделя по изучению экономики, исторической системы и цивилизаций И. Валлерстайна (1999), идея Г. как
универсального  этапа  мирового  развития  есть  "громадная  ошибка  современной  действительности,  обман,
навязанный нам властными группировками и нами же самими, часто в отчаянии... Мы можем думать об этом
переходе как о политической битве между двумя крупными лагерями: тех, кто хочет удержать привилегии
существующей неравноправной системы, и тех, кто хотел бы видеть создание новой системы, которая будет
более справедливой". 

\end{document}