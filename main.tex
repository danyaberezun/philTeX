\documentclass[12pt]{article}
\usepackage[left=2cm,right=2cm, top=2cm,bottom=2cm,bindingoffset=0cm]{geometry}
\usepackage{fontspec}
\usepackage{polyglossia}
\setdefaultlanguage{russian}
\setmainfont[Mapping=Tex]{CMU Serif}

\begin{document}

% \tableofcontents


\newpage
\section{Предмет философии в ее истории}
Слово  «философия»  –  греческого  происхождения  и  буквально  означает  «любовь  к  мудрости».  Философия
представляет собой систему взглядов на окружающую нас действительность, систему наиболее общих понятий
о мире и месте в нем человека. С момента своего возникновения она стремилась выяснить, что представляет
собой мир как единое целое, понять природу самого человека, определить, какое место занимает он в обществе,
может ли его разум проникнуть в тайны мироздания, познать и обратить на благо людей могущественные силы
природы. Философия таким образом, ставит самые общие и вместе с тем очень важные, коренные вопросы,
определяющие  подход  человека  к  самым  разнообразным  областям  жизни  и  знания.  На  все  эти  вопросы
философы давали самые различные, и даже взаимоисключающие ответы.
Предмет философии – философия – наука, которая изучает наиболее общие законы развития природы, общества
и познания. Философия рассматривает наиболее важные связи в системе «мир- человек»
Назначение философии – поиск удела человека, обеспечение его бытия в причудливом мире, а в конечном счете
в  возвышении  человека,  в  обеспечении  его  совершенствования.  Общую  структуру  философского  знания
составляют четыре основных раздела: онтология(учение о бытие), гносеология(учение о познании), человек,
общество.
Специфика философской мудрости состоит в ее нацеленности на смыслы максимально всеохватной (общей,
предельной, целостной) и вместе с тем фундаментальной значимости.


\newpage
\section{Структура философского знания и его функции}
Философия, как особый вид знания сформировался в 7-6 в. до н э в Древней Греции, Индии и Китае. Термин
"философия" встречается впервые у Пифагора и переводится как “любовь к мудрости”. Ф.- "учение об общих
принципах бытия и познания, об отношении человека к миру; наука о всеобщих законах развития природы,
общества и мышления." Философия - венец культуры, высшая мудрость (Панфилософизм, Гегель). Кьеркегор:
Обычному человеку с его тревогами и бедами нечего ждать от философии. “Я предъявляю к философии вполне
законные  требования  -  что  делать  человеку?  Как  жить?...Молчание  философии  является  в  данном  случае
уничтожающим  доводом  против  неё  самой?”  От  религии  философия  отличается  тем,  что  рационально
обосновывает  свои  принципы.Розанов:  Философия  объясняет,  религия  решает  проблемы  (проблему
преодоления  смерти).  Наука  -  частное  знание,  философия  общее.  В  науке  существует  целый  корпус  не
подвергающихся  сомнению  положений,  имеется  явное  поступательное  развитие.  Результаты  научной
деятельности  безличны.  Философия  не  удовлетворяет  принципам  верификации  и  фальсификации.  Рассел:
философия - размышление о предметах, знание о которых еще не возможно. Если человек ищет ответ на вопрос
как  жить,  и  хочет  обоснованного  ответа  то  он  обращается  к  философии.  Но  ответа  не  находит.
Мировоззренческая ф-я. Методологическая функция. 
Маркс.-лен  ф-я  -  есть  общая  методология  познания  и  практического  преобразования  объективного  мира.
"Философия содействует приросту научных знаний и создаёт предпосылки для научных открытий".
Соловьёв: “Что делает философия для человечества, какие блага ему даёт от каких зол избавляет?”. 
Философия “освобождала человеческую личность от внешнего насилия, и давала ей внутреннее содержание”.
Философия служит, соответственно, и материальному и божественному началу. Но “Эта двойственная сила и
этот двойственный  процесс,  составляя  сущность философии,  вместе с тем и составляют сущность самого
человека”. Поэтому окончательный ответ Соловьева на вопрос, что же делает философия: “Она делает человека
человеком”.
Внутри философии сформировались такие философские дисциплины: 
Онтология - учение о бытии или первоначалах всего сущего
Гносеология(эпистемология)  -  теория  познания,  занимающаяся  исследованием  природы  познания,  его
структуры, выясняющая условия его достоверности и истинности.
Логика- наука о формах правильного, т. е. последовательного, связного, доказательного мышления.
Этика- учение о морали или нравственности.
Эстетика- учение о прекрасном, о природе искусства.
Социальная  философия  исследует  общество  как  систему  надиндивидуальных  форм,  связей  и  отношений,
которые человек создаёт своей деятельностью. С ней непосредственно связана философия истории, которая
исследует  смысл,  закономерности  и  основные  направления  исторического  процесса;  философская
антропология,  которая  выясняет  сущность  человека  как  личности;  политическая  философия  (и  философия
права), которая выясняет природу власти и государства. Философия науки изучает строение научного знания,
механизмы  и  формы  его  развития.  Философия  религии  осмысляет  природу  и  функции  религии.  Всегда
предстаёт  либо  как  философское  религиоведение  либо  как  философская  теология.  Также  выделяются
философия культуры, философия техники, философия экономики, философия творчества, философия любви. 
Принципиально важной областью философии является история философии.
\subsection{Специфика философского знания}
Основная специфика философского знания заключается в его двойственности, так как оно:
• имеет очень много общего с научным знанием — предмет, методы, логико-понятийный аппарат;
• однако не является научным знанием в чистом виде.
Главное отличие философии от всех иных наук заключается в том, что философия является теоретическим
мировоззрением, предельным обобщением ранее накопленных человечеством знаний.
Предмет философии шире предмета исследования любой отдельной науки, философия обобщает, интегрирует
иные науки, но не поглощает их, не включает в себя все научное знание, не стоит над ним.
. Можно выделить следующие особенности философского знания:
• имеет сложную структуру (включает онтологию, гносеологию, логику и т. д.);
• носит предельно общий, теоретический характер;
• содержит базовые, основополагающие идеи и понятия, которые лежат в основе иных наук;
• во многом субъективно — несет в себе отпечаток личности и мировоззрения отдельных философов;
•  является  совокупностью  объективного  знания  и  ценностей,  нравственных  идеалов  своего  времени,
испытывает на себе влияние эпохи;
• изучает не только предмет познания, но и механизм самого познания;
•  имеет  качество  рефлексии  —  обращенности  мысли  на  саму  себя  (то  есть  знание  обращено  как  на  мир
предметов, так и само на себя);
• испытывает на себе сильное влияние доктрин, вырабатываемых прежними философами;
• в то же время динамично — постоянно развивается и обновляется;
• опирается на категории — предельно общие понятия;
• неисчерпаемо по своей сути;
•  ограничено  познавательными  способностями  человека  (познающего  субъекта),  имеет  неразрешимые,
"извечные"  проблемы  (происхождение  бытия,  первичность  материи  или  сознания,  происхождение  жизни,
бессмертие души, наличие либо отсутствие Бога, его влияние на мир), которые на сегодняшний день не могут
быть достоверно разрешены логическим путем.
\subsection{Плюрализм философского знания и его предпосылки}
Плюрализм – философская мировоззренческая позиция, согласно которой существует множество независимых
и несводимых друг к другу начал или видов бытия. 
Плюрализм (от лат. pluralis - множественный) - философская позиция, согласно которой существует множество
различных равноправных,  независимых и несводимых друг к другу форм знания и методологий познания
(эпистемологический  плюрализм)  либо  форм  бытия  (онтологический  плюрализм).  Плюрализм  занимает
оппонирующую позицию по отношению к монизму. 
Термин  «плюрализм»  был  введён  в  начале  XVIII  в.  Христианом  Вольфом,  последователем  Лейбница  для
описания  учений,  противостоящих  теории  монад  Лейбница,  в  первую  очередь  различных  разновидностей
дуализма.
В конце XIX - XX веке плюрализм получил распространение и развитие как в андроцентрических философских
конценциях,  абсолютизирующих  уникальность  личного  опыта  (персонализм,  экзистенциализм),  так  и  в
эпистемологии (прагматизм Уильяма Джеймса, философия науки Карла Поппера и, особенно, теоретический
плюрализм его последователя Пауля Фейерабенда).
Эпистемологический плюрализм как методологическкий подход в науке, подчёркивая субъективность знания и
примат  воли  в  процессе  познания  (Джемс),  историческую  (Поппер)  и  социальную  (Фейерабенд)
обусловленность знания, критикует классическую научную методологию и является одной из посылок ряда
антисциентистских течений.
Причины:
- многообразие действительности, которую постигает философия (Гегель)
- историческая обусловленность философии
- принципиальная неполнота и фрагментарность знаний о действительности.
- люди все разные.


\newpage
\section{Мифология, религия и философия как основные типы мировоззрения и их характерные особенности}
\subsection{Понятие мировоззрения. Типы мировоззрения и их характерные особенности}
Мировоззрение, миропонимание-взгляд на мир и положение человека в этом мире, оценка и характеристика
взаимоотношений человека и мира. Мировоззрение формируется веками и продолжает формироваться, поэтому
в ходе развития М нужно выделить различные этапы, т.е., характеризуя М как историческое.Исторические типы
М: ( Мифологический, Религиозный, Научный, Философский). 
Мировоззрение   исторически  конкретно,  оно  вырастает  на  почве  культуре  и  вместе  с  ней  претерпевает
изменения. МЗ каждой эпохи  реализуется во  множестве  групповых  и индивидуальных  вариантах.  МЗ  как
система включает в себя : знания( имеющих своей опорой истину), а наряду с этим и ценности. МЗ выработана
не только разумом, но и чувствами. Это значит, что МЗ состоит как бы из двух частей – Интеллектуальный и
Эмоциональный.  Э.  сторона  МЗ  представлена  мироощущением  и  мировосприятием.  И.-  миропониманием.
Соотношением И. и Э. стороны МЗ зависит от эпохи, от самого индивидуума. Так же бывает разная окраска
понимания мира, что выражается в чувствах. Второй уровень МЗ –миропонимание, опирающееся прежде всего
на знания, хотя МП и МО не даны просто так рядом друг с другом: они, как правило едины. МЗ- включает в
свою структуру уверенность и веру. МЗ делится на жизненно-повседневное и теоретическое. ЖП складывается
повседневно. Важно — нужно учитывать. Страдает 1) недостаточной широтой 2) своеобразным переплетением 
положений и установок с примитивными, мистическими, предрассудками 3) большой эмоциональностью. Эти
минусы преодолеваются на теоретическом уровне вировоззрения. Это философский уровень мировоззрения,
когда человек подходит к миру с позиции разума, действует, опираясь на логику, обосновывая свои выводы и
утверждения.
Философии как особому типу МЗ предшеств. Мифологический и религиозный типы МЗ.
Миф как особая форма сознания и мировоззрения представляет собой своеобразный сплав знаний, хотя и
весьма ограниченных, религиозных верований и различных видов искусств.
Дальнейшее развитие миропонимания пошло по двум линиям – по линии религии и по линии философии.
Религия  –  форма мировоззрения, в которой освоение мира осуществляется через его удвоение на земное,
естественные и потусторонний , сверхъестественный . При этом в отличии от науки , тоже создающий свой
второй  мир  в  виде  научной  картины  природы,  второй  мир  религии  основан  не  на  знании  ,  а  на  вере  в
сверхъестественные силы и их главенствующую роль в мире, в жизни людей. Религиозная вера – это особое
состояние сознания, отличное от уверенности ученого, которая базируется на рациональных основах.
Общее что роднит философию и религию это решение мировоззренческих проблем но пути и подходы решения
этих проблем у ник сильно отличаются.
\subsection{Мифология как первичная форма мировоззрения}
Мифологическое мировоззрение было древнейшей формой познания мира, космоса, общества и человека. Миф
по  необходимости  возник  из  потребности  индивида,  его  семьи,  рода  и  социума  в  целом,  в  осознании
окружающей природной и социальной стихии, сущности человека и передачи их единства через различные
символические системы. В мифологических системах человек и социум, как правило, не выделяют себя из
окружающего  мира.  Космос,  природа,  общество  и  человек  —  различные  проявления  одного  и  того  же
божественного  закона,  передаваемого  через  символическую  или  символико-мифологическую  системы.
Природа,  общество  и  человек  слиты  в  единое  целое,  неразрывное  и  единое,  однако  сами  они  внутри
неоднородны и уже авторитарны, авторитаризм общества перенесён на всю природу.
Мифологическое  сознание  мыслит  символами:  каждый  образ,  бог,  культурный  герой,  действующее  лицо
обозначает стоящее за ним явление или понятие. Это возможно потому, что в мифологическом мировоззрении
существует  постоянная  и  неразрывная  связь  между  «однотипными»  явлениями  и  объектами  в  социуме,
личности, природе и космосе.
Важнейшим аспектом Традиционной культуры и мифологического мировоззрения является и то, что мифы
изначально  живут  в  своём,  особенном  времени  —  времени  «первоначала»,  «первотворения»,  к  которому
неприложимы  линейные  представления  о  течении  времени.  Подобное  отношение  к  времени  хорошо
прослеживается  в  народных,  в  частности русских  сказках,  где время  действия  определяется как «давным-давно», «в стародавние времена» и т. д.
Кроме того, миф, особенно на начальных стадиях своего развития (в долитературном виде), мыслит образами,
живёт эмоциями, ему чужды доводы современной формальной логики. При этом он объясняет мир, исходя из
ежедневной  практики.  Данный  парадокс  объясним  тем,  что  социум,  где  преобладает  мифологическое
мировоззрение, напрямую соотносит особенности своего восприятия мира с реальным миром, индивидуальные
психические процессы с природными и социальными явлениями, зачастую не делая различия между причиной
и следствием, а часто меняя их местами.
По Традиционному мифологическому мировоззрению её последователь способен подняться до уровня бога, а
значит для человека, рода и социума миф о странствиях и подвигах культурного героя, в большинстве случаев,
читай «бога», был практически полезен и являлся руководством к действию.
Особенности, характерные для мифологической формы мировоззрения:
Антропоморфность  —  рассмотрение  явлений  природы  по  аналогии  с  человеком.  Явлениям  природы
приписываются все те свойства, которые есть у человека: ощущения, реакции на негативные факторы, желания,
страдание и т. п. 
Дескриптативность — стремление к объяснению событий, явлений в форме описательного рассказа, сказания,
легенды; среди действующих фигур — герои и боги в виде особых людей. 
Синкретизм (слитность, нерасчленённость) объективного и субъективного миров, что в значительной степени
объясняется антропоморфностью, пронизывающей все стороны этой формы мировоззрения. 
Связь  с  магией  свойственна  более  зрелому  первобытно-общинному  сознанию  и  выражается  в  действиях
колдунов, шаманов и других людей, вооружённых зачатками научных знаний о теле человека, о животных и
растениях. Наличие магического элемента в составе данной формы мировоззрения позволяет отвергнуть точку
зрения, будто это мировоззрение не связано с практикой, а является лишь пассивно-созерцательным. 
Апелляция  к  прецеденту  в  объяснении  событий,  определяющих  современный  порядок  вещей.  Например:
"человек стал смертным потому, что гонец (часто какое-то животное) неправильно передал волю божества",
"человек начал использовать огонь потому, что тот был украден у богов Прометеем" и т. д. 
Антиисторичность.  Время  не  понимается  как  процесс  прогрессивного  развития.  В  лучшем  случае  оно
допускается как обращённое вспять: движение от золотого века к серебрянному и медному, что само по себе
выражает желание видеть мир статичным, постоянно воспроизводящимся в том же самом виде. 

\newpage
\section{Генезис философии как переход от мифологического мышления к рациональному}
Нужно отметить, что генезис философии является проблемой для самой философии, развиваясь, она постоянно
сталкивается с проблемой собственного возникновения, ибо, только решив ее, философия сможет в полной мере
осознать свою сущность. 
Существуют  три  основополагающие  теории  возникновения  философии:  мифогенная,  гносеогенная  и  та,
которую, по ее характеру, можно назвать концепцией "качественного скачка". Любая из этих точек зрения, если
признать  только  ее  единственно  верной,  ограничена.  Философия,  ее  категориальный  аппарат  во  многом
наследует мифологемы и философское мировоззрение во многом вырастает из мифа, особенно у таких ранних
философов, как Ферекид, Эмпедокл и, даже, Платон. Вместе с тем, философское мировоззрение радикально
противоречит мифической картине мира. Поэтому нельзя говорить, что сущность философии вытекает из мифа:
она настолько же из него вытекает, насколько от него отталкивается в своем становлении. Миф у Ферекида и
даже у Гомера уже не тот миф, которому безусловно поклоняются и которому безусловно верят, - это уже
рефлектированный миф, с которого снят ореол святости. 
Если же говорить о происхождении философии из науки, то, как уже выяснено, это невозможно. Вместе с тем,
невозможно,  наверное,  и  развитие  философии  без  науки,  как  и  без  мифа.  Поэтому,  отвечая  на  вопрос,
происходит ли философия из мифа или из накопления научного знания, невозможно дать односложный ответ
"да" или "нет", здесь отношения более сложные и замысловатые. Ясно одно: без них бы философия не возникла,
и с ними она также могла бы не возникнуть. Так, мы имеем примеры в иных культурах, когда существовала
развитая  мифология  и  был достаточный  уровень научного знания, однако  философия  так и не  состоялась
(Египет, Вавилон и др. культуры). Получается, что наука и миф являются необходимыми, но недостаточными
предпосылками возникновения философии. Не хватает еще чего-то, что делает философию именно философией
и что невозможно найти ни в науке, ни в мифе. 
Третья концепция генезиса философии, концепция "качественного скачка", подчеркивает именно эту специфику
философского знания как его сущностное отличие от иных типов знания. Поясним: эта концепция не отрицает
важность  для  становления  философии  мифа  и  науки,  а  также  развития  социальных,  экономических,
политических  и  иных  связей.  Действительно,  без  совокупности  всех  этих  необходимых  предпосылок
философия  никак  бы  не  могла  появиться,  однако  весь  этот  перечень  предпосылок,  сколь  бы  длинным  и
объемным он ни был, не способен автоматически породить философию. В современном мире, как казалось бы,
всех этих предпосылок куда больше, и даже новых мифов создается больше, чем было старых у древних греков,
однако  Древняя  Греция  породила  такую  плеяду  оригинальных  философов,  которых  в  XX  веке  можно
пересчитать по пальцам одной руки. 
Концепция "качественного скачка" интересна, оригинальна и даже нова тем, что останавливает свое внимание
именно на внутреннем существе философского знания, которое несводимо ни к каким предпосылкам, которое
ниоткуда  невозможно  вывести,  как  только  из  самой  философии.  Философия  породила  саму  себя  в  акте
духовного  познания  и  самопознания,  она  -  совершенно  суверенная,  автономная  область  знания,  знания
предвечного, ибо его неспособна постичь ни мифология, ни наука, ни религия. Философия возникает сразу и
как бы вдруг, разорвав все путы мифологического, научного, поэтического, эпического и какого угодно иного
вида  знания.  Но  хотя  философия  возникает  вдруг  и  сразу,  она  не  сразу  и  не  вдруг  приходит  к  своему
самосознанию, она не сразу обнаруживает себя в своей первозданной чистоте. В истории античной философии
она как бы идет от себя к себе, выражаясь терминологией Гегеля, от в-себе-бытия к для-себя-бытию.

\newpage
\section{Основные концепции античной натурфилософии}
Первые философские представления о природе вещества и происхождении его свойств зародились практически
одновременно в разных цивилизациях около VII века до н.э. В Китае это были Конфуций и Лао Цзы, в Индии –
Будда, в Персии – Зороастр (Заратустра), в Греции – философы т.н. Милетской школы.
Все эти натурфилософские учения имеют общие черты:
1. Космологический подход. Учение о природе вещества и его свойств является частью учения о мироздании в
целом, причем свойства вещества с необходимостью следуют из свойств Вселенной.
2.   Дуализм.  Важнейшим  элементом  любого  натурфилософского  учения  является  существование  пар
противоположных мировых начал (Ян – Инь, светлое – тёмное, активное – пассивное, любовь – ненависть и
т.п.).
Отличительная особенность греческой натурфилософии – её в значительной степени светский (нерелигиозный)
характер. В греческой натурфилософии можно выделить два течения, выделившиеся по способу ответа на
вопрос о делимости материи: континуализм и атомизм.
Континуализм исходит из предположения, что материя непрерывна и делима до бесконечности; любая сколь
угодно малая часть материи тождественна тому телу, делением которого она получена.
Атомизм  утверждает,  что  материя  дискретна  и  состоит  из   множества  неделимых  частичек  –   атомов,  –
движущихся в пустоте.
Несмотря на принципиальные различия континуализма и атомизма в объяснении разнообразия веществ,  все
античные  натурфилософские  школы  имели  известную  общность  подхода:  в  любом  случае  многообразие
свойств считалось случайным проявление субстанции – неких абсолютных начал, хотя бы и дискретных.
В качестве основных черт натурфилософии можно отметить следующее:
1.  Умозрительность. Всякая античная натурфилософская концепция представляет собой абстракцию (порой
гениальную), лишённую каких-либо эмпирических основ. Чувственные данные всегда используются лишь как
иллюстрация для умозаключений.
2. Дедукция (рассуждение от общего к частному). Всякая античная натурфилософская концепция претендует на
всеобщее объяснение устройства Вселенной; свойства вещества логически вытекают из свойств Вселенной.
3. Выбор первоматерии (субстанции) в качестве объекта изучения.

\newpage
\section{Антропологический поворот в античной философии (софисты и Сократ)}
К досократикам причисляют софистов, которые, однако, осуществили в философии антропологический поворот
–  от  исследования  природы  они  обратились  к  изучению  человека  как  общественного  существа.  Время
деятельности  софистов,  которые  были  первыми  “учителями  мудрости”  для  всех  желающих,  разоблачали
традиционные мифические представления и поставили под вопрос традиционные моральные и религиозные
нормы,  сделав  их  предметом  сознательного  и  критического  отношения,  называют  также  “греческим
просвещением”. Софисты открыли принципы относительности и субъективности всякого знания о мире (так
называемый философский релятивизм), т. е. его изменчивости и зависимости от человека как субъекта, от его
способностей и потребностей: человек есть мера всех вещей. Иначе говоря, первоначалом всего, “архэ”, надо
признать человека, личность, его чувства и его ум. В философии софистов впервые находит своё выражение и
обоснование характерный для западной цивилизации принцип индивидуализма. 
Сократ – основоположник этики, т. е. философской теории морали, а также диалектики как такого искусства
ведения беседы, диалога, благодаря которому через столкновение противоположных мнений и противоречия
достигается общее понимание сути вещей, истина. Подвергнув критике релятивизм софистов, он искал общие
определения нравственных понятий, которые имеют силу для всех людей как разумных существ. Главный
предмет  всех  бесед  Сократа,  описанных  Платоном  и  Ксенофонтом,  –  разумная  жизнь,  добродетель,  благо
человека. 
Сократ обнаружил в своих беседах с людьми, что хотя сами они убеждены, что знают, в чем заключается их
благо и добродетель, в действительности они располагают лишь кажущимся знанием, которое не выдерживает
испытания  “логосом”  (разумом)  в  свободном  диалоге.  Он  придумал  определенный  метод  для  достижения
истинного, надежного знания о благе человека – повивально-иронический диалог. Метод Сократа вытекает из
полного доверия к “логосу”, разуму. Он был убеждён в том, что человек, как существо разумное, не должен
подчиняться ничему, кроме разума, который есть наилучшее, божественное во мне. Неосмысленная, неразумная
жизнь не имеет никакой ценности. Сократовское отождествление добродетели и знания называется этическим
рационализмом. Разум – и только он один – есть источник и мерило нравственности. Разум может и должен
подчинить себе жизнь. В этом и заключается назначение человека и высшее благо для него. 

\newpage
\section{Проблемы бытия и познания в философии Платона}
Платон – основоположник идеалистического направления философии.
Платон  является  основателем  идеализма.  Главными  положениями  его  идеалистического  учения  являются
следующие:
материальные вещи изменчивы, непостоянны и со временем прекращают свое существование;
окружающий  мир  («мир  вещей»  также  временен  и  изменчив  и  в  действительности  не  существует  как
самостоятельная субстанция;
реально существуют лишь чистые (бестелесные) идеи (эйдосы);
чистые (бестелесные) идеи истинны, вечны и постоянны;
любая существующая вещь является всего лишь материальным отображением первоначальной идеи (эйдоса)
данной вещи (например, кони рождаются и умирают, но они лишь являются воплощением идеи коня, которая
вечна и неизменна и т.д.);
весь мир является отображением чистых идей (эйдосов).
«Триада» Платона – «единое», «ум», «душа».
Также  Платон  выдвигает  философское  учение  о  триаде,  согласно  которому  все  сущее  состоит  из  трех
субстанций:
— «единого»;
— «ума»;
— «души».
«Единое»:
является основой всякого бытия;
не имеет никаких признаков (ни начала, ни конца, ни частей, ни целостности, ни формы, ни содержания, и т.д.);
есть ничто;
выше всякого бытия, выше всякого мышления, выше всякого ощущения;
первоначало всего – всех идей, всех вещей, всей явлений, всех свойств (как всего хорошего с точки зрения
человека, так и всего плохого).
«Ум»:
происходит от «единого»; разделен с «единым»; противоположен «единому»; является сущностью всех вещей;
есть обобщение всего живого на земле.
«Душа»:
подвижная субстанция, которая объединяет и связывает»единое – ничто» и «ум – все живое», а также связывает
между собой все вещи и все явления; также согласно Платону душа может быть мировой и душой отдельного
человека; при гилозоическом (одушевленном) подходе душу могут иметь также вещи и неживая природа;
душа человека (вещи) есть часть мировой души;
душа бессмертна;
при смерти человека умирает только тело, душа же, ответив в подземном царстве за свои земные поступки,
приобретает новую телесную оболочку;
постоянство души смена телесных форм – естественный закон Космоса.
Касаясь гносеологии (учения о познании), Платон исходит из созданной им идеалистической картины мира:
поскольку материальный мир является всего лишь отображением «мира идей», то предметом познания должны
стать прежде всего «чистые идеи»; «чистые идеи» невозможно познать с помощью чувственного познания
(такой тип познания дает не достоверное знание, а лишь мнение – «докса»); высшей духовной деятельностью
могут  заниматься  только  люди  подготовленные  –  образованные  интеллектуалы,  философы,  следовательно,
только они способны увидеть и осознать «чистые идеи».


\newpage
\section{Проблемы бытия и познания в философии Аристотеля}
Аристотель (384-322). Из семьи придворного врача. Был слушателем "Академии" Платона до смерти основателя
в теч. 20 лет. "Платон мне друг, но истина дороже.» Критиковал платонизм: сущность вещей в самих вещах, а не
в идеях. Странствует. В 50 лет возвращается в Афины. Открывает фил. школу ликей (рядом с храмом Аполлона
Ликейского). Стал учителем А.Македонского. После смерти АЛ-дра обвинен в богохульстве. Покидает Афины.
Создатель самой обширной науч. системы из существ. в античности. Создал вместе со своими учениками новое
научн. напр., систематизир. науку, определил предмет и методы отдельных наук."О философии", "Диалог о
счастье", "О небе","Политика", "Экономика", "Поэтика", "Риторика", "Метафизика". 
Для объяснения того, что существует, А. принимал 4 причины:
— сущность и суть бытия, в силу которой всякая вещь такова, какова она есть(формальная причина)
— материя и подлежащее (субстрат)- то, из чего что-либо возникает (материальная причина)
— движущая причина, начало движения;
— целевая причина-то, ради чего что-либо осуществляется
О предмете фил.: если предметом физики явл. материальные сущности, то фил. имеет право на самостоятельное
существование, если в ней есть эл-ты нематериальных причин, сверхчувств., неподвижные, вечные сущности. 
Это наиболее божественная наука в 2 смыслах: 
1.Владеть ею пристало скорее богу, чем человеку. 
2.Ее предметом являются божественные предметы. 
Поэтому А. называет свою фил. теологией, учением о боге. Дал определение понятиям сущего и сущности.
Сейчас: Сущее- то, что существует, что можно наблюдать. Сущность- то, что нельзя ощутить, а только понять
как  внутр.  связи  (Солн.  затм.  -  сущее,  понимание  его  -  сущность).  Основа  любого  бытия  -  проматерия,
перводвиг. всего - бог. Материя то, из чего вещь возникла. Вещь существует независимо от идеи, материя
первична. Материя - сама по себе не обозначается ни как определенное по существу, ни как определенное по
количеству, ни как обладающее каким-либо иным св-вом сущего. Такая материя включает и проматерию из
провеществ (вода, воздух, земля, огонь).
Познание: 1-чувств -овладение единичным 2-рацион-овлад. общим. 
Путь познания; ощущ.->предст.-> опыт-> искусство->наука
Об об-ве и гос-ве: "политика". Человек - обществ. сущ-во. Рабовладение - естественное состояние организации
общества. С рождением каждому определено быть или рвабом или рабовладельцем.
Общество свободных людей. 
1. Очень богаты - противоественный способ приобретения богатства - плохо. 
2.Средний класс 
3. Крайне бедные. 
Формы гос-ва: хорошие (монархия, аристокр.), плохие (тирания, олигаргия, демокр.) Задача- предотвращение
чрезмерного накопления богатства.


\newpage
\section{Характерные черты и этапы средневековой христианской философии}
Раннее  средневековье  характеризуется  становлением  Христианской  догматики  в  условиях  формирования
европейского  государства  в  результате  падения  Римской  империи.  В  условиях  жесткого  диктата  церкви  и
господства власти философия была объявлена служанкой богословия, которая должна была использовать свой
рацион.  Аппарат  для  подтверждения  догматов  христианства.  Эта  философия  получила  название
"схоластики"(опиралась на форм. логику Аристотеля)
Еще в 5 веке(христианство уже господствующая религия в Греции и Риме) было сильно влияние философия
неоплатонизма,  враждебного  христианству.(Нехристианские  философские  школы  были  закрыты  по  декрету
императора  Юстиана  в  529г.)  При  этом  одни  христианские  идеологи  склонялись  к  отрицанию,  другие  к 
использованию  учений  ф.  Идеалистов  древности.  Так  возникла  литература  апологетов(защитников)
христианства,  а  за  ней  возникает  патристика–сочинения  отцов  церкви,  писателей,  заложивших  основы
философии христианства.
Со 2 века Греческие апологеты обращались к императорам, преследовавшим христианство. Они стремились
доказать,  что  христианство  поднимает  такие  вопросы,  которые  ставила  и  предшествующая  греческая
философия , но дает более совершенное их разрешение. Видный апологет- Тертуллиан(из Карфагена, 2в.)- сущ.
непримиримое  разногласие  между  религией,  божественным  откровением,  священным  писанием  и  чел.
мудростью. Не создав ф. систем апологеты, однако , наметили круг вопросов, кот. стали основными для хр. ф.(о
боге,  о  сотворении  мира,  о  природе  человека  и  его  целях).  Апологетика  использует  логические  доводы,
обращенные к разуму, для доказательства бытия бога, бессмертия души. разбирает доводы, обращенные против
религии  и  отдельных  догматов.  Противоречие  в  том,  что  будучи  рациональной  по  форме,  апологетика
иррациональна по содержанию, т. е. обращаясь к разуму, говорит о непостижимости разумом религиозных
догматов.
Склонна к софистике и догматизму.
Античная философия космоцентрична, философия средневековья–теоцентрична(основная проблема–проблема
христианского бога). Христианство появилось примерно в середине 1 века и стимулировало развитие С.Ф.
Этапы развития средневековой философии:
1. Этап патристики(2- 8 век, конец этапа–деятельность Боэция–первого схоласта)
2. Этап становления схоластики((7–12 вв.)–Боэций, Эриуген, П.Абеяр)
3. Расцвет схоластики(13 век–Бэкон, Альберт Великий, Фома Аквинский)
Блаженный Августин. Все сущее, поскольку оно существует и именно потому что существует, есть благо. Зло –
не субстанция, а недостаток, порок и повреждение формы, небытие. Напротив, благо есть субстанция, «форма»
со всеми ее элементами: видом, мерой, числом, порядком. Бог есть источник бытия, чистая форма, источник
блага. Поддержание бытия мира есть постоянное творение его Богом вновь. Если бы творческая сила Бога
прекратилась, мир вернулся бы в небытие. Мир один, послед-ть миров – игра воображения. В мировом порядке
всякая вещь имеет свое место. Материя также имеет свое место в строе целого.
Душа – нематериальная субстанция, отличная от тела, а не простое св-во тела. Она бессмертна.
Вечность и время: Мир ограничен в пространстве, а бытие его ограничено во времени. Время и пространство
существуют только в мире и с миром. Начало творения мира – начало времени. Время есть мера движения и
изменения. А. пришел к гениальной идее: ни прошедшее, ни будущее не имеют реального существования –
действительное  существование  присуще  только  настоящему.  Нет  никакого  «пред  тем»,  никакого  «потом».
Прошедшее обязано своим существованием нашей памяти, а будущее – нашей надежде. «В вечном нет ни
приходящего, ни будущего, ибо что проходит, то перестает существовать, а что будет, то еще не начало быть.»
Фома  Аквинский  (Аквинат).  Систематизатор  ортодоксальной  схоластики,  основатель  томизма  –  одного  из
направлений сх-ки. Исходным принципом учения явл. божественное откровение: человеку для своего спасения
нужно знать такое, что ускользает от его разума, через божественное откровение.
Разграничил области ф. и теологии: предметом ф. явл. «истины разума», а второй – «истины откровения». Не
все «истины откровения» доступны рациональному док-ву. Религиозная истина не м.б. уязвима со стороны ф., в
жизненном, практически-нравственном отношении любовь к Богу важнее познания Бога.
Исходя из учения Аристотеля Ф. рассматривал Бога как первопричину и конечн. цель всего сущего, как «чистую
форму», «чистую актуальность». Сущность всего телесного – в единстве формы и материи. Они суть реальные
сверхчувственные внутренние принципы, образующие всякую реальную вещь, все телесное вообще. Материя –
только воспреемница сменяющих друг друга форм, «чистая потенциальность», ибо лишь благодаря форме вещь
явл-ся вещью опред-го рода и вида. Кр. того, форма – целевая причина образования вещи.
Индивидуальность человека – это личностное единство души и тела, именно душа обладает животворящей
силой  человеческого  организма.  Душа  нематериальна,  самосуща,  уникальна  и  бессмертна  –  субстанция,
обретающая полноту лишь в единстве с телом., которое  тоже участвует в духовно-душевной деятельности
человека.
Основной принцип познания, по Ф.А., — реальное существование всеобщего. В споре об универсалиях (см.
ниже) отстаивал позиции умеренного реализма, т.е. всеобщее существует трояко: «до вещей» (в разуме Бога как
идеи будущих идей, как вечные идеальные прообразы сущего), «в вещах», получив конкр-ое осуществление, и
«после вещей» — в мышлении человека после абстрагирования и обобщения. Человеку присуще 2 способности
познания – чувство и интеллект. Познание начинается с с чувств. опыта под действием внешн. объектов. Но
воспринимается не все бытие объекта, а лишь то в нем, что уподобляется субъекту. При вхождении в душу
познающего познаваемое теряет свою материальность и может войти в нее лишь как образ. Вещь сущ-ет вне
нас во всем своем бытии и внутри нас как образ.
Истина – соответствие интеллекта и вещи. При этом понятия, образуемые интеллектом, истинны в той мере, в
какой они соотв-ют своим понятиям, предшествующим в интеллекте Бога.
Универсалии.  Одна  из  особенностей  средневек.  ф.  проявилась  в  споре  между  реалистами  (realis  –
вещественный, действительный) и номиналистами (nomen – имя, наименование) о природе универсалий, т.е. о
природе общих понятий. Реалисты (Эриугенаи Фома Аквинский), основываясь на положении Аристотеля о том,
что общее существует в неразрывной связи с единичным, являясь с его формой, сформулировали концепцию о
трех видах существования универсалий. Универсалии существуют трояко: [см. выше про Фому]. Такое решение 
вопроса носит в истории ф. название «умеренного реализма» в отличие от «крайнего реализма», по которому
общее существует только вне вещей. Крайний реализм не мог быть принят ортодокс. церковью именно из-за
того, что материя была частично оправдана христианством как одна из двух природ Иисуса Христа.
Номиналисты  (в  частности  Росцелин)  довели  идею  отрицания  объективного  существования  общего  до
логического конца, считая, что универсалии существуют лишь в человеческом разуме, в мышлении, т.е. они
отрицали  не  только  наличие  общего  в  конкретной  единичной  вещи,  но  и  его  существование  «до  вещи».
Универсалии суть только имена вещей, и существование их сводится лишь у колебаниям голоса. Существует
только индивидуальное, и только оно может быть предметом познания.

\newpage
\section{Особенности философии эпохи Возрождения}
1.  Философией  эпохи  Возрождения  называется  совокупность  философских  направлений,  возникших  и
развивавшихся  в  Европе  в  XIV  —  XVII  вв.,  которые  объединяла  антицерковная  и  антисхоластическая
направленность,  устремленность  к  человеку,  вера  в  его  великий  физический  и  духовный  потенциал,
жизнеутверждающий и оптимистический характер.
Предпосылками возникновения философии и культуры эпохи Возрождения были:
• совершенствование орудий труда и производственных отношений;
• кризис феодализма;
• развитие ремесла и торговли;
• усиление городов, превращение их в торгово-ремесленные, военные, культурные и политические центры,
независимые от феодалов и Церкви;
• укрепление, централизация европейских государств, усиление светской власти;
• появление первых парламентов;
• отставание от жизни, кризис Церкви и схоластической (церковной) философии;
• повышение уровня образованности в Европе в целом;
• великие географические открытия (Колумба, Васко да Гамы, Магеллана);
•  научно-технические  открытия  (изобретение  пороха,  огнестрельного  оружия,  станков,  доменных  печей,
микроскопа, телескопа, книгопечатания, открытия в области медицины и астрономии, иные научно-технические
достижения).
2. Основными направлениями философии эпохи Возрождения являлись:
• гуманистическое (XIV - XV вв., представители: Данте Алигьери, Франческо Петрарка, Лоренцо Валли и др.) -в центр внимания ставило человека, воспевало его достоинство, величие и могущество, иронизировало над
догматами Церкви;
• неоплатоническое (сер. XV - XVI вв.), представители которого - Николай Кузанский, Пико делла Мирандола,
Парацельс и др. - развивали учение Платона, пытались познать природу, Космос и человека с точки зрения
идеализма;
• натурфилософское (XVI - нач. XVII вв), к которому принадлежали Николай Коперник, Джордано Бруно,
Галилео Галилей и др., пытавшиеся развенчать ряд положений учения Церкви о Боге, Вселенной, Космосе и
основах мироздания, опираясь на астрономические и научные открытия;
• реформационное (XVI - XVII вв.), представители которого -Мартин Лютер, Томас Монцер, Жан Кальвин,
Джон  Усенлиф,  Эразм  Роттердамский  и  др.  -  стремились  коренным  образом  пересмотреть  церковную
идеологию и взаимоотношение между верующими и Церковью;
• политическое (XV - XV] вв., Николо Макиавелли) - изучало проблемы управления государством, поведение
правителей;
• утопическо-социалистическое (XV - XVII вв., представители -Томас Мор, Томмазо Кампанелла и др.) - искало
идеально-фантастические  формы  построения  общества  и  государства,  основанные  на  отсутствии  частной
собственности и всеобщем уравнении, тотальном регулировании со стороны государственной власти.
3. К характерным чертам философии эпохи Возрождения относятся: 
• антропоцентризм и гуманизм — преобладание интереса к человеку, вера в его безграничные возможности и
достоинство;
• оппозиционность к Церкви и церковной идеологии (то есть отрицание не самой религии, Бога, а организации,
сделавшей себя посредником между Богом и верующими, а также застывшей догматической, обслуживающей
интересы Церкви философии — схоластики);
• перемещение основного интереса от формы идеи к ее содержанию;
•  принципиально  новое,  научно-материалистическое  понимание  окружающего  мира  (шарообразности,  а  не
плоскости  Земли,  вращения  Земли  вокруг  Солнца,  а  не  наоборот,  бесконечности  Вселенной,  новые
анатомические знания и т. д.);
• большой интерес к социальным проблемам, обществу и государству;
• торжество индивидуализма;
• широкое распространение идеи социального равенства


\newpage
\section{Традиция рационализма в философии Нового времени. Обоснование рационализма в философии Декарта}
Рассмотрение данного вопроса необходимо начать именно с самого яркого представителя рационалистического
направления – Рене Декарта. Несколько слов о его биографии. Он в восемь лет уходит на учебу в иезуитский
колледж Ла-Флеш. Здесь он получил основы образования. В ряде жизнеописаний Декарта указывается, что
сухое, педантичное обучение его не удовлетворяло. Отрицательное отношение к схоластическому пониманию
науки и философии проявилось у него, однако, позже, когда он как военный побывал в значительной части
Европы. В 1621 г. он уходит с военной службы и путешествует. Посетил Германию, Польшу, Швейцарию,
Италию и некоторое время жил во Франции. Наиболее интенсивно предавался исследованиям во время своего
сравнительно долгого пребывания в Голландии в 1629—1644 гг. В этот период он пишет большинство своих
работ. Годы 1644—1649 были наполнены стремлением отстоять, и не только теоретически, взгляды и идеи,
содержащиеся, в частности, в «Размышлениях о первой философии» и в «Началах философии». В 1643 г. в
Утрехте, а в 1647 г. в Лейдене (где сравнительно долго жил Декарт) было запрещено распространение его
воззрений, а его труды были сожжены. В этот период Декарт вновь несколько раз посещает Париж и думает
даже о возвращении во Францию. Однако затем он принимает приглашение шведской королевы Христины и
уезжает в Стокгольм, где вскоре умирает от простуды.
Наиболее  выдающиеся  из  его  философских  трудов  —  это  работы,  посвященные  (как  и  у  Бэкона)
методологической  проблематике.  К  ним  принадлежат,  прежде  всего,  «Правила  для  руководства  разума»,
написанные в 1628—1629 гг., в которых Декарт излагает методологию научного познания. С этой работой
связано и вышедшее в 1637 г. как введение к его трактату о геометрии «Рассуждение о методе». В 1640— 1641
гг.  Декарт  пишет  «Размышления  о  первой  философии»,  в  которых  вновь  возвращается  к  определен ным
аспектам своей новой методологии и одновременно придает ей более глубокое философское обоснование. В
1643 г. выходит его труд «Начала философии», в котором полно изложены его философские воззрения.
Естествознание XVI—XVII столетий еще не формулирует эти новые принципы познания (по крайней мере без
соответствующей степени общности). Оно скорее реализует их непосредственно в процессе овладения своим
предметом. Если философия  Бэкона является  предвестником  нового (его  философия скорее  симпатизирует
естествознанию Нового времени, чем создает для него философское обоснование), то в философии Декарта уже
закладываются  основания  (достаточно  общие)  новой  теории  света,  в  которой  не  только  обобщены,  но  и
философски  разработаны  и  оценены  все  полученные  к  тому  времени  результаты  нового  естествознания.
Поэтому философия Декарта представляет собой новый, цельный и рационально обоснованный образ мира, не
только соответствующий актуальному состоянию естествознания, но и полностью определяющий направление
его развития. Одновременно она вносит  и основополагающие  изменения в развитие  самого философского
мышления, новую ориентацию в философии.
Первую и исходную определенность всякой философии Декарт видит в определенности сознания — мышления.
Требование, что должно исходить лишь из мышления как такового, Декарт выражает словами: «во всем должно
сомневаться» - это абсолютное начало. Таким образом, первым условием философии он делает само отвер-жение всех определений.
Декартово сомнение и «отвержение всех определений» исходит, однако, не из предпосылки о принципиальной
невозможности существования этих определений. Это не скепсис, с которым мы встречались, например, в
античной философии. Принцип Декарта, согласно которому во всем следует сомневаться, выдвигает сомнение
не как цель, но лишь как средство.
Первичную достоверность Бэкон находит в чувственной очевидности, в эмпирическом, смысловом познании.
Для Декарта, однако, чувственная очевидность как основа, принцип достоверности познания неприемлема.
Декарт  ставит  вопрос  о  постижении  достоверности  самой  по  себе,  достоверности,  которая  должна  быть
исходной предпосылкой и поэтому сама не может опираться на другие предпосылки. Такую достоверность он
находит в мыслящем Я — в сознании, в его внутренней сознательной очевидности. «Если мы отбросим и
провозгласим ложным все, в чем можно каким-либо способом сомневаться, то легко предположить, что нет
бога, неба, тела, но нельзя сказать, что не существуем мы, которые таким образом - мыслим. Ибо является
противоестественным полагать, что то, что мыслит, не существует. А поэтому факт, выраженный словами: «я
мыслю, значит, существую» (cogito ergo sum), является наипервейшим из всех и наидостовернейшим из тех,
которые перед каждым, кто правильно философствует, предстанут».
С  проблематикой  познания  в  философии  Декарта тесно  связан  вопрос  о  способе конкретного достижения
наиболее истинного, т. е. наиболее достоверного, познания. Тем самым мы подходим к одной из важнейших
частей философского наследия Декарта — к рассуждениям о методе.
В «Рассуждении о методе» Декарт говорит, что его «умыслом не является учить здесь методу, которому каждый
должен следовать, чтобы правильно вести свой разум, но лишь только показать, каким способом я стремился
вести свой разум».
Правила, которых он придерживается и которые на основе своего опыта полагает важнейшими, он формулирует
следующим образом:
не  принимать  никогда любую  вещь за  истинную, если ты ее не познал как истинную с очевидностью;
избегать всякой поспешности и заинтересованности; не включать в свои суждения ничего, кроме того, что
предстало как ясное и видимее перед моим духом, чтобы не было никакой возможности сомневаться в этом;
разделить каждый из вопросов; которые следует изучить, на столько частей, сколько необходимо, чтобы эти 
вопросы лучше разрешить;
свои идеи располагать в надлежащей последовательности,    начиная    с    предметов    наипростейших и
наилегче познаваемых, продвигаться медленно, как бы со ступени на ступень, к знанию наиболее
сложных, предполагая порядок даже среди тех, которые естественно не следуют друг за другом;
совершать везде такие полные расчеты и такие полные обзоры, чтобы быть уверенным в том, что ты ничего не
обошел.
Правила  Декарта,  как  и  все  его  «Рассуждения  о  методе»,  имели  исключительное  значение  для  развития
философии и науки Нового времени.
Рационализм Декарта нашел много продолжателей. К наиболее выдающимся из тех, кто существенным образом
способствовал обогащению и развитию философской мысли, принадлежат, в частности, Б. Спиноза и Г. В.
Лейбниц.
Спиноза всю жизнь прожил в Голландии. И хотя, как уже говорилось, Голландия в то время была страной
прогрессивной,  он  и  здесь  не  избежал  жестокого  преследования  со  стороны  как  еврейских,  так  и
протестантских и католических религиозных кругов.
Основные  идеи  философии  Спинозы  изложены  в  его  главном  и  основном  труде  –  «Этике».  Рассуждения,
содержащиеся в «Этике», разделены на пять основных частей (о Боге, о природе и происхождении мысли, о
происхождении  и  аффекте,  о  человеческой   несвободе,   или   о   силе   аффектов,  о  силе  разума,  или  о
человеческой свободе). Как раз интерес для нашей темы представляет собой последняя, пятая часть «Этики».
Мышление  трактовалось  как  своего  рода  самосознание  природы.  Отсюда  принцип  познаваемости  мира  и
глубокий вывод: порядок и связь идей те же, что порядок и связь вещей. И те, и другие суть только следствия
божественной сущности: любить то, что не знает начала и не имеет конца, — значит любить Бога. Человек
может  лишь  постигнуть  ход  мирового  процесса,  чтобы  сообразовать  с  ним  свою  жизнь,  свои  желания  и
поступки. Мышление тем совершеннее, чем шире круг вещей, с которыми человек вступает в контакт, т.е. чем
активнее субъект. Мера совершенства мышления определяется мерой его согласия с общими законами природы,
а подлинными правилами мышления являются верно познанные общие формы и законы мира. Понимать вещь
—  значит  видеть  за  ее  индивидуальностью  универсальный  элемент,  идти  от  модуса  к  субстанции.  Разум
стремится постичь в природе внутреннюю гармонию причин и следствий. Эта гармония постижима, когда
разум, не довольствуясь непосредственными наблюдениями, исходит из всей совокупности впечатлений.
Готфрид  Вильгельм  Лейбниц  представляет  определенное  завершение  европейского  философского  рацио-нализма. В философии Лейбница отразились почти все философские импульсы его времени. Он был подробно
знаком  не только с  философией Декарта,  Спинозы,  Бойля, но  и с  философскими трудами представителей
эмпирически ориентированной английской философии, в частности с работами Дж. Локка, и давал им оценку.
Так,  он  был  не  согласен  с  концепцией  картезианского  дуализма,  отвергал  определенные  элементы  кар-тезианской теории познания, в частности тезис о врожденных идеях, возражал и против спинозовской единой
субстанции (бога), которая является и всем, и субстанцией.
Лейбниц не создал ни одного философского труда, в котором он представил бы или логически разработал свою
философскую систему. Его воззрения разбросаны по разным статьям и письмам.
Ядро философской системы Лейбница составляет учение о «монадах» — монадология. Монада — основное
понятие системы — характеризуется как простая, неделимая субстанция. Лейбниц отвергает учение Спинозы о
единой  субстанции,  которое,  по  его  представлениям,  вело  к  тому,  что  из  мира  исключаются  движение,
активность. Он утверждал, что субстанций бесконечное множество. Они, согласно его воззрениям, являются
носителями силы, имеют духовный характер.
Вопрос гармонии — важнейший в философии Лейбница. Она является неким внутренним порядком всего мира
монад и представляет собой принцип, преодолевающий изолированность монад.
Следующая характерная черта монад заключается в том, что каждая монада имеет собственную определенность
(является носителем определенных качеств), которой она отличается от всех остальных. В этой связи Лейбниц
формулирует и свой известный принцип тождества. Если бы две монады были полностью одинаковы, они были
бы тождественны, т. е. неразличимы.
По  степени  развития  он  различает  монады  трех  видов.  Низшая  форма,  или  монады  нижайшей  степени,
характеризуется «перцепцией» (пассивной способностью восприятия). Они способны образовывать неясные
представления. Монады высшей степени уже способны иметь ощущения и опирающиеся на них более ясные
представления. Эти монады Лейбниц определяет как монады-души. Монады наибольшей степени развития
способны к апперцепции (наделены сознанием). Их Лейбниц определяет как монады-духи.
Монады сами не имеют никаких пространственных (или каких-либо физических) характеристик, они, таким
образом,  не  являются  чувственно  постижимыми.  Мы  можем  их  постичь  лишь  разумом.  Чувственно  вос-принимаемые тела, т. е. соединения монад, различаются согласно тому, из каких монад они состоят. Тела,
содержащие лишь монады низшей степени развития (т. е. тела, в которых не содержатся монады, способ ные к
сознанию или ощущениям),-это тела физические (т. е. предметы неживой природы). Тела, в которых монады
способны к ощущениям и представлениям (содержат монады-души), являются биологическими объектами.
Человек  представляет  собой  такую  совокупность  монад,  в  которой  организующую  роль  играют  монады,
наделенные  сознанием.  Образование  совокупностей  монад  не  является  случайным.  Оно  определено
«предустановленной гармонией». При этом, однако, в каждой из монад потенциально заключена возможность
развития. Этим Лейбниц объясняет тот факт, что все монады постоянно изменяются, развиваются и при этом их 
развитие не «подвержено влиянию извне».
Отношение Лейбница к основным идеям сенсуалистской концепции познания более внимательное и осто-рожное, чем, например, отношение к ней Спинозы. Он не отвергает чувственного познания или роли опыта в
процессе познания. Он принимает главный тезис сенсуализма «ничего нет в разуме, что не прошло бы раньше
через чувства», но он дополняет его следующим положением — «кроме самого разума», т. е. врожденных
способностей к мышлению и образованию понятий или идей.
Чувственное  познание  выступает,  таким  образом,  как  определенная  низшая  ступень  или  предпосылка
рационального  познания.  Разумное,  рациональное  познание  раскрывает  действительное,  необходимое  и
существенное в мире, тогда как чувственное познание постигает лишь случайное и эмпирическое.
Философское мышление Лейбница представляет собой вершину европейской рационалистической философии.
Итак,  подводя  итог  вышесказанному,  мы  выделяем  следующие  характерные  черты  рационализма
новоевропейской философии:
- через достоверность мысли мыслящее существо идет к познанию окружающего мира;
- рационализм требует ясности и непротиворечивости мышления;
-  на  начальной  стадии  развития  неприятие  чувственного  метода  постижения  истины  как  такового,  с
последующим признанием, что он необходим как низшая ступень познания.


\newpage
\section{Традиция эмпиризма в философии Нового времени}
Европейскую философию 17 века условно принято называть философией Нового времени. Данный период
отличается неравномерностью социального развития. Так, например, а Англии происходит буржуазная
революция (1640–1688). Франция переживает период рассвета абсолютизма, а Италия вследствие победы
контрреформации оказывается надолго отброшенной с переднего края общественного развития. Общее
движение от феодализма к капитализму носило противоречивый характер и часто принимало драматические
формы. Расхождение между силой власти, права и денег приводит к тому, что сами жизненные условия для
человека становятся случайными.
В силу всего  перечисленного,  философия  Нового  времени  не  является тематически  и  содержательно
однородной, она представлена различными национальными школами и персоналиями. Но, несмотря на
все различия, сущность философских устремлений у всех одна: доказать, что между фактическим и
логическим положением дел существует принципиальное тождество. По вопросу о том, как реализуется
это тождество, существую две философские традиции: эмпиризм и рационализм.
Успешное  освоение  природы   в   рамках капиталистического способа производства было немыслимо без
развития наук о природе, а претворение в жизнь новых социально-политических идеалов предполагало
иную, по сравнению с теоцентризмом, модель человеческого участия в переустройстве мира. Новое время
входило  в  жизнь  и  развивалось под лозунгами  свободы,  равенства,  активности  индивида.

\subsection {Ф.Бэкон}
Главным
орудием реализации этих лозунгов явилось рациональное знание. Один из классиков философии Нового
времени,  Ф.Бэкон,  выразил  это  в  ставшем   знаменитым утверждении: «Знание есть сила, и  тот,  кто
овладеет знанием, тот будет могущественным».
Сын высокопоставленного английского сановника, Бэкон и сам стал со временем государственным канцлером
Англии. Главный  труд  Бэкона  – «Новый Органон»(1620).  Название это показывает, что  Бэкон сознательно
противопоставлял свое понимание науки и ее метода тому пониманию, на которое опирался «Органон» (свод
логических работ) Аристотеля. Другим важным сочинением Бэкона была утопия «Новая Атлантида».
В согласии с передовыми умами своего века Бэкон провозгласил высшей задачей познания завоевание природы
и усовершенствование человеческой жизни. Но это — практическое в последнем счете — назначение науки не
может, по Бэкону, означать, будто всякое научное исследование должно быть ограничено соображениями о его
возможной непосредственной пользе. Знание — сила, но действительной силой оно может стать, только если
оно истинно, основывается на выяснении истинных причин происходящих в природе явлений. Лишь та наука
способна побеждать природу и властвовать над ней, которая сама «повинуется» природе, т. е. руководится
познанием ее законов.
Поэтому Бэкон различает два вида опытов: 1) «плодоносные» и 2) «светоносные». Плодоносными он называет
опыты, цель которых — принесение непосредственной пользы человеку, светоносными — те, цель которых
не   непосредственная   польза,  а познание законов явлений и свойств вещей. Недостоверность известного
доселе знания обусловлена, по Бэкону, ненадежностью умозрительной формы умозаключения и доказательства.
Как  уже  говорилось,  творчество  Бэкона  характеризуется  определенным  подходом  к  методу  человеческого
познания и мышления. Исходным моментом любой познавательной деятельности являются для него прежде
всего  чувства.  Поэтому  его  часто  называют  основателем  эмпиризма  —  направления,  которое  строит  свои
гносеологические  посылки  преимущественно  на  чувственном  познании  и  опыте.  Принятие  этих
гносеологических  посылок  характерно  и  для  большинства  других  представителей  английской  философии
Нового времени. Основной принцип этой философской ориентации в области теории познания выражен в тезисе:
«Нет ничего в разуме, что бы до этого не прошло через чувства».
Понятия добываются обычно путем слишком поспешного и недостаточно обоснованного обобщения. Поэтому
первым условием реформы науки, прогресса знания является усовершенствование методов обобщения, образования понятий.
Так как процесс обобщения есть индукция, то логическим основанием реформы науки должна 
быть новая теория индукции.
До Бэкона философы, писавшие об индукции, обращали понимание главным образом на те случаи или
факты, которые подтверждают доказываемые или обобщаемые положения. Бэкон подчеркнул значение тех
случаев,   которые   опровергают обобщение, противоречат ему. Это так называемые негативные инстанции.
Уже  один-единственный  такой  случай  способен  полностью  или  по  крайней  мере  частично  опровергнуть
поспешное обобщение. По Бэкону, пренебрежение к отрицательным инстанциям есть главная причина ошибок,
суеверий и предрассудков.
Бэкон выставляет новую логику: «Моя логика, однако, существенно отличается от традиционной логики тремя
вещами: самой своей целью, способом доказательства и тем, где она начинает свое исследование. Целью моей
науки не является изобретение аргументов, но различные искусства; не вещи, что согласны с принципами, но
сами принципы; не некоторые правдоподобные отношения и упорядочения, но прямое изображение и описание
тел». Как видно, свою логику он подчиняет той же цели, что и философию. Основным рабочим методом своей
логики Бэкон считает индукцию. В этом он видит гарантию от недостатков не только в логике, но и во всем
познании  вообще.  Характеризует  он  ее  так:  «Под  индукцией  я  понимаю  форму  доказательства,  которая
присматривается к чувствам, стремится постичь естественный характер вещей, стремится к делам и почти с
ними сливается».
Бэкон,  однако,  останавливается  на  данном  состоянии  разработки  и  существующем  способе  использования
индуктивного подхода. Он отвергает ту индукцию, которая, как он говорит, осуществляется простым
перечислением. Такая индукция «ведет к неопределенному заключению, она подвержена опасностям, которые
ей угрожают со стороны противоположных случаев, если она обращает внимание лишь на то, что ей привычно,
и не приходит ни к какому выводу». Поэтому он подчеркивает необходимость переработки или, точнее говоря,
разработки индуктивного метода.
Условием реформы науки должно быть также очищение разума от заблуждений. Бэкон различает четыре вида
заблуждений, или препятствий, на пути познания — четыре вида   «идолов»   (ложных  образов)   или
призраков. Это — «идолы рода», «идолы пещеры», «идолы площади» и «идолы театра».
«Идолы рода» — препятствия, обусловленные общей для всех людей природой. Человек судит о природе по
аналогии с собственными свойствами. Отсюда возникает телеологическое представление о природе, ошибки,
проистекающие  из несовершенства   человеческих   чувств   под влиянием различных желаний, влечений.
«Идолы пещеры» — ошибки, которые присущи не всему человеческому роду, а только некоторым группам
людей вследствие субъективных предпочтений, симпатий, антипатий ученых: одни больше видят различия
между предметами, другие — их сходства; одни склонны верить в непогрешимый авторитет древности, другие,
наоборот, отдают предпочтение только новому. «Идолы  площади» — препятствия, возникающие вследствие
общения между людьми посредством слов. Во многих случаях значения слов были установлены не на
основе познания сущности предмета; а на основании совершенно случайного впечатления от этого предмета.
«Идолы театра» — препятствия, порождаемые в науке некритически усвоенными, ложными мнениями. «Идолы
театра» не врождены нашему уму, они возникают вследствие подчинения ума ошибочным воззрениям.
Этим Бэкон существенно способствовал формированию философского мышления Нового времени. И хотя
его  эмпиризм  был  исторически  и  гносеологически  ограничен,  а  с  точки  зрения  последующего  развития
познания его можно по многим  направлениям критиковать, он в свое время сыграл весьма положительную
роль.

\subsection{Д.Локк}
Продолжая  тему  развития  эмпиризма,  нельзя  не  упомянуть  представителя  английской  философской  линии
данного направления – Джона Локка. Главное внимание он уделяет проблематике познания. Уже в первой части
его  «Опыта  о  человеческом  разуме»  встречается  идея,  суть  которой  состоит  в  том,  что  предпосылкой
исследования всех разнообразнейших проблем является изучение способностей нашего собственного познания,
т. е. выяснение того, что оно может достичь, каковы его границы, а также каким образом оно получает знания о
внешнем мире.
Для  философских  и  гносеологических  воззрений  Локка  характерным  является  подчеркивание  чувственно
постигаемой эмпирии. Гегель (который, естественно, такой способ философствования оценивал не слишком
высоко)  подчеркивал,  что  «Локк  указал  на  то,  что  общее,  и  мышление  вообще,  покоится  на  чувственно
воспринимаемом сущем,  и  указал, что общее  и  истину  мы  получаем  из опыта».  В  признании приоритета
чувственного познания Локк близок эмпиризму Бэкона.
Локк  не  исключает  в  целом  роль  разума  (как  иногда  упрощенно  представляется),  но  признает  за  ним  в
«достижении истины» еще меньше простора, чем Т. Гоббс. В сущности роль разума он ограничивает лишь,
говоря нынешними терминами, простыми эмпирическими суждениями.
Философию Локка можно характеризовать как учение, которое прямо направлено против рационализма Декарта
(и  не  только  против  Декарта,  но  и  во  многом  против  систем  Спинозы  и  Лейбница).  Локк  отрицает
существование «врожденных идей», которые играли такую важную роль в теории познания Декарта, и концепцию
«врожденных принципов» Лейбница, которые представляли собственно некую естественную потенцию
понимания идей.
Человеческая мысль (душа), согласно Локку, лишена всяких врожденных идей, понятий, принципов либо еще
чего-нибудь  подобного.  Он  считает  душу  чистым  листом  бумаги  (tabula  rasa).  Лишь  опыт  (посредством
чувственного познания) этот чистый лист заполняет письменами.
Локк понимает опыт прежде всего как воздействие предметов окружающего мира на нас, наши чувственные 
органы. Поэтому для него ощущение является основой всякого познания. Однако в соответствии с одним из
своих  основных  тезисов  о  необходимости  изучения  способностей  и  границ  человеческого  познания  он
обращает внимание и на исследование собственно процесса познания, на деятельность мысли (души). Опыт,
который мы приобретаем при этом, он определяет как «внутренний» в отличие от опыта, обретенного при
посредстве восприятия чувственного мира. Идеи, возникшие на основе внешнего опыта (т. е. опосредованные
чувственными восприятиями), он называет чувственными (sensations); идеи, которые берут свое происхождение
из  внутреннего  опыта,  он  определяет  как  возникшие  «рефлексии».  Однако  опыт  —  как  внешний,  так  и
внутренний — непосредственно ведет лишь к возникновению простых (simple) идей. Для того, чтобы наша
мысль (душа) получила общие идеи, необходимо размышление. Размышление не является сущностью души
(мысли), но лишь ее свойством.
Размышление,  в  понимании  Локка,  является  процессом,  в  котором  из  простых  (элементарных)  идей
(полученных на основе внешнего и внутреннего опыта) возникают новые идеи, которые не могут появиться
непосредственно на основе чувств или рефлексии. Сюда относятся такие общие понятия, как пространство,
время и т. д.

Таким образом, мы рассмотрели основные черты эмпиризма философии Нового времени, которые заключаются
в следующем:
\begin{itemize}
\item исключительная значимость и необходимость наблюдений и опыта в обнаружении истины;
\item путем, ведущим к знанию, является наблюдение, анализ, сравнение, эксперимент;
\item исключительно все знания черпаются из опыта, ощущений.
\end{itemize}


\newpage
\section{Трансцендентальная философия Канта: структура познания и априоризм}
Главным философским произведением Канта является «Критика чистого разума». Исходной проблемой для
Канта  является  вопрос  «Как  возможно  чистое  знание?»  Прежде  всего,  это  касается  возможности  чистой
математики и чистого естествознания («чистый» означает «неэмпирический», то есть такой к которому не
примешивается  ощущение).  Указанный  вопрос  Кант  формулировал  в  несколько  неудачных  терминах
различения аналитических и синтетических суждений — «Как возможны синтетические суждения априори?»
Под  «синтетическими»  суждениями  Кант  понимал  суждения  с  приращением  содержания,  по  сравнению  с
содержанием  входящих в  суждение понятий, которые  отличал от  аналитических суждений, раскрывающих
смысл самих понятий. Термин «априори» означает «вне опыта», в противоположность термину «апостериори»
— «из опыта». Кант, вслед за Юмом, соглашается, что если наше познание начинается с опыта, то его связь —
всеобщность и необходимость не из него. Однако, если Юм из этого делает скептический вывод о том, что связь
опыта является всего лишь привычкой, то Кант эту связь относит к необходимой априорной деятельности
сознания. Выявлением этой деятельности сознания в опыте Кант называет трансцендентальным исследованием.
Вот как об этом пишет сам Кант: «Я называю трансцендентальным всякое познание, занимающееся не столько
предметами, сколько видами нашего познания предметов, поскольку это познание должно быть возможным
априори». Кант называет свою философию «критической» в противоположность догматической философии,
которая оставляет нерешенным вопрос возможности знания. Кант, по его словам, совершает Коперниканский
переворот в философии, тем, что первый указывает, что для обоснования возможности знания следует признать,
что не наши познавательные способности должны сообразовываться с миром, а мир должен сообразоваться с
нашими способностями, чтобы вообще могло состояться познание. Иначе говоря, наше сознание не просто
пассивно постигает мир как он есть на самом деле (догматизм), как бы это можно было доказать и обосновать?
Но  скорее,  наоборот,  мир  сообразуется  с  возможностями  нашего  познания,  а  именно:  сознание  является
активным  участником  становления  самого  мира,  данного  нам  в  опыте.  Опыт  по  сути  есть  синтез  того
содержания, материи, которое дается миром (ощущений) и той субъективной формы, в которой эти ощущения
постигаются сознанием. Единое синтетическое целое материи и формы Кант и называет опытом, который по
необходимости становится чем-то только субъективным. Именно поэтому Кант различает мир как он есть сам
по себе (то есть вне деятельности формирования сознания) — вещи-в-себе и мир как он дан в явлении, то есть в
опыте. В опыте выделяется два уровня формообразования (активности) сознания:
\begin{enumerate}
\item Субъективные формы
чувства — пространство и время. В созерцании, чувства (материя) постигаются нами в формах пространства и
времени, и тем самым опыт чувства становится чем-то необходимым и всеобщим. Это чувственный синтез.

\item
Категории рассудка, благодаря которому связываются созерцания. Это рассудочный синтез.
\end{enumerate}
Основой всякого
синтеза  является,  согласно  Канту,  самосознание  —  единство  апперцепции  (Лейбницевский  термин).  В
«Критике» много места уделяется тому, как понятия рассудка подводятся под представления. Здесь решающую
роль играет воображение и рассудочный категориальный схематизм. Наконец, описав эмпирическое применение
рассудка, Кант задается вопросом возможности чистого применения рассудка, которое он называет разумом.
Здесь возникает новый вопрос: «Как возможна метафизика?». В результате исследования чистого разума Кант
доказывает, что разум не может иметь конститутивного значения, то есть основывать на самом себе чистое
знание, которое должно было бы составить чистую метафизику, поскольку «запутывается» в паралогизмах и
неразрешимых антиномиях (противоречиях, каждое из утверждений которого одинаково обосновано), но только
регулятивное значение — как систему принципов, которым должно удовлетворять всякое знание. Собственно,
всякая будущая метафизика, согласно Канту, должна принимать во внимание его выводы.

Кант выделяет следующие категории рассудка:
\begin{enumerate}
\item Категории количества
    \begin{enumerate}
        \item Единство
        \item Множество
        \item Цельность
    \end{enumerate}
    
\item Категории качества
    \begin{enumerate}
        \item Реальность
        \item Отрицание
        \item Ограничение
    \end{enumerate}
\item Отношения
    \begin{enumerate}
        \item Субстанция и принадлежность
        \item Причина и следствие
        \item Взаимодействия
    \end{enumerate}
\item Категории модальности
    \begin{enumerate}
        \item Возможность и невозможность
        \item Существование и несуществование
        \item Предопределённость и случайность
    \end{enumerate}
\end{enumerate}

Знание даётся путём синтеза категорий и наблюдений. Кант впервые показал, что наше знание о мире не
является  пассивным  отображением  реальности,  а  является  результатом  активной  творческой  деятельности
сознания.

\subsection{Этическое учение Канта}
Этическое учение Канта изложено в «Критике практического разума». Этика Канта основана на принципе «как
если бы». Бога и свободу невозможно доказать, но надо жить как если бы они были. Практический разум — это
совесть,  руководящая  нашими  поступками  посредством  максим  (ситуативные  мотивы)  и  императивов
(общезначимые правила). Императивы бывают двух видов: категорические и гипотетические. Категорический
императив — требует соблюдения долга. Гипотетический императив — требует, чтобы наши действия были
полезны. Существует две формулировки категорического императива:
\begin{itemize}
\item «Поступай всегда так, чтобы максима (принцип) твоего поведения могла стать всеобщим законом (поступай
так, как ты бы мог пожелать, чтобы поступали все)»;
\item «Относись к человечеству в своем лице (так же, как и в лице всякого другого) всегда только как к цели и
никогда – как к средству».
\end{itemize}
В этическом учении человек рассматривается с двух точек зрения:
\begin{itemize}
\item Человек как явление;
\item Человек как вещь в себе.
\end{itemize}
Поведение  первого детерминировано исключительно  внешними факторами  и  подчиняется  гипотетическому
императиву. Второй — категорическому императиву — высшему априорному моральному принципу. Таким
образом, поведение может определяться практическими интересами и моральными принципами. Возникает два
тенденции: стремление к счастью (удовлетворению некоторых материальных потребностей) и стремление к
добродетели. Эти стремления могут противоречить друг другу и возникает «антиномия практического разума».


\newpage
\section{Система объективного идеализма и диалектический метод Гегеля}
Высшей  ступени  своего  развития  диалектика  в  идеалистической  форме  достигла  в  философии  Гегеля
(1770-1831), который был великим представителем объективного идеализма. Гегелевская система объективного
идеализма состоит из трех основных частей.
В первой части своей системы - в "Науке логики" - Гегель изображает мировой дух (называемый им здесь
"абсолютной идеей") таким, каким он был до возникновения природы, т.е. признает дух первичным.
Идеалистическое учение о природе изложено им во второй части системы - в "Философии природы". Природу
Гегель как идеалист считает вторичной, производной от абсолютной идеи.
Гегелевская идеалистическая теория общественной жизни составляет третью часть его системы - "Философию
духа". Здесь абсолютная идея становиться по Гегелю "абсолютным духом".
Таким образом, система взглядов Гегеля носила ярко выраженный идеалистический характер. Существенная
позитивная особенность идеалистической философии Гегеля состоит в том, что абсолютная идея, абсолютный
дух  рассматривается  им  в  движении,  в  развитии.  Учение  Гегеля  о  развитии  составляет  ядро  гегелевской
идеалистической диалектики и целиком направлено против метафизики. Особенное значение в диалектическом
методе  Гегеля  имели  три  принципа  развития,  понимаемые  им  как  движение  понятий,  а  именно:  переход
количества в качество, противоречие как источник развития и отрицание отрицания.
В этих трех принципах, хотя и в идеалистической форме, Гегель вскрыл всеобщие законы развития. Впервые в
истории философии Гегель учил, что источником развития являются противоречия, присущие явлениям. Мысль
Гегеля о внутренней противоречивости развития была драгоценным приобретением философии.
Выступая  против  метафизиков,  рассматривавших  понятия  вне  связи  друг  с  другом,  абсолютизировавших 
анализ, Гегель выдвинул диалектическое положение о том. что понятия взаимосвязаны между собой. Таким
образом, Гегель обогатил философию разработкой диалектического метода. В его идеалистической диалектике
заключалось глубокое рациональное отражение. Рассматривая основные понятия философии и естествознания,
он в известной мере диалектически подходил к истолкованию природы, хотя в своей системе он и отрицал
развитие природы во времени.
В  Гегелевской  философии  существует  противоречие  между  метафизической  системой  и  диалектическим
методом. Метафизическая система отрицает развитие в природе, а его диалектический метод признает развитие,
смену одних понятий другими, их взаимодействие и движение от простого к сложному.
Развитие общественной жизни Гегель видел лишь в прошлом. Он считал, что история общества завершится
конституционной  сословной  прусской  монархией,  а  венцом  всей  истории  философии  он  объявил  свою
идеалистическую систему объективного идеализма.
Так система Гегеля возобладала над его методом. Однако в гегелевской идеалистической теории общества
содержится много ценных диалектических идей о развитии общественной жизни. Гегель высказал мысль о
закономерностях  общественного  прогресса.  Гражданское  общество,  государство,  правовые,  эстетические,
религиозные,  философские  идеи,  согласно  гегелевской  диалектике,  прошли  длинный  путь  исторического
развития.
Если  идеалистическая  система  взглядов  Гегеля носила  консервативный  характер, то  диалектический метод
Гегеля  имел  огромное  положительное  значение  для  дальнейшего  развития  философии,  явился  одним  из
теоретических источников диалектико-материалистической философии.
Таким образом, историческая роль философских учений немецких философов конца 18 - начала 19 века, в
особенности  Гегеля,  состояла  в  развитии  этими  выдающимися  мыслителями  диалектического  метода.  Но
идеалистические  умозрительные  системы  немецких  философов  в  интересах  дальнейшего  развития
философской мысли требовалось преодолеть, удержав то ценное, что в них содержалось. Это в значительной
степени было достигнуто Л. Фейербахом (1804-1872).


\newpage
\section{Идеалистическое и материалистическое понимания истории}


\newpage
\section{Позитивизм: основные идеи учения и этапы эволюции}
Позитивизм  (лат.  positivus  —  положительный)  в  качестве  главной  проблемы  рассматривает  вопрос  о
взаимоотношении  философии  и  науки.  Главный  тезис  позитивизма  состоит  в  том,  что  подлинное
(положительное) знание о действительности может быть получено только лишь конкретными, специальными
науками.
Первая историческая форма позитивизма возникла в 30-40 г. XIX века как антитеза традиционной метафизике в
смысле философского учения о началах всего сущего, о всеобщих принципах бытия, знание о которых не может
быть дано в непосредственном чувственном опыте. Основателем позитивистской философии является Огюст
Конт (1798-1857), французский философ и социолог, который продолжил некоторые традиции Просвещения,
высказывал убеждение в способности науки к бесконечному развитию, придерживался классификации наук,
разработанной энциклопедистами.
Кант утверждал, что всякие попытки приспособить «метафизическую» проблематику к науке обречены на
провал, ибо наука не нуждается в какой-либо философии, а должна опираться на себя. «Новая философия»,
которая должна решительно порвать со старой, метафизической («революция в философии») своей главной
задачей должна считать обобщение научных данных, полученных в частных, специальных науках.
Вторая историческая форма позитивизма (рубеж XIX-XX вв.) связана с именами немецкого философа Рихарда
Авенариуса (1843-1896) и австрийского физика и философа Эрнста Маха (1838-1916). Основные течения —
махизм и эмпириокритицизм. Махисты отказывались от изучения внешнего источника знания в противовес
кантовс-кой идеи «вещи в себе» и тем самым возрождали традиции Беркли и Юма. Главную задачу философии
видели не в обобщении данных частных наук (Конт), а в создании теории научного познания. Рассматривали
научные  понятия  в  качестве  знака  (теория  иероглифов)  для  экономного  описания  элементов  опыта  —
ощущений.
В 10-20 гг. XX века появляется третья форма позитивизма — неопозитивизм или аналитическая философия,
имеющая несколько направлений.
Логический позитивизм или логический эмпиризм представлен именами Мори-ца Шлика (1882-1936), Рудольфа
Карнапа  (1891-1970)  и  других.  В  центре  внимания  проблема  эмпирической  осмысленности  научных
утверждений.  Философия,  утверждают  логические  позитивисты,  не  является  ни  теорией  познания,  ни
содержательной наукой о какой-либо реальности. Философия — это род деятельности по анализу естественных
и  искусственных  языков.  Логический  позитивизм  основывается  на  принципе  верификации  (лат.  verus  —
истинный; facere — делать), который означает эмпирическое подтверждение теоретических положений науки
путем  сопоставления  их  с  наблюдаемыми  объектами,  чувственными  данными,  экспериментом.  Научные
утверждения,  не  подтвержденные  опытом,  не  имеют  познавательного  значения,  являются  некорректными. 
Суждение о факте называется протоколом или протокольным предложением. Ограниченность верификации
впоследствии выявилась в том, что универсальные законы науки не сводимы к совокупности протокольных
предложений. Сам принцип проверяемости также не мог быть исчерпаем простой суммой Какого-либо опыта.
Поэтому сторонники лингвистического анализа—другого влиятельного направления неопозитивизма Джордж
Эдуард Мур (1873-1958) и Людвиг Витгенштейн (1889-1951), принципиально отказались от верификационной
теории значения и некоторых других тезисов.
Четвертая  форма  позитивизма  —  постпозитивизм  характеризуется  отходом  от  многих  принципиальных
положений  позитивизма.  Подобная  эволюция  характерна  для  творчества  Карла  Поппера  (1902-1988),
пришедшего к выводу, что философские проблемы не сводятся к анализу языка. Главную задачу философии он
видел в проблеме демаркации— разграничении научного знания от ненаучного. Метод демаркации основан на
принципе фальсификации, т.е. принципиальной опровержимости любого утверждения, относящегося к науке.
Если утверждение, концепция или теория не могут быть опровергнуты, то они относятся не к науке, а к
религии. Рост научного знания заключается в выдвижении смелых гипотез и их опровержении.


\newpage
\section{Иррационализм Шопенгауэра. Мир как представление и воля}
Для Артура Шопенгауэра, который наряду с Кьеркегором является основоположником иррационализма XIX
века, Гегель также был живым и едва ли не главным врагом.
Система Шопенгауэра изложена в 4 книгах его основного труда "Мир как воля и представление".
У Шопенгауэра основой и животворящим началом всего является не познавательная способность и активность
человека, а в о л я как слепая, бессознательная жизненная сила. Тем самым в "человеке разумном", в homo
sapiens, разум перестал считаться его родовой сущностью; ею становилась неразумная воля, а разум начинал
играть второстепенную, служебную роль.
Жизнь, жизненная сила, волевое напряжение - вот что вышло на авансцену, оттеснив интеллект, рацио на
задний план.
Наука, по Шопенгауэру, никогда не может обрести конечной цели, однако есть сфера, которая рассматривает
"единственную действительную сущность мира" - это ИСКУССТВО. Шопенгауэр говорит, что"обыкновенный
человек, этот фабричный товар природы" не способен на незаинтересованное созерцание так же, как и ученый,
и только гений способен на это. Искусство есть создание гения, и гений возможен только в искусстве. Искусство
воспроизводит постигнутые чистым созерцанием вечные идеи.
Согласно Шопенгауэру, высшее из искусств - это МУЗЫКА, имеющая своей целью уже не воспроизведение
идей, а непосредственное отражение самой воли.
Шопенгауэр  является  прямым  предшественником  философии  жизни  -  иррационалистического  течения
философии конца XIX - начала XX века, основными представителями которого являются Ницше, Дильтей,
Зиммель, Шпенглер, Бергсон и др.
Он не очень любил людей, трудно сходился, был самолюбив. Полная не признанность при жизни (чем-то похож
на Гераклита, хотя в горы Шопенгауэр не пошел ?). Получил большое наследство, занялся «мыслительством».
Последние 10 лет (1845-1846) к Шопенгауэру начинает приходить знаменитость. Произошло следующее: 40-е
годы – годы революций (Французской и Немецкой), которые закончились поражением. Народ приуныл, а тут
третье  издание  книги  Шопенгауэра.  Прямо  в  унисон  настоящему  времени,  кризиса  и  безысходности,
пессимизма. Шопенгауэр говорит, что это естественное состояние жизни. «Великий пессимист», «Философ
мировой скорби».  Мировоззрение Шопенгауэра формируется из трех источников (трех мировоззрений):
Грек Платон (с его миром идей и миром вещей, которая проникнута духом аристократизма, презрения к толпе)
Философия Канта
Философия древней Индии (философия Упанишады)
Общая  их  идея:  утверждается,  что  разум  (наше  мышление)  постигает  только  поверхность  (часть)  бытия.
Подлинные причиныпроисходящего всегда скрыты от нас. Есть иная (высшая) реальность, которая определяет
нашу жизнь (поэтому мы и терпим войны, неудачи и т.д.) Мы никогда не сможем вычислить и усмотреть всех
последствий, так как есть то, что нельзя сосчитать ложкой, …, разумом.
Фихте, Гегель, Шеллинг – три прославленных софиста (~болтуна) послекантовского учения, которые извратили
сущность его учения. Эти три глупца, помешавшиеся на разуме, завели философию в тупик. Но они остаются
правыми, ибо они призваны к философии министерством, а я – природой.
Кант ограничил познавательные возможности человека только разумом, только рациональностью. /* «Кто ясно
мыслит, тот ясно излагает» */. Такая философия (такой подход) лишает нас понимания жизни, так как она
(жизнь) не только рационально познаваема.
Шопенгауэр утверждает: существует иная способность, иной путь познания, интуиция.
Интуиция – реальная способность познавательной человеческой деятельности. /* Спиноза: везде логика, но
основные  положения  я  беру  в  виде  неких  аксиом,  очевидности.  Форма  их  данности  –  интуиция.  Она
нерациональна. */
Что же позволяет интуиция увидеть за завесой? За нашими представлениями (интуитивно открывается, что
мир) есть то, что в интуитивной данности он мне дан, как воля. «Вещь в себе» Канта и есть воля.
У А.Шопенгауэра термин воля относится, прежде всего, не к человеку и его психологии (сознанию) /* У нас
воля – психологическая характеристика человека */. Под волей он понимает некую основу нашего бытия.
Мировая воля: рационально мы о ней ничего не знаем. 
«Мир есть проявление мировой воли, или ее действие (динамика)». Мировая воля дана нам как динамика
(вечное движение и изменение) мироздания. О чем бы мы не стали думать, много у вещей не совпадающих
свойств, но у всего в мире есть одно общее свойство – они прибывают в вечном движении. Эта динамика есть
мировая воля в своем проявлении. Это изменение называется мировой волей.
У человека воля впервые приобретает осознанную форму.
Три ступени воли:
\begin{enumerate}
\item Силы природы (тяготение, текучесть, электричество, магнетизм)
\item Силы жизни (в жизни мировая воля проявляется как воля к жизни). Она самопротиворечива, так как пожирает
сама себя и в разных видах служит своею собственной пищей. /* В камне есть сила природы, но нет силы жизни
*/
\item Человек – это есть  тоже воля к жизни (самосохранение, самовоспроизведение),– но она уже всеядна и
пользуется (пожирает) уже все. Человек есть осознанная воля к жизни.
\end{enumerate}
Вершина всего – разум или интеллект человека (которая тоже есть воля).
?: Похоже на теорию эволюции???
Шопенгауэр: Фундаментальное свойство мировой воли в том, что она бесцельна => она бессмысленна!
«Воля есть бесконечное стремление самоуничтожения и самовоспроизведения вновь».
Человек с его интеллектом – это крайний случай в движении мировой воли. Не было у природы задачи создать
человека, просто в хаосе воли сошлись 1000 причин.
В мировоззрении Шопенгауэра не предполагается никакого высшего разума, бога, плана. Человек не продукт
эволюции, так как все случайно. Тот факт, что, планируя что-либо, мы, зачастую, терпим неудачи, служит
подтверждением отсутствия какого-то ни было направления.
Человек выше ее… Но он не выражает ее сущности, ее идеи.
Шопенгауэр: Что такое мир, в котором мы живем? Во-первых, мир есть мое представление, образованное по
законам моего сознания; и во-вторых, мир, на самом деле, есть беспричинная, бессознательная воля.
Что указывает человеку на существование мировой воли? Шопенгауэр приводит четыре аргумента:
\begin{enumerate}
\item Ссылается на Шеллинга (один из бездарных пачкунов). Он говорил о бессознательной основе бытия, а это
рационалист Шеллинг. 
\item Логический: сознание  – есть продукт естественного развития  организма,  оно  формируется жизненными
потребностями, но если это так, то тогда … но не из этих потребностей, не из сознания. Сформированного ими
нельзя объяснить стремление человека к духовному творчеству, ибо если  сознание есть приспособительный
механизм, то как объяснить избыточные способности? Здесь проявляется мировая воля (творчество из хаоса). 
\item Наблюдение  жизни.  Все  мы  стремимся  в  жизни  к  определенным  целям.  Но  все  планы  и  проекты
заканчиваются неудачей. Проекты разные, а результат отрицателен. => основа жизни не рациональна. 
\item Существование музыки. Музыка есть наиболее непосредственное выражение мировой воли.
\end{enumerate}
Вопрос: В чем сущность и положение человека? 
Шопенгауэр прибегает к метафоре дерева: воля – это корни; наша жизнь (с нашим интеллектом) – это то, что
над землей, ствол и крона. Мы, как дерево, не подозреваем о своих корнях. Участие воли или ее постоянное
присутствие, в конечном счете, деформирует и искажает работу интеллекта. => (в итоге) наш разум (интеллект)
дают временный успех. Интеллект слаб и беспомощен на фоне этой стихии – мировой воли.
Наше научное познание носит частичный и относительный характер, такое познание все равно, в конечном
итоге, не даст власти над обстоятельствами. Человек одновременно несчастен и безнравственен. Он несчастен,
ибо не понимает, не видит, цели и смысла своих поступков; одновременно он безнравственен, ибо увеличивает
страдание в мире, так как, в конечном счете, каждый из нас подчинен слепому безумию мировой воли.
«Мы здесь никому ненужные. Мы появились случайно, и ничего у нас не получится».
Позиция Шопенгауэра: Раз мы появились, то поскольку мы люди, постольку мы должны стремиться быть
людьми.  Все  свои  способности  направить  хоть  и  на  бесцельное,  но  самоутверждение  в  этом  мире.  Есть
культурные механизмы, которые помогают существовать в этом безумии. Надо находить тихие гавани.
\begin{itemize}
\item 1 путь – эстетический. Это путь творчества. Способ, который изобрело человечество – гавань, которую мы 
строим сами. Это дом защищает нас на некоторое время от мирового хаоса. Необязательно быть художником,
так как тот, кто воспринимает произведения искусства, тот тоже пребывает в акте творчества.
\item 2 путь – этический. Это путь, на котором человек стремиться максимально уединиться от мира, в котором много
проблем. В частности, это путь религиозного отшельника или одинокого философа. На таком пути вся суета
отдаляется,  но  он  (этот  путь)  для  избранных.  Для  большинства  жизнь  навсегда  остается  наполненной
неурядицами и т.п. А впереди – только смерть.
\end{itemize}
Философия и религия создают иллюзию жизненных ценностей и иллюзию смысла жизни, а потому,– спасибо
им. Философия и религия облегчают наше существование.
Философия ничего нам не разъясняет. Цель любой философии сделать жизнь выносимой. Философия есть
стремление познать  сквозь завесу  представления то,  что не есть  представление,  но ведь  не философия,  в
конечном счете, заставляет нас жить в любой ситуации?! Жить нас заставляет Воля, проявляющая себя как воля
к жизни. 
Воля – есть первая реальность или наше исходное состояние.
Шопенгауэр ярко проявляет свой иррационализм. «Человеческая жизнь начинается с желания и стремления. /*
Основатель  рационализма  нового  времени  –  Декарт:  «Я  мыслю,  значит  существую».  Иррационализм
Шопенгауэра: «Я желаю, хочу, стремлюсь. Значит, существую». */
А.Шопенгауэр – философ мировой скорби. Но это не унылая, не безысходная скорбь, скорее она напоминает
позицию античных стоиков.
Люди  страдают.  Страдание  порождают  вражду  и  злобу.  Чем  больше  люди  страдают,  тем  больше  злобы.
Противопоставить этому можно только одно: наше стремление к духовности и духовной культуре. 
/* Культура – это средство обуздания злобно-эгоистического начала в человеке. */
Парадокс: Шопенгауэр – классик современного европейского иррационализма, но, с другой стороны, в своих
сочинениях он выступает как чистый рационалист. (Его трактаты строги, без поэтических вольностей. Он
понятен.) /* Популярным в России Шопенгауэр становится в конце XIX-го, в начале XX-го века. */
Если мировая воля не имеет никакой логики, то и человеческая история один из случайных векторов. «История
– смутный кошмар человечества» => такой науки, как истории быть не может, ведь не может быть истории
хаоса! То, что читают в Университете есть иллюзия истории, набор сказок.
А.Шопенгауэр – великий скептик, элитарист. Не питает никаких иллюзий на счет человека. Государство –
жесткий намордник, который необходимо надевать, чтобы привести человечество к определенному порядку.
Государство должно быть наиболее жестким.


\newpage
\section{Философия Ницше: нигилизм и проблема переоценки ценностей}
Ницше  выступает  как  "радикальный  нигилист"  и  требует  кардинальной  переоценки  ценностей  культуры,
философии,  религии.  "Европейский  нигилизм"  Ницше  сводит  к  некоторым  основным  постулатам,
провозгласить которые с резкостью, без страха и лицемерия считает своим долгом. Эти тезисы: ничто больше не
является истинным; бог умер; нет морали; все позволено. Надо точно понять Ницше — он стремится, по его
собственным словам, заниматься не сетованиями и моралистическими пожеланиями, а "описывать грядущее",
которое не может не наступить. По его глубочайшему убеждению (которое, к сожалению, никак не опровергнет
история заканчивающегося XX в.), нигилизм станет реальностью по крайней мере для последующих двух
столетий.  Европейская  культура,  продолжает  Ницше  свое  рассуждение,  издавна  развивается  под  игом
напряжения, которое растет от столетия к столетию, приближая человечество и мир к катастрофе. Себя Ницше
объявляет "первым нигилистом Европы", "философом нигилизма и посланцем инстинкта" в том смысле, что он
изображает  нигилизм  как  неизбежность,  зовет  понять  его  суть.  Нигилизм  может  стать  симптомом
окончательного упадка воли, направленной против бытия. Это "нигилизм слабых". "Что дурно? — Все, что
вытекает  из  слабости"  ("Антихрист".  Афоризм  2).  А  "нигилизм  сильных"  может  и  должен  стать  знаком
выздоровления, пробуждения новой воли к бытию. Без ложной скромности Ницше заявляет, что по отношению
к "знакам упадка и начала" он обладает особым чутьем, большим, чем какой-либо другой человек. Я могу,
говорит о себе философ, быть для других людей учителем, ибо знаю оба полюса противоречия жизни; я и есть
само это противоречие... А то, что его философия, не понятая эпохой, принадлежит к числу "несвоевременных
размышлений", никого  не должно  смущать, ибо нет ничего  более  своевременного,  чем умение мыслителя
преодолеть свое время, диктат его ценностей. К переоценке ценностей Ницше звал своих читателей уже в
ранних работах. Так, "Человеческое, слишком человеческое" он начинает в Искренней, исповедальной манере.
Ницше рассказывает о своем духовном становлении, о страстном увлечении Вагнером и Шопенгауэром и столь
же страстном отказе от их (и других мыслителей) идей и доктрин. А это порождает вопрос, который Ницше
обращает к себе и к своим читателям: "...сколько лживости мне еще нужно, чтобы сызнова позволить себе
роскошь  моей  правдивости?".  В  чем  же  удел  мыслителя,  отказавшегося  от  лжи,  фальши  устаревших,
догматизированных  воззрений?  Стать  из-за  переоценки  ценностей  унылым,  лишенным  чувства  юмора
философом и морализатором-одиночкой? Нет, отвечает Ницше. Везде рождаются, хотя и в великих муках и
постепенно, "свободные умы" и обновленные души. Они движутся навстречу друг другу. "Какие узы крепче
всего?  Какие  путы  почти  неразрывны?  У  людей  высокой  избранной  породы  то  будут  обязанности  —
благоговение,  которое  присуще  юности,  и  нежность  ко  всему,  издревле  почитаемому  и  достойному, 
благодарность  почве,  из  которой  они  выросли,  руке,  которая  их  вела,  храму,  в  котором  они  научились
поклоняться...". Но потом приходит тяготение к "великому разрыву", выраженному в виде тревожного вопроса:
"...Нельзя ли перевернуть все ценности? и, может быть, добро есть зло? а Бог — выдумка и ухищрение дьявола?
И может быть, в последней основе все ложно? И если мы обмануты, то не мы ли, в силу того же самого, и
обманщики?". Намеченная здесь идея переоценки ценностей духовной аристократией нового типа развита в
последующих произведениях, особенно в "Заратустре".


\newpage
\section{Особенности развития и характерные черты русской философии}
Фил. мысль в Р. формировалась под влиянием общемировой фил. Однако специф Р фил во многом складывалась
под влиянием социально културных процессов, происходивших на Руси. Христианизация Р. сыграла огромную
роль  в  становлении  рус.  фил  мысли.  Поиски  рус  ф-й  мысли  (16-18в)  проходили  в  противоборстве  2-х
тенденций: 1)акцентировала вним-е на самобыт с неповторимым своеобраз-м рус дух жизни 2) выражала стрем-е вписать Р в процесс раз-я европ культуры, представит считали что поскольку Р встала на путь культ раз-я
позже др стран то она должна уч-ся у запада и пройти тот же истор путь. Своеобр направлением в Р фил
явились воззрения славянофилов Хомякова и Киреевского.  В центре их внимания судьба Р и ее роль в мир
истор процессе. В самобытности истор прошлого они видели залог всечеловеч. призвания Р., тем более, что по
их мнению, зап культура уже завершила круг своего развития и клонится к упадку, что выраж в порожденном
ею  чувстве  обманутой  надежды  и  безотрадной  пустоты.  Славяноф.  развивали  основанное  на  религиозных
представлениях учение о чел и обществе. Достижение  целостности чел и связанное с этим обновление общ
жизни они видели в идее общины, дух основа которой - церковь. 
Основные черты и проблемы русской философии 19 - начала 20 в.
- Начало: Достоевский, Толстой, Соловьев.
Эти писатели выдвинули и сформулировали идею о связи прогресса с христианской нравственностью. Только
то может быть принято обществом что прошло чистилище нравственности.
- Предельный гуманизм.
Знаменитое непротивление злу насилием. / Толстой /
Все достижения прогресса за слезу ребенка. / Достоевский/ 
- Вето на любой прогресс, если он связан с насилием над человеком.
Насилие понимается как превращение человека в производственную машину за счет удушения уникальной,
божественной души.
- Литература: жанр антиутопии.
Будущее основано на любви и искусстве и потому прекрасно. Пример: сны Веры Павловны. Чернышевский
"Что делать"
- Сверхнационалистический характер философствования. 
Включение всего человечества в ожидаемый прогресс прекрасного будущего.
- Постижение сущего дается лишь цельной жизнью духа, лишь в полноте жизни.
Полностью православно-христианский подход к человеку и его месту в природе. Есть Бог, есть человек и
божественное  предназначение  последнего.  Выполнить  это  предназначение  его  задача.  Задача  решается
обращением к Богу, вере, христианской нравственности.
- Антипод русской философии - европейский рационализм - Гегель.
- Русский иррационализм (вера, православие) VS западный (воля) -> деление на славянофилов и западников.
-  Славянофилы  (Киреевский,  Хомяков)  -  самобытный  путь  развития  России.  Земельная  община  и  артель.
Западные приобретения - скорее вред, так как оплачены потерей целостности человеческой личности.
- Западники (Герцен, Грановский, Боткин) - Россия с петровских времен привязана к западу. Прогресс не может
обойти Россию, прогресс поможет нам выбраться из отсталости.
- Примат христианского откровения, веры над рациональным знанием.
- Философия всеединства (Соловьев). Теория цельного знания. Сверхрационализм. Цельность - характеристика
и свойство человеческой души, отличающая человека от животных. 


\newpage
\section{Диалог славянофилов и западников в русской философии и культуре XIX века}
Первыми представителями «органической русской философии» были западники и славянофилы.
К западникам относятся: П.Л. Чаадаев, А.Л. Герцен, Т.М. Грановский, Н.Г. Чернышевский, В.П. Боткин и др.
Основная  идея  западников  заключается  в  признании  европейской  культуры  последним  словом  мировой
цивилизации, необходимости полного культурного воссоединения с Западом, использования опыта его развития 
для процветания России.
Особое место в русской философии XIX в. вообще, а в западничестве в частности занимает П.Я. Чаадаев,
мыслитель,  сделавший  первый  шаг  в  самостоятельном  философском  творчестве  в  России  XIX  столетия,
положивший начало идеям западников. Свое философское миропонимание он излагает в «Философических
письмах» и в работе «Апология сумасшедшего».
По-своему  понимал  Чаадаев  и  вопрос  о  сближении  России  и  Запада.  Он  видел  в  этом  сближении  не
механическое  заимствование  западноевропейского  опыта,  а  объединение  на  общей  христианской  основе,
требующей реформации, обновления православия. Это обновление Чаадаев видел не в подчинении православия
католицизму,  а  именно  в  обновлении,  освобождении  от  застывших  догм  и  придании  религиозной  вере
жизненности и активности, чтобы она могла способствовать обновлению всех сторон и форм жизни. Эта идея
Чаадаева позже была глубоко разработана виднейшим представителем славянофильства А. Хомяковым.
Второе направление в русской философии первой половины XIX в. — славянофильство. О сторонниках этого
направления сложилось устойчивое мнение как о представителях либерального дворянства, провозглашающих
особое историческое предназначение России, особые пути развития ее культуры и духовной жизни. Такое
одностороннее толкование славянофильства нередко приводило к тому, что это направление трактовалось как
реакционное  или,  в  лучшем  случае,  как  консервативное,  отсталое.  Подобная  оценка  далека  от  истины.
Славянофилы  действительно  противопоставляли  Восток  Западу,  остава-  46  ясь  в  своих  философских,
религиозных историко-философских воззрениях на русской почве. Но противопоставление Западу проявлялось
у  них  не  в  огульном  отрицании  его  достижений,  не  в  замшелом  национализме.  Напротив,  славянофилы
признавали и высоко ценили достоинства западноевропейской культуры, философии, духовной жизни в целом.
Они творчески восприняли философию Шеллинга, Гегеля, стремились использовать их идеи.
Славянофилы  отрицали  и  не  воспринимали  негативные  стороны  западной  цивилизации:  социальные
антагонизмы,  крайний  индивидуализм  и  меркантильность,  излишнюю  рациональность  и  т.п.  Истинное
противостояние  славянофильства  Западу  заключалось  в  различном  подходе  к  пониманию  основ,  «начал»
русской  и  западноевропейской  жизни.  Славянофилы  исходили  из  убеждения,  что  русский  народ  должен
обладать самобытными духовными ценностями, а не воспринимать огульно и пассивно духовную продукцию
Запада. И это мнение сохраняет свою актуальность и поныне.
В развитии славянофильства особую роль сыграли И.В. Киреевский, А.С. Хомяков, К.С. и И.С. Аксаковы, Ю.Ф.
Самарин.  Многообразие  их  взглядов  объединяет  общая  позиция:  признание  основополагающего  значения
православия,  рассмотрение  веры  как  источника  истинных  знаний.  В  основе  философского  мировоззрения
славянофильства лежит церковное сознание, выяснение сущности церкви. Наиболее полно эта основа раскрыта
Л.С. Хомяковым. Церковь для него не является системой или организацией, учреждением. Он воспринимает
Церковь как живой, духовный организм, воплощающий в себе истину и любовь, как духовное единство людей,
находящих в ней более совершенную, благодарную жизнь, чем вне ее. Основным принципом Церкви является
органическое, естественное, а не принудительное единение людей на общей духовной основе: бескорыстной
любви к Христу.
Итак,  западничество  и  славянофильство  —  две  противоположные,  но  и  вместе  с  тем  взаимосвязанные
тенденции в развитии русской философской мысли, наглядно показавшие самобытность и большой творческий
потенциал русской философии XIX в. 


\newpage
\section{Бытие как предмет онтологии. Метафизическое и физическое (натуралистическое) понимания бытия}
Онтология (новолат. ontologia от др.-греч. — сущее, то, что существует и учение, наука) — раздел философии,
изучающий проблемы бытия; наука о бытии.
Термин «Онтология» был предложен Р. Гоклениусом в 1613 году в его «Философском словаре» («Lexicon
philosophicum, quo tanquam clave philisophiae fores aperiunter. Fransofurti»), и чуть позже И. Клаубергом в 1656
году в работе «Metaphysika de ente, quae rectus Ontosophia», предложившем его (в варианте «онтософия») в
качестве  эквивалента  понятию  «метафизика».  В  практическом  употреблении  термин  был  закреплен  Х.
Вольфом, явно разделившим семантику терминов «онтология» и «метафизика».
Обычно под онтологией подразумевается эксплицитная, то есть явная, спецификация концептуализации, где в
качестве  концептуализации  выступает  описание  множества  объектов  и  связей  между  ними.  Формально
онтология состоит из понятий терминов, организованных в таксономию, их описаний и правил вывода.
Основной вопрос онтологии: что существует?
Основные  понятия  онтологии:  бытие,  структура,  свойства,  формы  бытия  (материальное,  идеальное,
экзистенциальное), пространство, время, движение.
Онтология,  таким  образом,  представляет  собой  попытку  наиболее  общего  описания  универсума
существующего, который не ограничивался бы данными отдельных наук и, возможно, не сводился бы к ним.
Иное понимание онтологии даёт американский философ Уиллард Куайн: в его терминах онтология — это
содержание  некоторой  теории,  то  есть  объекты,  которые  постулируются  данной  теорией  в  качестве
существующих.
Вопросы онтологии — это древнейшая тема европейской философии, восходящая к досократикам и особенно
Пармениду. Важнейший вклад в разработку онтологической проблематики внесли Платон и Аристотель. В
средневековой философии центральное место занимала онтологическая проблема существования абстрактных
объектов (универсалий).
В философии XX века специально онтологической проблематикой занимались такие философы как Николай
Гартман («новая онтология»), Мартин Хайдеггер («фундаментальная онтология») и другие. Особый интерес в
современной философии вызывает онтологические проблемы сознания.
Предмет онтологии
* Основным предметом онтологии является бытие, которое определяется как полнота и единство всех видов
реальности: объективной, физической, субъективной, социальной и виртуальной.
   * Реальность традиционно ассоциируется с материей и подразделяется на косную, живую и социальную
материю.
   * Бытие как то, что можно мыслить, противопоставляется немыслимому ничто (а также ещё-не-бытию
возможности  в  философии  аристотелизма).  Поскольку  мышлением  и  постижением  возможностей  бытия
обладает  только  человек,  то  в  последнее  время  (в  феноменологии  и  экзистенциализме)  именно  он
отождествляется бытием. Однако в классической метафизике под бытием понимается Бог. Человек как бытие
обладает свободой и волей.
Современная онтология
Современная  философия  рассматривает  бытие  как  единую  систему,  все  части  которой  взаимосвязаны  и
представляют собой некую целостность, единство. Вместе с тем мир разделен, дискретен и имеет четкую
структуру. В основе структуры мира 3 слоя реальности: бытие природы, бытие социальное, бытие идеальное.
Бытие природы
Бытие природы — первая форма реальности, универсума.
* Включает все существующее кроме человека.
* Является следствием длительной универсальной эволюции.
   * Системная организация мироустройства. Развитие мира — процесс преобразования и взаимодействия
образующих  его  систем.  Способность  всех  природных,  социальных  систем  к  самоорганизации,
самопроизвольному переходу на более высокий уровень организованности и упорядоченности.
* Подсистемы вещества и поля. Вещество — вид материи, обладающий массой покоя. Поле — основной вид
материи, связывающий частицы и тела. Частицы поля не имеют массы покоя, так как способ их существования
— движение.
   * Подсистемы неживой и живой природы. Неживая природа — движение элементарных частиц и полей,
атомов и молекул. Её уровни: вакуумный-микроэлементный-атомный-молекулярный-макроуровень-мегауровень
(планеты, галактики). Живая природа — биологические процессы и явления, происходит из неживой, включена
в неё, но представляет иной уровень развития. Её уровни: молекулярный-клеточный-микроорганизменный-тканевый-организменно-популяционнный-биогеоценотический-биосферный.
Бытие социальное
Бытие социальное — вторая форма реальности.
* Включает в себя бытие общества и бытие человека (экзистенция).
    *  Структура  социального  бытия  или  социума:  индивид,  семья,  коллектив,  класс,  этнос,  государство,
человечество.  По  сферам  общественной  жизни:  материальное  производство,  наука,  духовная  сфера,
политическая сфера, сфера обслуживания и т. д.
Бытие идеальное, духовное
Бытие идеальное, духовное — третья форма реальности.
* Тесно связано с бытием социальным, на своем уровне повторяет и воспроизводит структуру социума.
    *  Включает  неосознаваемые  духовные  структуры  индивидуального  и  коллективного  бессознательного
(архетипов),  сложившиеся  в  психике  людей  в  доцивилизационный  период.  Роль  этих  структур  признается
существенной и определяющей.
    *  Усиление  взаимодействия  всех  форм  духовной  жизни  с  производством,  практикой  (космонавтика,
биоинженерия и т. д.).
    *  Новые  информационные  технологии  и  средства  связи  сделали  духовное  бытие  более  динамичным,
подвижным.
«Белые пятна» современной онтологии
* Возникновение Вселенной — дата приблизительна, не ясна причина Большого взрыва.
* Появление жизни на Земле — не ясен процесс перехода от неживого к живому, от сложных органических
веществ к простейшим живым организмам, возникновение механизма наследственности.
Фрэнсис Крик, лауреат Нобелевской премии, английский биофизик: «Мы не видим пути от первичного бульона
до естественного отбора. Можно прийти к выводу, что происхождение жизни — чудо, но это свидетельствует
только о нашем незнании».
 * Не ясен механизм перехода от мира биологического к миру социальному, механизм появления человека.
Доказано, что человек — продукт эволюции, однако у человека как биологического вида нет родственников,
которые по прежнему оставались бы «детьми природы» и с которыми могла быть установлена генетическая
связь. Шимпанзе и гориллы связаны с человеком только общим предком, жившим более 7 миллионов лет назад.


\newpage
\section{Учение о материи в истории философии}
Уже в древности философы пытались представить  видимое многообразие вещей как проявление видимого
начала.  Это  общее,  несотворимая   и  не  уничтожимая  основа  всех  вещей  получила  название
субстанции.Формирование субстанции - это и формирование научного понимания материи.В древней Греции
под  субстанцией   материалисты  понимали  конкретное  вещество.  Древнегреческие  атомисты  считали,  что
субстанция это атом. Все что состоит из атомов и пустоты. В философии и естествознании нового времени и в
работах Ньютона, Ломоносова анатомические идеи получили развитие.
 С конца 16в.до начала 19в. господствовала механистическая картина мира. Материя рассматривалась  как
совокупность неделимых атомов, которые наделены геометрическими и механическими  свойствами: массой,
протяженностью, формой, непроницаемостью и способностью перемещаться.
Но в это время были высказаны и другие идеи о материи:1) о самодвижении в материи2) материя понималась
как  абстрактное  понятие,  в  котором  отображены  свойства  многообразных   веществ.3)  Материи  присуща
внутренне  мысль.Трактовка  материи  идеалистическими  направлениями   фил.Субъективные  идеалисты
отождествляли материю с совокупностью ощущений. Объективныеидеалисты считали, что материя пассивна,
инертна. Жизнь ей придает идею. Итак и для материализма  и для идеализма характерно отождествление
материи с веществом, с конкретными формами ее проявления; Не проводится различие м/у фил. пониманием
материи  и  естественноисторическими   взглядами  на  существующий  мир.  Маркс  и  Энгельс  понимали  под
материей субстанцию.
  Материя  -  это  всеобщая  носительница  св-в,  отношений,  изменений  конкретных  вещей.   Но  материя  не
существует вне конкретных вещей, а только через них. В начале 19-20вв. в естество знании было сделано ряд
открытий, которые опровергли старую, мех. картину мира. Открытие рентгеновских лучей подрывает идею
непроницаемости  материи.  Явление   дефекта  массы,   открытое  Беккерелем  при  изучении  радиоактивного
распада. Был сделан  вывод, что материя превращается в материю. Материя исчезает. Материализм ложен.
Открытие Томпсоном электрона отвергает идею о том, что атом не делим. Прежняя картина мира рушилась. В
естествознании наступил методологический кризис. Новые открытия не вписывались в старую картину мира.
Перед фил. стала задача уточнения понятия <материя>.
  Это  сделал  Ленин  в  работе  <Материализм  и  эмпириализм>.   Материя    -  философская  категория   для
обозначения объективной реальности, кот. дана ч-ку в ощущениях его, которая копируется, фотографируется,
отображается, нашими ощущениями,  существующая независимо  от  них.   В  этом  определении выделено 2
признака материи:
1) Признание первичности материи по отношении к сознанию (объективность ощущения) 
2) Признание принципиальной познаваемости мира. Ленин разграничивает философское понимание материи и
естественнонаучные  знания   о  существующем  мире.Ленин  способствовал  преодолению  кризиса  в  физике,
связанного с включением принципа структурности материи и делимости атомов в научную картину мира.


\newpage
\section{Движение и развитие; прогресс и регресс. Проблема социального прогресса, его факторы и критерии}
Характеризуя  развития общества, обществоведы обращаются к понятиям «прогресс» и «регресс». Прогресс
понимается  как  тип  направленного  развития,  связанный  переходом  от  низшего  к  высшему,   от  менее
совершенного  к  более  совершенному.  Представление  о  прогрессе   направленном   изменении  к  лучшему
возникло в древности. Однако поначалу оно носило в основном оценочный характер. Сначала прогресс был
отмечен в сфере научного познания, а затем идея  прогресса была распространена на область социальных
отношений. Одним из наиболее выдающихся  философов, разрабатывавших идею общественного прогресса,
был итальянский философ  Вико (17век). Он утверждал, что все народы проходят 3 этапа своего развития:
божественный,  героический и человеческий. Пройдя эти этапы, человечество осуществляет движение  по
нисходящей  линии.  В  целом  прогресс  характеризуется  нарастанием  темпов  развития   и  принимается  как
объективный закон исторической эволюции. Однако одной из существенных особенностей прогресса является
присутствие в нём регресса отдельных элементов, связей  и функций. С общественным прогрессом всегда
связывали  восходящее  поступательное  развитие   человеческого  общества  от  низших  ступеней  к  высшим.
Фаталистическое  понимание  прогресса,   когда  он  принимался  за  нечто  неизбежное,  настраивало  на
пассивность, созерцательность, безынициативность в отношении к жизни. Волюнтаризм понимал прогресс как
отрицание   объективных  законов  всемирной  истории,  религия   как  теологическое  представление  о
божественном  творчестве  и  конечном  состоянии  мира.  Прогресс   многогранное  явление.   В  качестве
конкретных  критериев  общественного  прогресса  выделяют  экономический,  социально   -  политический,
идеологический и гуманистический. С их помощью определяется прогрессивность тех или иных социальных  
систем. Многие социологи предлагают искать критерии прогресса в сфере сознания. Понятие прогресса носит
относительный характер. Возможность прогресса  определяется наличием конкретно  исторических условий.
Эти условия могут способствовать ускорению прогресса или же тормозить его, а иногда и препятствовать ему.
Типы общественного прогресса:
· Прогресс в антагонистических условиях (классовые, эксплуататорские общества) тип прогресса одинаков для
всех обществ и обусловлен господством частной собственности и эксплуатацией небольшой частью общества
большинства. Происходит развитие одних сторон общественной структуры за счёт других.
· Прогресс в частнособственнических формациях совершается за счёт усиления эксплуатации народных масс.
· Прогресс в феодальном обществе.
· Прогресс в капиталистических условиях «невидимые нити» капиталистической зависимости приковывают
рабочего к капиталисту.
· Прогресс в условиях социализма ликвидация всякого антагонизма (исчезновение классовых сил, тормозящих
прогресс), народные массы ведущая сила общественного прогресса.
В мире все находится в движении, от атомов до вселенной. Все пребывает в вечном стремлении к иному
состоянию,  и  не  по  принуждению,  а  по  собственной  природе.  Поскольку  движ  есть  сущностный  атрибут
материи, то оно, также как и сама материя, несотворимо и неуничтожимо. Движение - это способ сущ материи.
Движ заключено в самой природе материи. Одни формы движ превращ в другие и ни один вид не берется
ниокуда. 
Движение есть единство изменчивости и устойчивости, беспокойства и покоя. В потоке не прекращ движения
всегда  присутствуют  дискретные  моменты  покоя.,  проявляющиеся  прежде  всего  в  сохранении  внутренней
природы каждого данного движения, в виде равновесия движений и их относительно устойчивой формы, т.е.
относительного покоя. Покой, т.о., сущ как характеристика движ в какой-либо устойчивой форме. Каак бы не
изменялся предмет, но пока он сущ, он сохраняет свою определенность. Река остается рекой.
Абсолютный покой невозможен.
Существует несколько качественно различных форм движ материи: механ, физическая, хим, биологическая,
социальная... Качественное разнообразие одного уровня не м.б. объяснено кач разнообразием другого. Точное
описание движ частиц воздуха не может объяснить смысл чел речи. Однако необходимо иметь в виду и общие
закономерности, свойственные вскм уровням, а также их взаимодействие. Эта связь выраж в том, что высшее
включает низшее. (ДНК - хим соединение) Однако высшие формы не включены в низшие. (нет жизни в хим
соединениях)
Прослеживание связей между различными формами движ материи позволяет создать картину их развития во
вселенной.  На  его  разных  этапах  возникают  все  новые  уровни  организации  материи  и  соотв  им  формы
движения, причем появление каждой новой формы связано с сосотоянием Вселенной как целого. Сразу после Б.
Взрыва не было ни атомов ни соотв им форм движения. Хим и физ формы движения возникли на опред уровне
развития  Вселенной.  Также  на  опр  этапе  косм  эволюции  сформировались  планетные  системы,  возникли
условия  для  возникнов  жизни,т.е  биол  формы  движения.  В  этом  смысле  жизнь  надо  расматривать  как
космическое явление. В свою очередь только пройдя длшит этап эволюции, жив природа смогла породить
социально организованную материю., и тогда возникла социальная форма движ.
Совр наука показывает, что наша астрономическая вселенная, мир, в кот мы живем, по-видимому, явл только
одним  из  возможных  миров.  Причем  уже  в  особенностях  взаимод.  элементарных  частиц  заложены  опр
предпосылки, возможности для развертывания более сложных форм движения. (мировые константы)
В соврем космологии указанные идеи входят в содержание так называемого антропного принципа, согласнокот
наш мир устроен таким образом, что допускает возможность появл человека как закономерного итога эволюции
материи. Но возможны и др миры, с другими мир константами. Эти миры возможно бедны, пусты и допускают
только низшие формы движ материи, а возможно и наоборот. В этом смысле человек и чел общество предстают
как  такая  форма  организации  материи,  кот  обусловлена  свойствами  целого  нашей  Вселенной,
фундаментальными характеристиками космоса. 
Мировое развитие являет собой заклномерный поступательный процесс, противоречия которого представляют
собой источник, движ силу общ прогресса.Всемирная история постоянно выдвигала пробл противоречий общ
прогресса, и каждая ее эпоха свидетельствовала о катаклизмах, прерворотах и вместе с тем она являет собой
необходимый процесс движ человечества от одних форм своей соц организации к другим, более совершенным. 
Критерий  прогресса  -  общественно  историческая  практика,  в  кот  выделяются  два  ее  основных  вида:
производственная и социально-преобразующая. Ядром этой практики выступает развитие производ сил как
высшего критерия общ прогресса. Главное в произв силах - это человек Этим объясняется то, что в данном
критерии воплощаются и достижения науки, принципы управления, и социально-полит состояние общества, и
уровень образования, и образ жизни вплоть до мировоззрения, кот опосредовано влияют на эффективность
производства. Вот почему "развитие производ сил человечества означает прежде всего развитие богатства чел
природы как самоцель". Действительным ядром общ прогресса выступают способ производства.
Для  определения  подлинно  прогрессивного  есть  критерий,  выработанный  самой  историей  человечества.
Критерий этот, выраженный словом гуманизм, обозначает как специф свойства чел природы так и оценку этих
свойств как высшего начала общ жизни. Прогрессивно то, что способствует возвышению гуманизма.

\subsection{Проблема социального прогресса, его факторы и критерии}
Социальный  прогресс  –  направленный  процесс,  который  неуклонно  приводит  систему  все  ближе  к  более
предпочтительному,  лучшему  состоянию  (по  мнению  большинства  исследователей  –  к  реализации
определенных ценностей этического порядка: счастью, свободе, процветанию, знаниям).
Идея  прогресса  лежит  в  фундаментальной  особенности  человеческого  бытия  –  противоречии  между
реальностью  и  желаниями,  жизнью  и  мечтами.  Концепция  прогресса  смягчает  возникающее  при  этом
напряжение, порождая надежду на лучший мир в будущем и уверяя, что его приход гарантирован или, по
крайней  мере,  возможен  ("Мир  сегодня  верит  в  прогресс,  потому  что  единственной  альтернативой  будет
всеобщее отчаяние" (С. Поллард)).
Современное толкование социального прогресса основывается на нескольких фундаментальных идеях: 1) о
необратимом времени, текущем линейно и обеспечивающем непрерывность прошлого, настоящего и будущего
(прогресс  есть  положительно  оцениваемая  разница  между  прошлым  и  настоящим);  2)  о  направленном
движении, в котором ни одна стадия не повторяется; 3) о кумулятивном процессе, протекающем либо по
возрастающей, шаг за шагом, либо революционным путем; 4) о различии между типичными, необходимыми
стадиями, которые проходит процесс; 5) об эндогенных причинах, вызывающих самодвижение (саморазвитие)
процесса; 6) о признании неизбежного, необходимого, естественного характера процесса, который не может
быть остановлен или отвергнут; 7) об улучшении, усовершенствовании, отражающими тот факт, что каждая
последующая стадия лучше предыдущей.
Кульминацией прогресса должна стать полная реализация таких ценностей, как счастье, изобилие, свобода,
справедливость, равенство. Отсюда следует, что прогресс – ценностная категория. И каждая историческая эпоха
оценивает его исходя из своего понимания ценностей (в XIX в. критериями прогресса были индустриализация,
урбанизация, модернизация; в начале XXI в. они таковыми уже не считаются).
В  ХХ  в.  социальные  процессы  шли  крайне  противоречиво,  их  оценка  исследователями  и  общественным
мнением была амбивалентной (как позитивной, так и негативной). В конце концов, это привело к кризису идеи
социального прогресса.
Проявления  кризиса  идеи  прогресса:  1)  идея  прогресса  сменилась  распространением  мистицизма,  бунтом
против  рассудка  и  науки,  всеобщим  пессимизмом  в  образе  дегенерации,  разрушения  и  упадка;  2)  идея  о
необходимости постоянного экономического и технологического роста сменилась идеей пределов роста; 3) вера
в  рассудок  и  науку  сменилась  убеждением  в  доминирующей  роли  эмоций,  интуиции,  подсознательного  и
бессознательного, утверждении иррационализма; 4) утверждение о важности, высочайшей ценности жизни на
земле  сменилось  чувством  бессмысленности,  аномии  и  отчуждения;  5)  крушение  идей  утопизма
(окончательный  удар  по  утопическому  мышлению  нанесло  падение  коммунистической  системы);  6)
лейтмотивом конца ХХ – начала XXI в. стало повсеместное распространение идеи кризиса. При этом люди
склонны  рассматривать  социальный  кризис  как  хронический,  всеобщий  и  не  предвидят  его  будущего
ослабления.
Однако  не  все  исследователи  настроены  столь  пессимистично.  По  мнению  П.  Штомпки,  в  современном
обществе существуют реальные возможности и условия для обеспечения социального прогресса сегодня и в
обозримом  будущем.  Для  этого  есть  все  необходимые  предпосылки:  1)  наличие  в  обществе  творческих,
независимых, адекватно осознающих реальность деятелей; 2) богатые и гибкие общественные структуры; 3) 
благоприятные  и  активно  воспринимаемые  естественные  условия;  4)  долгая  и  уважаемая  традиция;  5)
оптимистичный, долгосрочный взгляд на будущее и его планирование. 


\newpage
\section{Понятия "пространство" и" время" в их естественнонаучной и социогуманитарной интерпретациях}
Материальный  мир  состоит  из  структурных  объектов,  которые  находятся  в  движении  и  развитии,
представляющие собой процессы, которые развертываются по определенным этапам.
Наиболее общая характеристика пространства — свойство объекта быть протяженным, занимать место среди
других, граничить с другими объектами.
Сравнение  различных  длительностей,  выражающих  скорость  развертывания  процессов,  их  ритм  и  темп
является понятием времени.
Категории  пространства  и  времени  выступают  как  формы  бытия  материи.  Существует  две  концепции
пространства и времени:
* субстанциальная — рассматривает пространство и время как особые сущности, которые существуют сами
по себе, независимо от материальных объектов (Демокрит, Эпикур, Ньютон);
* реляционная — рассматривает пространство и время как особые отношения между объектами и процессами
и вне их не существуют (Лейбниц).
Всеобщие  свойства  пространства  и  времени:  объективность  и  независимость  от  сознания  человека;
абсолютность как атрибутов материи; неразрывная связь друг с другом и с движением материи; зависимость от
структурных отношений и процессов развития в материальных системах; единство прерывного и непрерывного
в их структуре; количественная и качественная бесконечность.
Различают метрические (т.е. связанные с измерениями) и топологические (например, связность, симметрия
пространства и непрерывность, одномерность, необратимость времени) свойства пространства и времени.
Топологические  характеристики  описывают:  прерывность  и  непрерывность,  размерность,свяэность,
ориентируемость.
Метрические характеристики: кривизну, конечность и бес-конечносгь, изотропность, гомогенность.
Всеобщие  свойства  пространства:  протяженность,  означающая  рядоположенность  и  сосуществование
различных элементов (точек, отрезков, объемов и i:n.), возможность прибавления к каждому данному элементу
некоторого следующего элемента либо возможность уменьшения числа элементов; связность и непрерывность;
трехмерность.
С протяженностью пространства неразрывно связаны его метрические свойства, выражающие особенности
связи пространственных элементов, порядок и количественные законы этих связей.
Специфические (локальные) свойства пространства: симметрия и асимметрия, конкретная форма и размеры,
местоположение,  расстояние  между  телами,  пространственное  распределение  вещества  и  поля,  границы,
определяющие различные системы.
Всеобщие свойства времени: объективность; неразрывная связь с материей и ее атрибутами; длительность;
одномерность и ассиметричность; необратимость и направленность от прошлого к будущему.
Специфическими свойствами времени являются конкретные периоды существования тел от возникновения до
перехода в качественно иные формы, одновременность событий, которая всегда относительна, ритм процессов,
скорость изменения состояний, темпы развития, временные отношения между различными циклами в структуре
систем.
Теория А. Эйнштейна доказала, что в реальном физическом мире пространственные и временные интервалы
меняются при переходе от одной системы отсчета к другой.
Теория относительности вывела глубокую связь между пространством и временем, показав, что в природе
существует единое  пространство —  время,  а  отдельно  пространство и отдельно  время выступают  как  его
своеобразные проекции, на которые оно по-разному расщепляется в зависимости от характера движения тел. 


\newpage
\section{Экзистенциально-антропологический поворот в онтологии: «фундаментальная онтология» Хайдеггера}
\subsection{Философия экзистенциализма: основные направления и проблемы}
Экзистенциализм – философия существования. Иррационалистическая философия.
Наиболее крупные представители: М. Хейдеггер, религиозный (К. Ясперс, Г. Марсель), атеистический (Ж.П. 
Сартр, А. Камю), Н. Аббаньяно.
Экзистенциалисты.  поставили  вопрос  о  смысле  жизни,  о  судьбе  человечества,  о  выборе  и  личной
ответственности в условиях исторических катастроф и противоречий.
Исходный пункт философии экзистенциализма – изолированный, одинокий индивид, все интересы которого
сосредоточены  на нем же  самом,  на его  собственном ненадежном  и  бренном  существовании.  Отчуждение
человека от общества.
Экзистенциальные проблемы – это проблемы, которые возникают из самого факта существования человека. Для
Э. имеет значение только его собственное существование и его движение к небытию.
Хотя бытие вещей совершенно непонятно, но есть 1 вид бытия отлично нам знакомый – это наше собственное
бытие. Здесь то и открывается доступ к бытию как таковому, он идет через наше существование. Но это
существование  –  нечто  внутреннее  и  невыразимое  в  понятиях:  "существование  есть  то,  что  никогда  не
становится объектом", ибо мы никогда не можем взглянуть на себя со стороны.
Экзистенциализм  –  это  философия,  единственный  предмет  которой  –  человеческое  существование,  точнее
переживание  существования.  Среди  всех  способов  бытия  существования  Э  ищут  такой,  в  котором
существование раскрылось бы наиболее полно – это страх. Страх – это исходное переживание, лежащее в
основе всего существования. В конечном счете, это страх перед смертью.
Экзистенциалисты  объявляют  предметом  философии  бытие.  «Современная  философия,  как  и  в  прошлые
времена, занята бытием" (Сартр). Они утверждают, что понятие бытия является неопределимым, и что никакой
логический анализ его невозможен. Поэтому философия не может быть наукой о бытии и должна искать иных,
ненаучных, иррациональных путей для проникновения в него. Противопоставляя науку философии, говорят, что
наука  занимается  сущим,  а  философия  –  бытием.  Бытие  постигается  не  через  рассудочное  мышление,  а
непосредственно открываясь человеку через его экзистенцию.
Экзистенция  представляет  собой  центральное  ядро  человеческого  Я,  благодаря  чему  Я  выступает  не  как
отдельный  мыслящий  индивид  и  не  как  мыслящее  всеобщее,  а  как  отдельная  неповторимая  личность.
Экзистенция – не сущность человека, а открытая возможность. Важнейшее определение экзистенции – ее
необъективируемость. Можно объективировать способности и знания через материальный мир, рассматривать
психические акты и деятельность, единственное, что неподвластно объективации – экзистенция. В обыденной
жизни человек не осознает экзистенцию, для этого ему надо оказаться в пограничной ситуации.
Обретая себя как экзистенция, человек обретает свободу.
Свобода. Человек сам свободно выбирает свою сущность, он становится тем, кем он себя сделает. Человек – это
постоянная возможность, замысел, проект. Он свободно выбирает себя и несет полную ответственность за свой
выбор. Свобода составляет само человеческое существование, человек и есть свобода.
Однако  свобода  понимается  ими  как  нечто  неизъяснимое,  не  поддающееся  выражению  в  понятиях,
иррациональное. Свободу они мыслят как свободу вне общества. Это внутреннее состояние, настроенность,
переживание  индивида.  Свобода  противопоставляется  необходимости.  Такая  свобода,  противопоставленная
необходимости и отрешенная от общества, – есть пустой формальный принцип. Свобода – это свобода выбора
отношения  к  окружающей  действительности.  Раб  может  быть  свободным,  соответственно  самоопределяя
отношение к своему бытию. Свобода становится неотвратимым роком. «Человек осужден быть свободным».
Свобода есть мучительная необходимость.
Характерной  чертой  человеческого  существования  является  то,  что  он  не  сам  выбирает  условия  своего
существования, он заброшен в мир и подвластен судьбе. От человека не зависит время его рождения и смерти.
Это  приводит их  к мысли,  что помимо  чел  существования  существует потусторонняя  реальность,  которая
понимается как способ существования человека, состоящий в озабоченности человека, направленной куда-то
вне его. Внешний мир представляет среду, мир заботы человека, окружающий человеческое существование и
находящийся в неразрывной связи с ним. Пространство и время есть способы чел существования. Время – это
переживание  существованием  своей  ограниченности,  временности.  Представление  о  времени  до  моего
рождения  и  после  смерти  –  произвольная  экстраполяция.  Говорить  о  том,  что  будет  после  моей  смерти
бессмысленно.
Личность и общество. Общество – всеобщая безличная сила, подавляющая и разрушающая индивидуальность,
отнимающая у человека его бытие, навязывающая личности трафаретные вкусы, нравы, взгляды... Человек,
преследуемый страхом смерти, ищет прибежища в обществе. Растворяясь в нем, он утешает себя тем, что люди
смертны. Но жизнь в обществе не истинна. В глубине человека скрыто истинное, одинокое существование.
Каждый умирает в одиночку.

\subsection{Проблема человеческого существования в экзистенциализме}
Идейные истоки экзистенциализма — философия жизни, феноменология Гуссерля, религиозно-мистическое
учение  Кьеркегора.  Различают  экзистенциализм  религиозный  (Марсель,  Ясперс,  Бердяев,  Бубер)  и
атеистический  (Хайдеггер,  Сартр,  Камю).  В  философии  существования  нашёл  отражение  кризис
оптимистического  либерализма,  опирающегося  на  технический  прогресс,  но  бессильный  объяснить
неустойчивость,  неустроенность  человеческой  жизни,  присущие  человеку  чувство  страха,  отчаяния,
безысходности.
Экзистенциализм  —  это  иррациональная  реакция  на  рационализм  Просвещения  и  немецкой  классической
философии.  По  утверждениям  философов-экзистенциалистов,  основной  порок  рационального  мышления
состоит в том, что оно исходит из принципа противоположности субъекта и объекта, то есть разделяет мир на
две  сферы:  объективную  и  субъективную.  Всю  действительность,  в  том  числе  и  человека,  рациональное
мышление рассматривает только как предмет, как «сущность», познанием которой можно манипулировать в
терминах  субъекта-объекта.  Подлинная  философия  с  точки  зрения  экзистенциализма  должна  исходить  из
единства  объекта  и  субъекта.  Это  единство  воплощено  в  «экзистенции»,  то  есть  некой  иррациональной
реальности.
Согласно экзистенциалистскому учению, чтобы осознать себя как «экзистенцию», человек должен оказаться в
«пограничной ситуации», например перед лицом смерти. В результате мир становится для человека «интимно
близким». Истинным способом познания, способом проникновения в мир «экзистенции» объявляется интуиция
(«экзистенциальный опыт» у Марселя, «понимание» у Хайдеггера, «экзистенциальное озарение» у Ясперса),
которая являет собой иррационалистически истолкованный феноменологический метод Гуссерля.
Значительное  место  в  экзистенциализме  занимает  постановка  и  решение  проблемы  свободы,  которая
определяется  как  «выбор»  личностью  одной  из  бесчисленных  возможностей.  Предметы  и  животные  не
обладают свободой, поскольку сразу обладают «сущим», эссенцией. Человек же постигает своё сущее в течение
всей жизни и несёт ответственность за каждое совершённое им действие, не может объяснять свои ошибки
«обстоятельствами». Таким образом, человек мыслится экзистенциалистами как самостроящий себя «проект».
В конечном итоге идеальная свобода человека это свобода личности от общества.

\subsection{Фундаментальная онтология Хайдеггера}
В конце 30х гг. становится ректором университета и принимает активное участие в фашистском движении.
Гуссерль:  сознание  –  интенционально,  т.е.  сознание  чего-то.  В  сознании  имеем  не  реальные  вещи,  а
конструкторы (интенциональные объекты).
Хайдеггер:  интенциональные  объекты  есть  сами  вещи,  данные  сознанию  так,  как  они  существуют  в
действительности.
Он пишет: истина по гречески «алетейа», что буквально переводится как «несокрытость». Т.е. истинно то, что
для нас не скрыто. Т.е. древние греки считали, что вещи могут непосредственно контактировать с сознанием,
т.е. вещи нам непосредственно даны. Но они даны нам не в научном познании, а в интуиции (в жизненной
интуиции).
Искусственность  схемы  гуссерля  по  Хайдеггеру  в  том,  что  он  познание  отрывает  от  деятельности.  По
Хайдеггеру познание – лишь одна из функций интеллекта, человек – существо прежде всего действующее и
живущее. Мир дан непосредственно, и дан не в познании прежде всего, а в действии.
Вещы  нам  даны  непосредственно  как  феномены.  С  греческого  «феномен»  –  то,  что  показывает  себя.  Но
показывает себя полностью (а не какой-то стороной) ? нет никаких кантовских вещей в себе. Это придумка,
возникающая при отрыве познания от деятельности.
Таким образом, феномены – это вещи в действительности, в их данности нашему сознанию, но не вещи сами по
себе. Т.е. вещи и есть формы бытия (или бытие в различных его формах).
Исходной формой бытия является вещь, которую он называет Dasein – это особая исходная форма бытия,
которую он выражает словами: «бытие здесь» – это мое личное существование. Это я сам, как форма бытия.
(Хайдеггер «Бытие и время»)
Основное свойство dasein – его существование в чем-то, которое мы называем миром. Мир -–это наличные для
dasein вещи, среди которых dasein не является вещью. Но dasein и мир образуют неразделимое единство.
Познание по Хайдеггеру – лишь одна из форм существования dasein. А человек прежде всего не теоретик, а
практик и познание – вторичный элемент в его деятельности, а первичный элемент – практика.
Устройство бытия.
Мир, бытие имеет центр – dasein.
Дальше Хайдеггер вводит понятие Werkwelt – «подручный мир» или «подручные вещи» – мир повседневных,
ближайших интересов и занятий. Umwelt – окружающий мир – то, что входит в область моего зрения, но не
очень меня затрагивающий.
Окружающий мир по Хайдеггеру отличается от физического мира науки. Если физическое пространство не
имеет никакого центра, то Umwelt имеет центр – dasein. Природа по сравнению с живым миром моего “я”
является сухой абстракцией. Природа, как ее видит наука – то, что остается от мира, если воспринимать его
чисто созерцательно. Чтобы получить “подлинную” картину природы, ученые стараются свести к минимуму
фактор наблюдателя в опыте. Т.е. наука ? к «объективности». И тогда природа – это Umwelt, ставший жертвой
нашего стремления я к объективности. Научная картина мира становится неподлинной и непонятной.
Поэтому мир науки – это мир искаженный, поэтому это мир непонятный. Понятно только то, что м.б. встречено.
Понятно, как наука строит свою картину мира, понятны будут ее теории и доказательства, но общий результат
всего этого нам совершенно непонятен (что такое мир в целом, как он устроен). Научная картина мира –
любопытная, сложная абстрактная конструкция, которая что-то проясняет; но она не совпадает с миром, в
котором мы живем.
Хайдеггер ставит вопрос о бытии независимо от познания (вне проблемы субъект-объект).
Хайдеггер оказал огромное влияние на развитие экзистенциализма. Но экзистенциализм исходит из другого
понятия бытия: чувственного бытия.
Мир Хайдеггера все же рационален.


\newpage
\section{Основные концепции сознания и его структура}
Сознание — высшая, свойственная лишь человеку форма отражения обьективной действительности. Сознание
представляет  собой  единство  психических  процессов,активно  участвующих  в  осмыслении  человеком
обьективного  мира  и  своего  собственного  бытия.  С  самого  рождения  человек  попадает  в  мир  предметов,
созданных предыдущими поколениями и формируется как таковой лишь в процесе обучения целенаправленому
их использользованию, которое происходит лишь в процессе общения. Имено потому, что человек относится к
объектам с пониманием, со знанием, способ его отношения к миру называется сознанием. Любое ощущение
или чувство является частью сознания так как обладает значением и смыслом. Однако сознание не есть только
знание  или  языковое  мышление.  С  другой  стороны  нельзя  отождествлять  сознание  и  психику,  т.к.  не  все
психические  процессы  включаются  в  данный  момент  в  сознание.  Оно  возникло  в  процессе  общественно-производственной деятельности человека и неразрывно связанно с языком. Сознание существует в 2 формах —
общественном  и  индивидуальном.  Основными  подходами  к  происхождению  сознания  были  следующие:
Платон: «тело  человека  — вместилище  бессмертной  души и ее  раб,  бестелесная душа управляет всем  во
вселенной».  Христианство:  «разум  человека,  его  мышление  —  искорка  божественного  разума.  Именно  он
мыслит, желает, чувствует в человеческом сознании». Декарт: ’’Сознание — внепространственная субстанция,
впервые  расnматривает  проблему  самосознания’’.  Гегель:  С.  —  одно  из  воплощений  всемирного  разума’’.
Впервые рассматривает социально — историческую природу сознания, говорит о принципе историзма. В ХХ
веке возникает теория отражения. Согласно этой теории сознание это высщая форма отражения. Таким образом
сознание формируется деятельностью чтобы затем влиять на эту деятельность, определяя и регулируя ее.
Сознание  структурно  организованно,  представляет  систему  элементов,  находящихся  между  собой  в
закономерных  отношениях.  В  структуре  сознания  наиболее  отчетливо  выделяются  такие  элементы  как
осознание вещей, а также переживание, т.е. отношение к содержанию того что отражается. Развитие сознания
предполагает  прежде  всего  обогащение  его  новыми  знаниями.  Познание  вещей  имеет  разные  уровни
проникновения и степень ясности понимания. Отсюда обыденное, философское, научное, и.т.д. осознание мира
а также чувственный и рациональный уровень сознания. Ощущения, понятия, восприятия, мышление образуют
ядро  сознания.  Но  они  не  исчерпывают  всей  структурной  полноты  сознания:  оно  включает  в  себя  и  акт
внимания как свой необходимый компонет. Именно благодаря сосредоточенности внимания определенный круг
объектов находится в фокусе сознания. Воздействующие на нас предметы, события вызывают у нас не только
познавательные  образы,  но  и  эмоции.  Богатейшая  сфера  эмоциональной  жизни  человека  включает  в  себя
собственно чувства, настроение, или эмоциональное самочувствие и аффекты (ярость, ужас и.т.д.). Чувства,
эмоции  суть  компоненты  сознания  Сознание  не  ограничивается  познавательными  процессами,
направленностью на обьект, эмоциональной сферой. Наши намерения претворяются в жизнь благодаря усилиям
воли. Однако сознание — это не сумма множества составляющих его элементов, а их интегральное сложно
структурированное целое.
Наиболее  общеизвестное  определение  сознания  можно  сформулировать  следующим  образом.  Сознание
является комбинацией следующих аспектов:
1.Восприятие и ощущения 
2.Абстрактное мышление
3.Память
4.Воображение
5.Эмоции 
6.Воля (свобода выбора)
Личность – целостная комбинация этих категорий. Для человека характерно наличие и развитость абстрактного
мышления  и  воображения.  Существуют  два  различных  подхода  к  изучению  сознания:  физиологический  и
психический. 
Совершенно очевидно, что мозг – материальный субстрат сознания. И возникает извечный вопрос – в каком
отношении  находятся  сознание  и  мозг?  Всегда  существовали  материалистическая  и  идеалистическая
тенденции. 
7.Демокрит.  Первая  материалистическая  тенденция.  Он  говорил,  что  сознание  –  это  сферические  атомы, 
сдерживаемые  в  теле  человека  внешним  воздухом.  Бесспорно,  чистый  материализм.  Мысли  объяснялись
движением атомов. Потом, гораздо позже, в эпоху Возрождения появилась теория, что мысли – выделения мозга
(типа желчи). Еще позже появилась концепция биотоков и т.п. Например, Павлов говорил, что если бы череп
был прорачный,  а возбужденные  клетки  светились,  то мы видели  бы чередование  мерцание  –  мысли.  Он
отождествлял психические и физиологические процессы 
8.Идеалитисческая (?) модель. Гласит, что психические процессы протекают параллельно физиологическим, что
есть душа человека и она бессмертная. Мысль – не функция мозга, и носителем психики является идеальная
субстанция – наподобие духовной субстанции Спинозы, которая есть в человеке помимо материальной. Но
психические процессы связаны с физиологическими т.к. по мере усложнения нервной системы усложняются
психические процессы (имеется в виду переход ребенок - взрослый)
Как бы там ни было нельзя ни отождествлять физиологию с психикой, не разделять их. Их надо рассматривать в
тесной взаимосвязи, природа которой до сих пор не ясна.
Можно вести критерии сознания, свидетельствующие о его наличии. Критерии сознания следующие:
9.Способность отражения окружающего мира (но для этого должен быть эталон, для сравнения)
10.Способность самоотражения
11.Оценка окружающего мира 
12.Самооценка (самооценкой называется применение моральных законов к самому себе. Это присуще только
человеку.)
13.Свобода выбора (на основе оценки окружающего мира и самооценки)
Таким образом мы видим, что сознание человека по меньшей мере пятигранно. Ганди принадлежат слова:
“Душа человека подобна драгоценному камню, на каждой грани которого горят огни…”
Происхождение сознания – не только продукт биологической эволюции но и социальной в том числе.


\newpage
\section{Проблема соотношения сознания и бессознательного в психоанализе Фрейда и в аналитической психологии Юнга}
С  вопросами  биологического  и  социального,  сущности  и  существования  тесно  связана  и  проблема
бессознательного и сознательного в философской антропологии, отражающая важную сторону существования
человека.
Длительное  время  в  философии  доминировал  принцип  антропологического  рационализма  —  человек,  его
мотивы поведения и само бытие рассматривались только как проявление сознательной жизни. Этот взгляд
нашел свое яркое воплощение в знаменитом картезианском тезисе «cogito ergo sum» («мыслю, следовательно,
существую»). Человек в этом плане выступал лишь как «человек разумный». Но, начиная с Нового времени, в
философской антропологии все большее место занимает проблема бессознательного.
Лейбниц, Кант, Кьеркегор, Гартман, Шопенгауэр, Ницше с разных сторон и позиций начинают анализировать
роль и значение психических процессов, не осознающихся человеком.
Но определяющее влияние на разработку этой проблемы оказал 3. Фрейд, открывший целое направление в
философской антропологии и утвердивший бессознательное как важнейший фактор человеческого измерения и
существования. Он представил бессознательное как могущественную силу, которая противостоит сознанию.
Согласно его концепции, психика человека состоит из трех пластов. Самый нижний и самый мощный слой —
«Оно» (Id) находится за пределами сознания. По своему объему он сравним с подводной частью айсберга. В
нем  сосредоточены  различные  биологические влечения и страсти,  прежде  всего сексуального  характера, и
вытесненные из сознания идеи. Затем следует сравнительно небольшой слой сознательного — это «Я» (Ego)
человека. Верхний пласт человеческого духа — «Сверх-Я» (Super Ego) — это идеалы и нормы общества, сфера
долженствования  и  моральная  цензура.  По  Фрейду,  личность,  человеческое  «Я»  вынуждено  постоянно
терзаться и разрываться между Сциллой и Харибдой — неосознанными осуждаемыми «Оно» и нравственно-культурной цензурой «Сверх-Я». Таким образом, оказывается, что собственное «Я» — сознание человека — не
является  «хозяином  в  своем  собственном  доме».  Именно  сфера  «Оно»,  всецело  подчиненная  принципу
удовольствия и наслаждения, оказывает, по Фрейду, решающее влияние на мысли, чувства и поступки человека.
Человек — это прежде всего существо, управляемое и движимое сексуальными устремлениями и сексуальной
энергией (либидо).
Драматизм  человеческого  существования  у  Фрейда  усиливается  тем,  что  среди  бессознательных  влечений
имеется и врожденная склонность к разрушению и агрессии, которая находит свое предельное выражение в
«инстинкте смерти», противостоящем «инстинкту жизни». Внутренний мир человека оказался, следовательно,
еще и ареной борьбы между двумя этими влечениями. В конце концов Эрос и Танатос рассматриваются им как
две наиболее могущественные силы, определяющие поведение человека.
Таким образом, фрейдовский человек оказался сотканным из целого ряда противоречий между биологическими
влечениями и сознательными социальными нормами, сознательным и бессознательным, инстинктом жизни и
инстинктом смерти. Но в итоге биологическое бессознательное начало оказывается у него определяющим.
Человек, по Фрейду, — это прежде всего эротическое существо, управляемое бессознательными инстинктами.
Проблема бессознательного интересовала и швейцарского психиатра К.-Г. Юнга. Однако он выступил против
трактовки человека как существа эротического и попытался более глубоко дифференцировать фрейдовское
«Оно». В частности, Юнг выделил в нем помимо личностного бессознательного, как отражение в психике
индивидуального  опыта,  еще  и  более  глубокий  слой  —  коллективное  бессознательное,  которое  является 
отражением опыта предшествующих поколений. Содержание коллективного бессознательного составляют, по
Юнгу, общечеловеческие первообразы — архетипы (например, образ матери-родины, народного героя, богатыря
и  т.д.).  Совокупность  архетипов  образует  опыт  предшествующих  поколений,  который  наследуется  новыми
поколениями.  Архетипы  лежат  в  основе  мифов,  сновидений,  символики  художественного  творчества.
Сущностное  ядро  личности  составляет  единство  индивидуального  и  коллективного  бессознательного,  но
основное  значение  имеет  все-таки  последнее.  Человек,  таким  образом,  —  это  прежде  всего  существо
архетипное.
Проблема  бессознательного  и  сознательного  развивалась  и  другими  представителями  психоанализа  —
последователями  Фрейда,  которые  уточняли  и  развивали  его  учение,  внося  в  него  свои  коррективы.  Так,
австрийский  психиатр  А.  Адлер  подверг  критике  учение  Фрейда,  преувеличивающего  биологическую  и
эротическую  детерминацию  человека.  По  Адлеру,  человек  —  не  только  биологическое,  но  и  социальное
существо,  жизнедеятельность  которого  связана  с  сознательными  интересами,  поэтому  «бессознательное  не
противоречит сознанию», как это имеет место у Фрейда. Таким образом, Адлер в определенной степени уже
социологизирует  бессознательное  и  пытается  снять  противоречие  между  бессознательным  и  сознанием  в
рассмотрении человека.
Американский неофрейдист, социальный психолог и социолог Э. Фромм выступил против биологизации и
эротизации бессознательного и подверг критике теорию Фрейда об антагонизме между сущностью человека и
культурой.  Но  вместе  с  тем  он  отверг  и  социологизаторские  трактовки  человека.  По  его  собственному
признанию,  его  точка  зрения  является  «не  биологической,  и  не  социальной».  Одним  из  наиболее  важных
факторов  развития  человека,  по  Фромму,  является  противоречие,  вытекающее  из  двойственной  природы
человека,  который  является  частью  природы  и  подчинен  ее  законам,  но  одновременно  это  и  субъект,
наделенный разумом, существо социальное. Это противоречие он называет экзистенциальной дихотомией. Она
связана  с  тем,  что  ввиду  отсутствия  сильных  инстинктов,  которые  помогают  в  жизни  животным,  человек
должен принимать решения, руководствуясь своим сознанием. Но получается так, что результаты при этом не
всегда оказываются продуктивными, что порождает тревогу и беспокойство. Поэтому «цена, которую человек
платит за сознание», — это неуверенность его.


\newpage
\section{Проблема познаваемости мира. Основные гносеологические концепции}


\newpage
\section{Понятие субъекта и объекта в гносеологии. Познание как «отражение» и как «конструирование» действительности}

\subsection{Субъект познания, основные концепции субъекта}
Под  субъектом  познания  следует  понимать  наделенного  сознанием  человека,  включенного  в  систему
социокультурных связей, чья активность направлена на постижение тайн противостоящего ему объекта. 
Основные концепции субъекта
   * Психологический субъект познания (изолированный субъект). В данной концепции субъект напрямую
отождествляется с человеческим индивидом, осуществляющим познавательный акт. Такая позиция близка к
нашему  повседневно-реалистическому  опыту  и  наиболее  распространена.  В  рамках  данной  концепции
познающий чаще всего рассматривается как пассивный регистратор внешних воздействий, с той или иной
степенью адекватности отражающий объект. Такой подход не учитывает активный и конструктивный характер
поведения субъекта — того, что последний способен не только отражать, но и формировать объект познания.
   * Трансцедентальный субъект познания. Данная концепция утверждает, что существует инвариантное и
устойчивое "познавательное ядро" в каждом человеке, которое обеспечивает единство познания в контексте
различных  эпох  и  культур.  Его  выявление  составляет  важную  часть  всей  теоретико-познавательной
деятельности. Такая трактовка субъекта восходит к И.Канту
    *  Коллективный  субъект  познания  практически  реализуется  посредством  совместных  усилий  многих
индивидуальных психологических субъектов. Такой субъект не сводится к простой сумме индивидуальных
субъектов  и  относительно  от  них  автономен.  Примером  подобного  субъекта  может  служить  научно-исследовательский коллектив, профессиональное сообщество или даже человеческое общество в целом.

\subsection{Объект познания, основные концепции объекта}
Объект — это та сторона действительности, на которую направлено познание.
Объект  познания  обычно  определяется  путем  выделения  части  объективной  реальности,  вовлеченной  в
человеческую производственную и познавательную деятельность. Однако понятия “объект” и “объективная
реальность” не вполне правомерно рассматривать на основе включения одного в другое. Если принять во
внимание,  что  человек,  человечество,  которые  обычно  характеризуются  как  субъект,  сами  включаются  в
процесс  познания  в  качестве  объекта,  то  правомерность  указанной  выше  логической  операции  станет
сомнительной.  Не  всякая  объективная  реальность  есть  объект  познания,  но  и  не  всякий  объект  познания
является объективной реальностью.
Далее.  Категория  объекта  часто  рассматривается  как  соотносительная  категории  субъекта.  Эта
соотносительность  не  обозначает  зависимости  существования  объективной  реальности  от  существования
субъекта. Но от существования и деятельности субъекта зависит вовлеченность части объективной реальности в
качестве объекта в его практику и познание. Следовательно, объективная реальность и объект познания не
тождественны.
Несовпадение объекта познания с объективной реальностью не только в том, что он представляет собой лишь
ее  часть,  но  и  в  том,  что  логическое  познание,  составляющее  необходимую  сторону  познания  вообще,
направлено на исследование идеализированных систем, построенных с помощью абстрагирования отдельных
сторон, свойств и отношений реальности. Это так называемые вторичные объекты. Е. К. Войшвилло дает
характеристику такого вторичного объекта: “Объектом мысли могут быть и сама мысль (понятие, суждение),
представление, образ вообще, и даже отдельные стороны, свойства мысли, образа (логическая форма мысли,
знаковая форма ее выражения)...”.30 В результате образования таких идеализированных систем происходит
определенное расхождение объекта познания и объективной реальности. Известно, что геометрия, механика и
др. науки изучают свойства объектов, которых как таковых в данном виде в самой действительности нет.
Исходя  из  сказанного,  следует  заключить,  что  объективная  реальность  лишь  в  конечном  счете  является
объектом познания. Если попытаться более полно и конкретно очертить объект познания, то следует выделить
относительно самостоятельные его стороны. Сюда входят: 1) предметы и явления природы, вовлеченные в
сферу субъективной деятельности в широком смысле, как практической, так и теоретической, познавательной;
2) человеческие отношения, общество во всех его аспектах; 3) отношение знания к явлениям материального
мира; 4) отношение средств выражения знания (знаковых систем) к знанию; 5) отношение средств выражения
знания (знаковых систем) к явлениям материального мира; 6) отношение элементов знания между собой; 7)
отношение  средств  выражения  знания  (знаковых  систем)  между  собой;  8)  отношение  средств  познания  к
явлениям материального мира; 9) отношение средств познания к знанию и т.п. Разумеется, это неполная и самая
общая градация, нуждающаяся в развитии и дальнейшей конкретизации. Таким образом, лишь в конечном
счете, в пределе объектом познания выступают предметы и явления Материального мира.
Сфера  объекта  познания  по  мере  развития  практической  и  познавательной  деятельности  постоянно
расширяется.  Познание,  особенно  научное,  в  не  меньшей  степени  является  фактором  расширения  сферы
объекта, нежели практика, хотя последняя и является исходом, началом и основой выделения объекта и для
познания. Выделение объекта познания, превращение субъекта в объект познания произошло одновременно с
возникновением общественного сознания, когда человек стал отделять себя от окружающей среды. Касаясь этой
особенности  человека,  К.  Маркс  писал:  “Животное  непосредственно  тождественно  со  своей
жизнедеятельностью. Оно не отличает себя от своей жизнедеятельности. Оно и есть эта жизнедеятельность.
Человек же делает сам свою жизнедеятельность предметом своей воли и своего сознания”.31
Объект  и  субъект  рассматриваются  в  познании  как  некие  противоположности.  Но  они  определены  и
отграничены не раз и навсегда. В разных отношениях, в разных аспектах, в разное историческое время они
могут выступать и в качестве субъекта, и в качестве объекта, конечно, исключая объективную реальность,
материальный мир, который не может быть субъектом в той своей части, которая не относится к человеку, к
обществу.
Идея противоречивого тождества субъекта и объекта, в абстрактной форме заявленная Гегелем и чрезвычайно
важная для понимания истины как процесса, получила свое онтологическое обоснование в категории практики,
выдвинутой диалектическим материализмом. Термин “практика” имеет емкое содержание. Это прежде всего
материальный  процесс  специфического  “обмена  веществ”  общества  и  природы,  предметно-чувственная
деятельность человека. Особенность данного процесса в том, что он включает в себя идеальный момент, ибо
осуществляется  людьми,  обладающими  сознанием,  имеющими  свои  цели,  планы,  представляющими
технологию  их  реализации,  прогнозирующими  результаты.  Наконец,  практика  —  процесс  социально-исторический,  не  только  потому,  что  его  субъектом  выступает  не  отдельное  существо,  а  организованное
человеческое  сообщество,  но  и  потому,  что  данное  сообщество  носит  исторический  характер.  Практика
воплощает в себе единство живого и овеществленного труда, выражающего преемственность развивающейся
материальной и идеальной культуры, субъектом которой выступает определенным образом обобществленное
человечество.
Категория практики имеет сложную структуру. Она включает в себя целый ряд необходимых соотношений —
идеального  и  материального,  индивидуального  и  социального,  прошлого  и  будущего,  преемственности  и
новизны. Указанные соотношения полностью теряют свой смысл и значение без основной диалектической
связи — субъекта и объекта.
В  современной  философии  эта  связь  оказалась  предметом  всеобъемлющей  критики.  В  постмодернизме
возникла тенденция размывания субъект-объектной оппозиции, на том якобы основании, что она перестает
выполнять роль ведущей оси, организующей мыслительное пространство. Теряется значение крайних терминов
оппозиции,  особенно  субъекта.  По  свидетельству  французского  философа  М.  Фуко,  понятие  “субъект”
употребляется разве что как своеобразная дань классической традиции. На самом деле можно говорить только о
том, что в определенных условиях некий индивид выполняет функцию субъекта. Постмодернистская культура,
по сути, декларирует бессубъектную философию. Однако “новизна”, выраженная в метафоре “смерть субъекта”,
представляет  тупиковую  ветвь  философской  эволюции.  Не  случайно  “After  постмодернизм”  (поздний
постмодернизм) заговорил о “возрождении субъекта”.
Категория практики, выступающая в своей онтологической функции, выражающая основу и способ бытия
человека в мире, указывает на немыслимость игнорирования философией понятия субъект. И именно практика,
представляющая  собой  систему  определенного  множества  деятельностей,  обусловливает  относительную
самостоятельность  Развития  ее  (практики)  идеальных  и  материальных  сторон,  индивидуального  и
общественного  в  реальном  процессе  преобразования  природы,  общества  и  человека,  субъективного  и
объективного.

\subsection{Познание как «отражение» и как «конструирование» действительности}
Чувственное и эмпирическое познание - не одно и то же. Чувственное знание - это знание в виде ощущений и
восприятии свойств вещей, непосредственно данных органам чувств
Эмпирическое знание может быть отражением данного не непосредственно, а опосредованно. Например, я
вижу  показание  прибора  или  кривую  электрокардиограммы,  информирующие  меня  о  состоянии
соответствующего  объекта,  которого  я  не  вижу.  Иначе  говоря,  эмпирический  уровень  познания  связан  с
использованием  всевозможных  приборов;  он  предполагает  наблюдение,  описание  наблюдаемого,  ведение
протоколов, использование документов, например историк работает с архивами и иными источниками. Словом,
это более высокий уровень познания, чем просто чувственное познание.
Исходным  чувственным  образом  в  познавательной  деятельности  является  ощущение  -  простейший
чувственный образ, отражение, копия или своего рода снимок отдельных свойств предметов. Многообразие
ощущений верно отображает объективный характер качественного многообразия мира и вызвано им. Потеря
способности ощущать неизбежно влечет за собой потерю сознания.
Специфичность органов чувств не только не препятствует правильному познанию внешнего мира, как это
пытались представить “физиологические” идеалисты, но, напротив, она обеспечивает наиболее полное и точное
отражение объективных свойств предметов.
Положение  об  ощущении  как  субъективном  образе  объективного  мира  направлено  своим  острием  против
механического деления качеств на первичные и вторичные. С этой точки зрения первичные качества (форма,
объем и т.д.) являются отражением объективно существующих особенностей предметов, а вторичные (цвет, звук
и т.д.) носят чисто субъективный характер. Отсюда ложный вывод: цвет, запах - это не свойства предметов, а
наши ощущения (Э.  Мах);  словом  “цвет” обозначается  определенный  класс психических переживаний  (В.
Оствальд). Мир же беззвучен, лишен красок, запахов.
Тот  или  иной  предмет  воздействует  на  органы  чувств  человека  какое-то  определенное  время.  Затем  это
воздействие  прекращается.  Но  образ  предмета  не  исчезает  сразу  же  бесследно.  Он  запечатлевается  и
сохраняется в памяти. Следовательно, мыслить что-то можно и по его исчезновении: ведь о нем остается
определенное представление. Душа обретает возможность оперировать образами вещей, не имея их в поле
чувственного восприятия.
Можно ли рассуждать о познании, игнорируя память? Конечно, нет: душа без памяти - что сеть без рыбы.
Никакое познание немыслимо без этого чудесного феномена. Процессы ощущения и восприятия оставляют
после себя “следы” в мозгу, суть которых состоит в способности воспроизводить образы предметов, которые в
данный  момент  не  воздействуют  на  человека.  Память  играет  очень  важную  познавательную  роль.  Она
объединяет прошедшее и настоящее в одно органическое целое, где имеется их взаимное проникновение.
Человек может творчески комбинировать и относительно свободно создавать новые образы. Представление -это  промежуточное  звено  между  восприятием  и  теоретическим  мышлением.  Познание  невозможно  без
воображения:  оно  есть  свойство  человеческого  духа  величайшей  ценности.  Воображение  восполняет
недостаток наглядности в потоке отвлеченной мысли. Сила воображения не только снова вызывает имеющиеся
в опыте образы, но и связывает их друг с другом и, таким образом, поднимает их до общих представлений.
Воспроизведение образов осуществляется силой воображения произвольно и без помощи непосредственного
созерцания.
Люди стремятся познать то, чего они еще не знают. Но для начала они должны, хотя бы в самом общем виде,
знать, чего же они не знают и что они хотят знать. “Не всякий знает, как много надо знать, чтобы знать, как мало
мы знаем”, - гласит восточное изречение.
Правильная  постановка  проблемы  направляет  поиски  ее  решения.  Пытаться  найти  решение  поставленной
проблемы можно двумя путями: искать нужную информацию в существующей литературе или самостоятельно
исследовать проблему с помощью наблюдений, экспериментов и теоретического мышления.
Важными методами исследования в науке, особенно в естествознании, являются наблюдение и эксперимент.
Наблюдение представляет собой преднамеренное, планомерное восприятие, осуществляемое с целью выявить
существенные свойства  и отношения  объекта  познания.  Эксперимент  это  метод  исследования, с  помощью
которого объект или воспроизводится искусственно, или ставится в определенные условия, отвечающие целям
исследования.  Особую  форму  познания  составляет  мысленный  эксперимент,  который  совершается  над
воображаемой моделью. Для него характерно тесное взаимодействие воображения и мышления. Основным
методом эксперимента является метод изменения условий, в которых обычно находится исследуемый предмет.
Он дает возможность вскрыть причинную зависимость между условиями и свойствами исследуемого объекта, а
также характер изменения этих свойств в связи с изменением условии.
В  ходе  и  в  результате  наблюдения  и  эксперимента  осуществляется  описание  или  протоколирование.  Оно
производится  и  в  виде  отчета  с  использованием  общепринятых  терминов,  и  наглядным  образом  в  виде
графиков, рисунков, фото- и кинопленок, и символически в виде математических, химических формул и т.п.
Основное научное требование к описанию - это достоверность, точность воспроизведения данных наблюдений
и эксперимента. 


\newpage
\section{Основные формы чувственного познания}
Всякое познание начинается с живого созерцания, с ощуще­ний, чувственных восприятии.
Предметы воздействуют на наши органы чувств и вызывают в них ощущения, которые восприни­маются мозгом.
Других средств приема сигналов из внешнего мира для передачи их в мозг, кроме органов чувств, у человека нет.

Формами чувственного познания являются
\begin{enumerate}
\item \textit{ощущения}.
	Ощущение - это отражение отдель­ных свойств предметов или явлений материального мира, непосредственно воздействующих на органы чувств (например, ощущения горького, соленого, теплого, красного, круглого, гладкого и т.д.).
	Каждый предмет имеет не одно, а множество свойств.
	В ощу­щениях и отражаются различные свойства предметов.
	Ощущения как субъективный образ объективного мира возникает в коре больших полушарий головного мозга.
	Чувствительность органов чувств повышается в зависимости от тренировки.
	Обычный че­ловек различает, например, 3-4 оттенка черного цвета, профес­сионалы - до 40 оттенков.
	В ощущениях осуществляется связь сознания с внешним ми­ром.
	Ощущения возникают в результате воздействия предметов на различные органы чувств - органы зрения, слуха, обоняния, осязания, вкуса.
	Если человек лишен одного или нескольких орга­нов чувств (как, например, у слепоглухонемых), то остальные ор­ганы чувств значительно обостряются и частично восполняют функции недостающих.
	Пьеса Гибсона “Сотворившая чудо” рассказывает о детстве и обучении американской слепоглухонемой девочки Элен Келлер. 
	Эта пьеса очень ярко передает всю трудность общения с Элен и методику ее обучения.
	Когда девоч­ка произнесла первое слово - вода, это было воспринято как чудо. Она научилась говорить, хотя сама не слышала своего голоса.
\item \textit{восприятие}.
	Восприятие есть целостное отражение внешнего материаль­ного предмета, непосредственно воздействующего на органы чувств (например, образы автобуса, пшеничного поля, электро­станции, книги и т. д.).
	Восприятия слагаются из ощущений.
	Так, восприятие апельсина слагается из таких ощущений: шарообраз­ный, оранжевый, сладкий, ароматный и др.
	Восприятия, хотя и являются чувственным образом в отражении предмета, который воздействует на человека в данный момент, но во многом зави­сят от прошлого опыта.
	Полнота, целенаправленность восприятия, например, зеленого луга, будет различной у ребенка, у взрослого, художника, биолога или крестьянина (художник восхитится его красотой, биолог увидит на нем виды некоторых лекарственных или нелекарственных растений, крестьянин прикинет, сколько же с него можно скосить травы, получить сена и т. д.).
	Насколько сильно восприятия переплетаются с прежним опы­том и знаниями, видно из следующей истории.
	Рассказывают, что один европеец, путешествуя по Центральной Африке, оста­новился в негритянской деревушке, жители которой не имели представления о книгах и газетах.
	Пока ему меняли лошадей, он раскрыл газету и начал ее читать.
	Вокруг него собралась толпа и внимательно следила за ним.
	Когда путешественник уже приготовился ехать дальше, к нему подошли местные жители и попросили продать газету за большие деньги.
	На вопрос путеше­ственника, зачем нужна им газета, они ответили, что они видели, как он долго смотрел на черные изображения на ней и, очевидно, лечил свои глаза, и они хотели бы иметь у себя это лечебное средство.
	Так, жители этой деревни, не зная, что такое чтение, и рассуждая на основе своего прежнего опыта, восприняли газету как лечебное средство.
\item \textit{представление}.
	Представление - это чувственный образ предмета, в данный момент нами не воспринимаемого, но который ранее в той или иной форме воспринимался.
	Представление может быть воспроизво­дящим (например, у каждого есть сейчас образ своего дома, сво­его рабочего места, образы некоторых знакомых и родных людей, которых мы сейчас не видим).
	Представление может быть и твор­ческим, в том числе фантастическим.
	Творческое представление у человека может возникнуть и по словесному описанию.
	Так, мы можем по описанию представить себе тундру или джунгли, хотя там не были ни разу, или полярное сияние, хотя не были на севере и не видели его.
	По описанию внешнего облика какого-то реального человека или литературного героя мы стараемся зрительно создать его образ, представить его внешность.
	Приведем пример. Вспом­ним в этой связи сцену из кинофильма Чарли Чаплина “Граф”.
	Мнимый граф Чарли попал в затруднительное положение.
	Когда перед ним положили большой кусок арбуза, он по неведению атаковал его без ножа и вилки.
	Как и следовало ожидать, вы­грызать мякоть арбуза вскоре стало неудобно.
	Острые и жест­кие края корки залезали даже в уши.
	Чтобы избежать этого, Чарли подвязал щеки салфеткой.
	Это действие уже смешно - ведь куда проще было разрезать или разломить кусок арбуза.
	Но оно по­влекло за собой и вторичный комический эффект: с подвязанной салфеткой вокруг головы Чарли приобрел вид человека, страда­ющего от зубной боли.
	Так для создания комического эффекта Чаплин использует простые явления реальной жизни, представ­ленные в неожиданном, а потому смешном освещении.
\end{enumerate}

Путем чувственного отражения мы познаем явление, но не сущ­ность, отражаем отдельные предметы во всей их наглядности.
Законы мира, сущность предметов и явлений, общее в них мы познаем посредством абстрактного мышления - более сложной формы познания.
Абстрактное, или рациональное, мышление от­ражает мир и его процессы глубже и полнее, чем чувственное познание.
Переход от чувственного познания к абстрактному мыш­лению представляет собой скачок в процессе познания.
Это --- скачок от познания фактов к познанию законов.


\newpage
\section{Основные формы рационального познания}

Формы абстрактного мышления:
\begin{enumerate}
\item \textit{Понятие}.
	Понятие - форма мышления, в которой отражаются сущест­венные признаки одноэлементного класса или класса однород­ных предметов.
	Понятия в языке выражаются отдельными сло­вами (“портфель”, “трапеция”) или группой слов, т. е. словосо­четаниями (“студент медицинского института”, “производитель материальных благ”, “река Нил”, “ураганный ветер” и др.).
\item \textit{Суждение}.
	Суждение - форма мышления, в которой что-либо утвержда­ется или отрицается о предметах, их свойствах или отношениях. 
	Суждение выражается в форме повествовательного предложения.
	Суждения могут быть простыми и сложными. Например:
	“Саранча опустошает поля” - простое суждение, а суждение “Наступила весна, прилетели грачи” - сложное, состоящее из двух простых.
\item \textit{Умозаключение}.
	Умозаключение - форма мышления, посредством которой из одного или нескольких суждений, называемых посылками, мы по определенным правилам вывода получаем заключение. Видов умо­заключений много; их изучает логика.
\end{enumerate}

В процессе познания мы стремимся достичь истинного зна­ния.
\textit{Истина} есть адекватное отражение в сознании человека яв­лений и процессов природы, общества и мышления'.
Истинность знания есть соответствие его действительности.
Законы науки представляют собой истину. Истину могут дать нам и формы чувственного познания - ощущения и восприятия.
Понимание истины как соответствия знания вещам восходит к мыслителям древности, в частности, к Аристотелю.

Как отличить истину от заблуждения? Критерием истины яв­ляется практика.
Под практикой понимают всю общественную и производственную деятельность людей в определенных истори­ческих условиях, т.е. это материальная, производственная дея­тельность людей в области промышленности и сельского хозяй­ства, а также политическая деятельность, борьба за мир, соци­альные революции и реформы, научный эксперимент и т. д.

“...Практика человека и человечества есть проверка, крите­рий объективного познания”.
Так, прежде чем пустить машину в массовое производство, ее проверяют на практике, в дейст­вии, самолеты испытывают летчики-испытатели, действие ме­дицинских препаратов сначала проверяют на животных, потом, убедившись в их пригодности, используют для лечения людей.
Прежде чем послать в космос человека, советские ученые про­вели серию испытаний с животными.

\subsection*{Особенности абстрактного мышления}
С помощью рационального (от лат. ratio - разум) мышления люди открывают законы мира, обнаруживают тенденции развития событий, анализируют общее и особенное в любом предмете, строят планы на будущее и т. д.

Выделяют следующие особенности абстрактного мышления:
\begin{enumerate}
\item Мышление отражает действительность в обобщен­ных формах.
	В отличие от чувственного познания абстрактное мышление, отвлекаясь от единичного, выделяет в сходных пред­метах только общее, существенное, повторяющееся (например, выделяя общие признаки, присущие всем инертным газам, мы образуем понятие “инертный газ”).
	С помощью абстрактного мышления создаются научные понятия (именно так были соз­даны следующие понятия: “материя”, “сознание”, “движение”, “государство”, “наследственность”, “ген” и др.).
\item Абстрактное мышление - форма опосредованного отражения мира.
	Человек может получать новую информа­цию без непосредственной помощи органов чувств, лишь на ос­нове имеющихся у него знаний (например, по уликам юристы судят о происшедшем преступлении, строят свои умозаключе­ния и выдвигают различные версии о предполагаемом преступ­нике или преступниках).
\item Абстрактное мышление - процесс активного отраже­ния действительности.
	Человек, определяя цель, способы и ставя сроки осуществления своей деятельности, активно преобразует мир.
	Активность мышления проявляется в творческой деятельности человека, его способности к воображению, в на­учной, художественной и другой фантазии.
\item Абстрактное мышление неразрывно связано с языком.
	Язык - способ выражения мысли, средство закрепления и пере­дачи мыслей другим людям.
	Познание направлено на получение истинного знания, к которому приводит как чувственное позна­ние, так и абстрактное мышление.
	Мышление представляет со­бой отражение объективной реальности.
\end{enumerate}


\newpage
\section{Мышление и язык. Основные функции языка}
Важно отметить, что связи мыслительных процессов с лингвистическими структурами, широко обсуждается
сегодня  представителями  различных  школ  и  учений  философии  -  структурализмом,   постпозитивизмом
(лингвистический позитивизм), герменевтикой и др. Представители постпозитивизма обсуждают, как правило,
отношения  между  мышлением  и  языком  в  рамках  проблемы  духовного  и  телесного  (“ментального”  и
“физического”).  Одним  из  наиболее  активных  защитников  идеи  “экстралингвистического  знания”  является
К.Хуккер.  Он  исходит  из  того,  что  лингвистические  структуры  -  это  подкласс  информационных  структур,
поэтому недопустимо, по его мнению, отождествлять мысль и речь. Справедливо отмечая, более широкий
характер информационных структур по сравнению с лингвистическими К.Хуккер склонен к абсолютизации их,
придания  им  статуса  бытийности.  Из  этой  идеи  исходит  и  другая  идея  постпозитивизма  -  о  тождестве
“ментального” и “физического”, эту идею пропагандируют  “элининативныке материалисты”. Они полагают,
что “ментальные термины” теории языка и мышления должны быть элиминированы, как ненаучные и заменены
терминами нейрофизиологии. Чтобы решить эту задачу, нужно, прежде всего, как они полагают, отвергнуть
“миф  данного”,  т.е.  утверждение  о  том,  что  мы  располагаем  некоторым  непосредственным  и  мгновенным
знанием  о  собственных  “ментальных”  процессах.  Пожалуй,  самым   решительным  образом  отрицает
“непосредственно  данное”  П.Фейерабент.  По  его  убеждению, “непосредственно  данное”  является  вовсе не
фактом природы, а “результатом того способа, которым любой род занятий (или мнения) относительно сознания
воплощен  и  воплощается  в  языке”.  Этот  “якобы  факт  природы”  есть  типичная  кажимость,  обусловленная
“бедностью содержания ментальных терминов по сравнению с физическими терминами”/
Заметим, что отрицание “непосредственного данного” означает, что знание существует только тогда, когда оно
вербализовано, т.е. выражено словами. Если этот факт “непосредственно данного” признается, то вопрос о его
отношении  языку  и  речи  решается  по-разному.  Встречается  точка  зрения,  что  непосредственное  знание  о
собственных сознательных состояниях всегда так или иначе вербализовано.  Например, Г.Фейгл говорит о
наличии  сугубо  личного  языка,  с  помощью  которого  субъект  выражает  для  себя  указанное  знание.
Непосредственное знание, прямой опыт он называет “сырыми чувствами”. Последние и выступают в форме
“личного  языка”,  который  в  процессе  общения  переводится  на  интерсубъективный,  обыденный  язык.  Эти
примеры  свидетельствуют  о  междисциплинарном  характере  проблемы  соотношения  языка  и  мышления  и
возможности различных трактовок этого взаимодействия.
Язык - главная из знаковых систем человека, важнейшее средство человеческого общения. К. Маркс, например,
назвал язык “непосредственной действительностью мысли”. С помощью слов можно интерпретировать другие
знаковые системы (например, можно описать картину). Язык - универсальный материал, который используется
людьми при объяснении мира и формировании той или иной его модели. Хотя художник может это сделать и
при помощи зрительных образов, а музыкант - при помощи звуков, но все они вооружены, прежде всего,
знаками универсального кода - языка. 
Язык - это особая знаковая система. Любой язык состоит из различных слов, то есть условных звуковых знаков,
обозначающих  различные  предметы  и  процессы,  а  также  из  правил,  позволяющих  строить  из  этих  слов
предложения.  Именно  предложения  являются  средством  выражения  мысли.  С  помощью  вопросительных
предложений люди спрашивают, выражают свое недоумение или незнание, с помощью повелительных - отдают
приказы, повествовательные предложения служат для описания окружающего мира, для передачи и выражения
знаний о нем. Совокупность слов того или иного языка образует его словарь. Словари наиболее развитых
современных языков насчитывают десятки тысяч слов. С их помощью благодаря правилам комбинирования и
объединения слов в предложения можно написать и произнести  неограниченное количество осмысленных
фраз, заполнив ими сотни миллионов статей, книг и файлов. В силу этого язык позволяет выражать самые
разные мысли, описывать чувства и переживания людей, формулировать математические теоремы и т.д.
Язык может быть устным и письменным, он возникает в человеческом сообществе, выполняя  важнейшие
функции:  выражения  мысли  или  сознания;хранения  и  передачи  информации;средства  общения  или
коммуникативную.


\newpage
\section{Проблема истины. Основные концепции и критерии}
Проблема истины является ведущей в гносеологии. Все проблемы теории познания касаются либо средств и
путей достижения истины, либо форм существования истины.В философии существуют следующие концепции
истины:1)  классическая  теория  истин.  Истина  -  это  правильное  отражение  предмета,   процесса  в
индивидуальном познании.- самая древнейшая из концепций, её поддерживали  Аристотель, Платон, Гегель,
Фейербах.2)  когерентная  концепция  -  рассматривает  истину   как  соответствие  одних  знаний  другим.3)
прагматическая концепция. Истиной считается  то, что полезно для человека.4) конвенциальная концепция.
Истина - это то, что считает большинство.5) экзистенциальная концепция (Хайдеггер). Истина есть свобода,
это процесс, с одной стороны, в котором мир открывается нам с одной стороны, а с другой - человек сам волен
выбирать, каким способом и чем можно познать этот мир.6) неатомистическая концепция. Истина - это божье  
откровение.Характерной чертой истины является наличие в ней объективной по содержанию и субъективной
по форме сторон.
Истина объективна - это значит, что содержание человеческих представлений не зависит  от субъекта, не
зависит  ни  от  человека,  ни  от  человечества.  В  идеалистических  системах   истина  понимается  как  вечно
неизменное  и  абсолютное  свойство  идеальных  объектов.   Точка  зрения  сторонников  субъективно-идеалистического  эмпиризма  состоит  в  понимании   истинности  как  соответствие  мышления  ощущениям
субъекта или как соответствие идей  стремлениям личности к достижению успеха. На каждом историческом
этапе человечество располагает относительной истиной - приблизительно адекватным, неполным, содержащим
заблуждения знанием. Истинное знание каждой эпохи содержит элементы абсолютной  истины. Абсолютная
истина - такое знание, которое полностью исчерпывает предмет познания и не может быть опровергнуто при
дальнейшем  развитии  познания.  Всякая  относительная   истина  содержит  элемент  абсолютного  знания.
Абсолютная истина складывается из суммы относительных истин. Критерий истины находится не в мышлении
самом по себе и не в действительности, взятой вне субъекта, а заключается в практике.
Истина  –  это  адекватная  информация  об  объекте,  получаемая  посредством  его  чувственного  или
интеллектуального постижения либо сообщения о нем и характеризуемая с точки зрения ее достоверности.
Таким  образом,  истина  существует  не  как  объективная,  а  как  субъективная,  духовная  реальность  в  ее
информационном и ценностном аспектах.
Истина относительна, ибо она отражает объект не полностью, не целиком, не исчерпывающим образом, а в
изветсных  пределах,  условиях, отношениях,  которые  постоянно  изменяются  и  развиваются.  Относительная
истина есть ограниченно верное знание о чем-либо. Гора познания не имеет вершины. Истины, познанные
наукой на том или ином историческим этапе, не могут считаться окончательными. Они по необходимости
являются относительными, т.е. истинами, котрые нуждаются в дальнейшем развитии, углублении, уточнении.
Абсолютная истина – это такое содержание знания, которое не опровергается последующим развитием науки, а
обогащается  и  постоянно  подтверждается  жизнью. Под  абсолютной  истиной  в  науке  имеют  в  виду
исчерпывающее, предельное знание об объекте, как бы достижение тех границ, за которыми уже больше нечего
познавать.  Процесс  развития  науки  можно  представить  в  виде  ряда  последовательных  приближений  к
абсолютной истине, каждое из которых точнее, чем предыдущие.
Конкретность – это свойство истины, основанное на знании реальных связей, взаимодействия всех сторон
объекта,  главных,  существенных  свойств,  тенденций  его  развития.  Принцип  конкретности  истины  требует
подходить к фактам не с общими формулами и схемами, а с учетом конкретной обстановки, реальных условий.
Что дает людям гарантию истинности их знаний, служит основанием для отличения истины от заблуждения и
ошибок?
Р. Декарт, Б. Спиноза, Г. Лейбниц предлагали в качестве критерия истины ясность и отчетливость мыслимого.
Ясно то, что открыто для наблюдающего разума и с очевидностью признается таковым, не возбуждая сомнений.
Пример такой истины – «квадрат имеет четыре стороны». Подобного рода истины – результат «естественного
света разума». Однако, этот критерий не гарантирует надежности.
Выдвигался  и  такой  критерий  истины  как  общезначимость:  истинно  то,  что  соответствует  мнению
большинства. Однако еще Декарт заметил, что вопрос об истинности не решается большинством голосов. Из
истории науки мы знаем, что первооткрыватели, отсаивая истину, как правило, оказывались в одиночестве.
В некоторых философских системах существует такой критерий истины, как принцип прагматизма, т.е. теории
узкоутилитарного понимания истины, игнорирующего ее предметные основания и ее объективную значимость.
Истиной прагматизм признает то, что лучше всего «работает» на нас.
Один из фундаментальных принципов научного мышления гласит: некоторое положение является истинным в
том случае, если  можно  доказать, применимо ли оно в той или  иной  конкретной ситуации. Это принцип
выражается  термином  «реализуемость».  Посредством  реализации  идеи  в  практическом  действии  знание
соизмеряется,  сопоставляется  со  своим  объектом,  выявляя  тем  самым  настоящую  меру  объективности,
истинности своего содержания. В знании истинно то, что прямо или косвенно подтверждено на практике, т.е.
результативно осуществлено в практике.


\newpage
\section{Феноменологическая концепция познания, ее основные принципы и понятия}


\newpage
\section{Проблема человека в истории философии}
Философия Древнего Востока о человеке.
Первые  представления о человеке  возникают задолго до  самой  философии.  На начальных этапах истории
людям  присущи  мифологические  и  религиозные  формы  самосознания.  В  преданиях,  сказаниях,  мифах
раскрывается  понимание  природы,  предназначения  и  смысла  человека  и  его  бытия.  Кристаллизация
философского понимания человека происходит как раз на базе заложенных в них представлений, идей, образов
и понятий и в диалоге между формирующейся философией и мифологией. Именно таким образом и возникают
первые учения о человеке в государствах Древнего Востока.
Древнеиндийская философия человека представлена прежде всего в памятнике древнеиндийской литературы —
Ведах,  в  которых  выражено  одновременно  мифологическое,  религиозное  и  философское  мировоззрение.
Повышенный интерес к человеку и в примыкающих к Ведам текстах — упанишадах. В них раскрываются
проблемы нравственности человека, а также пути и способы освобождения его от мира объектов и страстей.
Человек  считается  тем  совершеннее  и  нравственнее,  чем  больше  он  достигает  успеха  в  деле  такого 
освобождения. Последнее, в свою очередь, осуществляется посредством растворения индивидуальной души
(атмана) в мировой душе, в универсальном принципе мира (брахмане).
Человек  в  философии  Древней  Индии  мыслится  как  часть  мировой  души.  В  учении  о  переселении  душ
(сансаре)  граница  между  живыми  существами  (растениями,  животными,  человеком)  и  богами  оказывается
проходимой  и  подвижной.  Но  важно  заметить,  что  только  человеку  присуще  стремление  к  свободе,  к
избавлению от страстей и пут эмпирического бытия с его законом сансары-кармы. В этом пафос упанишад.
Упанишады оказали огромное влияние на развитие всей философии человека в Индии. В частности, велико их
влияние  на  учения  джайнизма,  буддизма,  индуизма,  санкхьи,  йоги.  Это  влияние  сказалось  и  на  взглядах
известного индийского философа М.К. Ганди.
Философия Древнего Китая создала также самобытное учение о человеке. Один из наиболее значительных ее
представителей — Конфуций разработал концепцию «неба», которое означает не только часть природы, но и
высшую духовную силу, определяющую развитие мира и человека. Но в центре его философии находится не
небо,  не  природный  мир  вообще,  а  человек,  его  земная  жизнь  и  существование,  т.е.  она  носит
антропоцентристский характер.
Обеспокоенный  разложением  современного  ему  общества,  Конфуций  обращает  внимание  прежде  всего  на
нравственное поведение человека. Он писал, что наделенный небом определенными этическими качествами,
человек обязан поступать в согласии с моральным законом — дао и совершенствовать эти качества в процессе
обучения. Целью обучения является достижение уровня «идеального человека», «благородного мужа» (цзюнь-цзы), концепцию которого впервые разработал Конфуций. Чтобы приблизиться к цзюнь-цзы, каждый должен
следовать целому ряду этических принципов. Центральное место среди них принадлежит концепции жэнь
(человечность, гуманность, любовь к людям), которая выражает закон идеальных отношений между людьми в
семье и государстве в соответствии с правилом «не делай людям того, чего не пожелаешь себе». Это правило в
качестве нравственного императива в разных вариантах будет встречаться потом и в учениях «семи мудрецов» в
Древней Греции, в Библии, у Канта, у Вл. Соловьева и других. Особое внимание Конфуций уделяет принципу
сяо (сыновняя почтительность и уважение к родителям и старшим), являющемуся основой других добродетелей
и самым эффективным методом управления страной, рассматриваемой как «большая семья». Значительное
внимание он уделял также таким принципам поведения, как ли (этикет), и (справедливость) и др.
Наряду с учением Конфуция и его последователей в древнекитайской философии следует отметить и другое
направление — даосизм. Основателем его является Лао-цзы. Исходной идеей даосизма служит учение о дао
(путь, дорога) — это невидимый, вездесущий, естественный и спонтанный закон природы, общества, поведения
и. мышления отдельного человека. Человек должен следовать в своей жизни принципу дао, т.е. его поведение
должно  согласовываться  с  природой  человека  и  вселенной.  При  соблюдении  принципа  дао  возможно
бездействие, недеяние, приводящее тем не менее к полной свободе, счастью и процветанию.
Характеризуя древневосточную философию человека, отметим, что важнейшей чертой ее является ориентация
личности на крайне почтительное и гуманное отношение как к социальному, так и природному миру. Вместе с
тем эта философская традиция ориентирована на совершенствование внутреннего мира человека. Улучшение
общественной жизни, порядков, нравов, управления и т. д. связывается прежде всего с изменением индивида и
приспособлением его к обществу, а не с изменением внешнего мира и обстоятельств. Человек сам определяет
пути  своего  совершенствования  и  является  своим  богом  и  спасителем.  Нельзя  при  этом  забывать,  что
характерной чертой философского антропологизма является трансцендентализм — человек, его мир и судьба
непременно связываются с трансцендентным (запредельным) миром.
Философия человека Древнего Востока оказала огромное влияние на последующее развитие учений о человеке,
а также на формирование образа жизни, способа мышления, культурных образцов и традиций стран Востока.
Общественное и  индивидуальное сознание  людей в этих странах до  сих  пор находится под  воздействием
образцов, представлений и идей, сформулированных в тот далекий период.
Проблема человека в философии Древней Греции.
Античная  Греция  положила  начало  западноевропейской  философской  традиции  вообще  и  философской
антропологии в частности. древнегреческой философии первоначально человек не существует сам по себе, а
лишь в системе определенных отношений, воспринимаемых как абсолютный порядок и космос. Со всей своей
природной  и  социальной  средой,  соседями  и  полисом,  неодушевленными  и  одушевленными  предметами,
животными и богами он живет в едином, нераздельном мире. Даже боги, также находящиеся внутри космоса,
являются  для  людей  реальными  действующими  лицами.  Само  понятие  космоса  здесь  имеет  человеческий
смысл,  вместе  с  тем  человек  мыслится  как  часть  космоса,  как  микрокосм,  являющийся  отражением
макрокосмоса,  понимаемого  как  живой  организм.  Именно  таковы  взгляды  на  человека  у  представителей
милетской школы, стоящих на позициях гилозоизма, т.е. отрицавших границу между живым и неживым и
полагавших всеобщую одушевленность универсума.
Поворот  к  собственно  антропологической  проблематике  связан  с  критической  и  просветительской
деятельностью софистов и создателем философской этики Сократом.
Исходный принцип софистов, сформулированный их лидером Протагором, следующий: «Мера всех вещей —
человек, существующих, что они существуют, а несуществующих, что они не существуют».
В концепции софистов следует обратить внимание прежде всего на три момента:
- релятивизм и субъективизм в понимании таких этических феноменов, как благо, добродетель, справедливость  
и т.д.;
- в бытие как главное действующее лицо они вводят человека;
- впервые процесс познания они наполняют экзистенциальным смыслом и обосновывают экзистенциальный
характер истины.
Для Сократа основной интерес представляет внутренний мир человека, его душа и добродетели. Он впервые
обосновывает принцип этического рационализма, утверждая, что «добродетель есть знание». Поэтому человек,
познавший что такое добро и справедливость, не будет поступать дурно и несправедливо. Задача человека как
раз и состоит в том, чтобы всегда стремиться к нравственному совершенству на основе познания истины. И
прежде всего она сводится к познанию самого себя, своей нравственной сущности и ее реализации.
Всей своей жизнью Сократ старался реализовать нравственный пафос своей философии человека, а сама его
смерть, когда он ради утверждения справедливости отказался от жизни, явилась апофео¬зом его нравственной
философии.
Демокрит — представитель материалистического монизма в учении о человеке. Человек, по Демокриту, — это
часть природы, и, как вся природа, он состоит из атомов. Из атомов же состоит и душа человека. Вместе со
смертью  тела уничтожается  и душа.  В отличие  от такого вульгарно-материалистического  взгляда  на душу
человека его этическая концепция носит более деликатный характер. Цель жизни, по нему, — счастье, но оно не
сводится к телесным наслаждениям и эгоизму. Счастье — это прежде всего радостное и хорошее расположение
духа — эвтюмия. Важнейшее условие ее — мера, соблюсти которую помогает человеку разум. Как утверждал
Демокрит, «желать чрезмерно подобает ребенку, а не мужу», мужественным же человеком является тот, кто
сильнее своих страстей.
В отличие от Демокрита Платон стоит на позиции антропологического дуализма души и тела. Но именно душа
является субстанцией, которая делает человека человеком, а тело рассматривается как враждебная ей материя.
Поэтому от качества души зависит и общая характеристика человека, его предназначение и социальный статус.
На первом месте в иерархии душ находится душа философа, на последнем — душа тирана. Это объясняется
тем, что душа философа наиболее мудра и восприимчива к знанию, а это и является главным в характеристике
сущности человека и его отличия от животного.
Человеческая душа постоянно тяготеет к трансцендентному миру идей, она вечна, тело же смертно. Это учение
о двойственном характере человека оказало влияние на средневековое религиозное учение о нем. В единстве и
противоположности  души  и  тела  заключен,  по  Платону,  вечный  трагизм  человеческого  существования.
Телесность ставит человека  в  животный  мир,  душа  возвышает его  над этим  миром, тело  — это материя,
природа, душа же устремлена в мир идей. Позднее этот трагизм станет одним из существенных моментов
русской религиозной философской антропологии.
В концепции Аристотеля человек рассматривается как существо общественное, государственное, политическое.
И эта социальная природа человека отличает его и от животного, и от «недоразвитых в нравственном смысле
существ», и от «сверхчеловека». По этому поводу он пишет, что «тот, кто не способен вступать в общение или,
считая  себя  существом  самодовлеющим,  не  чувствует  потребности  ни  в  чем,  уже  не  составляет  элемента
государства, становясь либо животным, либо божеством».
Еще один отличительный признак человека — его разумность, «человек и есть в первую очередь ум». Таким
образом,  человек,  по  Аристотелю,  —  это  общественное  животное,  наделенное  разумом.  Социальность  и
разумность — две основные характеристики, отличающие его от животного.
К этому следует добавить, что Аристотель вплотную подходит к формулировке положения о деятельностной
сущности  человека.  Он,  в  частности,  пишет,  что  добродетельная  жизнь  человека  имеет  проявление  в
деятельности, в которой заключена и единственная возможность самореализации личности.
Новая сторона философского антропологизма обнаруживается в эпоху разложения древнегреческого общества.
На  первый  план  здесь  выступают  проблемы  человека,  связанные  с  социальным  и  нравственным  упадком,
утратой экзистенциальных ценностей и смысла жизни людей. В этой ситуации на передний план выдвигается
интеллектуально-терапевтическая  функция  философии,  т.е.  та  функция,  которую  В.  Франкл  назвал
логотерапевтической. Особенно ярко она выражена в учении Эпикура, который утверждал, что подобно тому,
как  медицина  помогает  лечить  тело  человека,  философия  должна  помогать  лечить  его  душу.  В  плане
соотношения  индивида  и  общества  Эпикур  стоит  на  позициях  методологического  и  социально-этического
индивидуализма. Исходный пункт рассмотрения общества и человека — это индивид. Социум — это лишь
средство для удовлетворения потребностей отдельного человека, его желаний и блага.
В заключение отметим, что древнегреческая философская антропология, как и древневосточная, несет на себе
печать мифологии и религии и развивается в непосредственном диалоге с ними.
Так же, как древневосточная философия человека оказала огромное влияние на все последующее ее развитие в
рамках  восточной  традиции,  древнегреческая  философская  антропология  является  начатом  и  источником
западноевропейской традиции в философии человека.
Средневековая христианская концепция человека.
В средние века человек рассматривается прежде всего, как часть мирового порядка, установленного Богом. А
представление о нем самом, как оно выражено в христианстве, сводится к тому, что человек есть «образ и
подобие Бога». Но согласно этой точки зрения в реальности этот человек внутренне раздвоен вследствие его
грехопадения,  поэтому  он  рассматривается  как  единство  божественной  и  человеческой  природы,  которое 
находит свое выражение в личности Христа. Поскольку каждый изначально обладает божественной природой,
он  имеет  возможность  внутреннего  приобщения  к  божественной  «благодати»  и  тем  самым  сделаться
«сверхчеловеком».  В  этом  смысле  концепция  сверхчеловека  часто  развивается  и  в  русской  религиозной
философии.
В социальном плане в Средние века человек провозглашается пассивным участником божественного порядка и
является  существом  тварным  и  ничтожным  по  отношению  к  Богу.  В  отличие  от  античных  богов,  как  бы
родственных  человеку,  христианский  бог  стоит  над  природой  и  человеком,  является  их  трансцендентным
творцом и творческим началом. Главная задача для человека состоит в том, чтобы приобщиться к богу и
обрести спасение в день страшного суда. Поэтому вся драма человеческой истории выражается в парадигме:
грехопадение — искупление. И каждый человек призван реализовать это, соизмеряя свои поступки с Богом. В
христианстве каждый сам за себя отвечает перед Богом.
Видным представителем средневековой христианской философии является Августин Блаженный. Не только его
онтология и учение о боге как абсолютном бытии, но и учение о человеке многое берет от Платона. Человек —
это противоположность души и тела, которые являются независимыми. Однако именно душа делает человека
человеком. Это собственная, имманентная субстанция его. То, что Августин вносит нового по этому вопросу, —
развитие  человеческой  личности,  которое  он  рассматривает  в  «Исповеди».  Она  представляет
автобиографическое  исследование,  описывающее  внутреннее  становление  автора  как  личности.  Здесь  мы
находим и психологический самоанализ, и показ противоречивого характера развития личности, и указание на
темные  бездны  души.  Учение  Августина  повлияло  на  последующее  формирование  экзистенциализма,
представители которого рассматривают его как своего предшественника.
В  отличие  от  Августина  Фома  Аквинский  использует  для  обоснования  христианского  учения  о  человеке
философию  Аристотеля.  Человек  —  это  промежуточное  существо  между  животными  и  ангелами.  Он
представляет  единство  души  и  тела,  но  именно  душа  является  «двигателем»  тела  и  определяет  сущность
человека.  В  отличие  от  Августина,  для  которого  душа  является  не  зависимой  от  тела  и  тождественной  с
человеком, для Фомы Аквинского человек есть личностное единство того и другого. Душа — нематериальная
субстанция, но получает свое окончательное осуществление только через тело.
Человек в философии эпохи Возрождения.
Философская антропология эпохи Возрождения формируется под влиянием зарождающихся капиталистических
отношении, научного знания и новой культуры, получившей название гуманизм.
Если религиозная философия Средневековья решала проблему человека в мистическом плане, то философия
эпохи Возрождения (Ренессанса) ставит человека на земную основу и на этой почве пытается решить его
проблемы. В противоположность учению об изначальной греховности человека она утверждает естественное
стремление  его  к  добру,  счастью  и  гармонии.  Ей  органически  присущи  гуманизм  и  антропоцентризм.  В
философии этого периода Бог не отрицается полностью. Но, несмотря на пантеизм, философы делают своим
знамением  не  его,  а  человека.  Вся  философия  оказывается  проникнута  пафосом  гуманизма,  автономии
человека, верой в его безграничные возможности.
Так, согласно Пико делла Мирандоле (1463—1494), человек занимает центральное место в мироздании. Это
происходит потому, что он причастен всему земному и небесному. Астральный детерминизм он отвергает в
пользу свободы воли человека. Свобода выбора и творческие способности обусловливают то, что каждый сам
является  творцом  своего  счастья  или  несчастья  и  способен  дойти  как  до  животного  состояния,  так  и
возвыситься до богоподобного существа.
В  философской  антропологии  этого  периода  уже  достаточно  отчетливо  слышны  мотивы  приближающего
индивидуализма, эгоизма и утилитаризма, связанные с нарождающимися капиталистическими общественными
отношениями и господством частного интереса. Так, Лоренцо Балла (1406 — 1457) со всей определенностью
заявляет, что благоразумие и справедливость сводятся к выгоде индивида, на первом месте должны стоять свои
собственные интересы, а на последнем — родины. И вообще, по его мнению, сохраняет «свою силу славнейшее
изречение «там для меня родина, где хорошо».
Человек Нового времени в европейской философии.
Влияние  господства  частного  интереса  на  Представления  о  человеке,  мотивы  его  ведения  и  жизненные
установки  со всей  очевидностью  выражены в  концепции  Т.  Гоббса.  В противоположность  Аристотелю он
утверждает, что человек по природе своей — существо не общественное. Напротив, «человек человеку — волк»
(homo  homini  lupus  est),  а  «война  всех  против  всех»  является  естественным  состоянием  общества.  Его
методологический  индивидуализм  и  номинализм  тесно  связаны  с  социологическим  и  этическим
индивидуализмом. Глубинной же основой такого состояния является всеобщая конкуренция между людьми в
условиях новых экономических отношений. Сам он в этой связи пишет:
Человеческая жизнь может быть сравнима с состязанием в беге... единственная цель и единственная награда
каждого из участников, это — оказаться впереди своих конкурентов.
Влияние развития науки на представления о человеке и обусловленный им антропологический рационализм
ярко обнаруживаются в философских взглядах Б. Паскаля, который утверждал, что все величие и достоинство
человека «в его способности мыслить».
Однако основателем новоевропейского рационализма вообше и антропологического рационализма в частности 
по праву считается Р. Декарт. Согласно ему, мышление является единственно достоверным свидетельством
человеческого существования, что вытекает уже из его основополагающего тезиса: «мыслю, следовательно,
существую» («cogito ergo sum»). Кроме того, у философа наблюдается антропологический дуализм души и тела,
рассмотрение  их  как  двух  разнокачественных  субстанций,  имевших  большое  значение  для  разработки
психофизической  проблемы.  Согласно  Декарту,  тело  является  своего  рода  машиной,  тогда  как  сознание
воздействует на него и, в свою очередь, испытывает на себе его влияние.
Этот  механистический  взгляд  на  человека,  рассматриваемого  в  качестве  машины,  получил  широкое
распространение в тот период. Знаменем такой концепции может служить название работы Ж. Ламетри —
«Человек-машина»,  в  которой  представлена  точка  зрения  механистического  материализма  на  человека.
Согласно  ему,  существует  лишь  единая  материальная  субстанция,  а  человеческий  организм  —  это
самостоятельно заводящаяся машина, подобная часовому механизму.
Подобный взгляд характерен для всех французских материалистов XVIII в. (Гольбах, Гельвеции, Дидро).
Другая, отличительная черта их философской антропологии — рассмотрение человека как продукта природы,
абсолютно детерминированного ее законами, так что он «не может — даже в мысли выйти из природы». Стоя на
принципах  последовательного  механистического  детерминизма,  они,  конечно,  не  могли  ни  в  какой  мере
признать свободу воли человека. Еще одна характерная черта этих мыслителей состояла в том, что, критикуя
христианскую догматику об изначальной греховности человека, они утверждали, что человек по своей природе
изначально добр и не греховен.
Немецкая классическая философия
Основоположник  немецкой  классической  философии  И.  Кант  ставит  человека  в  центр  философских
исследований. Для него вопрос «Что такое человек?» является основным вопросом философии, а сам человек
— «самый главный предмет в мире». Подобно Декарту, Кант стоит на позиции антропологического дуализма,
но его дуализм — это не дуализм души и тела, а нравственно-природный дуализм. Человек, по Канту, с одной
стороны,  принадлежит  природной  необходимости,  а  с  другой  —  нравственной  свободе  и  абсолютным
ценностям.  Как  составная  часть  чувственного  мира  явлений  он  подчинен  необходимости,  а  как  носитель
духовности — он свободен. Но главная роль отводится Кантом нравственной деятельности человека.
Кант  стремится  утвердить  человека  в  качестве  автономного  и  независимого  начала  и  законодателя  своей
теоретической  и  практической  деятельности.  При  этом  исходным  принципом  поведения  должен  быть
категорический императив — формальное внутреннее повеление, требование, основанное на том, что всякая
личность является самоцелью и самодостаточна и поэтому не должна рассматриваться ни в коем случае как
средство осуществления каких бы то ни было даже очень благих задач.
Человек, пишет Кант, «по природе зол», но вместе с тем он обладает и задатками добра. Задача нравственного
воспитания и состоит в том, чтобы добрые задатки смогли одержать верх над изначально присущей человеку
склонностью ко злу. Хотя зло изначально преобладает, но задатки добра дают о себе знать в виде чувства вины,
которое овладевает людьми. Поэтому нормальный человек, по Канту, «никогда не свободен от вины», которая
составляет основу морали. Человек, который всегда прав и у которого всегда спокойная совесть, такой человек,
не может быть моральным. Основное отличие человека от других существ — самосознание. Из этого факта
вытекает и эгоизм как природное свойство человека, но философ выступает против эгоизма, в каких бы формах
он не проявлялся.
Антропологическая концепция Гегеля, как и вся его философия, проникнута рационализмом. Само отличие
человека от животного заключается прежде всего в мышлении, которое сообщает всему человеческому его
человечность. Он с наибольшей силой выразил положение о человеке как субъекте духовной деятельности и
носителе общезначимого духа и разума. Личность, в отличие от индивида, начинается только с осознания
человеком себя как существа «бесконечного, всеобщего и свободного». В социальном плане его учение ярко
выражает методологический и социологический коллективизм, то есть принцип приоритета социального целого
над  индивидом.  В  отличие  от  немецкого  идеализма  материалист  Л.  Фейербах  утверждает  самоценность  и
значимость  живого,  эмпирического  человека,  которого  он  понимает,  прежде  всего,  как  часть  природы,
чувственно-телесное  существо.  Антропологический  принцип,  являющийся  стержнем  всей  его  философии,
предполагает  именно  такое понимание человека.  Антропологический монизм  Фейербаха направлен  против
идеалистического понимания человека и дуализма души и тела и связан с утверждением материалистического
взгляда на его природу. Но самого человека Фейербах понимает слишком абстрактно. Его человек оказывается
изолированным  от  реальных  социальных  связей,  отношений  и  деятельности.  В  основе  его  философской
антропологии  лежат  отношения  между  Я  и  Ты,  при  этом  особенно  важными  в  этом  плане  оказываются
отношения между мужчиной и женщиной.
Антропологическая проблема в русской философии.
В истории русской философии можно в русской философии выделить два основных направления, касающихся
человека:
1) материалистические учения революционных демократов (Белинского, Герцена, Чернышевского и др.);
2) концепции представителей религиозной философии (Федорова, Вл. Соловьева, Бердяева и др.).
В развитии философских взглядов В.Г. Белинского проблема человека постепенно приобретает первостепенное
значение. В письме к Боткину от 1 марта 1841 г. он отмечает, что «судьба субъекта, индивидуума, личности  
важнее  судеб  всего  мира».  При  этом  достижение  свободы  и  независимости  личности  он  связывает  с
социальными преобразованиями, утверждая, что они возможны только в обществе, «основанном на правде и
доблести». Обоснование и утверждение необходимости развития личности и ее защиты приводят Белинского к
критике капитализма и религии и защите идей утопического социализма и атеизма.
Защиту идей «русского социализма» исходя из необходимости освобождения трудящегося человека, прежде
всего «мужика», предпринял А.И. Герцен. Его антропология рационалистична: человек вышел из «животного
сна» именно благодаря разуму. И чем больше соответствие между разумом и деятельностью, тем больше он
чувствует себя свободным. В вопросе о формировании личности он стоял на позиции ее взаимодействия с
социальной  средой.  В  частности,  он  писал,  что  личность  «создается  средой  и  событиями,  но  и  события
осуществляются личностями и носят на себе их печать; тут взаимодействие».
В  работе  «Антропологический  принцип  в  философии»  Н.Г.  Черны¬шевский  утверждает  природно-монистическую сущность человека. Человек — высшее произведение природы. На взгляды Чернышевского
оказало влияние учение Фейербаха, и многие недостатки последнего свойственны также и Чернышевскому.
Хотя,  в  отличие  от  Фейербаха,  он  вводит  в  учение  о  человеке  социальные  аспекты  человеческого
существования,  в  частности  связывает  решение  проблемы  человека  с  преобразованием  общества  на
социалистических началах. Как и всем представителям натуралистического направления философии человека,
ему присуща и натуралистическая трактовка духовной жизнедеятельности человека.
В концепциях русских религиозных философов антропологическая проблематика занимает центральное место.
Это особенно относится к периоду развития русской философии, начиная с Ф.М. Достоевского, являющегося
мыслителем экзистенциального склада и внесшего в развитие этого направления значительный вклад. И хотя
представители  этого  направления  постоянно  обращаются  к  Богу,  однако  в  центре  их  внимания  находится
человек, его предназначение и судьба. Слова Бердяева о Достоевском: «Его мысль занята антропологией, а не
теологией», можно отнести ко многим представителям русской религиозной философии.
В  основе  учения  о  человеке  в  русской  религиозной  философии  находится  вопрос  о  природе  и  сущности
человека. Его решение часто видится на пути дуализма души и тела, свободы и необходимости, добра и зла,
божественного и земного. Так, антропологические взгляды Достоевского зиждятся на той предпосылке, что
человек в своей глубинной сущности содержит два полярных начала — бога и дьявола, добро и зло, которые
проявляются особенно сильно, когда человек «отпущен на свободу».
Это  трагическое  противоречие  двух  начал  в  человеке  лежит  и  в  основе  философской  антропологии  Вл.
Соловьева.
Человек, — пишет он, — совмещает в себе всевозможные противоположности, которые все сводятся к одной
великой  противоположности  между  безусловным  и  условным,  между  абсолютною  и  вечною  сущностью  и
преходящим явлением или видимостью. Человек есть вместе и божество и ничтожество.
В не меньшей степени эта проблема души и тела отражена и в философии Н.А. Бердяева, который отмечает:
Человек есть микрокосм и микротеос. Он сотворен по образу и подобию Бога. Но в то же самое время человек
есть существо природное и ограниченное. В человеке есть двойственность: человек есть точка пересечения
двух миров, он отражает в себе мир высший и мир низший... В качестве существа плотского он связан со всем
круговоротом мировой жизни, как существо духовное он связан с миром духовным и с Богом».
В силу этой изначальной раздвоенности и дуализма человека его судьба оказывается трагичной по самой своей
сути.
Весь трагизм жизни, — пишет Бердяев, — происходит от столкновения конечного и бесконечного, временного и
вечного,  от  несоответствия  между  человеком,  как  духовным  существом,  и  человеком,  как  природным
существом, живущим в природном мире.
С  точки  зрения  представителей  этого  направления,  главное  для  человека  имеет  духовная,  божественная
субстанция, а подлинный смысл человека и его существования заключается в том, чтобы соединить человека с
Богом.  В  русской  религиозной  философии  вопрос  о  человеке  органически  превращается  в  божественный
вопрос, а вопрос о Боге — в человеческий. Человек раскрывает свою подлинную сущность в Боге, а Бог
проявляется в человеке. Отсюда одна из центральных проблем этого направления — проблема богочеловека,
или сверхчеловека. В отличие от концепции Ницше, у которого сверхчеловек — это человекобог, в русской
философии  сверхчеловек  —  это  богочеловек.  Ее  антропология  носит  сугубо  гуманистический  характер,
утверждая превосходство добра над злом и бога над дьяволом.


\newpage
\section{Проблема человека в экзистенциализме. Подлинное и неподлинное существование.}


\newpage
\section{Специфичность человеческого существования в мире (М.Шелер, Х.Плеснер, А.Гелен). }


\newpage
\section{Проблема свободы в истории европейской философии. Свобода и ответственность}
Все общественные отношения разделяются на первичные (материальные) и вторичные (духовно-практические).
В общественной жизни объективное и субъективное, практическое и духовное неразделимы.
Свободно-волевая деятельность и закономерности в обществе:
   * М.Вебер (субъективный идеализм): абсолютизировал неповторимость исторических событий и на этой
основе отверг наличие какой-либо закономерности в обществе. Но если в обществе нет объективной тенденции,
то невозможно прогнозирование событий, а стало быть, теряет смысл и цели социальное бытие.
    *  Марксизм:  в  обществе  имеют  место  объективная  необходимость,  причинная  обусловленность  и
повторяемость (т.е. общественная жизнь детерминирована) но существуют особенности общественных законов.
Человечество  всегда должно  будет корректировать  линию своего поведения, считаясь с законами  живой  и
неживой природы. Маркс: "люди не свободны в выборе своих производительных сил, которые образуют основу
всей их истории, потому что всякая производительная сила есть приобретенная сила, продукт предшествующей
деятельности". Свобода есть деятельность на основе познанной необходимости. Необходимость отражает нечто
устойчивое, упорядоченное, что и отражается в законах сохранения. Свобода же отражает развитие, появление
нового, разнообразного, новых возможностей. Необходимость выражает наличное, показывает, каков мир есть,
а свобода отражает будущее - каким мир должен быть. Развитие общества и есть постоянный процесс перехода
необходимости в свободу.
* Гегель: мировая история - процесс возрастания свободы. Свобода многолика, но сущность ее одна - наличие
разнообразных возможностей, следовательно, она - наибольшая ценность. Маркс: лишь при условии обретении
социальной свободы начинается развитие человеческих сил, которое является самоцелью.


\newpage
\section{Понятие культуры. Культура и природа. Культура и цивилизация}
Культура (лат. cultura, от корня colere — «возделывать») — обобщающее понятие для форм жизнедеятельности
человека, созданных и создаваемых нами в процессе эволюции. Культура — это нравственные, моральные и
материальные ценности, умения, знания, обычаи, традиции. 
Во многом, современное понятие «культуры» как цивилизации сформировалось в XVIII — начале XIX веков в
Западной  Европе.  В  дальнейшем,  это  понятие,  с  одной  стороны,  стало  включать  отличия  между  разными
группами людей в самой Европе, а с другой стороны — различия между метрополиями и их колониями по
всему миру. Отсюда то, что в данном случае понятие «культуры» является эквивалентом «цивилизации», то есть
антипода понятию «природы». Используя такое определение, можно с лёгкостью классифицировать отдельных
людей и даже целые страны по уровню цивилизованности. Отдельные авторы даже определяют культуру просто
как «всё лучшее в мире, что было создано и сказано» (Мэтью Арнольд), а всё что не попадает в это определение
— хаос и анархия. С этой точки зрения, культура тесно связана с социальным развитием и прогрессом в
обществе.  Арнольд  последовательно  использует  своё  определение:  «…культура  является  результатом
постоянного совершенствования, вытекающего из процессов получения знаний обо всём, что нас касается, её
составляет всё лучшее, что было сказано и помыслено» (Арнольд, 1882).
На практике, понятие культуры относится ко всем лучшим изделиям и поступкам, в том числе в области моды,
искусства и классической музыки. С этой точки зрения, в понятие «культурный» попадают люди, каким-либо
образом  связанные  с этими  областями. При  этом  люди, причастные к  классической  музыке, находятся  по
определению  на  более  высоком  уровне,  чем  любители  панк-рока  из  рабочих  кварталов  или  аборигены
Австралии.
Люди, поддерживающие такую точку зрения, зачастую отвергают множественное понимание культуры. Они не
верят в существование различных культур, каждой со своей логикой и своими ценностями. Фактически, для них
существует только один стандарт, который необходимо применять ко всем без исключения. Таким образом,
согласно  этому  мировоззрению,  люди,  не  укладывающиеся  в  общие  рамки,  сразу  причисляются  к
«некультурным», у них отбирается право на наличие «своей» культуры.
Однако,  в  рамках  такого  мировоззрения,  существует  своё  течение  —  где  менее  «культурные»  люди
рассматриваются,  во  многом,  как  более  «естественные»,  а  «высокой»  культуре  приписывается  подавление
«человеческой природы». Такая точка зрения встречается в работах многих авторов уже начиная с XVIII-го
века. Они, например, подчёркивают, что народная музыка (как созданная простыми людьми) честнее выражает
естественный образ  жизни,  в то  время как  классическая  музыка выглядит  поверхностной и  декадентской.
Следуя такому мнению, люди за пределами «западной цивилизации» — «благородные дикари», не испорченные
сильно расслоённым капитализмом Запада.
Сегодня большинство исследователей отвергают обе крайности. Они не принимают и понятие «единственно
правильной»  культуры,  так  и  полное  противопоставление  её  природе.  В  данном  случае  признаётся,  что
«неэлитарное» может обладать столь же высокой культурой, что и «элитарное», а «незападные» жители могут
быть  столь  же  культурными,  просто  культура  которых  выражается  другими  способами.  Однако  в  данной
концепции проводится различие между «высокой» культурой, как культурой элит, и «массовой» культурой,
подразумевающей  товары  и  произведения,  направленные  на  потребности  простых  людей.  Следует  также
отметить, что в некоторых работах оба вида культуры, «высокая» и «низкая», относятся просто к различным
субкультурам.
Одна из основных оппозиций философии, культурологии, социологии культуры. Оппозиция «К. — П.» является
предпосылкой подразделения всех наук на науки о культуре (К.) и науки о природе (П.), она лежит в основе
определения  предмета  культурологии  и  предмета  социологии  К.  Самым  общим  образом  К.  может  быть
охарактеризована  как  все  то,  что  создано  руками  и  умом  человека  в  процессе  его  исторической
жизнедеятельности, соответственно, П. — это все существующее и не созданное человеком. К. представляет
собой то, что не имело бы места и не способно было бы удерживаться в дальнейшем без постоянных усилий и
поддержки человека; П. — то, что не является результатом человеческой деятельности и может существовать
независимо от нее. Очевидно, что с расширением сферы К. область П. соответствующим образом сужается;
если К. хиреет, съеживается и гаснет, сфера природного расширяется.
К.включает материальную и духовную части. К К. относятся как созданные человеком здания, машины, каналы,
предметы повседневной жизни и т.п., так и созданные им идеи, ценности, религии, научные теории, нормы,
традиции, правила грамматики и ритуала, и т.п. К. находится в постоянной динамике, она радикально меняется
от эпохи к эпохе. Это означает, что постоянно меняется и понятие П., противопоставляемое понятию К. 
Широкое понимание К. как противоположности П. необходимо при обсуждении общих проблем становления и
развития К. и, в частности, проблемы видения П. конкретными К. Это понимание является в известном смысле
классическим, хотя оно не единственно возможное.


\newpage
\section{Культура элитарная и массовая. Феномен «восстания масс»}
Массовая культура формирует иную, ту, что называют высокой, или лучше — элитарной. Причем по разным
оценкам потребителями элитарной культуры в Европе на протяжении нескольких веков остается примерно одна
и та же доля населения — что-то около одного процента. Именно массовая культура — индикатор многих
сторон жизни общества и одновременно коллективный пропагандист и организатор его, общества, настроений.
Внутри  массовой  культуры  существует  своя  иерархия  ценностей  и  иерархия  персон.  Взвешенная  система
оценок и, наоборот, скандальные потасовки, драка за место у престола.
Массовая  культура  —  это  часть  общей  культуры,  отделенная  от  элитарной  лишь  большим  количеством
потребителей и социальной востребованностью. Эта определенность не строга, более того, объекты переходят
через эту условную границу довольно часто. Все остальные признаки подобного отделения только следуют из
количественного фактора.
Музыка Моцарта в зале филармонии остается явлением элитарной культуры, а та же мелодия в упрощенном
варианте, звучащая как сигнал вызова мобильного телефона,— явление культуры массовой.
Итак,  в  отношении  субъекта  творчества  -  восприятия  можно  выделить  народную  культуру,  элитарную  и
массовую (см.: массовая культура). При этом народная культура находится практически в стадии музеефикации
- консервации или превращается в сувенирный бизнес.
Элитарность и массовость имеют равное отношение как к феноменам Культуры. В самой массовой культуре
можно выделить, например, стихийно складывающуюся культуру под воздействием массы внешних факторов:
культуру  тоталитарную,  навязанную  массам  тем  или  иным  тоталитарным  режимом  (советским  в  СССР,
нацистским в Германии) и всячески поддерживаемую им. Искусство социалистического реализма является
одной из главных разновидностей такого искусства.
Возможна также фиксация внимания на функционировании и модификации традиционных видов искусства и
появлении  новых.  К  последним  относятся  фотоискусство,  кино,  телевидение,  видео-,  различные  виды
электронных искусств, компьютерное искусство и их всевозможные взаимо-соединения и комбинации. 

\subsection{Массовое общество и феномен «восстания масс»}
"Массовое  общество"  (англ.  mass  society),  понятие,  употребляемое  немарксистскими  социологами  и
философами  для  обозначения  ряда  специфических  черт  современного  общества.  В  области  социально-экономической "М. о."  связывается с индустриализацией и урбанизацией,  стандартизацией  производства  и 
массовым  потреблением,  бюрократизацией  общественной  жизни,  распространением  средств  массовой
коммуникации и "массовой культуры".
"ВОССТАНИЕ МАСС" ("La Rebelion de las masas", 1930) — работа Ортеги-и-Гассета. Философ констатирует,
что в современной Европе происходит явление "полного захвата массами общественной власти". "Масса", как
полагает Ортега-и-Гассет, есть "совокупность лиц, не выделенных ничем". По его мысли, плебейство и гнет
массы даже в традиционно элитарных кругах — характерный признак современности: "заурядные души, не
обманываясь насчет собственной заурядности, безбоязненно утверждают свое право на нее и навязывают ее
всем и всюду". Новоявленные политические режимы оказываются результатом "политического диктата масс". В
то же время, согласно убеждению Ортеги-и-Гассета, чем общество "аристократичней, тем в большей степени
оно  общество, как и  наоборот".  Массы, достигнув  сравнительно высокого  жизненного уровня, "вышли  из
повиновения, не подчиняются никакому меньшинству, не следуют за ним и не только не считаются с ним, но и
вытесняют его и  сами его замешают". Автор  акцентирует  призвание  людей "вечно быть  осужденными на
свободу, вечно решать, чем ты станешь в этом мире. И решать без устали и без передышки". Представителю же
массы  жизнь  представляется  "лишенной  преград":  "средний  человек  усваивает  как  истину,  что  все  люди
узаконенно равны". "Человек массы" получает удовлетворение от ощущения идентичности с себе подобными.
Его  душевный  склад  суть  типаж  избалованного  ребенка.  По  мысли  Ортеги-и-Гассета,  благородство
определяется  "требовательностью  и  долгом,  а  не  правами".  Личные  права  суть  "взятый  с  бою  рубеж".
"Всеобщие" же права типа "прав человека и гражданина", "обретаются по инерции, даром и за чужой счет,
раздаются  всем  поровну  и  не  требуют  усилий...  Всеобщими  правами  владеют,  а  личными  непрестанно
завладевают". Массовый человек полагает себя совершенным, "тирания пошлости в общественной жизни, быть
может,  самобытнейшая  черта  современности,  наименее  сопоставимая  с  прошлым.  Прежде  в  европейской
истории чернь никогда не заблуждалась насчет собственных идей касательно чего бы то ни было. Она ...не
присваивала себе умозрительных суждений — например, о политике или искусстве — и не определяла, что они
такое и чем должны стать... Никогда ей не взбредало в голову ни противопоставлять идеям политика свои, ни
даже  судить  их,  опираясь  на  некий  свод  идей,  признанных  своими...  Плебей  не  решался  даже  отдаленно
участвовать  почти  ни  в  какой  общественной  жизни,  по  большей  части  всегда  концептуальной.  Сегодня,
напротив,  у  среднего  человека  самые  неукоснительные  представления  обо  всем,  что  творится  и  должно
твориться во Вселенной". Как подчеркивает Ортега-и-Гассет, это "никоим образом" не прогресс: идеи массового
человека не есть культура, "культурой он не обзавелся": в Европе возникает "тип человека, который не желает
ни признавать, ни доказывать правоту, а намерен просто-напросто навязать свою волю". Это "Великая Хартия"
одичания: это агрессивное завоевание "права не быть правым". Человек, не желающий, не умеющий "ладить с
оппозицией", есть "дикарь, внезапно всплывший со дна цивилизации". 19 в. утратил "историческую культуру":
большевизм и фашизм... отчетливо представляют собой, согласно Ортеге-и-Гассету, движение вспять. Свою
долю исторической истины они используют "допотопно", антиисторически. Едва возникнув, они оказываются
"реликтовыми": "произошедшее в России исторически невыразительно, и не знаменует собой начало новой
жизни". Философ пишет: "Обе попытки — это ложные зори, у которых не будет завтрашнего утра". Ибо
"европейская история впервые оказалась отданной на откуп заурядности... Заурядность, прежде подвластная,
решила  властвовать".  "Специалисты",  узко  подготовленные  "ученые-невежды",  —  наитипичнейшие
представители "массового сознания". "Суть же достижений современной Европы в либеральной демократии и
технике. Главная же опасность Европы 1930-х, по мысли Ортеги-и-Гассета, "полностью огосударствленная
жизнь, экспансия власти, поглощение государством всякой социальной самостоятельности". Человека массы
вынудят жить для государственной машины. Высосав из него все соки, она умрет "самой мертвой из смертей —
ржавой смертью механизма".


\end{document}